% Options for packages loaded elsewhere
% Options for packages loaded elsewhere
\PassOptionsToPackage{unicode}{hyperref}
\PassOptionsToPackage{hyphens}{url}
\PassOptionsToPackage{dvipsnames,svgnames,x11names}{xcolor}
%
\documentclass[
  letterpaper,
  DIV=11,
  numbers=noendperiod]{scrreprt}
\usepackage{xcolor}
\usepackage{amsmath,amssymb}
\setcounter{secnumdepth}{-\maxdimen} % remove section numbering
\usepackage{iftex}
\ifPDFTeX
  \usepackage[T1]{fontenc}
  \usepackage[utf8]{inputenc}
  \usepackage{textcomp} % provide euro and other symbols
\else % if luatex or xetex
  \usepackage{unicode-math} % this also loads fontspec
  \defaultfontfeatures{Scale=MatchLowercase}
  \defaultfontfeatures[\rmfamily]{Ligatures=TeX,Scale=1}
\fi
\usepackage{lmodern}
\ifPDFTeX\else
  % xetex/luatex font selection
\fi
% Use upquote if available, for straight quotes in verbatim environments
\IfFileExists{upquote.sty}{\usepackage{upquote}}{}
\IfFileExists{microtype.sty}{% use microtype if available
  \usepackage[]{microtype}
  \UseMicrotypeSet[protrusion]{basicmath} % disable protrusion for tt fonts
}{}
\makeatletter
\@ifundefined{KOMAClassName}{% if non-KOMA class
  \IfFileExists{parskip.sty}{%
    \usepackage{parskip}
  }{% else
    \setlength{\parindent}{0pt}
    \setlength{\parskip}{6pt plus 2pt minus 1pt}}
}{% if KOMA class
  \KOMAoptions{parskip=half}}
\makeatother
% Make \paragraph and \subparagraph free-standing
\makeatletter
\ifx\paragraph\undefined\else
  \let\oldparagraph\paragraph
  \renewcommand{\paragraph}{
    \@ifstar
      \xxxParagraphStar
      \xxxParagraphNoStar
  }
  \newcommand{\xxxParagraphStar}[1]{\oldparagraph*{#1}\mbox{}}
  \newcommand{\xxxParagraphNoStar}[1]{\oldparagraph{#1}\mbox{}}
\fi
\ifx\subparagraph\undefined\else
  \let\oldsubparagraph\subparagraph
  \renewcommand{\subparagraph}{
    \@ifstar
      \xxxSubParagraphStar
      \xxxSubParagraphNoStar
  }
  \newcommand{\xxxSubParagraphStar}[1]{\oldsubparagraph*{#1}\mbox{}}
  \newcommand{\xxxSubParagraphNoStar}[1]{\oldsubparagraph{#1}\mbox{}}
\fi
\makeatother

\usepackage{color}
\usepackage{fancyvrb}
\newcommand{\VerbBar}{|}
\newcommand{\VERB}{\Verb[commandchars=\\\{\}]}
\DefineVerbatimEnvironment{Highlighting}{Verbatim}{commandchars=\\\{\}}
% Add ',fontsize=\small' for more characters per line
\usepackage{framed}
\definecolor{shadecolor}{RGB}{241,243,245}
\newenvironment{Shaded}{\begin{snugshade}}{\end{snugshade}}
\newcommand{\AlertTok}[1]{\textcolor[rgb]{0.68,0.00,0.00}{#1}}
\newcommand{\AnnotationTok}[1]{\textcolor[rgb]{0.37,0.37,0.37}{#1}}
\newcommand{\AttributeTok}[1]{\textcolor[rgb]{0.40,0.45,0.13}{#1}}
\newcommand{\BaseNTok}[1]{\textcolor[rgb]{0.68,0.00,0.00}{#1}}
\newcommand{\BuiltInTok}[1]{\textcolor[rgb]{0.00,0.23,0.31}{#1}}
\newcommand{\CharTok}[1]{\textcolor[rgb]{0.13,0.47,0.30}{#1}}
\newcommand{\CommentTok}[1]{\textcolor[rgb]{0.37,0.37,0.37}{#1}}
\newcommand{\CommentVarTok}[1]{\textcolor[rgb]{0.37,0.37,0.37}{\textit{#1}}}
\newcommand{\ConstantTok}[1]{\textcolor[rgb]{0.56,0.35,0.01}{#1}}
\newcommand{\ControlFlowTok}[1]{\textcolor[rgb]{0.00,0.23,0.31}{\textbf{#1}}}
\newcommand{\DataTypeTok}[1]{\textcolor[rgb]{0.68,0.00,0.00}{#1}}
\newcommand{\DecValTok}[1]{\textcolor[rgb]{0.68,0.00,0.00}{#1}}
\newcommand{\DocumentationTok}[1]{\textcolor[rgb]{0.37,0.37,0.37}{\textit{#1}}}
\newcommand{\ErrorTok}[1]{\textcolor[rgb]{0.68,0.00,0.00}{#1}}
\newcommand{\ExtensionTok}[1]{\textcolor[rgb]{0.00,0.23,0.31}{#1}}
\newcommand{\FloatTok}[1]{\textcolor[rgb]{0.68,0.00,0.00}{#1}}
\newcommand{\FunctionTok}[1]{\textcolor[rgb]{0.28,0.35,0.67}{#1}}
\newcommand{\ImportTok}[1]{\textcolor[rgb]{0.00,0.46,0.62}{#1}}
\newcommand{\InformationTok}[1]{\textcolor[rgb]{0.37,0.37,0.37}{#1}}
\newcommand{\KeywordTok}[1]{\textcolor[rgb]{0.00,0.23,0.31}{\textbf{#1}}}
\newcommand{\NormalTok}[1]{\textcolor[rgb]{0.00,0.23,0.31}{#1}}
\newcommand{\OperatorTok}[1]{\textcolor[rgb]{0.37,0.37,0.37}{#1}}
\newcommand{\OtherTok}[1]{\textcolor[rgb]{0.00,0.23,0.31}{#1}}
\newcommand{\PreprocessorTok}[1]{\textcolor[rgb]{0.68,0.00,0.00}{#1}}
\newcommand{\RegionMarkerTok}[1]{\textcolor[rgb]{0.00,0.23,0.31}{#1}}
\newcommand{\SpecialCharTok}[1]{\textcolor[rgb]{0.37,0.37,0.37}{#1}}
\newcommand{\SpecialStringTok}[1]{\textcolor[rgb]{0.13,0.47,0.30}{#1}}
\newcommand{\StringTok}[1]{\textcolor[rgb]{0.13,0.47,0.30}{#1}}
\newcommand{\VariableTok}[1]{\textcolor[rgb]{0.07,0.07,0.07}{#1}}
\newcommand{\VerbatimStringTok}[1]{\textcolor[rgb]{0.13,0.47,0.30}{#1}}
\newcommand{\WarningTok}[1]{\textcolor[rgb]{0.37,0.37,0.37}{\textit{#1}}}

\usepackage{longtable,booktabs,array}
\usepackage{calc} % for calculating minipage widths
% Correct order of tables after \paragraph or \subparagraph
\usepackage{etoolbox}
\makeatletter
\patchcmd\longtable{\par}{\if@noskipsec\mbox{}\fi\par}{}{}
\makeatother
% Allow footnotes in longtable head/foot
\IfFileExists{footnotehyper.sty}{\usepackage{footnotehyper}}{\usepackage{footnote}}
\makesavenoteenv{longtable}
\usepackage{graphicx}
\makeatletter
\newsavebox\pandoc@box
\newcommand*\pandocbounded[1]{% scales image to fit in text height/width
  \sbox\pandoc@box{#1}%
  \Gscale@div\@tempa{\textheight}{\dimexpr\ht\pandoc@box+\dp\pandoc@box\relax}%
  \Gscale@div\@tempb{\linewidth}{\wd\pandoc@box}%
  \ifdim\@tempb\p@<\@tempa\p@\let\@tempa\@tempb\fi% select the smaller of both
  \ifdim\@tempa\p@<\p@\scalebox{\@tempa}{\usebox\pandoc@box}%
  \else\usebox{\pandoc@box}%
  \fi%
}
% Set default figure placement to htbp
\def\fps@figure{htbp}
\makeatother





\setlength{\emergencystretch}{3em} % prevent overfull lines

\providecommand{\tightlist}{%
  \setlength{\itemsep}{0pt}\setlength{\parskip}{0pt}}



 


\KOMAoption{captions}{tableheading}
\makeatletter
\@ifpackageloaded{bookmark}{}{\usepackage{bookmark}}
\makeatother
\makeatletter
\@ifpackageloaded{caption}{}{\usepackage{caption}}
\AtBeginDocument{%
\ifdefined\contentsname
  \renewcommand*\contentsname{Table of contents}
\else
  \newcommand\contentsname{Table of contents}
\fi
\ifdefined\listfigurename
  \renewcommand*\listfigurename{List of Figures}
\else
  \newcommand\listfigurename{List of Figures}
\fi
\ifdefined\listtablename
  \renewcommand*\listtablename{List of Tables}
\else
  \newcommand\listtablename{List of Tables}
\fi
\ifdefined\figurename
  \renewcommand*\figurename{Figure}
\else
  \newcommand\figurename{Figure}
\fi
\ifdefined\tablename
  \renewcommand*\tablename{Table}
\else
  \newcommand\tablename{Table}
\fi
}
\@ifpackageloaded{float}{}{\usepackage{float}}
\floatstyle{ruled}
\@ifundefined{c@chapter}{\newfloat{codelisting}{h}{lop}}{\newfloat{codelisting}{h}{lop}[chapter]}
\floatname{codelisting}{Listing}
\newcommand*\listoflistings{\listof{codelisting}{List of Listings}}
\makeatother
\makeatletter
\makeatother
\makeatletter
\@ifpackageloaded{caption}{}{\usepackage{caption}}
\@ifpackageloaded{subcaption}{}{\usepackage{subcaption}}
\makeatother
\usepackage{bookmark}
\IfFileExists{xurl.sty}{\usepackage{xurl}}{} % add URL line breaks if available
\urlstyle{same}
\hypersetup{
  pdftitle={The Little Book of Python},
  pdfauthor={Duc-Tam Nguyen},
  colorlinks=true,
  linkcolor={blue},
  filecolor={Maroon},
  citecolor={Blue},
  urlcolor={Blue},
  pdfcreator={LaTeX via pandoc}}


\title{The Little Book of Python}
\usepackage{etoolbox}
\makeatletter
\providecommand{\subtitle}[1]{% add subtitle to \maketitle
  \apptocmd{\@title}{\par {\large #1 \par}}{}{}
}
\makeatother
\subtitle{Version 0.1.0}
\author{Duc-Tam Nguyen}
\date{2025-09-15}
\begin{document}
\maketitle

\renewcommand*\contentsname{Table of contents}
{
\hypersetup{linkcolor=}
\setcounter{tocdepth}{2}
\tableofcontents
}

\bookmarksetup{startatroot}

\chapter{The Little Book of Python}\label{the-little-book-of-python}

\section{Chapter 1. Basics of Python}\label{chapter-1.-basics-of-python}

\subsection{1. What is Python?}\label{what-is-python}

Python is a high-level, general-purpose programming language that
emphasizes simplicity and readability. Created by Guido van Rossum in
the early 1990s, Python was designed to make programming more accessible
by using a clean, English-like syntax. Unlike lower-level languages such
as C or assembly, Python abstracts away many technical details, allowing
developers to focus on solving problems rather than managing memory or
dealing with system-level operations.

Python is both interpreted and dynamically typed. Being interpreted
means that Python executes code line by line without requiring
compilation into machine code beforehand. This makes it very
beginner-friendly, as you can test and run your code quickly without
extra steps. Being dynamically typed means you do not need to declare
variable types explicitly---Python determines them at runtime, which
speeds up development.

The language is also cross-platform. A Python program written on macOS
can usually run on Linux or Windows with little to no changes, as long
as the environment has Python installed. Combined with its vast
ecosystem of libraries and frameworks, Python has become one of the most
popular languages worldwide, used in areas ranging from web development
to artificial intelligence.

Python's design philosophy emphasizes readability. For example, instead
of curly braces (\texttt{\{\}}) to mark blocks of code, Python uses
indentation (spaces or tabs). This enforces clean code structure and
makes programs easier to read and maintain.

\subsubsection{Deep Dive}\label{deep-dive}

\begin{itemize}
\item
  Versatility: Python is sometimes called a ``glue language'' because it
  can integrate with other systems and languages easily. You can call C
  or C++ libraries, run shell commands, or embed Python into other
  applications.
\item
  Community and ecosystem: With millions of developers worldwide, Python
  has a massive community. This means a wealth of tutorials, open-source
  projects, and support forums are available for learners and
  professionals.
\item
  Libraries and frameworks: Python has specialized libraries for nearly
  every domain:

  \begin{itemize}
  \tightlist
  \item
    Data Science \& AI: NumPy, Pandas, TensorFlow, PyTorch.
  \item
    Web Development: Django, Flask, FastAPI.
  \item
    Automation \& Scripting: Selenium, BeautifulSoup, \texttt{os} and
    \texttt{shutil} modules.
  \item
    Systems Programming: \texttt{subprocess}, \texttt{asyncio},
    threading tools.
  \end{itemize}
\item
  Design Philosophy: The ``Zen of Python'' (accessible by running
  \texttt{import\ this} in a Python shell) summarizes guiding
  principles, such as ``Simple is better than complex'' and
  ``Readability counts.''
\end{itemize}

Python's balance of simplicity and power makes it an excellent first
language for beginners, yet powerful enough for advanced engineers
building production-grade systems.

\subsubsection{Tiny Code}\label{tiny-code}

\begin{Shaded}
\begin{Highlighting}[]
\CommentTok{\# A simple Python program}
\BuiltInTok{print}\NormalTok{(}\StringTok{"Hello, World!"}\NormalTok{)}

\CommentTok{\# Variables don\textquotesingle{}t require type declarations}
\NormalTok{x }\OperatorTok{=} \DecValTok{10}       \CommentTok{\# integer}
\NormalTok{y }\OperatorTok{=} \FloatTok{3.14}     \CommentTok{\# float}
\NormalTok{name }\OperatorTok{=} \StringTok{"Ada"} \CommentTok{\# string}

\CommentTok{\# Control flow example}
\ControlFlowTok{if}\NormalTok{ x }\OperatorTok{\textgreater{}} \DecValTok{5}\NormalTok{:}
    \BuiltInTok{print}\NormalTok{(}\SpecialStringTok{f"}\SpecialCharTok{\{}\NormalTok{name}\SpecialCharTok{\}}\SpecialStringTok{, x is greater than 5!"}\NormalTok{)}
\end{Highlighting}
\end{Shaded}

\subsubsection{Why it Matters}\label{why-it-matters}

Python matters because it lowers the barrier to entry into programming.
Its readability and straightforward syntax make it an ideal starting
point for newcomers, while its depth and ecosystem allow professionals
to tackle complex problems in machine learning, finance, cybersecurity,
and more. Learning Python often serves as a gateway to the broader world
of computer science and software engineering.

\subsubsection{Try It Yourself}\label{try-it-yourself}

\begin{enumerate}
\def\labelenumi{\arabic{enumi}.}
\item
  Open a terminal or Python shell and type
  \texttt{print("Hello,\ Python!")}.
\item
  Assign a number to a variable and print it. Example:

\begin{Shaded}
\begin{Highlighting}[]
\NormalTok{age }\OperatorTok{=} \DecValTok{25}
\BuiltInTok{print}\NormalTok{(}\StringTok{"I am"}\NormalTok{, age, }\StringTok{"years old."}\NormalTok{)}
\end{Highlighting}
\end{Shaded}
\item
  Run \texttt{import\ this} in the Python shell and read through the Zen
  of Python. Which line resonates with you most, and why?
\end{enumerate}

This exercise introduces you to Python's core design philosophy while
letting you experience the simplicity of writing and running your first
code.

\subsection{2. Installing Python \& Running
Scripts}\label{installing-python-running-scripts}

Python is available for almost every operating system, and installing it
is the first step before you can write and execute your own programs.
Most modern computers already come with Python preinstalled, but often
it is not the latest version. For development, it is generally
recommended to use the most recent stable release (for example, Python
3.12).

\subsubsection{Deep Dive}\label{deep-dive-1}

Download and Install:

\begin{itemize}
\tightlist
\item
  On Windows, download the installer from the official website
  \href{https://www.python.org}{python.org}. During installation, make
  sure to check the box \emph{``Add Python to PATH''} so you can run
  Python from the command line.
\item
  On macOS, you can use Homebrew (\texttt{brew\ install\ python}) or
  download from python.org.
\item
  On Linux, Python is usually preinstalled. If not, use your package
  manager (\texttt{sudo\ apt\ install\ python3} on Ubuntu/Debian,
  \texttt{sudo\ dnf\ install\ python3} on Fedora).
\end{itemize}

After installation, open your terminal (or command prompt) and type:

\begin{Shaded}
\begin{Highlighting}[]
\ExtensionTok{python3} \AttributeTok{{-}{-}version}
\end{Highlighting}
\end{Shaded}

This should display something like \texttt{Python\ 3.14.0}. If it
doesn't, the installation or PATH configuration may need adjustment.

Running the Interpreter (REPL):

You can enter interactive mode by typing \texttt{python} or
\texttt{python3} in your terminal. This launches the Read-Eval-Print
Loop (REPL), where you can execute code line by line:

\begin{Shaded}
\begin{Highlighting}[]
\OperatorTok{\textgreater{}\textgreater{}\textgreater{}} \DecValTok{2} \OperatorTok{+} \DecValTok{3}
\DecValTok{5}
\OperatorTok{\textgreater{}\textgreater{}\textgreater{}} \BuiltInTok{print}\NormalTok{(}\StringTok{"Hello, Python!"}\NormalTok{)}
\NormalTok{Hello, Python}\OperatorTok{!}
\end{Highlighting}
\end{Shaded}

Running Scripts:

While the REPL is good for quick experiments, most real programs are
saved in files with a \texttt{.py} extension. You can create a file
\texttt{hello.py} containing:

\begin{Shaded}
\begin{Highlighting}[]
\BuiltInTok{print}\NormalTok{(}\StringTok{"Hello from a script!"}\NormalTok{)}
\end{Highlighting}
\end{Shaded}

Then run it from your terminal:

\begin{Shaded}
\begin{Highlighting}[]
\ExtensionTok{python3}\NormalTok{ hello.py}
\end{Highlighting}
\end{Shaded}

IDEs and Editors:

Beginners often start with editors like IDLE (which comes with Python)
or more advanced ones like VS Code or PyCharm, which provide syntax
highlighting, debugging tools, and project management.

Environment Management:

Installing libraries for one project can affect others. To avoid
conflicts, Python provides virtual environments (\texttt{venv}). This
isolates project dependencies:

\begin{Shaded}
\begin{Highlighting}[]
\ExtensionTok{python3} \AttributeTok{{-}m}\NormalTok{ venv myenv}
\BuiltInTok{source}\NormalTok{ myenv/bin/activate   }\CommentTok{\# On Linux/macOS}
\ExtensionTok{myenv\textbackslash{}Scripts\textbackslash{}activate}      \CommentTok{\# On Windows}
\end{Highlighting}
\end{Shaded}

\subsubsection{Tiny Code}\label{tiny-code-1}

\begin{Shaded}
\begin{Highlighting}[]
\CommentTok{\# File: hello.py}
\NormalTok{name }\OperatorTok{=} \StringTok{"Ada"}
\BuiltInTok{print}\NormalTok{(}\StringTok{"Hello,"}\NormalTok{, name)}
\end{Highlighting}
\end{Shaded}

To run:

\begin{Shaded}
\begin{Highlighting}[]
\ExtensionTok{python3}\NormalTok{ hello.py}
\end{Highlighting}
\end{Shaded}

\subsubsection{Why it Matters}\label{why-it-matters-1}

Understanding how to install Python and run scripts is fundamental
because it gives you control over your development environment. Without
mastering this, you can't progress to building real applications.
Installing properly also ensures you have access to the latest features
and security updates.

\subsubsection{Try It Yourself}\label{try-it-yourself-1}

\begin{enumerate}
\def\labelenumi{\arabic{enumi}.}
\tightlist
\item
  Install the latest version of Python on your computer.
\item
  Verify your installation with \texttt{python3\ -\/-version}.
\item
  Open the REPL and try basic arithmetic (\texttt{5\ *\ 7},
  \texttt{10\ /\ 2}).
\item
  Write a script called \texttt{greeting.py} that prints your name and
  favorite color.
\item
  Run the script from your terminal.
\end{enumerate}

This exercise ensures you can not only experiment interactively but also
save and execute complete programs.

\subsection{3. Python Syntax \&
Indentation}\label{python-syntax-indentation}

Python's syntax is designed to be simple and human-readable. Unlike many
other programming languages that use braces \texttt{\{\}} or keywords to
define code blocks, Python uses indentation (spaces or tabs). This is
not optional---correct indentation is part of Python's grammar. The
focus on clean and consistent code is one of the reasons why Python is
popular both in education and professional development.

\subsubsection{Deep Dive}\label{deep-dive-2}

\begin{itemize}
\item
  Indentation Instead of Braces: In languages like C, C++, or Java, you
  often see:

\begin{Shaded}
\begin{Highlighting}[]
\ControlFlowTok{if} \OperatorTok{(}\NormalTok{x }\OperatorTok{\textgreater{}} \DecValTok{0}\OperatorTok{)} \OperatorTok{\{}
\NormalTok{    printf}\OperatorTok{(}\StringTok{"Positive}\SpecialCharTok{\textbackslash{}n}\StringTok{"}\OperatorTok{);}
\OperatorTok{\}}
\end{Highlighting}
\end{Shaded}

  In Python, the same block is defined by indentation:

\begin{Shaded}
\begin{Highlighting}[]
\ControlFlowTok{if}\NormalTok{ x }\OperatorTok{\textgreater{}} \DecValTok{0}\NormalTok{:}
    \BuiltInTok{print}\NormalTok{(}\StringTok{"Positive"}\NormalTok{)}
\end{Highlighting}
\end{Shaded}

  The colon (\texttt{:}) signals the start of a new block, and the
  indented lines that follow belong to that block.
\item
  Consistency Matters: Python requires consistency in indentation. You
  cannot mix tabs and spaces within the same block. The most common
  convention is 4 spaces per indentation level.
\item
  Nested Indentation: Blocks can be nested by increasing indentation
  further:

\begin{Shaded}
\begin{Highlighting}[]
\ControlFlowTok{if}\NormalTok{ x }\OperatorTok{\textgreater{}} \DecValTok{0}\NormalTok{:}
    \ControlFlowTok{if}\NormalTok{ x }\OperatorTok{\%} \DecValTok{2} \OperatorTok{==} \DecValTok{0}\NormalTok{:}
        \BuiltInTok{print}\NormalTok{(}\StringTok{"Positive and even"}\NormalTok{)}
    \ControlFlowTok{else}\NormalTok{:}
        \BuiltInTok{print}\NormalTok{(}\StringTok{"Positive and odd"}\NormalTok{)}
\end{Highlighting}
\end{Shaded}
\item
  Syntax Simplicity: Python syntax avoids clutter. For example:

  \begin{itemize}
  \tightlist
  \item
    No need for semicolons (\texttt{;}) at the end of lines (though
    allowed).
  \item
    Parentheses are optional in control statements unless needed for
    clarity.
  \item
    Whitespace and line breaks matter, which encourages writing readable
    code.
  \end{itemize}
\item
  Line Continuation: Long lines can be split with
  \texttt{\textbackslash{}} or by wrapping expressions inside
  parentheses:

\begin{Shaded}
\begin{Highlighting}[]
\NormalTok{total }\OperatorTok{=}\NormalTok{ (}\DecValTok{100} \OperatorTok{+} \DecValTok{200} \OperatorTok{+} \DecValTok{300} \OperatorTok{+}
         \DecValTok{400} \OperatorTok{+} \DecValTok{500}\NormalTok{)}
\end{Highlighting}
\end{Shaded}
\item
  Comments: Python uses \texttt{\#} for single-line comments and triple
  quotes (\texttt{"""\ ...\ """}) for docstrings or multi-line comments.
\end{itemize}

\subsubsection{Tiny Code}\label{tiny-code-2}

\begin{Shaded}
\begin{Highlighting}[]
\CommentTok{\# Proper indentation example}
\NormalTok{score }\OperatorTok{=} \DecValTok{85}

\ControlFlowTok{if}\NormalTok{ score }\OperatorTok{\textgreater{}=} \DecValTok{60}\NormalTok{:}
    \BuiltInTok{print}\NormalTok{(}\StringTok{"Pass"}\NormalTok{)}
    \ControlFlowTok{if}\NormalTok{ score }\OperatorTok{\textgreater{}=} \DecValTok{90}\NormalTok{:}
        \BuiltInTok{print}\NormalTok{(}\StringTok{"Excellent"}\NormalTok{)}
    \ControlFlowTok{else}\NormalTok{:}
        \BuiltInTok{print}\NormalTok{(}\StringTok{"Good job"}\NormalTok{)}
\ControlFlowTok{else}\NormalTok{:}
    \BuiltInTok{print}\NormalTok{(}\StringTok{"Fail"}\NormalTok{)}
\end{Highlighting}
\end{Shaded}

\subsubsection{Why it Matters}\label{why-it-matters-2}

Indentation rules enforce consistency across all Python code. This
reduces errors caused by messy formatting and makes programs easier to
read, especially when working in teams. Python's syntax philosophy
ensures beginners learn clean habits from the start and professionals
maintain readability in large projects.

\subsubsection{Try It Yourself}\label{try-it-yourself-2}

\begin{enumerate}
\def\labelenumi{\arabic{enumi}.}
\tightlist
\item
  Write a program that checks if a number is positive, negative, or zero
  using proper indentation.
\item
  Experiment by removing indentation or mixing spaces and tabs---notice
  how Python raises an \texttt{IndentationError}.
\item
  Write nested \texttt{if} statements to check whether a number is
  divisible by both 2 and 3.
\end{enumerate}

This will help you experience firsthand why Python enforces indentation
and how it guides you to write clean, structured code.

\subsection{4. Variables \& Assignment}\label{variables-assignment}

In Python, a variable is like a box with a name where you can store
information. You can put numbers, text, or other kinds of data inside
that box, and later use the name of the box to get the value back.

Unlike some languages, you don't need to say what kind of data will go
inside the box---Python figures it out for you automatically.

\subsubsection{Deep Dive}\label{deep-dive-3}

\begin{itemize}
\item
  Creating a Variable: You just choose a name and use the equals sign
  \texttt{=} to assign a value:

\begin{Shaded}
\begin{Highlighting}[]
\NormalTok{age }\OperatorTok{=} \DecValTok{20}
\NormalTok{name }\OperatorTok{=} \StringTok{"Alice"}
\NormalTok{height }\OperatorTok{=} \FloatTok{1.75}
\end{Highlighting}
\end{Shaded}
\item
  Reassigning a Variable: You can change the value at any time:

\begin{Shaded}
\begin{Highlighting}[]
\NormalTok{age }\OperatorTok{=} \DecValTok{21}   \CommentTok{\# overwrites the old value}
\end{Highlighting}
\end{Shaded}
\item
  Naming Rules:

  \begin{itemize}
  \tightlist
  \item
    Names can include letters, numbers, and underscores (\texttt{\_}).
  \item
    They cannot start with a number.
  \item
    They are case-sensitive: \texttt{Age} and \texttt{age} are
    different.
  \item
    Use meaningful names, like \texttt{temperature}, instead of
    \texttt{t}.
  \end{itemize}
\item
  Dynamic Typing: Python does not require you to declare the type. The
  same variable can hold different types of data at different times:

\begin{Shaded}
\begin{Highlighting}[]
\NormalTok{x }\OperatorTok{=} \DecValTok{10}      \CommentTok{\# integer}
\NormalTok{x }\OperatorTok{=} \StringTok{"hello"} \CommentTok{\# now it\textquotesingle{}s a string}
\end{Highlighting}
\end{Shaded}
\item
  Multiple Assignments: You can assign several variables in one line:

\begin{Shaded}
\begin{Highlighting}[]
\NormalTok{a, b, c }\OperatorTok{=} \DecValTok{1}\NormalTok{, }\DecValTok{2}\NormalTok{, }\DecValTok{3}
\end{Highlighting}
\end{Shaded}
\item
  Swapping Values: Python makes it easy to swap values without a
  temporary variable:

\begin{Shaded}
\begin{Highlighting}[]
\NormalTok{a, b }\OperatorTok{=}\NormalTok{ b, a}
\end{Highlighting}
\end{Shaded}
\end{itemize}

\subsubsection{Tiny Code}\label{tiny-code-3}

\begin{Shaded}
\begin{Highlighting}[]
\CommentTok{\# Assign variables}
\NormalTok{name }\OperatorTok{=} \StringTok{"Ada"}
\NormalTok{age }\OperatorTok{=} \DecValTok{25}

\CommentTok{\# Print them}
\BuiltInTok{print}\NormalTok{(}\StringTok{"My name is"}\NormalTok{, name)}
\BuiltInTok{print}\NormalTok{(}\StringTok{"I am"}\NormalTok{, age, }\StringTok{"years old"}\NormalTok{)}
\end{Highlighting}
\end{Shaded}

\subsubsection{Why it Matters}\label{why-it-matters-3}

Variables let you store and reuse information in your programs. Without
variables, you would have to repeat values everywhere, making your code
harder to read and change. They are the foundation of all programming.

\subsubsection{Try It Yourself}\label{try-it-yourself-3}

\begin{enumerate}
\def\labelenumi{\arabic{enumi}.}
\item
  Create a variable called \texttt{color} and assign your favorite color
  as text.
\item
  Make a variable \texttt{number} and assign it any number you like.
\item
  Print both values in a sentence, like:

\begin{verbatim}
My favorite color is blue and my number is 7
\end{verbatim}
\item
  Try changing the values and run the program again.
\end{enumerate}

This will show you how variables make your code flexible and easy to
update.

\subsection{4. Variables \& Assignment}\label{variables-assignment-1}

In Python, a variable is like a box with a name where you can store
information. You can put numbers, text, or other kinds of data inside
that box, and later use the name of the box to get the value back.

Unlike some languages, you don't need to say what kind of data will go
inside the box---Python figures it out for you automatically.

\subsubsection{Deep Dive}\label{deep-dive-4}

To create a variable, you simply choose a name and use the equals sign
\texttt{=} to assign a value. For example:

\begin{Shaded}
\begin{Highlighting}[]
\NormalTok{age }\OperatorTok{=} \DecValTok{20}
\NormalTok{name }\OperatorTok{=} \StringTok{"Alice"}
\NormalTok{height }\OperatorTok{=} \FloatTok{1.75}
\end{Highlighting}
\end{Shaded}

You can also change the value at any time. For instance:

\begin{Shaded}
\begin{Highlighting}[]
\NormalTok{age }\OperatorTok{=} \DecValTok{21}   \CommentTok{\# overwrites the old value}
\end{Highlighting}
\end{Shaded}

Variable names have a few rules. They can include letters, numbers, and
underscores (\texttt{\_}), but they cannot start with a number. They are
also case-sensitive, so \texttt{Age} and \texttt{age} are considered
different. It's a good habit to use meaningful names, like
\texttt{temperature} instead of just \texttt{t}.

Python uses dynamic typing, which means you don't have to declare the
type of data in advance. A single variable can hold different types of
data at different times:

\begin{Shaded}
\begin{Highlighting}[]
\NormalTok{x }\OperatorTok{=} \DecValTok{10}      \CommentTok{\# integer}
\NormalTok{x }\OperatorTok{=} \StringTok{"hello"} \CommentTok{\# now it\textquotesingle{}s a string}
\end{Highlighting}
\end{Shaded}

You can even assign several variables in one line, like this:

\begin{Shaded}
\begin{Highlighting}[]
\NormalTok{a, b, c }\OperatorTok{=} \DecValTok{1}\NormalTok{, }\DecValTok{2}\NormalTok{, }\DecValTok{3}
\end{Highlighting}
\end{Shaded}

And if you ever need to swap the values of two variables, Python makes
it very easy without needing a temporary helper:

\begin{Shaded}
\begin{Highlighting}[]
\NormalTok{a, b }\OperatorTok{=}\NormalTok{ b, a}
\end{Highlighting}
\end{Shaded}

\subsubsection{Tiny Code}\label{tiny-code-4}

\begin{Shaded}
\begin{Highlighting}[]
\CommentTok{\# Assign variables}
\NormalTok{name }\OperatorTok{=} \StringTok{"Ada"}
\NormalTok{age }\OperatorTok{=} \DecValTok{25}

\CommentTok{\# Print them}
\BuiltInTok{print}\NormalTok{(}\StringTok{"My name is"}\NormalTok{, name)}
\BuiltInTok{print}\NormalTok{(}\StringTok{"I am"}\NormalTok{, age, }\StringTok{"years old"}\NormalTok{)}
\end{Highlighting}
\end{Shaded}

\subsubsection{Why it Matters}\label{why-it-matters-4}

Variables let you store and reuse information in your programs. Without
variables, you would have to repeat values everywhere, making your code
harder to read and change. They are the foundation of all programming.

\subsubsection{Try It Yourself}\label{try-it-yourself-4}

\begin{enumerate}
\def\labelenumi{\arabic{enumi}.}
\item
  Create a variable called \texttt{color} and assign your favorite color
  as text.
\item
  Make a variable \texttt{number} and assign it any number you like.
\item
  Print both values in a sentence, like:

\begin{verbatim}
My favorite color is blue and my number is 7
\end{verbatim}
\item
  Try changing the values and run the program again.
\end{enumerate}

This will show you how variables make your code flexible and easy to
update.

\subsection{5. Data Types Overview}\label{data-types-overview}

Every piece of information in Python has a data type. A data type tells
Python what kind of thing the value is---whether it's a number, text, a
list of items, or something else. Understanding data types is important
because it helps you know what you can and cannot do with a value.

\subsubsection{Deep Dive}\label{deep-dive-5}

Python has several basic data types you'll use all the time.

Numbers are used for math. Python has three main kinds of numbers:
integers (\texttt{int}) for whole numbers, floating-point numbers
(\texttt{float}) for decimals, and complex numbers (\texttt{complex})
which are used less often, mostly in math and engineering.

Strings (\texttt{str}) represent text. Anything inside quotes, either
single (\texttt{\textquotesingle{}hello\textquotesingle{}}) or double
(\texttt{"hello"}), is treated as a string. Strings can hold words,
sentences, or even whole paragraphs.

Booleans (\texttt{bool}) represent truth values---either \texttt{True}
or \texttt{False}. These are useful for decision making in programs,
like checking if a condition is met.

Collections let you store multiple values in a single variable. Lists
(\texttt{list}) are ordered, changeable collections of items, like
\texttt{{[}1,\ 2,\ 3{]}}. Tuples (\texttt{tuple}) are like lists but
cannot be changed after creation, such as \texttt{(1,\ 2,\ 3)}. Sets
(\texttt{set}) are collections of unique, unordered items. Dictionaries
(\texttt{dict}) store data as key--value pairs, like
\texttt{\{"name":\ "Alice",\ "age":\ 25\}}.

There are also special types like \texttt{NoneType}, which only has the
value \texttt{None}. This represents ``nothing'' or ``no value.''

Python figures out the type of a variable automatically. If you want to
check a variable's type, you can use the built-in \texttt{type()}
function:

\begin{Shaded}
\begin{Highlighting}[]
\NormalTok{x }\OperatorTok{=} \DecValTok{42}
\BuiltInTok{print}\NormalTok{(}\BuiltInTok{type}\NormalTok{(x))  }\CommentTok{\# \textless{}class \textquotesingle{}int\textquotesingle{}\textgreater{}}
\end{Highlighting}
\end{Shaded}

\subsubsection{Tiny Code}\label{tiny-code-5}

\begin{Shaded}
\begin{Highlighting}[]
\CommentTok{\# Examples of different data types}
\NormalTok{number }\OperatorTok{=} \DecValTok{10}          \CommentTok{\# int}
\NormalTok{pi }\OperatorTok{=} \FloatTok{3.14}            \CommentTok{\# float}
\NormalTok{name }\OperatorTok{=} \StringTok{"Ada"}         \CommentTok{\# str}
\NormalTok{is\_student }\OperatorTok{=} \VariableTok{True}    \CommentTok{\# bool}
\NormalTok{items }\OperatorTok{=}\NormalTok{ [}\DecValTok{1}\NormalTok{, }\DecValTok{2}\NormalTok{, }\DecValTok{3}\NormalTok{]    }\CommentTok{\# list}
\NormalTok{point }\OperatorTok{=}\NormalTok{ (}\DecValTok{2}\NormalTok{, }\DecValTok{3}\NormalTok{)       }\CommentTok{\# tuple}
\NormalTok{unique }\OperatorTok{=}\NormalTok{ \{}\DecValTok{1}\NormalTok{, }\DecValTok{2}\NormalTok{, }\DecValTok{3}\NormalTok{\}   }\CommentTok{\# set}
\NormalTok{person }\OperatorTok{=}\NormalTok{ \{}\StringTok{"name"}\NormalTok{: }\StringTok{"Ada"}\NormalTok{, }\StringTok{"age"}\NormalTok{: }\DecValTok{25}\NormalTok{\}  }\CommentTok{\# dict}
\NormalTok{nothing }\OperatorTok{=} \VariableTok{None}       \CommentTok{\# NoneType}

\BuiltInTok{print}\NormalTok{(}\BuiltInTok{type}\NormalTok{(name))    }\CommentTok{\# check type}
\end{Highlighting}
\end{Shaded}

\subsubsection{Why it Matters}\label{why-it-matters-5}

Data types are the foundation of programming logic. Knowing the type of
data tells you what operations you can perform. For example, you can add
two numbers but not a number and a string without converting one of
them. This prevents errors and helps you design programs correctly.

\subsubsection{Try It Yourself}\label{try-it-yourself-5}

\begin{enumerate}
\def\labelenumi{\arabic{enumi}.}
\tightlist
\item
  Create a variable \texttt{city} with the name of your city.
\item
  Make a list called \texttt{colors} with three of your favorite colors.
\item
  Create a dictionary \texttt{book} with keys \texttt{title} and
  \texttt{author}.
\item
  Print out the type of each variable using \texttt{type()}.
\item
  Try combining different types (like adding a string and a number) and
  see what error appears.
\end{enumerate}

This will give you a feel for how Python handles different data and why
types matter.

\subsection{6. Numbers (int, float,
complex)}\label{numbers-int-float-complex}

Numbers are one of the most basic building blocks in Python. They allow
you to do math, represent quantities, and calculate results in your
programs. Python has three main types of numbers: integers
(\texttt{int}), floating-point numbers (\texttt{float}), and complex
numbers (\texttt{complex}).

\subsubsection{Deep Dive}\label{deep-dive-6}

Number Types in Python

\begin{longtable}[]{@{}
  >{\raggedright\arraybackslash}p{(\linewidth - 4\tabcolsep) * \real{0.0900}}
  >{\raggedright\arraybackslash}p{(\linewidth - 4\tabcolsep) * \real{0.1500}}
  >{\raggedright\arraybackslash}p{(\linewidth - 4\tabcolsep) * \real{0.7600}}@{}}
\toprule\noalign{}
\begin{minipage}[b]{\linewidth}\raggedright
Type
\end{minipage} & \begin{minipage}[b]{\linewidth}\raggedright
Example
\end{minipage} & \begin{minipage}[b]{\linewidth}\raggedright
Description
\end{minipage} \\
\midrule\noalign{}
\endhead
\bottomrule\noalign{}
\endlastfoot
\texttt{int} & \texttt{-3}, \texttt{0}, \texttt{42} & Whole numbers, no
decimal part. Can be very large (only limited by memory). \\
\texttt{float} & \texttt{3.14}, \texttt{-0.5} & Numbers with decimal
points, often used for measurements or precision math. \\
\texttt{complex} & \texttt{2\ +\ 3j} & Numbers with real and imaginary
parts, useful in math, physics, engineering. \\
\end{longtable}

Common Arithmetic Operators

\begin{longtable}[]{@{}
  >{\raggedright\arraybackslash}p{(\linewidth - 6\tabcolsep) * \real{0.1311}}
  >{\raggedright\arraybackslash}p{(\linewidth - 6\tabcolsep) * \real{0.1311}}
  >{\raggedright\arraybackslash}p{(\linewidth - 6\tabcolsep) * \real{0.0984}}
  >{\raggedright\arraybackslash}p{(\linewidth - 6\tabcolsep) * \real{0.6393}}@{}}
\toprule\noalign{}
\begin{minipage}[b]{\linewidth}\raggedright
Operator
\end{minipage} & \begin{minipage}[b]{\linewidth}\raggedright
Example
\end{minipage} & \begin{minipage}[b]{\linewidth}\raggedright
Result
\end{minipage} & \begin{minipage}[b]{\linewidth}\raggedright
Meaning
\end{minipage} \\
\midrule\noalign{}
\endhead
\bottomrule\noalign{}
\endlastfoot
\texttt{+} & \texttt{5\ +\ 2} & \texttt{7} & Addition \\
\texttt{-} & \texttt{5\ -\ 2} & \texttt{3} & Subtraction \\
\texttt{*} & \texttt{5\ *\ 2} & \texttt{10} & Multiplication \\
\texttt{/} & \texttt{5\ /\ 2} & \texttt{2.5} & Division (always
float) \\
\texttt{//} & \texttt{5\ //\ 2} & \texttt{2} & Floor division (whole
number part only) \\
\texttt{\%} & \texttt{5\ \%\ 2} & \texttt{1} & Modulo (remainder) \\
`\texttt{\textbar{}}2 3\texttt{\textbar{}}8` & Exponent (raise to a
power) & & \\
\end{longtable}

Type Conversion

\begin{longtable}[]{@{}lll@{}}
\toprule\noalign{}
Function & Example & Result \\
\midrule\noalign{}
\endhead
\bottomrule\noalign{}
\endlastfoot
\texttt{int()} & \texttt{int(3.9)} & \texttt{3} \\
\texttt{float()} & \texttt{float(7)} & \texttt{7.0} \\
\texttt{complex()} & \texttt{complex(2,\ 3)} & \texttt{2+3j} \\
\end{longtable}

You can check the type of any number with the \texttt{type()} function:

\begin{Shaded}
\begin{Highlighting}[]
\NormalTok{x }\OperatorTok{=} \DecValTok{42}
\BuiltInTok{print}\NormalTok{(}\BuiltInTok{type}\NormalTok{(x))  }\CommentTok{\# \textless{}class \textquotesingle{}int\textquotesingle{}\textgreater{}}
\end{Highlighting}
\end{Shaded}

\subsubsection{Tiny Code}\label{tiny-code-6}

\begin{Shaded}
\begin{Highlighting}[]
\CommentTok{\# Integers}
\NormalTok{a }\OperatorTok{=} \DecValTok{10}
\NormalTok{b }\OperatorTok{=} \OperatorTok{{-}}\DecValTok{3}

\CommentTok{\# Floats}
\NormalTok{pi }\OperatorTok{=} \FloatTok{3.14}
\NormalTok{g }\OperatorTok{=} \FloatTok{9.81}

\CommentTok{\# Complex}
\NormalTok{z }\OperatorTok{=} \DecValTok{2} \OperatorTok{+} \OtherTok{3j}

\CommentTok{\# Operations}
\BuiltInTok{print}\NormalTok{(a }\OperatorTok{+}\NormalTok{ b)    }\CommentTok{\# 7}
\BuiltInTok{print}\NormalTok{(a }\OperatorTok{/} \DecValTok{2}\NormalTok{)    }\CommentTok{\# 5.0}
\BuiltInTok{print}\NormalTok{(a }\OperatorTok{//} \DecValTok{2}\NormalTok{)   }\CommentTok{\# 5}
\BuiltInTok{print}\NormalTok{(a }\OperatorTok{\%} \DecValTok{3}\NormalTok{)    }\CommentTok{\# 1}
\BuiltInTok{print}\NormalTok{(}\DecValTok{2}  \DecValTok{3}\NormalTok{)   }\CommentTok{\# 8}

\CommentTok{\# Type checking}
\BuiltInTok{print}\NormalTok{(}\BuiltInTok{type}\NormalTok{(pi)) }\CommentTok{\# \textless{}class \textquotesingle{}float\textquotesingle{}\textgreater{}}
\BuiltInTok{print}\NormalTok{(}\BuiltInTok{type}\NormalTok{(z))  }\CommentTok{\# \textless{}class \textquotesingle{}complex\textquotesingle{}\textgreater{}}
\end{Highlighting}
\end{Shaded}

\subsubsection{Why it Matters}\label{why-it-matters-6}

Numbers are essential for everything from simple calculations to complex
algorithms. Understanding the different numeric types and how they
behave allows you to choose the right one for each situation. Use
integers for counting, floats for precise measurements, and complex
numbers for specialized scientific work.

\subsubsection{Try It Yourself}\label{try-it-yourself-6}

\begin{enumerate}
\def\labelenumi{\arabic{enumi}.}
\tightlist
\item
  Create two integers and try all the arithmetic operators (\texttt{+},
  \texttt{-}, \texttt{*}, \texttt{/}, \texttt{//}, \texttt{\%}, ``).
\item
  Make a float variable for your height (like \texttt{1.75}) and
  multiply it by 2.
\item
  Experiment with \texttt{int()}, \texttt{float()}, and
  \texttt{complex()} to convert between number types.
\item
  Write a complex number and print both its real and imaginary parts
  using \texttt{.real} and \texttt{.imag}.
\end{enumerate}

This will help you see how Python handles different numeric types in
practice.

\subsection{7. Strings (creation \&
basics)}\label{strings-creation-basics}

A string in Python is a sequence of characters---letters, numbers,
symbols, or even spaces---enclosed in quotes. Strings are used whenever
you want to work with text, such as names, sentences, or file paths.

\subsubsection{Deep Dive}\label{deep-dive-7}

Creating Strings

You can create strings using either single quotes or double quotes:

\begin{Shaded}
\begin{Highlighting}[]
\NormalTok{name }\OperatorTok{=} \StringTok{\textquotesingle{}Alice\textquotesingle{}}
\NormalTok{greeting }\OperatorTok{=} \StringTok{"Hello, world!"}
\end{Highlighting}
\end{Shaded}

For multi-line text, you can use triple quotes:

\begin{Shaded}
\begin{Highlighting}[]
\NormalTok{paragraph }\OperatorTok{=} \StringTok{"""This is a }
\StringTok{multi{-}line string."""}
\end{Highlighting}
\end{Shaded}

Basic String Operations

\begin{longtable}[]{@{}lll@{}}
\toprule\noalign{}
Operation & Example & Result \\
\midrule\noalign{}
\endhead
\bottomrule\noalign{}
\endlastfoot
Concatenation & \texttt{"Hello"\ +\ "\ "\ +\ "Bob"} &
\texttt{"Hello\ Bob"} \\
Repetition & \texttt{"ha"\ *\ 3} & \texttt{"hahaha"} \\
Indexing & \texttt{"Python"{[}0{]}} &
\texttt{\textquotesingle{}P\textquotesingle{}} \\
Negative Indexing & \texttt{"Python"{[}-1{]}} &
\texttt{\textquotesingle{}n\textquotesingle{}} \\
Slicing & \texttt{"Python"{[}0:4{]}} & \texttt{"Pyth"} \\
Length & \texttt{len("Python")} & \texttt{6} \\
\end{longtable}

Escape Characters

Sometimes you need special characters inside a string:

\begin{longtable}[]{@{}
  >{\raggedright\arraybackslash}p{(\linewidth - 6\tabcolsep) * \real{0.1803}}
  >{\raggedright\arraybackslash}p{(\linewidth - 6\tabcolsep) * \real{0.1967}}
  >{\raggedright\arraybackslash}p{(\linewidth - 6\tabcolsep) * \real{0.3279}}
  >{\raggedright\arraybackslash}p{(\linewidth - 6\tabcolsep) * \real{0.2951}}@{}}
\toprule\noalign{}
\begin{minipage}[b]{\linewidth}\raggedright
Escape Code
\end{minipage} & \begin{minipage}[b]{\linewidth}\raggedright
Meaning
\end{minipage} & \begin{minipage}[b]{\linewidth}\raggedright
Example
\end{minipage} & \begin{minipage}[b]{\linewidth}\raggedright
Result
\end{minipage} \\
\midrule\noalign{}
\endhead
\bottomrule\noalign{}
\endlastfoot
\texttt{\textbackslash{}n} & New line &
\texttt{"Hello\textbackslash{}nWorld"} & \texttt{Hello}\texttt{World} \\
\texttt{\textbackslash{}t} & Tab & \texttt{"A\textbackslash{}tB"} &
\texttt{A\ \ \ \ B} \\
\texttt{\textbackslash{}\textquotesingle{}} & Single quote &
\texttt{\textquotesingle{}It\textbackslash{}\textquotesingle{}s\ fine\textquotesingle{}}
& \texttt{It\textquotesingle{}s\ fine} \\
\texttt{\textbackslash{}"} & Double quote &
\texttt{"He\ said\ \textbackslash{}"Hi\textbackslash{}""} &
\texttt{He\ said\ "Hi"} \\
\texttt{\textbackslash{}\textbackslash{}} & Backslash &
\texttt{"C:\textbackslash{}\textbackslash{}Users\textbackslash{}\textbackslash{}Alice"}
& \texttt{C:\textbackslash{}Users\textbackslash{}Alice} \\
\end{longtable}

\subsubsection{Tiny Code}\label{tiny-code-7}

\begin{Shaded}
\begin{Highlighting}[]
\CommentTok{\# Creating strings}
\NormalTok{word }\OperatorTok{=} \StringTok{"Python"}
\NormalTok{sentence }\OperatorTok{=} \StringTok{\textquotesingle{}I love coding\textquotesingle{}}
\NormalTok{multiline }\OperatorTok{=} \StringTok{"""This is}
\StringTok{a string that spans}
\StringTok{multiple lines."""}

\CommentTok{\# Operations}
\BuiltInTok{print}\NormalTok{(word[}\DecValTok{0}\NormalTok{])        }\CommentTok{\# \textquotesingle{}P\textquotesingle{}}
\BuiltInTok{print}\NormalTok{(word[}\OperatorTok{{-}}\DecValTok{1}\NormalTok{])       }\CommentTok{\# \textquotesingle{}n\textquotesingle{}}
\BuiltInTok{print}\NormalTok{(word[}\DecValTok{0}\NormalTok{:}\DecValTok{3}\NormalTok{])      }\CommentTok{\# \textquotesingle{}Pyt\textquotesingle{}}
\BuiltInTok{print}\NormalTok{(word }\OperatorTok{+} \StringTok{" 3.12"}\NormalTok{) }\CommentTok{\# \textquotesingle{}Python 3.12\textquotesingle{}}
\BuiltInTok{print}\NormalTok{(}\StringTok{"ha"} \OperatorTok{*} \DecValTok{4}\NormalTok{)       }\CommentTok{\# \textquotesingle{}hahaha\textquotesingle{}}

\CommentTok{\# Escape characters}
\NormalTok{path }\OperatorTok{=} \StringTok{"C:}\CharTok{\textbackslash{}\textbackslash{}}\StringTok{Users}\CharTok{\textbackslash{}\textbackslash{}}\StringTok{Alice"}
\BuiltInTok{print}\NormalTok{(path)}
\end{Highlighting}
\end{Shaded}

\subsubsection{Why it Matters}\label{why-it-matters-7}

Strings are everywhere---whether you're printing messages, reading
files, sending data across the internet, or handling user input.
Mastering how to create and manipulate strings is essential for building
real-world Python programs.

\subsubsection{Try It Yourself}\label{try-it-yourself-7}

\begin{enumerate}
\def\labelenumi{\arabic{enumi}.}
\tightlist
\item
  Create a string with your full name and print the first letter and the
  last letter.
\item
  Write a sentence and use slicing to print only the first 5 characters.
\item
  Use string concatenation to join \texttt{"Hello"} and your name with a
  space in between.
\item
  Make a string with an escape sequence, like
  \texttt{"Line1\textbackslash{}nLine2"}, and print it.
\end{enumerate}

This practice will help you understand how Python treats text as data
you can store, manipulate, and display.

\subsection{8. Booleans and Truth
Values}\label{booleans-and-truth-values}

Booleans are the simplest type of data in Python. They represent only
two values: \texttt{True} or \texttt{False}. Booleans are often the
result of comparisons or conditions in a program, and they control the
flow of logic, such as deciding which branch of an \texttt{if} statement
should run.

\subsubsection{Deep Dive}\label{deep-dive-8}

Boolean Values

In Python, the boolean type is \texttt{bool}. There are only two
possible values:

\begin{Shaded}
\begin{Highlighting}[]
\NormalTok{is\_sunny }\OperatorTok{=} \VariableTok{True}
\NormalTok{is\_raining }\OperatorTok{=} \VariableTok{False}
\end{Highlighting}
\end{Shaded}

Notice that \texttt{True} and \texttt{False} are capitalized---writing
\texttt{true} or \texttt{false} will cause an error.

Comparisons That Produce Booleans

\begin{longtable}[]{@{}lll@{}}
\toprule\noalign{}
Expression & Example & Result \\
\midrule\noalign{}
\endhead
\bottomrule\noalign{}
\endlastfoot
Equal & \texttt{5\ ==\ 5} & \texttt{True} \\
Not equal & \texttt{5\ !=\ 3} & \texttt{True} \\
Greater than & \texttt{7\ \textgreater{}\ 10} & \texttt{False} \\
Less than & \texttt{2\ \textless{}\ 5} & \texttt{True} \\
Greater/Equal & \texttt{3\ \textgreater{}=\ 3} & \texttt{True} \\
Less/Equal & \texttt{4\ \textless{}=\ 2} & \texttt{False} \\
\end{longtable}

Boolean Logic

Python also supports logical operators that combine boolean values:

\begin{longtable}[]{@{}lll@{}}
\toprule\noalign{}
Operator & Example & Result \\
\midrule\noalign{}
\endhead
\bottomrule\noalign{}
\endlastfoot
\texttt{and} & \texttt{True\ and\ False} & \texttt{False} \\
\texttt{or} & \texttt{True\ or\ False} & \texttt{True} \\
\texttt{not} & \texttt{not\ True} & \texttt{False} \\
\end{longtable}

Truthiness in Python

Not just \texttt{True} and \texttt{False} are considered booleans. Many
values in Python have an implicit boolean value:

\begin{longtable}[]{@{}ll@{}}
\toprule\noalign{}
Value Type & Considered as \\
\midrule\noalign{}
\endhead
\bottomrule\noalign{}
\endlastfoot
\texttt{0}, \texttt{0.0}, \texttt{0j} & \texttt{False} \\
Empty string \texttt{""} & \texttt{False} \\
Empty list \texttt{{[}{]}} & \texttt{False} \\
Empty dict \texttt{\{\}} & \texttt{False} \\
\texttt{None} & \texttt{False} \\
Everything else & \texttt{True} \\
\end{longtable}

You can test this with the \texttt{bool()} function:

\begin{Shaded}
\begin{Highlighting}[]
\BuiltInTok{print}\NormalTok{(}\BuiltInTok{bool}\NormalTok{(}\DecValTok{0}\NormalTok{))     }\CommentTok{\# False}
\BuiltInTok{print}\NormalTok{(}\BuiltInTok{bool}\NormalTok{(}\StringTok{"hi"}\NormalTok{))  }\CommentTok{\# True}
\end{Highlighting}
\end{Shaded}

\subsubsection{Tiny Code}\label{tiny-code-8}

\begin{Shaded}
\begin{Highlighting}[]
\NormalTok{x }\OperatorTok{=} \DecValTok{10}
\NormalTok{y }\OperatorTok{=} \DecValTok{20}

\BuiltInTok{print}\NormalTok{(x }\OperatorTok{\textless{}}\NormalTok{ y)          }\CommentTok{\# True}
\BuiltInTok{print}\NormalTok{(x }\OperatorTok{==}\NormalTok{ y)         }\CommentTok{\# False}
\BuiltInTok{print}\NormalTok{((x }\OperatorTok{\textless{}}\NormalTok{ y) }\KeywordTok{and}\NormalTok{ (y }\OperatorTok{\textgreater{}} \DecValTok{5}\NormalTok{))  }\CommentTok{\# True}
\BuiltInTok{print}\NormalTok{(}\KeywordTok{not}\NormalTok{ (x }\OperatorTok{\textgreater{}}\NormalTok{ y))    }\CommentTok{\# True}

\CommentTok{\# Truthiness}
\BuiltInTok{print}\NormalTok{(}\BuiltInTok{bool}\NormalTok{(}\StringTok{""}\NormalTok{))       }\CommentTok{\# False}
\BuiltInTok{print}\NormalTok{(}\BuiltInTok{bool}\NormalTok{(}\StringTok{"Python"}\NormalTok{)) }\CommentTok{\# True}
\end{Highlighting}
\end{Shaded}

\subsubsection{Why it Matters}\label{why-it-matters-8}

Booleans are the foundation of decision-making in programming. They let
you write programs that can react differently depending on
conditions---like checking if a user is logged in, if there is enough
money in a bank account, or if a file exists. Without booleans, all
programs would just run straight through without making choices.

\subsubsection{Try It Yourself}\label{try-it-yourself-8}

\begin{enumerate}
\def\labelenumi{\arabic{enumi}.}
\tightlist
\item
  Assign a boolean variable \texttt{is\_python\_fun\ =\ True} and print
  it.
\item
  Compare two numbers (like \texttt{5\ \textgreater{}\ 3}) and store the
  result in a variable. Print the variable.
\item
  Test the truthiness of an empty list \texttt{{[}{]}} and a non-empty
  list \texttt{{[}1,\ 2,\ 3{]}} with \texttt{bool()}.
\item
  Write an expression using \texttt{and}, \texttt{or}, and \texttt{not}
  together.
\end{enumerate}

This practice will help you see how conditions and logic form the
backbone of Python programs.

\subsection{9. Comments in Python}\label{comments-in-python}

Comments are notes you add to your code that Python ignores when running
the program. They're meant for humans, not the computer. Comments
explain what your code does, why you wrote it a certain way, or leave
reminders for yourself and others.

\subsubsection{Deep Dive}\label{deep-dive-9}

Single-Line Comments In Python, the \texttt{\#} symbol marks the start
of a comment. Everything after it on the same line is ignored by Python:

\begin{Shaded}
\begin{Highlighting}[]
\CommentTok{\# This is a single{-}line comment}
\NormalTok{x }\OperatorTok{=} \DecValTok{10}  \CommentTok{\# You can also put a comment after code}
\end{Highlighting}
\end{Shaded}

Multi-Line Comments (Docstrings) Python doesn't have a special syntax
just for multi-line comments, but programmers often use triple quotes
(\texttt{"""} or
\texttt{\textquotesingle{}\textquotesingle{}\textquotesingle{}}). These
are usually used for docstrings (documentation strings), but they can
serve as block comments if not assigned to a variable:

\begin{Shaded}
\begin{Highlighting}[]
\CommentTok{"""}
\CommentTok{This is a multi{-}line comment.}
\CommentTok{You can use triple quotes}
\CommentTok{to write long explanations.}
\CommentTok{"""}
\end{Highlighting}
\end{Shaded}

Docstrings for Functions and Classes Triple quotes are more commonly
used as docstrings to document functions, classes, or modules. They are
placed right after the definition line:

\begin{Shaded}
\begin{Highlighting}[]
\KeywordTok{def}\NormalTok{ greet(name):}
    \CommentTok{"""}
\CommentTok{    This function takes a name}
\CommentTok{    and prints a greeting.}
\CommentTok{    """}
    \BuiltInTok{print}\NormalTok{(}\StringTok{"Hello,"}\NormalTok{, name)}
\end{Highlighting}
\end{Shaded}

You can read docstrings later using the \texttt{help()} function.

Why Comments Are Useful

\begin{longtable}[]{@{}
  >{\raggedright\arraybackslash}p{(\linewidth - 2\tabcolsep) * \real{0.3444}}
  >{\raggedright\arraybackslash}p{(\linewidth - 2\tabcolsep) * \real{0.6556}}@{}}
\toprule\noalign{}
\begin{minipage}[b]{\linewidth}\raggedright
Purpose
\end{minipage} & \begin{minipage}[b]{\linewidth}\raggedright
Example
\end{minipage} \\
\midrule\noalign{}
\endhead
\bottomrule\noalign{}
\endlastfoot
Explain code logic & \texttt{\#\ Loop\ through\ items\ in\ the\ list} \\
Clarify tricky parts &
\texttt{\#\ Using\ floor\ division\ to\ ignore\ decimals} \\
Leave reminders (TODOs, FIXMEs) &
\texttt{\#\ TODO:\ handle\ negative\ numbers} \\
Provide documentation & Docstrings that explain functions, classes, or
entire files \\
\end{longtable}

Good comments don't just repeat the code; they explain the why, not just
the what.

\subsubsection{Tiny Code}\label{tiny-code-9}

\begin{Shaded}
\begin{Highlighting}[]
\CommentTok{\# Store a user\textquotesingle{}s age}
\NormalTok{age }\OperatorTok{=} \DecValTok{25}

\CommentTok{\# Check if age is greater than 18}
\ControlFlowTok{if}\NormalTok{ age }\OperatorTok{\textgreater{}} \DecValTok{18}\NormalTok{:}
    \BuiltInTok{print}\NormalTok{(}\StringTok{"Adult"}\NormalTok{)}

\KeywordTok{def}\NormalTok{ square(x):}
    \CommentTok{"""Return the square of a number."""}
    \ControlFlowTok{return}\NormalTok{ x }\OperatorTok{*}\NormalTok{ x}

\BuiltInTok{print}\NormalTok{(square(}\DecValTok{4}\NormalTok{))  }\CommentTok{\# prints 16}
\end{Highlighting}
\end{Shaded}

\subsubsection{Why it Matters}\label{why-it-matters-9}

Comments make your code easier to understand for both yourself and
others. Six months from now, you might forget why you wrote something.
Clear comments act like a guidebook. In teams, comments and docstrings
are essential for collaboration, as they make the codebase easier to
maintain.

\subsubsection{Try It Yourself}\label{try-it-yourself-9}

\begin{enumerate}
\def\labelenumi{\arabic{enumi}.}
\tightlist
\item
  Write a small program that calculates the area of a rectangle. Add
  comments explaining what each step does.
\item
  Use a triple-quoted docstring to describe what the whole program does
  at the top of your file.
\item
  Add a \texttt{TODO} comment to remind yourself to improve the program
  later (for example, adding user input).
\end{enumerate}

This will show you how comments make programs not just for computers,
but for people too.

\subsection{\texorpdfstring{10. Printing Output (\texttt{print}
function)}{10. Printing Output (print function)}}\label{printing-output-print-function}

The \texttt{print()} function is one of the most commonly used tools in
Python. It lets you display information on the screen so you can see the
result of your program, check values, or interact with users.

\subsubsection{Deep Dive}\label{deep-dive-10}

Basic Printing The simplest use of \texttt{print()} is to show text:

\begin{Shaded}
\begin{Highlighting}[]
\BuiltInTok{print}\NormalTok{(}\StringTok{"Hello, world!"}\NormalTok{)}
\end{Highlighting}
\end{Shaded}

Printing Variables You can print variables directly by passing them to
\texttt{print()}:

\begin{Shaded}
\begin{Highlighting}[]
\NormalTok{name }\OperatorTok{=} \StringTok{"Ada"}
\NormalTok{age }\OperatorTok{=} \DecValTok{25}
\BuiltInTok{print}\NormalTok{(name)}
\BuiltInTok{print}\NormalTok{(age)}
\end{Highlighting}
\end{Shaded}

Printing Multiple Values \texttt{print()} can take multiple arguments
separated by commas. Python will add spaces between them automatically:

\begin{Shaded}
\begin{Highlighting}[]
\BuiltInTok{print}\NormalTok{(}\StringTok{"Name:"}\NormalTok{, name, }\StringTok{"Age:"}\NormalTok{, age)}
\end{Highlighting}
\end{Shaded}

String Formatting There are several ways to make your output more
readable:

\begin{longtable}[]{@{}
  >{\raggedright\arraybackslash}p{(\linewidth - 4\tabcolsep) * \real{0.2368}}
  >{\raggedright\arraybackslash}p{(\linewidth - 4\tabcolsep) * \real{0.4868}}
  >{\raggedright\arraybackslash}p{(\linewidth - 4\tabcolsep) * \real{0.2763}}@{}}
\toprule\noalign{}
\begin{minipage}[b]{\linewidth}\raggedright
Method
\end{minipage} & \begin{minipage}[b]{\linewidth}\raggedright
Example
\end{minipage} & \begin{minipage}[b]{\linewidth}\raggedright
Output
\end{minipage} \\
\midrule\noalign{}
\endhead
\bottomrule\noalign{}
\endlastfoot
f-strings (modern) &
\texttt{print(f"\{name\}\ is\ \{age\}\ years\ old")} &
\texttt{Ada\ is\ 25\ years\ old} \\
\texttt{.format()} method &
\texttt{print("\{\}\ is\ \{\}".format(name,\ age))} &
\texttt{Ada\ is\ 25} \\
Old \texttt{\%} style & \texttt{print("\%s\ is\ \%d"\ \%\ (name,\ age))}
& \texttt{Ada\ is\ 25} \\
\end{longtable}

End and Separator Options By default, \texttt{print()} ends with a new
line (\texttt{\textbackslash{}n}). You can change this using the
\texttt{end} parameter:

\begin{Shaded}
\begin{Highlighting}[]
\BuiltInTok{print}\NormalTok{(}\StringTok{"Hello"}\NormalTok{, end}\OperatorTok{=}\StringTok{" "}\NormalTok{)}
\BuiltInTok{print}\NormalTok{(}\StringTok{"World"}\NormalTok{)}
\CommentTok{\# Output: Hello World}
\end{Highlighting}
\end{Shaded}

You can also change the separator between multiple items using
\texttt{sep}:

\begin{Shaded}
\begin{Highlighting}[]
\BuiltInTok{print}\NormalTok{(}\StringTok{"apple"}\NormalTok{, }\StringTok{"banana"}\NormalTok{, }\StringTok{"cherry"}\NormalTok{, sep}\OperatorTok{=}\StringTok{", "}\NormalTok{)}
\CommentTok{\# Output: apple, banana, cherry}
\end{Highlighting}
\end{Shaded}

Printing Special Characters You can print new lines or tabs with escape
sequences:

\begin{Shaded}
\begin{Highlighting}[]
\BuiltInTok{print}\NormalTok{(}\StringTok{"Line1}\CharTok{\textbackslash{}n}\StringTok{Line2"}\NormalTok{)}
\BuiltInTok{print}\NormalTok{(}\StringTok{"A}\CharTok{\textbackslash{}t}\StringTok{B"}\NormalTok{)}
\end{Highlighting}
\end{Shaded}

\subsubsection{Tiny Code}\label{tiny-code-10}

\begin{Shaded}
\begin{Highlighting}[]
\NormalTok{name }\OperatorTok{=} \StringTok{"Grace"}
\NormalTok{language }\OperatorTok{=} \StringTok{"Python"}
\NormalTok{year }\OperatorTok{=} \DecValTok{1991}

\BuiltInTok{print}\NormalTok{(}\StringTok{"Hello, world!"}\NormalTok{)}
\BuiltInTok{print}\NormalTok{(}\StringTok{"My name is"}\NormalTok{, name)}
\BuiltInTok{print}\NormalTok{(}\SpecialStringTok{f"}\SpecialCharTok{\{}\NormalTok{name}\SpecialCharTok{\}}\SpecialStringTok{ created }\SpecialCharTok{\{}\NormalTok{language}\SpecialCharTok{\}}\SpecialStringTok{ in }\SpecialCharTok{\{}\NormalTok{year}\SpecialCharTok{\}}\SpecialStringTok{?"}\NormalTok{)}
\BuiltInTok{print}\NormalTok{(}\StringTok{"apple"}\NormalTok{, }\StringTok{"orange"}\NormalTok{, }\StringTok{"grape"}\NormalTok{, sep}\OperatorTok{=}\StringTok{" | "}\NormalTok{)}
\end{Highlighting}
\end{Shaded}

\subsubsection{Why it Matters}\label{why-it-matters-10}

Printing is the most direct way to see what your program is doing. It
helps you understand results, debug mistakes, and communicate with
users. Even professional developers rely heavily on \texttt{print()}
when testing and exploring code quickly.

\subsubsection{Try It Yourself}\label{try-it-yourself-10}

\begin{enumerate}
\def\labelenumi{\arabic{enumi}.}
\tightlist
\item
  Print your name and your favorite hobby in one sentence.
\item
  Create two numbers and print their sum with a clear message.
\item
  Use \texttt{sep} to print three words separated by dashes
  (\texttt{-}).
\item
  Use \texttt{end} to print two words on the same line without spaces.
\end{enumerate}

This will show you how flexible \texttt{print()} is for displaying
information in Python.

\section{Chapter 2. Control Flow}\label{chapter-2.-control-flow}

\subsection{11. Comparison Operators}\label{comparison-operators}

Comparison operators let you compare two values and return a boolean
result (\texttt{True} or \texttt{False}). They are the foundation for
making decisions in Python programs---without them, you couldn't check
conditions like ``Is this number bigger than that number?'' or ``Are
these two things equal?''

\subsubsection{Deep Dive}\label{deep-dive-11}

Comparison operators work on numbers, strings, and many other types.
They allow you to check equality, inequality, and order.

Basic Comparison Operators

\begin{longtable}[]{@{}llll@{}}
\toprule\noalign{}
Operator & Example & Meaning & Result \\
\midrule\noalign{}
\endhead
\bottomrule\noalign{}
\endlastfoot
\texttt{==} & \texttt{5\ ==\ 5} & Equal to & \texttt{True} \\
\texttt{!=} & \texttt{5\ !=\ 3} & Not equal to & \texttt{True} \\
\texttt{\textgreater{}} & \texttt{7\ \textgreater{}\ 3} & Greater than &
\texttt{True} \\
\texttt{\textless{}} & \texttt{2\ \textless{}\ 5} & Less than &
\texttt{True} \\
\texttt{\textgreater{}=} & \texttt{3\ \textgreater{}=\ 3} & Greater than
or equal to & \texttt{True} \\
\texttt{\textless{}=} & \texttt{4\ \textless{}=\ 2} & Less than or equal
to & \texttt{False} \\
\end{longtable}

Comparisons always return \texttt{True} or \texttt{False}, which can be
stored in variables or used directly inside control flow statements
(\texttt{if}, \texttt{while}).

Chained Comparisons Python allows chaining comparisons for readability:

\begin{Shaded}
\begin{Highlighting}[]
\NormalTok{x }\OperatorTok{=} \DecValTok{5}
\BuiltInTok{print}\NormalTok{(}\DecValTok{1} \OperatorTok{\textless{}}\NormalTok{ x }\OperatorTok{\textless{}} \DecValTok{10}\NormalTok{)  }\CommentTok{\# True}
\BuiltInTok{print}\NormalTok{(}\DecValTok{10} \OperatorTok{\textless{}}\NormalTok{ x }\OperatorTok{\textless{}} \DecValTok{20}\NormalTok{) }\CommentTok{\# False}
\end{Highlighting}
\end{Shaded}

This is equivalent to writing
\texttt{(1\ \textless{}\ x)\ and\ (x\ \textless{}\ 10)}.

Comparisons with Strings Strings are compared alphabetically
(lexicographically), based on Unicode values:

\begin{Shaded}
\begin{Highlighting}[]
\BuiltInTok{print}\NormalTok{(}\StringTok{"apple"} \OperatorTok{==} \StringTok{"apple"}\NormalTok{)  }\CommentTok{\# True}
\BuiltInTok{print}\NormalTok{(}\StringTok{"apple"} \OperatorTok{\textless{}} \StringTok{"banana"}\NormalTok{)  }\CommentTok{\# True}
\BuiltInTok{print}\NormalTok{(}\StringTok{"Zebra"} \OperatorTok{\textless{}} \StringTok{"apple"}\NormalTok{)   }\CommentTok{\# True (uppercase letters come first)}
\end{Highlighting}
\end{Shaded}

\subsubsection{Tiny Code}\label{tiny-code-11}

\begin{Shaded}
\begin{Highlighting}[]
\NormalTok{x }\OperatorTok{=} \DecValTok{10}
\NormalTok{y }\OperatorTok{=} \DecValTok{20}

\BuiltInTok{print}\NormalTok{(x }\OperatorTok{==}\NormalTok{ y)   }\CommentTok{\# False}
\BuiltInTok{print}\NormalTok{(x }\OperatorTok{!=}\NormalTok{ y)   }\CommentTok{\# True}
\BuiltInTok{print}\NormalTok{(x }\OperatorTok{\textgreater{}}\NormalTok{ y)    }\CommentTok{\# False}
\BuiltInTok{print}\NormalTok{(x }\OperatorTok{\textless{}=}\NormalTok{ y)   }\CommentTok{\# True}

\CommentTok{\# Chain comparisons}
\BuiltInTok{print}\NormalTok{(}\DecValTok{5} \OperatorTok{\textless{}}\NormalTok{ x }\OperatorTok{\textless{}} \DecValTok{15}\NormalTok{)  }\CommentTok{\# True}
\end{Highlighting}
\end{Shaded}

\subsubsection{Why it Matters}\label{why-it-matters-11}

Without comparisons, programs couldn't make choices. They are the basis
for decisions like checking passwords, validating input, controlling
loops, or comparing values in data. Every real-world Python program
relies on comparison operators to ``decide what to do next.''

\subsubsection{Try It Yourself}\label{try-it-yourself-11}

\begin{enumerate}
\def\labelenumi{\arabic{enumi}.}
\tightlist
\item
  Write a program that compares two numbers (\texttt{a\ =\ 7},
  \texttt{b\ =\ 12}) and prints whether \texttt{a} is less than, greater
  than, or equal to \texttt{b}.
\item
  Create two strings and check if they are equal.
\item
  Use a chained comparison to check if a number \texttt{n\ =\ 15} is
  between 10 and 20.
\item
  Experiment with \texttt{\textless{}} and \texttt{\textgreater{}} on
  strings like \texttt{"cat"} and \texttt{"dog"} to see how Python
  compares text.
\end{enumerate}

\subsection{12. Logical Operators}\label{logical-operators}

Logical operators combine boolean values (\texttt{True} or
\texttt{False}) to form more complex conditions. They are essential when
you want to check multiple things at once, like ``Is the number positive
and even?'' or ``Is this user an admin or a guest?''

\subsubsection{Deep Dive}\label{deep-dive-12}

Python has three main logical operators:

\begin{longtable}[]{@{}
  >{\raggedright\arraybackslash}p{(\linewidth - 6\tabcolsep) * \real{0.1194}}
  >{\raggedright\arraybackslash}p{(\linewidth - 6\tabcolsep) * \real{0.2388}}
  >{\raggedright\arraybackslash}p{(\linewidth - 6\tabcolsep) * \real{0.1045}}
  >{\raggedright\arraybackslash}p{(\linewidth - 6\tabcolsep) * \real{0.5373}}@{}}
\toprule\noalign{}
\begin{minipage}[b]{\linewidth}\raggedright
Operator
\end{minipage} & \begin{minipage}[b]{\linewidth}\raggedright
Example
\end{minipage} & \begin{minipage}[b]{\linewidth}\raggedright
Result
\end{minipage} & \begin{minipage}[b]{\linewidth}\raggedright
Meaning
\end{minipage} \\
\midrule\noalign{}
\endhead
\bottomrule\noalign{}
\endlastfoot
\texttt{and} & \texttt{True\ and\ False} & \texttt{False} & True only if
both sides are True \\
\texttt{or} & \texttt{True\ or\ False} & \texttt{True} & True if at
least one side is True \\
\texttt{not} & \texttt{not\ True} & \texttt{False} & Flips the truth
value (True → False) \\
\end{longtable}

Truth Tables

\texttt{and} operator:

\begin{longtable}[]{@{}lll@{}}
\toprule\noalign{}
A & B & A and B \\
\midrule\noalign{}
\endhead
\bottomrule\noalign{}
\endlastfoot
True & True & True \\
True & False & False \\
False & True & False \\
False & False & False \\
\end{longtable}

\texttt{or} operator:

\begin{longtable}[]{@{}lll@{}}
\toprule\noalign{}
A & B & A or B \\
\midrule\noalign{}
\endhead
\bottomrule\noalign{}
\endlastfoot
True & True & True \\
True & False & True \\
False & True & True \\
False & False & False \\
\end{longtable}

\texttt{not} operator:

\begin{longtable}[]{@{}ll@{}}
\toprule\noalign{}
A & not A \\
\midrule\noalign{}
\endhead
\bottomrule\noalign{}
\endlastfoot
True & False \\
False & True \\
\end{longtable}

Combining Conditions Logical operators are often used in \texttt{if}
statements:

\begin{Shaded}
\begin{Highlighting}[]
\NormalTok{age }\OperatorTok{=} \DecValTok{20}
\NormalTok{is\_student }\OperatorTok{=} \VariableTok{True}

\ControlFlowTok{if}\NormalTok{ age }\OperatorTok{\textgreater{}} \DecValTok{18} \KeywordTok{and}\NormalTok{ is\_student:}
    \BuiltInTok{print}\NormalTok{(}\StringTok{"Eligible for student discount"}\NormalTok{)}
\end{Highlighting}
\end{Shaded}

Short-Circuiting Python stops evaluating as soon as the result is known:

\begin{itemize}
\tightlist
\item
  For \texttt{and}, if the first condition is False, Python won't check
  the second.
\item
  For \texttt{or}, if the first condition is True, Python won't check
  the second.
\end{itemize}

\subsubsection{Tiny Code}\label{tiny-code-12}

\begin{Shaded}
\begin{Highlighting}[]
\NormalTok{x }\OperatorTok{=} \DecValTok{10}
\NormalTok{y }\OperatorTok{=} \DecValTok{5}

\BuiltInTok{print}\NormalTok{(x }\OperatorTok{\textgreater{}} \DecValTok{0} \KeywordTok{and}\NormalTok{ y }\OperatorTok{\textgreater{}} \DecValTok{0}\NormalTok{)   }\CommentTok{\# True}
\BuiltInTok{print}\NormalTok{(x }\OperatorTok{\textgreater{}} \DecValTok{0} \KeywordTok{or}\NormalTok{ y }\OperatorTok{\textless{}} \DecValTok{0}\NormalTok{)    }\CommentTok{\# True}
\BuiltInTok{print}\NormalTok{(}\KeywordTok{not}\NormalTok{ (x }\OperatorTok{==}\NormalTok{ y))      }\CommentTok{\# True}

\CommentTok{\# Short{-}circuit example}
\BuiltInTok{print}\NormalTok{(}\VariableTok{False} \KeywordTok{and}\NormalTok{ (}\DecValTok{10}\OperatorTok{/}\DecValTok{0}\NormalTok{))  }\CommentTok{\# False, no error (second part skipped)}
\BuiltInTok{print}\NormalTok{(}\VariableTok{True} \KeywordTok{or}\NormalTok{ (}\DecValTok{10}\OperatorTok{/}\DecValTok{0}\NormalTok{))    }\CommentTok{\# True, no error (second part skipped)}
\end{Highlighting}
\end{Shaded}

\subsubsection{Why it Matters}\label{why-it-matters-12}

Logical operators allow your programs to make more complex decisions by
combining multiple conditions. They're at the heart of all real-world
logic, from validating form inputs to controlling access in
applications.

\subsubsection{Try It Yourself}\label{try-it-yourself-12}

\begin{enumerate}
\def\labelenumi{\arabic{enumi}.}
\tightlist
\item
  Write a condition that checks if a number is both positive and less
  than 100.
\item
  Check if a variable \texttt{name} is either \texttt{"Alice"} or
  \texttt{"Bob"}.
\item
  Use \texttt{not} to test if a list is empty (\texttt{not\ my\_list}).
\item
  Experiment with short-circuiting by combining \texttt{and} or
  \texttt{or} with expressions that would normally cause an error.
\end{enumerate}

\subsection{\texorpdfstring{13. \texttt{if}
Statements}{13. if Statements}}\label{if-statements}

An \texttt{if} statement lets your program make decisions. It checks a
condition, and if that condition is \texttt{True}, it runs a block of
code. If the condition is \texttt{False}, the block is skipped. This is
the most basic form of control flow in Python.

\subsubsection{Deep Dive}\label{deep-dive-13}

Basic Structure

\begin{Shaded}
\begin{Highlighting}[]
\ControlFlowTok{if}\NormalTok{ condition:}
    \CommentTok{\# code runs only if condition is True}
\end{Highlighting}
\end{Shaded}

The colon (\texttt{:}) signals the start of the block, and indentation
shows which lines belong to the \texttt{if}.

Example

\begin{Shaded}
\begin{Highlighting}[]
\NormalTok{x }\OperatorTok{=} \DecValTok{10}
\ControlFlowTok{if}\NormalTok{ x }\OperatorTok{\textgreater{}} \DecValTok{5}\NormalTok{:}
    \BuiltInTok{print}\NormalTok{(}\StringTok{"x is greater than 5"}\NormalTok{)}
\end{Highlighting}
\end{Shaded}

Since \texttt{x\ \textgreater{}\ 5} is \texttt{True}, the message is
printed.

Condition Can Be Any Boolean Expression The expression inside
\texttt{if} must evaluate to \texttt{True} or \texttt{False}. This can
come from comparisons, logical operators, or truthy/falsy values:

\begin{Shaded}
\begin{Highlighting}[]
\ControlFlowTok{if} \StringTok{"hello"}\NormalTok{:     }\CommentTok{\# non{-}empty string is True}
    \BuiltInTok{print}\NormalTok{(}\StringTok{"This runs"}\NormalTok{)}
\end{Highlighting}
\end{Shaded}

Indentation is Required All code inside the \texttt{if} block must be
indented the same amount. Without correct indentation, Python will raise
an \texttt{IndentationError}.

\subsubsection{Tiny Code}\label{tiny-code-13}

\begin{Shaded}
\begin{Highlighting}[]
\NormalTok{temperature }\OperatorTok{=} \DecValTok{30}

\ControlFlowTok{if}\NormalTok{ temperature }\OperatorTok{\textgreater{}} \DecValTok{25}\NormalTok{:}
    \BuiltInTok{print}\NormalTok{(}\StringTok{"It\textquotesingle{}s a hot day!"}\NormalTok{)}

\ControlFlowTok{if}\NormalTok{ temperature }\OperatorTok{\textless{}} \DecValTok{0}\NormalTok{:}
    \BuiltInTok{print}\NormalTok{(}\StringTok{"It\textquotesingle{}s freezing!"}\NormalTok{)}
\end{Highlighting}
\end{Shaded}

\subsubsection{Why it Matters}\label{why-it-matters-13}

Without \texttt{if} statements, programs would always run the same way.
Conditions make programs dynamic and responsive---whether it's checking
user input, validating data, or making choices in games, \texttt{if} is
the starting point for logic in Python.

\subsubsection{Try It Yourself}\label{try-it-yourself-13}

\begin{enumerate}
\def\labelenumi{\arabic{enumi}.}
\tightlist
\item
  Write an \texttt{if} statement that prints \texttt{"Positive"} if a
  number is greater than zero.
\item
  Test what happens if the number is zero---does the code run or not?
\item
  Use an \texttt{if} statement to check if a string is empty, and print
  \texttt{"Empty\ string"} when it is.
\item
  Change the indentation in your code incorrectly and observe Python's
  error message.
\end{enumerate}

\subsection{\texorpdfstring{14.
\texttt{if...else}}{14. if...else}}\label{if...else}

The \texttt{if...else} structure lets your program choose between two
paths. If the condition is \texttt{True}, the \texttt{if} block runs. If
the condition is \texttt{False}, the \texttt{else} block runs instead.
This ensures that one of the two blocks always executes.

\subsubsection{Deep Dive}\label{deep-dive-14}

Basic Structure

\begin{Shaded}
\begin{Highlighting}[]
\ControlFlowTok{if}\NormalTok{ condition:}
    \CommentTok{\# code runs if condition is True}
\ControlFlowTok{else}\NormalTok{:}
    \CommentTok{\# code runs if condition is False}
\end{Highlighting}
\end{Shaded}

Example

\begin{Shaded}
\begin{Highlighting}[]
\NormalTok{age }\OperatorTok{=} \DecValTok{16}

\ControlFlowTok{if}\NormalTok{ age }\OperatorTok{\textgreater{}=} \DecValTok{18}\NormalTok{:}
    \BuiltInTok{print}\NormalTok{(}\StringTok{"You can vote"}\NormalTok{)}
\ControlFlowTok{else}\NormalTok{:}
    \BuiltInTok{print}\NormalTok{(}\StringTok{"You are too young to vote"}\NormalTok{)}
\end{Highlighting}
\end{Shaded}

Here, if \texttt{age} is 18 or more, the first message is printed.
Otherwise, the second one runs.

\texttt{if...else} with Variables You can use the result of conditions
to assign values:

\begin{Shaded}
\begin{Highlighting}[]
\NormalTok{x }\OperatorTok{=} \DecValTok{10}
\NormalTok{y }\OperatorTok{=} \DecValTok{20}

\NormalTok{bigger }\OperatorTok{=}\NormalTok{ x }\ControlFlowTok{if}\NormalTok{ x }\OperatorTok{\textgreater{}}\NormalTok{ y }\ControlFlowTok{else}\NormalTok{ y}
\BuiltInTok{print}\NormalTok{(bigger)   }\CommentTok{\# 20}
\end{Highlighting}
\end{Shaded}

This is called a ternary expression (or conditional expression).

Only One \texttt{else} An \texttt{if} statement can have at most one
\texttt{else}, and it always comes last.

\subsubsection{Tiny Code}\label{tiny-code-14}

\begin{Shaded}
\begin{Highlighting}[]
\NormalTok{score }\OperatorTok{=} \DecValTok{75}

\ControlFlowTok{if}\NormalTok{ score }\OperatorTok{\textgreater{}=} \DecValTok{60}\NormalTok{:}
    \BuiltInTok{print}\NormalTok{(}\StringTok{"Pass"}\NormalTok{)}
\ControlFlowTok{else}\NormalTok{:}
    \BuiltInTok{print}\NormalTok{(}\StringTok{"Fail"}\NormalTok{)}
\end{Highlighting}
\end{Shaded}

\subsubsection{Why it Matters}\label{why-it-matters-14}

The \texttt{if...else} structure makes programs capable of handling two
outcomes: one when a condition is met, and another when it isn't. It's
essential for branching logic---without it, you could only run code when
conditions are true, not handle the ``otherwise'' case.

\subsubsection{Try It Yourself}\label{try-it-yourself-14}

\begin{enumerate}
\def\labelenumi{\arabic{enumi}.}
\tightlist
\item
  Write a program that checks if a number is even or odd using
  \texttt{if...else}.
\item
  Create a variable \texttt{temperature} and print \texttt{"Warm"} if
  it's 20 or above, otherwise \texttt{"Cold"}.
\item
  Use a conditional expression to set \texttt{status\ =\ "adult"} if
  \texttt{age\ \textgreater{}=\ 18}, else \texttt{"minor"}.
\item
  Change the condition to test different inputs and see how the output
  changes.
\end{enumerate}

\subsection{\texorpdfstring{15.
\texttt{if...elif...else}}{15. if...elif...else}}\label{if...elif...else}

The \texttt{if...elif...else} structure lets you check multiple
conditions in order. The program will run the first block where the
condition is \texttt{True}, and then skip the rest. If none of the
conditions are true, the \texttt{else} block runs.

\subsubsection{Deep Dive}\label{deep-dive-15}

Basic Structure

\begin{Shaded}
\begin{Highlighting}[]
\ControlFlowTok{if}\NormalTok{ condition1:}
    \CommentTok{\# runs if condition1 is True}
\ControlFlowTok{elif}\NormalTok{ condition2:}
    \CommentTok{\# runs if condition1 is False AND condition2 is True}
\ControlFlowTok{elif}\NormalTok{ condition3:}
    \CommentTok{\# runs if above are False AND condition3 is True}
\ControlFlowTok{else}\NormalTok{:}
    \CommentTok{\# runs if none of the above are True}
\end{Highlighting}
\end{Shaded}

Example

\begin{Shaded}
\begin{Highlighting}[]
\NormalTok{score }\OperatorTok{=} \DecValTok{85}

\ControlFlowTok{if}\NormalTok{ score }\OperatorTok{\textgreater{}=} \DecValTok{90}\NormalTok{:}
    \BuiltInTok{print}\NormalTok{(}\StringTok{"Excellent"}\NormalTok{)}
\ControlFlowTok{elif}\NormalTok{ score }\OperatorTok{\textgreater{}=} \DecValTok{75}\NormalTok{:}
    \BuiltInTok{print}\NormalTok{(}\StringTok{"Good"}\NormalTok{)}
\ControlFlowTok{elif}\NormalTok{ score }\OperatorTok{\textgreater{}=} \DecValTok{60}\NormalTok{:}
    \BuiltInTok{print}\NormalTok{(}\StringTok{"Pass"}\NormalTok{)}
\ControlFlowTok{else}\NormalTok{:}
    \BuiltInTok{print}\NormalTok{(}\StringTok{"Fail"}\NormalTok{)}
\end{Highlighting}
\end{Shaded}

Here, Python checks each condition in order. Since
\texttt{score\ \textgreater{}=\ 75} is true, it prints \texttt{"Good"}
and skips the rest.

Order Matters Conditions are checked from top to bottom. As soon as one
is \texttt{True}, Python stops checking further. For example:

\begin{Shaded}
\begin{Highlighting}[]
\NormalTok{x }\OperatorTok{=} \DecValTok{100}
\ControlFlowTok{if}\NormalTok{ x }\OperatorTok{\textgreater{}} \DecValTok{50}\NormalTok{:}
    \BuiltInTok{print}\NormalTok{(}\StringTok{"Bigger than 50"}\NormalTok{)}
\ControlFlowTok{elif}\NormalTok{ x }\OperatorTok{\textgreater{}} \DecValTok{10}\NormalTok{:}
    \BuiltInTok{print}\NormalTok{(}\StringTok{"Bigger than 10"}\NormalTok{)}
\end{Highlighting}
\end{Shaded}

Only \texttt{"Bigger\ than\ 50"} is printed, even though
\texttt{x\ \textgreater{}\ 10} is also true.

Optional Parts

\begin{itemize}
\tightlist
\item
  The \texttt{elif} can appear as many times as needed.
\item
  The \texttt{else} is optional---you don't need it if you only want to
  handle some cases.
\end{itemize}

\subsubsection{Tiny Code}\label{tiny-code-15}

\begin{Shaded}
\begin{Highlighting}[]
\NormalTok{day }\OperatorTok{=} \StringTok{"Wednesday"}

\ControlFlowTok{if}\NormalTok{ day }\OperatorTok{==} \StringTok{"Monday"}\NormalTok{:}
    \BuiltInTok{print}\NormalTok{(}\StringTok{"Start of the week"}\NormalTok{)}
\ControlFlowTok{elif}\NormalTok{ day }\OperatorTok{==} \StringTok{"Friday"}\NormalTok{:}
    \BuiltInTok{print}\NormalTok{(}\StringTok{"Almost weekend"}\NormalTok{)}
\ControlFlowTok{elif}\NormalTok{ day }\OperatorTok{==} \StringTok{"Saturday"} \KeywordTok{or}\NormalTok{ day }\OperatorTok{==} \StringTok{"Sunday"}\NormalTok{:}
    \BuiltInTok{print}\NormalTok{(}\StringTok{"Weekend!"}\NormalTok{)}
\ControlFlowTok{else}\NormalTok{:}
    \BuiltInTok{print}\NormalTok{(}\StringTok{"Midweek day"}\NormalTok{)}
\end{Highlighting}
\end{Shaded}

\subsubsection{Why it Matters}\label{why-it-matters-15}

Most real-life decisions aren't just yes-or-no. The
\texttt{if...elif...else} chain lets you handle multiple possibilities
in an organized way, making your code more flexible and readable.

\subsubsection{Try It Yourself}\label{try-it-yourself-15}

\begin{enumerate}
\def\labelenumi{\arabic{enumi}.}
\tightlist
\item
  Write a program that checks a number and prints \texttt{"Positive"},
  \texttt{"Negative"}, or \texttt{"Zero"}.
\item
  Create a grading system: \texttt{90+\ =\ A}, \texttt{75–89\ =\ B},
  \texttt{60–74\ =\ C}, below \texttt{60\ =\ F}.
\item
  Write code that prints which day of the week it is, based on a
  variable \texttt{day}.
\item
  Experiment by changing the order of conditions and observe how the
  output changes.
\end{enumerate}

\subsection{16. Nested Conditions}\label{nested-conditions}

A nested condition means putting one \texttt{if} statement inside
another. This allows your program to make more specific decisions by
checking an additional condition only when the first one is true.

\subsubsection{Deep Dive}\label{deep-dive-16}

Basic Structure

\begin{Shaded}
\begin{Highlighting}[]
\ControlFlowTok{if}\NormalTok{ condition1:}
    \ControlFlowTok{if}\NormalTok{ condition2:}
        \CommentTok{\# runs if both condition1 and condition2 are True}
    \ControlFlowTok{else}\NormalTok{:}
        \CommentTok{\# runs if condition1 is True but condition2 is False}
\ControlFlowTok{else}\NormalTok{:}
    \CommentTok{\# runs if condition1 is False}
\end{Highlighting}
\end{Shaded}

Example

\begin{Shaded}
\begin{Highlighting}[]
\NormalTok{age }\OperatorTok{=} \DecValTok{20}
\NormalTok{is\_student }\OperatorTok{=} \VariableTok{True}

\ControlFlowTok{if}\NormalTok{ age }\OperatorTok{\textgreater{}=} \DecValTok{18}\NormalTok{:}
    \ControlFlowTok{if}\NormalTok{ is\_student:}
        \BuiltInTok{print}\NormalTok{(}\StringTok{"Adult student"}\NormalTok{)}
    \ControlFlowTok{else}\NormalTok{:}
        \BuiltInTok{print}\NormalTok{(}\StringTok{"Adult, not a student"}\NormalTok{)}
\ControlFlowTok{else}\NormalTok{:}
    \BuiltInTok{print}\NormalTok{(}\StringTok{"Minor"}\NormalTok{)}
\end{Highlighting}
\end{Shaded}

Here, the second check (\texttt{is\_student}) only happens if the first
check (\texttt{age\ \textgreater{}=\ 18}) is true.

Why Nesting is Useful Nested conditions let you handle cases that depend
on multiple layers of logic. However, too much nesting can make code
hard to read. In such cases, logical operators (\texttt{and},
\texttt{or}) are often better:

\begin{Shaded}
\begin{Highlighting}[]
\ControlFlowTok{if}\NormalTok{ age }\OperatorTok{\textgreater{}=} \DecValTok{18} \KeywordTok{and}\NormalTok{ is\_student:}
    \BuiltInTok{print}\NormalTok{(}\StringTok{"Adult student"}\NormalTok{)}
\end{Highlighting}
\end{Shaded}

Best Practice

\begin{itemize}
\tightlist
\item
  Use nesting when the second condition should only be checked if the
  first one is true.
\item
  For readability, avoid deep nesting---prefer combining conditions with
  logical operators when possible.
\end{itemize}

\subsubsection{Tiny Code}\label{tiny-code-16}

\begin{Shaded}
\begin{Highlighting}[]
\NormalTok{x }\OperatorTok{=} \DecValTok{15}

\ControlFlowTok{if}\NormalTok{ x }\OperatorTok{\textgreater{}} \DecValTok{0}\NormalTok{:}
    \ControlFlowTok{if}\NormalTok{ x }\OperatorTok{\%} \DecValTok{2} \OperatorTok{==} \DecValTok{0}\NormalTok{:}
        \BuiltInTok{print}\NormalTok{(}\StringTok{"Positive even number"}\NormalTok{)}
    \ControlFlowTok{else}\NormalTok{:}
        \BuiltInTok{print}\NormalTok{(}\StringTok{"Positive odd number"}\NormalTok{)}
\ControlFlowTok{else}\NormalTok{:}
    \BuiltInTok{print}\NormalTok{(}\StringTok{"Zero or negative number"}\NormalTok{)}
\end{Highlighting}
\end{Shaded}

\subsubsection{Why it Matters}\label{why-it-matters-16}

Nested conditions add depth to decision-making. They let you structure
logic in layers, which is closer to how we reason in real life---for
example, ``If the shop is open, then check if I have enough money.''

\subsubsection{Try It Yourself}\label{try-it-yourself-16}

\begin{enumerate}
\def\labelenumi{\arabic{enumi}.}
\tightlist
\item
  Write a program that checks if a number is positive. If it is, then
  check if it's greater than 100.
\item
  Make a program that checks if someone is eligible to drive: first
  check if \texttt{age\ \textgreater{}=\ 18}, then check if
  \texttt{has\_license\ ==\ True}.
\item
  Rewrite one of your nested conditions using \texttt{and} instead, and
  compare which version is easier to read.
\end{enumerate}

\subsection{\texorpdfstring{17. \texttt{while}
Loop}{17. while Loop}}\label{while-loop}

A \texttt{while} loop lets your program repeat a block of code as long
as a condition is \texttt{True}. It's useful when you don't know in
advance how many times you need to loop---for example, waiting for user
input or running until some condition changes.

\subsubsection{Deep Dive}\label{deep-dive-17}

Basic Structure

\begin{Shaded}
\begin{Highlighting}[]
\ControlFlowTok{while}\NormalTok{ condition:}
    \CommentTok{\# code runs as long as condition is True}
\end{Highlighting}
\end{Shaded}

Example

\begin{Shaded}
\begin{Highlighting}[]
\NormalTok{count }\OperatorTok{=} \DecValTok{1}
\ControlFlowTok{while}\NormalTok{ count }\OperatorTok{\textless{}=} \DecValTok{5}\NormalTok{:}
    \BuiltInTok{print}\NormalTok{(}\StringTok{"Count is:"}\NormalTok{, count)}
\NormalTok{    count }\OperatorTok{+=} \DecValTok{1}
\end{Highlighting}
\end{Shaded}

This loop prints numbers from 1 to 5. Each time, \texttt{count}
increases by 1 until the condition \texttt{count\ \textless{}=\ 5} is no
longer true.

Infinite Loops If the condition never becomes \texttt{False}, the loop
will run forever. For example:

\begin{Shaded}
\begin{Highlighting}[]
\ControlFlowTok{while} \VariableTok{True}\NormalTok{:}
    \BuiltInTok{print}\NormalTok{(}\StringTok{"This never ends!"}\NormalTok{)}
\end{Highlighting}
\end{Shaded}

You must stop such loops manually (Ctrl+C in most terminals).

Using \texttt{break} to Stop Early You can break out of a \texttt{while}
loop when needed:

\begin{Shaded}
\begin{Highlighting}[]
\NormalTok{x }\OperatorTok{=} \DecValTok{0}
\ControlFlowTok{while}\NormalTok{ x }\OperatorTok{\textless{}} \DecValTok{10}\NormalTok{:}
    \ControlFlowTok{if}\NormalTok{ x }\OperatorTok{==} \DecValTok{5}\NormalTok{:}
        \ControlFlowTok{break}
    \BuiltInTok{print}\NormalTok{(x)}
\NormalTok{    x }\OperatorTok{+=} \DecValTok{1}
\end{Highlighting}
\end{Shaded}

Using \texttt{continue} to Skip The \texttt{continue} keyword skips to
the next iteration without finishing the rest of the loop body.

Common Use Cases

\begin{itemize}
\tightlist
\item
  Waiting for user input until valid
\item
  Repeating a task until a condition is met
\item
  Infinite background tasks with break conditions
\end{itemize}

\subsubsection{Tiny Code}\label{tiny-code-17}

\begin{Shaded}
\begin{Highlighting}[]
\CommentTok{\# Print even numbers less than 10}
\NormalTok{num }\OperatorTok{=} \DecValTok{0}
\ControlFlowTok{while}\NormalTok{ num }\OperatorTok{\textless{}} \DecValTok{10}\NormalTok{:}
\NormalTok{    num }\OperatorTok{+=} \DecValTok{1}
    \ControlFlowTok{if}\NormalTok{ num }\OperatorTok{\%} \DecValTok{2} \OperatorTok{==} \DecValTok{1}\NormalTok{:}
        \ControlFlowTok{continue}
    \BuiltInTok{print}\NormalTok{(num)}
\end{Highlighting}
\end{Shaded}

\subsubsection{Why it Matters}\label{why-it-matters-17}

The \texttt{while} loop gives your program flexibility to keep running
until something changes. It's a powerful way to model ``keep doing this
until\ldots{}'' logic that often appears in real-world problems.

\subsubsection{Try It Yourself}\label{try-it-yourself-17}

\begin{enumerate}
\def\labelenumi{\arabic{enumi}.}
\tightlist
\item
  Write a loop that counts down from 10 to 1.
\item
  Create a loop that keeps asking the user for a password until the
  correct one is entered.
\item
  Use \texttt{while\ True} with a \texttt{break} to simulate a simple
  menu system (e.g., type \texttt{q} to quit).
\item
  Experiment with \texttt{continue} to skip printing odd numbers.
\end{enumerate}

\subsection{\texorpdfstring{18. \texttt{for} Loop
(range)}{18. for Loop (range)}}\label{for-loop-range}

A \texttt{for} loop in Python is used to repeat a block of code a
specific number of times. Unlike the \texttt{while} loop, which runs as
long as a condition is true, the \texttt{for} loop usually goes through
a sequence of values---often created with the built-in \texttt{range()}
function.

\subsubsection{Deep Dive}\label{deep-dive-18}

Basic Structure

\begin{Shaded}
\begin{Highlighting}[]
\ControlFlowTok{for}\NormalTok{ variable }\KeywordTok{in}\NormalTok{ sequence:}
    \CommentTok{\# code runs for each item in the sequence}
\end{Highlighting}
\end{Shaded}

Using \texttt{range()} The \texttt{range()} function generates a
sequence of numbers.

\begin{itemize}
\tightlist
\item
  \texttt{range(stop)} → numbers from \texttt{0} up to
  \texttt{stop\ -\ 1}
\item
  \texttt{range(start,\ stop)} → numbers from \texttt{start} up to
  \texttt{stop\ -\ 1}
\item
  \texttt{range(start,\ stop,\ step)} → numbers from \texttt{start} up
  to \texttt{stop\ -\ 1}, moving by \texttt{step}
\end{itemize}

Examples:

\begin{Shaded}
\begin{Highlighting}[]
\ControlFlowTok{for}\NormalTok{ i }\KeywordTok{in} \BuiltInTok{range}\NormalTok{(}\DecValTok{5}\NormalTok{):}
    \BuiltInTok{print}\NormalTok{(i)       }\CommentTok{\# 0, 1, 2, 3, 4}

\ControlFlowTok{for}\NormalTok{ i }\KeywordTok{in} \BuiltInTok{range}\NormalTok{(}\DecValTok{2}\NormalTok{, }\DecValTok{6}\NormalTok{):}
    \BuiltInTok{print}\NormalTok{(i)       }\CommentTok{\# 2, 3, 4, 5}

\ControlFlowTok{for}\NormalTok{ i }\KeywordTok{in} \BuiltInTok{range}\NormalTok{(}\DecValTok{0}\NormalTok{, }\DecValTok{10}\NormalTok{, }\DecValTok{2}\NormalTok{):}
    \BuiltInTok{print}\NormalTok{(i)       }\CommentTok{\# 0, 2, 4, 6, 8}
\end{Highlighting}
\end{Shaded}

Looping with \texttt{else} A \texttt{for} loop can have an optional
\texttt{else} block that runs if the loop finishes normally (not stopped
by \texttt{break}).

\begin{Shaded}
\begin{Highlighting}[]
\ControlFlowTok{for}\NormalTok{ i }\KeywordTok{in} \BuiltInTok{range}\NormalTok{(}\DecValTok{3}\NormalTok{):}
    \BuiltInTok{print}\NormalTok{(i)}
\ControlFlowTok{else}\NormalTok{:}
    \BuiltInTok{print}\NormalTok{(}\StringTok{"Loop finished"}\NormalTok{)}
\end{Highlighting}
\end{Shaded}

Common Patterns

\begin{itemize}
\tightlist
\item
  Counting a fixed number of times
\item
  Iterating over list indexes
\item
  Generating sequences for calculations
\end{itemize}

\subsubsection{Tiny Code}\label{tiny-code-18}

\begin{Shaded}
\begin{Highlighting}[]
\CommentTok{\# Print squares of numbers from 1 to 5}
\ControlFlowTok{for}\NormalTok{ n }\KeywordTok{in} \BuiltInTok{range}\NormalTok{(}\DecValTok{1}\NormalTok{, }\DecValTok{6}\NormalTok{):}
    \BuiltInTok{print}\NormalTok{(n, }\StringTok{"squared is"}\NormalTok{, n }\OperatorTok{*}\NormalTok{ n)}
\end{Highlighting}
\end{Shaded}

\subsubsection{Why it Matters}\label{why-it-matters-18}

The \texttt{for} loop is the most common way to repeat actions in Python
when you know how many times to loop. It's simpler and clearer than a
\texttt{while} loop for counting and iterating over ranges.

\subsubsection{Try It Yourself}\label{try-it-yourself-18}

\begin{enumerate}
\def\labelenumi{\arabic{enumi}.}
\tightlist
\item
  Write a loop that prints numbers 1 through 10.
\item
  Use \texttt{range()} with a step of 2 to print even numbers up to 20.
\item
  Write a loop that prints \texttt{"Python"} five times.
\item
  Create a loop with \texttt{range(10,\ 0,\ -1)} to count down from 10
  to 1.
\end{enumerate}

\subsection{\texorpdfstring{19. Loop Control (\texttt{break},
\texttt{continue})}{19. Loop Control (break, continue)}}\label{loop-control-break-continue}

Sometimes you need more control over loops. Python provides two special
keywords---\texttt{break} and \texttt{continue}---to change how a loop
behaves. These allow you to stop a loop early or skip parts of it.

\subsubsection{Deep Dive}\label{deep-dive-19}

\texttt{break} --- Stop the Loop The \texttt{break} statement ends the
loop immediately, even if the loop condition or range still has more
values.

\begin{Shaded}
\begin{Highlighting}[]
\ControlFlowTok{for}\NormalTok{ i }\KeywordTok{in} \BuiltInTok{range}\NormalTok{(}\DecValTok{10}\NormalTok{):}
    \ControlFlowTok{if}\NormalTok{ i }\OperatorTok{==} \DecValTok{5}\NormalTok{:}
        \ControlFlowTok{break}
    \BuiltInTok{print}\NormalTok{(i)}
\CommentTok{\# Output: 0, 1, 2, 3, 4}
\end{Highlighting}
\end{Shaded}

\texttt{continue} --- Skip to Next Iteration The \texttt{continue}
statement skips the rest of the loop body and moves to the next
iteration.

\begin{Shaded}
\begin{Highlighting}[]
\ControlFlowTok{for}\NormalTok{ i }\KeywordTok{in} \BuiltInTok{range}\NormalTok{(}\DecValTok{5}\NormalTok{):}
    \ControlFlowTok{if}\NormalTok{ i }\OperatorTok{==} \DecValTok{2}\NormalTok{:}
        \ControlFlowTok{continue}
    \BuiltInTok{print}\NormalTok{(i)}
\CommentTok{\# Output: 0, 1, 3, 4}
\end{Highlighting}
\end{Shaded}

Using with \texttt{while} Loops Both \texttt{break} and
\texttt{continue} work the same way in \texttt{while} loops.

\begin{Shaded}
\begin{Highlighting}[]
\NormalTok{x }\OperatorTok{=} \DecValTok{0}
\ControlFlowTok{while}\NormalTok{ x }\OperatorTok{\textless{}} \DecValTok{5}\NormalTok{:}
\NormalTok{    x }\OperatorTok{+=} \DecValTok{1}
    \ControlFlowTok{if}\NormalTok{ x }\OperatorTok{==} \DecValTok{3}\NormalTok{:}
        \ControlFlowTok{continue}
    \ControlFlowTok{if}\NormalTok{ x }\OperatorTok{==} \DecValTok{5}\NormalTok{:}
        \ControlFlowTok{break}
    \BuiltInTok{print}\NormalTok{(x)}
\CommentTok{\# Output: 1, 2, 4}
\end{Highlighting}
\end{Shaded}

When to Use

\begin{itemize}
\tightlist
\item
  \texttt{break} is useful when you find what you're looking for and
  don't need to continue looping.
\item
  \texttt{continue} is useful when you want to skip over certain cases
  but still keep looping.
\end{itemize}

\subsubsection{Tiny Code}\label{tiny-code-19}

\begin{Shaded}
\begin{Highlighting}[]
\CommentTok{\# Find first multiple of 7}
\ControlFlowTok{for}\NormalTok{ n }\KeywordTok{in} \BuiltInTok{range}\NormalTok{(}\DecValTok{1}\NormalTok{, }\DecValTok{20}\NormalTok{):}
    \ControlFlowTok{if}\NormalTok{ n }\OperatorTok{\%} \DecValTok{7} \OperatorTok{==} \DecValTok{0}\NormalTok{:}
        \BuiltInTok{print}\NormalTok{(}\StringTok{"Found:"}\NormalTok{, n)}
        \ControlFlowTok{break}

\CommentTok{\# Print only odd numbers}
\ControlFlowTok{for}\NormalTok{ n }\KeywordTok{in} \BuiltInTok{range}\NormalTok{(}\DecValTok{1}\NormalTok{, }\DecValTok{10}\NormalTok{):}
    \ControlFlowTok{if}\NormalTok{ n }\OperatorTok{\%} \DecValTok{2} \OperatorTok{==} \DecValTok{0}\NormalTok{:}
        \ControlFlowTok{continue}
    \BuiltInTok{print}\NormalTok{(n)}
\end{Highlighting}
\end{Shaded}

\subsubsection{Why it Matters}\label{why-it-matters-19}

Without loop control, you would have to add extra complicated logic or
duplicate code. \texttt{break} and \texttt{continue} give you
fine-grained control, making loops cleaner, more efficient, and easier
to understand.

\subsubsection{Try It Yourself}\label{try-it-yourself-19}

\begin{enumerate}
\def\labelenumi{\arabic{enumi}.}
\tightlist
\item
  Write a loop that prints numbers from 1 to 100, but stops when it
  reaches 42.
\item
  Write a loop that prints numbers from 1 to 10, but skips multiples of
  3.
\item
  Combine both: loop through numbers 1 to 20, skip evens, and stop
  completely if you find 15.
\end{enumerate}

\subsection{\texorpdfstring{20. Loop with
\texttt{else}}{20. Loop with else}}\label{loop-with-else}

In Python, a \texttt{for} or \texttt{while} loop can have an optional
\texttt{else} block. The \texttt{else} part runs only if the loop
finishes normally---that is, it isn't stopped early by a \texttt{break}.
This feature is unique to Python and is often used when searching for
something.

\subsubsection{Deep Dive}\label{deep-dive-20}

Basic Structure

\begin{Shaded}
\begin{Highlighting}[]
\ControlFlowTok{for}\NormalTok{ item }\KeywordTok{in}\NormalTok{ sequence:}
    \CommentTok{\# loop body}
\ControlFlowTok{else}\NormalTok{:}
    \CommentTok{\# runs if loop finishes without break}
\end{Highlighting}
\end{Shaded}

Example with \texttt{for}

\begin{Shaded}
\begin{Highlighting}[]
\ControlFlowTok{for}\NormalTok{ i }\KeywordTok{in} \BuiltInTok{range}\NormalTok{(}\DecValTok{5}\NormalTok{):}
    \BuiltInTok{print}\NormalTok{(i)}
\ControlFlowTok{else}\NormalTok{:}
    \BuiltInTok{print}\NormalTok{(}\StringTok{"Loop finished"}\NormalTok{)}
\end{Highlighting}
\end{Shaded}

This prints numbers 0 through 4, then prints \texttt{"Loop\ finished"}.

Using with \texttt{break} If the loop ends because of \texttt{break},
the \texttt{else} block is skipped:

\begin{Shaded}
\begin{Highlighting}[]
\ControlFlowTok{for}\NormalTok{ i }\KeywordTok{in} \BuiltInTok{range}\NormalTok{(}\DecValTok{5}\NormalTok{):}
    \ControlFlowTok{if}\NormalTok{ i }\OperatorTok{==} \DecValTok{3}\NormalTok{:}
        \ControlFlowTok{break}
    \BuiltInTok{print}\NormalTok{(i)}
\ControlFlowTok{else}\NormalTok{:}
    \BuiltInTok{print}\NormalTok{(}\StringTok{"Finished without break"}\NormalTok{)}
\CommentTok{\# Output: 0, 1, 2}
\end{Highlighting}
\end{Shaded}

Example with \texttt{while}

\begin{Shaded}
\begin{Highlighting}[]
\NormalTok{x }\OperatorTok{=} \DecValTok{0}
\ControlFlowTok{while}\NormalTok{ x }\OperatorTok{\textless{}} \DecValTok{3}\NormalTok{:}
    \BuiltInTok{print}\NormalTok{(x)}
\NormalTok{    x }\OperatorTok{+=} \DecValTok{1}
\ControlFlowTok{else}\NormalTok{:}
    \BuiltInTok{print}\NormalTok{(}\StringTok{"While loop ended normally"}\NormalTok{)}
\end{Highlighting}
\end{Shaded}

Practical Use Case: Searching The \texttt{else} block is handy when
searching for an item. If you find the item, \texttt{break} ends the
loop; if not, the \texttt{else} runs.

\begin{Shaded}
\begin{Highlighting}[]
\NormalTok{numbers }\OperatorTok{=}\NormalTok{ [}\DecValTok{1}\NormalTok{, }\DecValTok{2}\NormalTok{, }\DecValTok{3}\NormalTok{, }\DecValTok{4}\NormalTok{, }\DecValTok{5}\NormalTok{]}

\ControlFlowTok{for}\NormalTok{ n }\KeywordTok{in}\NormalTok{ numbers:}
    \ControlFlowTok{if}\NormalTok{ n }\OperatorTok{==} \DecValTok{7}\NormalTok{:}
        \BuiltInTok{print}\NormalTok{(}\StringTok{"Found 7!"}\NormalTok{)}
        \ControlFlowTok{break}
\ControlFlowTok{else}\NormalTok{:}
    \BuiltInTok{print}\NormalTok{(}\StringTok{"7 not found"}\NormalTok{)}
\end{Highlighting}
\end{Shaded}

\subsubsection{Tiny Code}\label{tiny-code-20}

\begin{Shaded}
\begin{Highlighting}[]
\ControlFlowTok{for}\NormalTok{ char }\KeywordTok{in} \StringTok{"Python"}\NormalTok{:}
    \ControlFlowTok{if}\NormalTok{ char }\OperatorTok{==} \StringTok{"x"}\NormalTok{:}
        \BuiltInTok{print}\NormalTok{(}\StringTok{"Found x!"}\NormalTok{)}
        \ControlFlowTok{break}
\ControlFlowTok{else}\NormalTok{:}
    \BuiltInTok{print}\NormalTok{(}\StringTok{"No x in string"}\NormalTok{)}
\end{Highlighting}
\end{Shaded}

\subsubsection{Why it Matters}\label{why-it-matters-20}

The \texttt{else} clause on loops lets you handle the ``nothing found''
case cleanly without needing extra flags or checks. It makes code
shorter and easier to understand when searching or checking conditions
across a loop.

\subsubsection{Try It Yourself}\label{try-it-yourself-20}

\begin{enumerate}
\def\labelenumi{\arabic{enumi}.}
\tightlist
\item
  Write a loop that searches for the number \texttt{10} in a list of
  numbers. If found, print \texttt{"Found"}. If not, let the
  \texttt{else} print \texttt{"Not\ found"}.
\item
  Create a \texttt{while} loop that counts from 1 to 5 and uses an
  \texttt{else} block to print \texttt{"Done\ counting"}.
\item
  Experiment with adding \texttt{break} inside your loop to see how it
  changes whether the \texttt{else} runs.
\end{enumerate}

\section{Chapter 3. Data Structures}\label{chapter-3.-data-structures}

\subsection{21. Lists (creation \& basics)}\label{lists-creation-basics}

A list in Python is an ordered collection of items. Think of it like a
container where you can store multiple values in a single
variable---numbers, strings, or even other lists. Lists are one of the
most commonly used data structures in Python because they're flexible
and easy to work with.

\subsubsection{Deep Dive}\label{deep-dive-21}

Creating Lists You create a list by putting values inside square
brackets \texttt{{[}{]}}, separated by commas:

\begin{Shaded}
\begin{Highlighting}[]
\NormalTok{fruits }\OperatorTok{=}\NormalTok{ [}\StringTok{"apple"}\NormalTok{, }\StringTok{"banana"}\NormalTok{, }\StringTok{"cherry"}\NormalTok{]}
\NormalTok{numbers }\OperatorTok{=}\NormalTok{ [}\DecValTok{1}\NormalTok{, }\DecValTok{2}\NormalTok{, }\DecValTok{3}\NormalTok{, }\DecValTok{4}\NormalTok{, }\DecValTok{5}\NormalTok{]}
\NormalTok{mixed }\OperatorTok{=}\NormalTok{ [}\DecValTok{1}\NormalTok{, }\StringTok{"hello"}\NormalTok{, }\FloatTok{3.14}\NormalTok{, }\VariableTok{True}\NormalTok{]}
\end{Highlighting}
\end{Shaded}

Lists can also be empty:

\begin{Shaded}
\begin{Highlighting}[]
\NormalTok{empty }\OperatorTok{=}\NormalTok{ []}
\end{Highlighting}
\end{Shaded}

Lists Are Ordered The items keep the order you put them in. If you
create a list \texttt{{[}10,\ 20,\ 30{]}}, Python remembers that order
unless you change it.

Lists Can Be Changed (Mutable) Unlike strings or tuples, lists can be
modified after creation---you can add, remove, or replace elements.

Accessing Elements Each item in a list has an index (position), starting
at 0:

\begin{Shaded}
\begin{Highlighting}[]
\NormalTok{fruits }\OperatorTok{=}\NormalTok{ [}\StringTok{"apple"}\NormalTok{, }\StringTok{"banana"}\NormalTok{, }\StringTok{"cherry"}\NormalTok{]}
\BuiltInTok{print}\NormalTok{(fruits[}\DecValTok{0}\NormalTok{])   }\CommentTok{\# "apple"}
\BuiltInTok{print}\NormalTok{(fruits[}\DecValTok{2}\NormalTok{])   }\CommentTok{\# "cherry"}
\end{Highlighting}
\end{Shaded}

Length of a List You can find out how many items a list has with
\texttt{len()}:

\begin{Shaded}
\begin{Highlighting}[]
\BuiltInTok{print}\NormalTok{(}\BuiltInTok{len}\NormalTok{(fruits))  }\CommentTok{\# 3}
\end{Highlighting}
\end{Shaded}

Quick Summary Table

\begin{longtable}[]{@{}lll@{}}
\toprule\noalign{}
Operation & Example & Result \\
\midrule\noalign{}
\endhead
\bottomrule\noalign{}
\endlastfoot
Create a list & \texttt{nums\ =\ {[}1,\ 2,\ 3{]}} &
\texttt{{[}1,\ 2,\ 3{]}} \\
Empty list & \texttt{empty\ =\ {[}{]}} & \texttt{{[}{]}} \\
Access by index & \texttt{nums{[}0{]}} & \texttt{1} \\
Last element & \texttt{nums{[}-1{]}} & \texttt{3} \\
Length of list & \texttt{len(nums)} & \texttt{3} \\
\end{longtable}

\subsubsection{Tiny Code}\label{tiny-code-21}

\begin{Shaded}
\begin{Highlighting}[]
\NormalTok{colors }\OperatorTok{=}\NormalTok{ [}\StringTok{"red"}\NormalTok{, }\StringTok{"green"}\NormalTok{, }\StringTok{"blue"}\NormalTok{]}

\BuiltInTok{print}\NormalTok{(colors)        }\CommentTok{\# [\textquotesingle{}red\textquotesingle{}, \textquotesingle{}green\textquotesingle{}, \textquotesingle{}blue\textquotesingle{}]}
\BuiltInTok{print}\NormalTok{(colors[}\DecValTok{1}\NormalTok{])     }\CommentTok{\# \textquotesingle{}green\textquotesingle{}}
\BuiltInTok{print}\NormalTok{(}\BuiltInTok{len}\NormalTok{(colors))   }\CommentTok{\# 3}
\end{Highlighting}
\end{Shaded}

\subsubsection{Why it Matters}\label{why-it-matters-21}

Lists let you store and organize multiple values in one place. Without
lists, you'd need a separate variable for each value, which quickly
becomes messy. Lists are the foundation for handling collections of data
in Python.

\subsubsection{Try It Yourself}\label{try-it-yourself-21}

\begin{enumerate}
\def\labelenumi{\arabic{enumi}.}
\tightlist
\item
  Create a list of five animals and print the whole list.
\item
  Print the first and last element of your list using indexes.
\item
  Make an empty list called \texttt{shopping\_cart} and check its
  length.
\item
  Try storing mixed types in one list (like a number, string, and
  boolean) and print it.
\end{enumerate}

\subsection{22. List Indexing \& Slicing}\label{list-indexing-slicing}

Lists in Python are ordered, which means each item has a position
(index). You can use indexes to get specific elements, or slices to get
parts of the list.

\subsubsection{Deep Dive}\label{deep-dive-22}

Indexing Basics Indexes start at 0 for the first element:

\begin{Shaded}
\begin{Highlighting}[]
\NormalTok{fruits }\OperatorTok{=}\NormalTok{ [}\StringTok{"apple"}\NormalTok{, }\StringTok{"banana"}\NormalTok{, }\StringTok{"cherry"}\NormalTok{, }\StringTok{"date"}\NormalTok{]}
\BuiltInTok{print}\NormalTok{(fruits[}\DecValTok{0}\NormalTok{])   }\CommentTok{\# "apple"}
\BuiltInTok{print}\NormalTok{(fruits[}\DecValTok{2}\NormalTok{])   }\CommentTok{\# "cherry"}
\end{Highlighting}
\end{Shaded}

Negative indexes count from the end:

\begin{Shaded}
\begin{Highlighting}[]
\BuiltInTok{print}\NormalTok{(fruits[}\OperatorTok{{-}}\DecValTok{1}\NormalTok{])  }\CommentTok{\# "date"}
\BuiltInTok{print}\NormalTok{(fruits[}\OperatorTok{{-}}\DecValTok{2}\NormalTok{])  }\CommentTok{\# "cherry"}
\end{Highlighting}
\end{Shaded}

Slicing Basics Slicing lets you grab a portion of a list. The syntax is:

\begin{Shaded}
\begin{Highlighting}[]
\BuiltInTok{list}\NormalTok{[start:stop]}
\end{Highlighting}
\end{Shaded}

It includes \texttt{start} but stops just before \texttt{stop}.

\begin{Shaded}
\begin{Highlighting}[]
\BuiltInTok{print}\NormalTok{(fruits[}\DecValTok{1}\NormalTok{:}\DecValTok{3}\NormalTok{])   }\CommentTok{\# [\textquotesingle{}banana\textquotesingle{}, \textquotesingle{}cherry\textquotesingle{}]}
\end{Highlighting}
\end{Shaded}

If you leave out \texttt{start}, Python begins at the start of the list:

\begin{Shaded}
\begin{Highlighting}[]
\BuiltInTok{print}\NormalTok{(fruits[:}\DecValTok{2}\NormalTok{])    }\CommentTok{\# [\textquotesingle{}apple\textquotesingle{}, \textquotesingle{}banana\textquotesingle{}]}
\end{Highlighting}
\end{Shaded}

If you leave out \texttt{stop}, Python goes to the end:

\begin{Shaded}
\begin{Highlighting}[]
\BuiltInTok{print}\NormalTok{(fruits[}\DecValTok{2}\NormalTok{:])    }\CommentTok{\# [\textquotesingle{}cherry\textquotesingle{}, \textquotesingle{}date\textquotesingle{}]}
\end{Highlighting}
\end{Shaded}

Slicing with Step You can add a third number for step size:

\begin{Shaded}
\begin{Highlighting}[]
\NormalTok{numbers }\OperatorTok{=}\NormalTok{ [}\DecValTok{0}\NormalTok{, }\DecValTok{1}\NormalTok{, }\DecValTok{2}\NormalTok{, }\DecValTok{3}\NormalTok{, }\DecValTok{4}\NormalTok{, }\DecValTok{5}\NormalTok{]}
\BuiltInTok{print}\NormalTok{(numbers[::}\DecValTok{2}\NormalTok{])   }\CommentTok{\# [0, 2, 4]}
\BuiltInTok{print}\NormalTok{(numbers[}\DecValTok{1}\NormalTok{::}\DecValTok{2}\NormalTok{])  }\CommentTok{\# [1, 3, 5]}
\end{Highlighting}
\end{Shaded}

Reversing a list is easy with step \texttt{-1}:

\begin{Shaded}
\begin{Highlighting}[]
\BuiltInTok{print}\NormalTok{(numbers[::}\OperatorTok{{-}}\DecValTok{1}\NormalTok{])  }\CommentTok{\# [5, 4, 3, 2, 1, 0]}
\end{Highlighting}
\end{Shaded}

Quick Summary Table

\begin{longtable}[]{@{}lll@{}}
\toprule\noalign{}
Operation & Example & Result \\
\midrule\noalign{}
\endhead
\bottomrule\noalign{}
\endlastfoot
First element & \texttt{fruits{[}0{]}} & \texttt{"apple"} \\
Last element & \texttt{fruits{[}-1{]}} & \texttt{"date"} \\
Slice (index 1--2) & \texttt{fruits{[}1:3{]}} &
\texttt{{[}\textquotesingle{}banana\textquotesingle{},\ \textquotesingle{}cherry\textquotesingle{}{]}} \\
From start to 2 & \texttt{fruits{[}:2{]}} &
\texttt{{[}\textquotesingle{}apple\textquotesingle{},\ \textquotesingle{}banana\textquotesingle{}{]}} \\
From 2 to end & \texttt{fruits{[}2:{]}} &
\texttt{{[}\textquotesingle{}cherry\textquotesingle{},\ \textquotesingle{}date\textquotesingle{}{]}} \\
Every 2nd element & \texttt{numbers{[}::2{]}} &
\texttt{{[}0,\ 2,\ 4{]}} \\
Reverse list & \texttt{numbers{[}::-1{]}} &
\texttt{{[}5,\ 4,\ 3,\ 2,\ 1,\ 0{]}} \\
\end{longtable}

\subsubsection{Tiny Code}\label{tiny-code-22}

\begin{Shaded}
\begin{Highlighting}[]
\NormalTok{colors }\OperatorTok{=}\NormalTok{ [}\StringTok{"red"}\NormalTok{, }\StringTok{"green"}\NormalTok{, }\StringTok{"blue"}\NormalTok{, }\StringTok{"yellow"}\NormalTok{]}

\BuiltInTok{print}\NormalTok{(colors[}\DecValTok{0}\NormalTok{])     }\CommentTok{\# red}
\BuiltInTok{print}\NormalTok{(colors[}\OperatorTok{{-}}\DecValTok{1}\NormalTok{])    }\CommentTok{\# yellow}
\BuiltInTok{print}\NormalTok{(colors[}\DecValTok{1}\NormalTok{:}\DecValTok{3}\NormalTok{])   }\CommentTok{\# [\textquotesingle{}green\textquotesingle{}, \textquotesingle{}blue\textquotesingle{}]}
\BuiltInTok{print}\NormalTok{(colors[::}\OperatorTok{{-}}\DecValTok{1}\NormalTok{])  }\CommentTok{\# [\textquotesingle{}yellow\textquotesingle{}, \textquotesingle{}blue\textquotesingle{}, \textquotesingle{}green\textquotesingle{}, \textquotesingle{}red\textquotesingle{}]}
\end{Highlighting}
\end{Shaded}

\subsubsection{Why it Matters}\label{why-it-matters-22}

Indexing and slicing make it easy to get exactly the parts of a list you
need. Whether you're grabbing one item, a range of items, or reversing
the list, these tools are essential for working with collections of
data.

\subsubsection{Try It Yourself}\label{try-it-yourself-22}

\begin{enumerate}
\def\labelenumi{\arabic{enumi}.}
\tightlist
\item
  Make a list of 6 numbers and print the first, third, and last
  elements.
\item
  Slice your list to get the middle three elements.
\item
  Use slicing with a step of 2 to get every other number.
\item
  Reverse the list using slicing and print the result.
\end{enumerate}

\subsection{\texorpdfstring{23. List Methods (\texttt{append},
\texttt{extend},
etc.)}{23. List Methods (append, extend, etc.)}}\label{list-methods-append-extend-etc.}

Lists in Python come with built-in methods that make it easy to add,
remove, and modify items. These methods are powerful tools for managing
collections of data.

\subsubsection{Deep Dive}\label{deep-dive-23}

Adding Items

\begin{itemize}
\tightlist
\item
  \texttt{append(x)} → adds a single item to the end of the list.
\item
  \texttt{extend(iterable)} → adds multiple items from another list (or
  any iterable).
\item
  \texttt{insert(i,\ x)} → inserts an item at a specific position.
\end{itemize}

\begin{Shaded}
\begin{Highlighting}[]
\NormalTok{fruits }\OperatorTok{=}\NormalTok{ [}\StringTok{"apple"}\NormalTok{, }\StringTok{"banana"}\NormalTok{]}
\NormalTok{fruits.append(}\StringTok{"cherry"}\NormalTok{)     }\CommentTok{\# [\textquotesingle{}apple\textquotesingle{}, \textquotesingle{}banana\textquotesingle{}, \textquotesingle{}cherry\textquotesingle{}]}
\NormalTok{fruits.extend([}\StringTok{"date"}\NormalTok{, }\StringTok{"fig"}\NormalTok{])  }\CommentTok{\# [\textquotesingle{}apple\textquotesingle{}, \textquotesingle{}banana\textquotesingle{}, \textquotesingle{}cherry\textquotesingle{}, \textquotesingle{}date\textquotesingle{}, \textquotesingle{}fig\textquotesingle{}]}
\NormalTok{fruits.insert(}\DecValTok{1}\NormalTok{, }\StringTok{"kiwi"}\NormalTok{)    }\CommentTok{\# [\textquotesingle{}apple\textquotesingle{}, \textquotesingle{}kiwi\textquotesingle{}, \textquotesingle{}banana\textquotesingle{}, \textquotesingle{}cherry\textquotesingle{}, \textquotesingle{}date\textquotesingle{}, \textquotesingle{}fig\textquotesingle{}]}
\end{Highlighting}
\end{Shaded}

Removing Items

\begin{itemize}
\tightlist
\item
  \texttt{remove(x)} → removes the first occurrence of \texttt{x}.
\item
  \texttt{pop(i)} → removes and returns the item at index \texttt{i}
  (defaults to last).
\item
  \texttt{clear()} → removes all items.
\end{itemize}

\begin{Shaded}
\begin{Highlighting}[]
\NormalTok{fruits.remove(}\StringTok{"banana"}\NormalTok{)   }\CommentTok{\# [\textquotesingle{}apple\textquotesingle{}, \textquotesingle{}kiwi\textquotesingle{}, \textquotesingle{}cherry\textquotesingle{}, \textquotesingle{}date\textquotesingle{}, \textquotesingle{}fig\textquotesingle{}]}
\NormalTok{fruits.pop(}\DecValTok{2}\NormalTok{)             }\CommentTok{\# removes \textquotesingle{}cherry\textquotesingle{}}
\NormalTok{fruits.clear()            }\CommentTok{\# []}
\end{Highlighting}
\end{Shaded}

Finding and Counting

\begin{itemize}
\tightlist
\item
  \texttt{index(x)} → returns the position of the first occurrence of
  \texttt{x}.
\item
  \texttt{count(x)} → returns how many times \texttt{x} appears.
\end{itemize}

\begin{Shaded}
\begin{Highlighting}[]
\NormalTok{nums }\OperatorTok{=}\NormalTok{ [}\DecValTok{1}\NormalTok{, }\DecValTok{2}\NormalTok{, }\DecValTok{2}\NormalTok{, }\DecValTok{3}\NormalTok{]}
\BuiltInTok{print}\NormalTok{(nums.index(}\DecValTok{2}\NormalTok{))  }\CommentTok{\# 1}
\BuiltInTok{print}\NormalTok{(nums.count(}\DecValTok{2}\NormalTok{))  }\CommentTok{\# 2}
\end{Highlighting}
\end{Shaded}

Sorting and Reversing

\begin{itemize}
\tightlist
\item
  \texttt{sort()} → sorts the list in place.
\item
  \texttt{reverse()} → reverses the order of items in place.
\item
  \texttt{sorted(list)} → returns a new sorted list without changing the
  original.
\end{itemize}

\begin{Shaded}
\begin{Highlighting}[]
\NormalTok{letters }\OperatorTok{=}\NormalTok{ [}\StringTok{"b"}\NormalTok{, }\StringTok{"a"}\NormalTok{, }\StringTok{"d"}\NormalTok{, }\StringTok{"c"}\NormalTok{]}
\NormalTok{letters.sort()       }\CommentTok{\# [\textquotesingle{}a\textquotesingle{}, \textquotesingle{}b\textquotesingle{}, \textquotesingle{}c\textquotesingle{}, \textquotesingle{}d\textquotesingle{}]}
\NormalTok{letters.reverse()    }\CommentTok{\# [\textquotesingle{}d\textquotesingle{}, \textquotesingle{}c\textquotesingle{}, \textquotesingle{}b\textquotesingle{}, \textquotesingle{}a\textquotesingle{}]}
\end{Highlighting}
\end{Shaded}

Quick Summary Table

\begin{longtable}[]{@{}
  >{\raggedright\arraybackslash}p{(\linewidth - 4\tabcolsep) * \real{0.1618}}
  >{\raggedright\arraybackslash}p{(\linewidth - 4\tabcolsep) * \real{0.5294}}
  >{\raggedright\arraybackslash}p{(\linewidth - 4\tabcolsep) * \real{0.3088}}@{}}
\toprule\noalign{}
\begin{minipage}[b]{\linewidth}\raggedright
Method
\end{minipage} & \begin{minipage}[b]{\linewidth}\raggedright
Purpose
\end{minipage} & \begin{minipage}[b]{\linewidth}\raggedright
Example
\end{minipage} \\
\midrule\noalign{}
\endhead
\bottomrule\noalign{}
\endlastfoot
\texttt{append(x)} & Add one item at the end & \texttt{lst.append(5)} \\
\texttt{extend()} & Add many items & \texttt{lst.extend({[}6,7{]})} \\
\texttt{insert()} & Insert at a position &
\texttt{lst.insert(1,\ "hi")} \\
\texttt{remove(x)} & Remove first matching value &
\texttt{lst.remove("hi")} \\
\texttt{pop(i)} & Remove by index (or last by default) &
\texttt{lst.pop(0)} \\
\texttt{clear()} & Empty the list & \texttt{lst.clear()} \\
\texttt{index(x)} & Find index of first match & \texttt{lst.index(2)} \\
\texttt{count(x)} & Count how many times value appears &
\texttt{lst.count(2)} \\
\texttt{sort()} & Sort list in place & \texttt{lst.sort()} \\
\texttt{reverse()} & Reverse order in place & \texttt{lst.reverse()} \\
\end{longtable}

\subsubsection{Tiny Code}\label{tiny-code-23}

\begin{Shaded}
\begin{Highlighting}[]
\NormalTok{colors }\OperatorTok{=}\NormalTok{ [}\StringTok{"red"}\NormalTok{, }\StringTok{"blue"}\NormalTok{]}

\NormalTok{colors.append(}\StringTok{"green"}\NormalTok{)}
\NormalTok{colors.extend([}\StringTok{"yellow"}\NormalTok{, }\StringTok{"purple"}\NormalTok{])}
\NormalTok{colors.insert(}\DecValTok{2}\NormalTok{, }\StringTok{"orange"}\NormalTok{)}

\BuiltInTok{print}\NormalTok{(colors)  }\CommentTok{\# [\textquotesingle{}red\textquotesingle{}, \textquotesingle{}blue\textquotesingle{}, \textquotesingle{}orange\textquotesingle{}, \textquotesingle{}green\textquotesingle{}, \textquotesingle{}yellow\textquotesingle{}, \textquotesingle{}purple\textquotesingle{}]}

\NormalTok{colors.remove(}\StringTok{"blue"}\NormalTok{)}
\NormalTok{last }\OperatorTok{=}\NormalTok{ colors.pop()}
\BuiltInTok{print}\NormalTok{(last)    }\CommentTok{\# \textquotesingle{}purple\textquotesingle{}}

\BuiltInTok{print}\NormalTok{(colors.count(}\StringTok{"red"}\NormalTok{))   }\CommentTok{\# 1}
\NormalTok{colors.sort()}
\BuiltInTok{print}\NormalTok{(colors)  }\CommentTok{\# [\textquotesingle{}green\textquotesingle{}, \textquotesingle{}orange\textquotesingle{}, \textquotesingle{}red\textquotesingle{}, \textquotesingle{}yellow\textquotesingle{}]}
\end{Highlighting}
\end{Shaded}

\subsubsection{Why it Matters}\label{why-it-matters-23}

List methods are essential for real-world programming, where data is
always changing. Being able to add, remove, and reorder items makes
lists versatile tools for tasks like managing to-do lists, processing
datasets, or handling user inputs.

\subsubsection{Try It Yourself}\label{try-it-yourself-23}

\begin{enumerate}
\def\labelenumi{\arabic{enumi}.}
\tightlist
\item
  Start with a list of three numbers. Add two more using
  \texttt{append()} and \texttt{extend()}.
\item
  Insert a number at the beginning of the list.
\item
  Remove one number using \texttt{remove()}, then use \texttt{pop()} to
  remove the last one.
\item
  Sort your list and then reverse it. Print the result at each step.
\end{enumerate}

\subsection{24. Tuples}\label{tuples}

A tuple is an ordered collection of items, just like a list, but with
one big difference: tuples are immutable. This means once you create a
tuple, you cannot change its contents---no adding, removing, or
modifying items. Tuples are useful when you want to store data that
should not be altered.

\subsubsection{Deep Dive}\label{deep-dive-24}

Creating Tuples You create a tuple using parentheses \texttt{()} instead
of square brackets:

\begin{Shaded}
\begin{Highlighting}[]
\NormalTok{numbers }\OperatorTok{=}\NormalTok{ (}\DecValTok{1}\NormalTok{, }\DecValTok{2}\NormalTok{, }\DecValTok{3}\NormalTok{)}
\NormalTok{fruits }\OperatorTok{=}\NormalTok{ (}\StringTok{"apple"}\NormalTok{, }\StringTok{"banana"}\NormalTok{, }\StringTok{"cherry"}\NormalTok{)}
\end{Highlighting}
\end{Shaded}

Tuples can hold mixed data types just like lists:

\begin{Shaded}
\begin{Highlighting}[]
\NormalTok{mixed }\OperatorTok{=}\NormalTok{ (}\DecValTok{1}\NormalTok{, }\StringTok{"hello"}\NormalTok{, }\FloatTok{3.14}\NormalTok{, }\VariableTok{True}\NormalTok{)}
\end{Highlighting}
\end{Shaded}

For a single-item tuple, you must include a trailing comma:

\begin{Shaded}
\begin{Highlighting}[]
\NormalTok{single }\OperatorTok{=}\NormalTok{ (}\DecValTok{5}\NormalTok{,)}
\BuiltInTok{print}\NormalTok{(}\BuiltInTok{type}\NormalTok{(single))   }\CommentTok{\# \textless{}class \textquotesingle{}tuple\textquotesingle{}\textgreater{}}
\end{Highlighting}
\end{Shaded}

Accessing Elements Tuples use the same indexing and slicing as lists:

\begin{Shaded}
\begin{Highlighting}[]
\BuiltInTok{print}\NormalTok{(fruits[}\DecValTok{0}\NormalTok{])     }\CommentTok{\# "apple"}
\BuiltInTok{print}\NormalTok{(fruits[}\OperatorTok{{-}}\DecValTok{1}\NormalTok{])    }\CommentTok{\# "cherry"}
\BuiltInTok{print}\NormalTok{(fruits[}\DecValTok{0}\NormalTok{:}\DecValTok{2}\NormalTok{])   }\CommentTok{\# ("apple", "banana")}
\end{Highlighting}
\end{Shaded}

Immutability You cannot modify a tuple after it's created:

\begin{Shaded}
\begin{Highlighting}[]
\NormalTok{fruits[}\DecValTok{0}\NormalTok{] }\OperatorTok{=} \StringTok{"pear"}   \CommentTok{\# ❌ Error: TypeError}
\end{Highlighting}
\end{Shaded}

Tuple Packing and Unpacking You can pack multiple values into a tuple
and unpack them into variables:

\begin{Shaded}
\begin{Highlighting}[]
\NormalTok{point }\OperatorTok{=}\NormalTok{ (}\DecValTok{3}\NormalTok{, }\DecValTok{4}\NormalTok{)}
\NormalTok{x, y }\OperatorTok{=}\NormalTok{ point}
\BuiltInTok{print}\NormalTok{(x, y)   }\CommentTok{\# 3 4}
\end{Highlighting}
\end{Shaded}

Use Cases

\begin{itemize}
\tightlist
\item
  Returning multiple values from a function.
\item
  Fixed collections of data (e.g., coordinates, RGB colors).
\item
  Keys in dictionaries (since tuples are hashable, lists are not).
\end{itemize}

Quick Summary Table

\begin{longtable}[]{@{}lll@{}}
\toprule\noalign{}
Feature & List & Tuple \\
\midrule\noalign{}
\endhead
\bottomrule\noalign{}
\endlastfoot
Syntax & \texttt{{[}1,\ 2,\ 3{]}} & \texttt{(1,\ 2,\ 3)} \\
Mutability & Mutable (can change) & Immutable (cannot) \\
Methods & Many (\texttt{append}, etc.) & Few (\texttt{count},
\texttt{index}) \\
Performance & Slower & Faster (lightweight) \\
\end{longtable}

\subsubsection{Tiny Code}\label{tiny-code-24}

\begin{Shaded}
\begin{Highlighting}[]
\NormalTok{colors }\OperatorTok{=}\NormalTok{ (}\StringTok{"red"}\NormalTok{, }\StringTok{"green"}\NormalTok{, }\StringTok{"blue"}\NormalTok{)}

\BuiltInTok{print}\NormalTok{(colors[}\DecValTok{1}\NormalTok{])       }\CommentTok{\# green}
\BuiltInTok{print}\NormalTok{(}\BuiltInTok{len}\NormalTok{(colors))     }\CommentTok{\# 3}

\CommentTok{\# Unpacking}
\NormalTok{r, g, b }\OperatorTok{=}\NormalTok{ colors}
\BuiltInTok{print}\NormalTok{(r, b)            }\CommentTok{\# red blue}

\CommentTok{\# Methods}
\BuiltInTok{print}\NormalTok{(colors.index(}\StringTok{"blue"}\NormalTok{))  }\CommentTok{\# 2}
\BuiltInTok{print}\NormalTok{(colors.count(}\StringTok{"red"}\NormalTok{))   }\CommentTok{\# 1}
\end{Highlighting}
\end{Shaded}

\subsubsection{Why it Matters}\label{why-it-matters-24}

Tuples give you a safe way to group data that should not be changed,
protecting against accidental modifications. They are also slightly
faster than lists, making them useful when performance matters and
immutability is desired.

\subsubsection{Try It Yourself}\label{try-it-yourself-24}

\begin{enumerate}
\def\labelenumi{\arabic{enumi}.}
\tightlist
\item
  Create a tuple with three of your favorite foods and print the second
  one.
\item
  Try changing one element---observe the error.
\item
  Use unpacking to assign a tuple \texttt{(10,\ 20,\ 30)} into variables
  \texttt{a,\ b,\ c}.
\item
  Create a dictionary where the key is a tuple of coordinates
  \texttt{(x,\ y)} and the value is a place name.
\end{enumerate}

\subsection{25. Sets}\label{sets}

A set in Python is an unordered collection of unique items. Sets are
useful when you need to store data without duplicates or when you want
to perform mathematical operations like union and intersection.

\subsubsection{Deep Dive}\label{deep-dive-25}

Creating Sets You can create a set using curly braces \texttt{\{\}} or
the \texttt{set()} function:

\begin{Shaded}
\begin{Highlighting}[]
\NormalTok{fruits }\OperatorTok{=}\NormalTok{ \{}\StringTok{"apple"}\NormalTok{, }\StringTok{"banana"}\NormalTok{, }\StringTok{"cherry"}\NormalTok{\}}
\NormalTok{numbers }\OperatorTok{=} \BuiltInTok{set}\NormalTok{([}\DecValTok{1}\NormalTok{, }\DecValTok{2}\NormalTok{, }\DecValTok{3}\NormalTok{, }\DecValTok{2}\NormalTok{, }\DecValTok{1}\NormalTok{])  }\CommentTok{\# duplicates removed}
\BuiltInTok{print}\NormalTok{(numbers)  }\CommentTok{\# \{1, 2, 3\}}
\end{Highlighting}
\end{Shaded}

No Duplicates If you try to add duplicates, Python automatically ignores
them:

\begin{Shaded}
\begin{Highlighting}[]
\NormalTok{colors }\OperatorTok{=}\NormalTok{ \{}\StringTok{"red"}\NormalTok{, }\StringTok{"blue"}\NormalTok{, }\StringTok{"red"}\NormalTok{\}}
\BuiltInTok{print}\NormalTok{(colors)  }\CommentTok{\# \{\textquotesingle{}red\textquotesingle{}, \textquotesingle{}blue\textquotesingle{}\}}
\end{Highlighting}
\end{Shaded}

Unordered Sets do not preserve order. You cannot access elements by
index (\texttt{set{[}0{]}} ❌).

Adding and Removing Items

\begin{itemize}
\tightlist
\item
  \texttt{add(x)} → adds an item.
\item
  \texttt{update(iterable)} → adds multiple items.
\item
  \texttt{remove(x)} → removes an item (error if not found).
\item
  \texttt{discard(x)} → removes an item (no error if not found).
\item
  \texttt{pop()} → removes and returns a random item.
\item
  \texttt{clear()} → removes all items.
\end{itemize}

\begin{Shaded}
\begin{Highlighting}[]
\NormalTok{s }\OperatorTok{=}\NormalTok{ \{}\DecValTok{1}\NormalTok{, }\DecValTok{2}\NormalTok{\}}
\NormalTok{s.add(}\DecValTok{3}\NormalTok{)           }\CommentTok{\# \{1, 2, 3\}}
\NormalTok{s.update([}\DecValTok{4}\NormalTok{, }\DecValTok{5}\NormalTok{])   }\CommentTok{\# \{1, 2, 3, 4, 5\}}
\NormalTok{s.remove(}\DecValTok{2}\NormalTok{)        }\CommentTok{\# \{1, 3, 4, 5\}}
\NormalTok{s.discard(}\DecValTok{10}\NormalTok{)      }\CommentTok{\# no error}
\end{Highlighting}
\end{Shaded}

Membership Test Checking if an item exists is fast:

\begin{Shaded}
\begin{Highlighting}[]
\BuiltInTok{print}\NormalTok{(}\StringTok{"apple"} \KeywordTok{in}\NormalTok{ fruits)  }\CommentTok{\# True}
\end{Highlighting}
\end{Shaded}

Set Operations Sets are great for math-like operations:

\begin{Shaded}
\begin{Highlighting}[]
\NormalTok{a }\OperatorTok{=}\NormalTok{ \{}\DecValTok{1}\NormalTok{, }\DecValTok{2}\NormalTok{, }\DecValTok{3}\NormalTok{\}}
\NormalTok{b }\OperatorTok{=}\NormalTok{ \{}\DecValTok{3}\NormalTok{, }\DecValTok{4}\NormalTok{, }\DecValTok{5}\NormalTok{\}}

\BuiltInTok{print}\NormalTok{(a }\OperatorTok{|}\NormalTok{ b)   }\CommentTok{\# union → \{1, 2, 3, 4, 5\}}
\BuiltInTok{print}\NormalTok{(a }\OperatorTok{\&}\NormalTok{ b)   }\CommentTok{\# intersection → \{3\}}
\BuiltInTok{print}\NormalTok{(a }\OperatorTok{{-}}\NormalTok{ b)   }\CommentTok{\# difference → \{1, 2\}}
\BuiltInTok{print}\NormalTok{(a }\OperatorTok{\^{}}\NormalTok{ b)   }\CommentTok{\# symmetric difference → \{1, 2, 4, 5\}}
\end{Highlighting}
\end{Shaded}

Quick Summary Table

\begin{longtable}[]{@{}
  >{\raggedright\arraybackslash}p{(\linewidth - 6\tabcolsep) * \real{0.2667}}
  >{\raggedright\arraybackslash}p{(\linewidth - 6\tabcolsep) * \real{0.1867}}
  >{\raggedright\arraybackslash}p{(\linewidth - 6\tabcolsep) * \real{0.3867}}
  >{\raggedright\arraybackslash}p{(\linewidth - 6\tabcolsep) * \real{0.1600}}@{}}
\toprule\noalign{}
\begin{minipage}[b]{\linewidth}\raggedright
Operation
\end{minipage} & \begin{minipage}[b]{\linewidth}\raggedright
Example
\end{minipage} & \begin{minipage}[b]{\linewidth}\raggedright
Result
\end{minipage} & \begin{minipage}[b]{\linewidth}\raggedright
\end{minipage} \\
\midrule\noalign{}
\endhead
\bottomrule\noalign{}
\endlastfoot
Create set & \texttt{\{1,\ 2,\ 3\}} & \texttt{\{1,\ 2,\ 3\}} & \\
Add item & \texttt{s.add(4)} & \texttt{\{1,\ 2,\ 3,\ 4\}} & \\
Remove item & \texttt{s.remove(2)} & error if not found & \\
Discard item & \texttt{s.discard(2)} & safe remove & \\
Union & `a & b` & combine sets \\
Intersection & \texttt{a\ \&\ b} & common items & \\
Difference & \texttt{a\ -\ b} & items only in \texttt{a} & \\
Symmetric difference & \texttt{a\ \^{}\ b} & items in \texttt{a} or
\texttt{b}, not both & \\
\end{longtable}

\subsubsection{Tiny Code}\label{tiny-code-25}

\begin{Shaded}
\begin{Highlighting}[]
\NormalTok{numbers }\OperatorTok{=}\NormalTok{ \{}\DecValTok{1}\NormalTok{, }\DecValTok{2}\NormalTok{, }\DecValTok{3}\NormalTok{, }\DecValTok{3}\NormalTok{, }\DecValTok{2}\NormalTok{\}}
\BuiltInTok{print}\NormalTok{(numbers)   }\CommentTok{\# \{1, 2, 3\}}

\NormalTok{numbers.add(}\DecValTok{4}\NormalTok{)}
\NormalTok{numbers.discard(}\DecValTok{1}\NormalTok{)}
\BuiltInTok{print}\NormalTok{(numbers)   }\CommentTok{\# \{2, 3, 4\}}

\NormalTok{odds }\OperatorTok{=}\NormalTok{ \{}\DecValTok{1}\NormalTok{, }\DecValTok{3}\NormalTok{, }\DecValTok{5}\NormalTok{\}}
\NormalTok{evens }\OperatorTok{=}\NormalTok{ \{}\DecValTok{2}\NormalTok{, }\DecValTok{4}\NormalTok{, }\DecValTok{6}\NormalTok{\}}
\BuiltInTok{print}\NormalTok{(odds }\OperatorTok{|}\NormalTok{ evens)   }\CommentTok{\# \{1, 2, 3, 4, 5, 6\}}
\end{Highlighting}
\end{Shaded}

\subsubsection{Why it Matters}\label{why-it-matters-25}

Sets make it easy to eliminate duplicates and perform operations like
union or intersection, which are common in data analysis, algorithms,
and everyday programming tasks. They are also optimized for fast
membership testing.

\subsubsection{Try It Yourself}\label{try-it-yourself-25}

\begin{enumerate}
\def\labelenumi{\arabic{enumi}.}
\tightlist
\item
  Create a set of your favorite fruits and add a new one.
\item
  Try adding the same fruit again---see how duplicates are ignored.
\item
  Make two sets of numbers and print their union, intersection, and
  difference.
\item
  Use \texttt{in} to check if an element is in the set.
\end{enumerate}

\subsection{26. Set Operations (union,
intersection)}\label{set-operations-union-intersection}

Sets in Python shine when you use them for mathematical-style
operations. They let you combine, compare, and filter items in powerful
ways. These operations are very fast and are often used in data
processing, searching, and analysis.

\subsubsection{Deep Dive}\label{deep-dive-26}

Union (\texttt{\textbar{}} or \texttt{.union()}) The union of two sets
contains all unique items from both.

\begin{Shaded}
\begin{Highlighting}[]
\NormalTok{a }\OperatorTok{=}\NormalTok{ \{}\DecValTok{1}\NormalTok{, }\DecValTok{2}\NormalTok{, }\DecValTok{3}\NormalTok{\}}
\NormalTok{b }\OperatorTok{=}\NormalTok{ \{}\DecValTok{3}\NormalTok{, }\DecValTok{4}\NormalTok{, }\DecValTok{5}\NormalTok{\}}
\BuiltInTok{print}\NormalTok{(a }\OperatorTok{|}\NormalTok{ b)           }\CommentTok{\# \{1, 2, 3, 4, 5\}}
\BuiltInTok{print}\NormalTok{(a.union(b))      }\CommentTok{\# \{1, 2, 3, 4, 5\}}
\end{Highlighting}
\end{Shaded}

Intersection (\texttt{\&} or \texttt{.intersection()}) The intersection
contains only items present in both sets.

\begin{Shaded}
\begin{Highlighting}[]
\BuiltInTok{print}\NormalTok{(a }\OperatorTok{\&}\NormalTok{ b)           }\CommentTok{\# \{3\}}
\BuiltInTok{print}\NormalTok{(a.intersection(b))  }\CommentTok{\# \{3\}}
\end{Highlighting}
\end{Shaded}

Difference (\texttt{-} or \texttt{.difference()}) The difference
contains items in the first set but not the second.

\begin{Shaded}
\begin{Highlighting}[]
\BuiltInTok{print}\NormalTok{(a }\OperatorTok{{-}}\NormalTok{ b)           }\CommentTok{\# \{1, 2\}}
\BuiltInTok{print}\NormalTok{(b }\OperatorTok{{-}}\NormalTok{ a)           }\CommentTok{\# \{4, 5\}}
\end{Highlighting}
\end{Shaded}

Symmetric Difference (\texttt{\^{}} or
\texttt{.symmetric\_difference()}) The symmetric difference contains
items in either set, but not both.

\begin{Shaded}
\begin{Highlighting}[]
\BuiltInTok{print}\NormalTok{(a }\OperatorTok{\^{}}\NormalTok{ b)           }\CommentTok{\# \{1, 2, 4, 5\}}
\BuiltInTok{print}\NormalTok{(a.symmetric\_difference(b))  }\CommentTok{\# \{1, 2, 4, 5\}}
\end{Highlighting}
\end{Shaded}

Subset and Superset Checks

\begin{itemize}
\tightlist
\item
  \texttt{a\ \textless{}=\ b} → checks if \texttt{a} is a subset of
  \texttt{b}.
\item
  \texttt{a\ \textgreater{}=\ b} → checks if \texttt{a} is a superset of
  \texttt{b}.
\end{itemize}

\begin{Shaded}
\begin{Highlighting}[]
\NormalTok{x }\OperatorTok{=}\NormalTok{ \{}\DecValTok{1}\NormalTok{, }\DecValTok{2}\NormalTok{\}}
\NormalTok{y }\OperatorTok{=}\NormalTok{ \{}\DecValTok{1}\NormalTok{, }\DecValTok{2}\NormalTok{, }\DecValTok{3}\NormalTok{\}}
\BuiltInTok{print}\NormalTok{(x }\OperatorTok{\textless{}=}\NormalTok{ y)   }\CommentTok{\# True (x is subset of y)}
\BuiltInTok{print}\NormalTok{(y }\OperatorTok{\textgreater{}=}\NormalTok{ x)   }\CommentTok{\# True (y is superset of x)}
\end{Highlighting}
\end{Shaded}

Quick Summary Table

\begin{longtable}[]{@{}
  >{\raggedright\arraybackslash}p{(\linewidth - 10\tabcolsep) * \real{0.2632}}
  >{\raggedright\arraybackslash}p{(\linewidth - 10\tabcolsep) * \real{0.0789}}
  >{\raggedright\arraybackslash}p{(\linewidth - 10\tabcolsep) * \real{0.1053}}
  >{\raggedright\arraybackslash}p{(\linewidth - 10\tabcolsep) * \real{0.3026}}
  >{\raggedright\arraybackslash}p{(\linewidth - 10\tabcolsep) * \real{0.0395}}
  >{\raggedright\arraybackslash}p{(\linewidth - 10\tabcolsep) * \real{0.2105}}@{}}
\toprule\noalign{}
\begin{minipage}[b]{\linewidth}\raggedright
Operation
\end{minipage} & \begin{minipage}[b]{\linewidth}\raggedright
Symbol
\end{minipage} & \begin{minipage}[b]{\linewidth}\raggedright
Example
\end{minipage} & \begin{minipage}[b]{\linewidth}\raggedright
Result
\end{minipage} & \begin{minipage}[b]{\linewidth}\raggedright
\end{minipage} & \begin{minipage}[b]{\linewidth}\raggedright
\end{minipage} \\
\midrule\noalign{}
\endhead
\bottomrule\noalign{}
\endlastfoot
Union & ` & ` & `a & b` & all unique items \\
Intersection & \texttt{\&} & \texttt{a\ \&\ b} & common items & & \\
Difference & \texttt{-} & \texttt{a\ -\ b} & in \texttt{a} not
\texttt{b} & & \\
Symmetric difference & \texttt{\^{}} & \texttt{a\ \^{}\ b} & in
\texttt{a} or \texttt{b}, not both & & \\
Subset & \texttt{\textless{}=} & \texttt{a\ \textless{}=\ b} &
True/False & & \\
Superset & \texttt{\textgreater{}=} & \texttt{a\ \textgreater{}=\ b} &
True/False & & \\
\end{longtable}

\subsubsection{Tiny Code}\label{tiny-code-26}

\begin{Shaded}
\begin{Highlighting}[]
\NormalTok{a }\OperatorTok{=}\NormalTok{ \{}\DecValTok{1}\NormalTok{, }\DecValTok{2}\NormalTok{, }\DecValTok{3}\NormalTok{\}}
\NormalTok{b }\OperatorTok{=}\NormalTok{ \{}\DecValTok{3}\NormalTok{, }\DecValTok{4}\NormalTok{, }\DecValTok{5}\NormalTok{\}}

\BuiltInTok{print}\NormalTok{(}\StringTok{"Union:"}\NormalTok{, a }\OperatorTok{|}\NormalTok{ b)              }\CommentTok{\# \{1, 2, 3, 4, 5\}}
\BuiltInTok{print}\NormalTok{(}\StringTok{"Intersection:"}\NormalTok{, a }\OperatorTok{\&}\NormalTok{ b)       }\CommentTok{\# \{3\}}
\BuiltInTok{print}\NormalTok{(}\StringTok{"Difference:"}\NormalTok{, a }\OperatorTok{{-}}\NormalTok{ b)         }\CommentTok{\# \{1, 2\}}
\BuiltInTok{print}\NormalTok{(}\StringTok{"SymDiff:"}\NormalTok{, a }\OperatorTok{\^{}}\NormalTok{ b)            }\CommentTok{\# \{1, 2, 4, 5\}}

\BuiltInTok{print}\NormalTok{(}\StringTok{"Subset?"}\NormalTok{, \{}\DecValTok{1}\NormalTok{, }\DecValTok{2}\NormalTok{\} }\OperatorTok{\textless{}=}\NormalTok{ a)       }\CommentTok{\# True}
\BuiltInTok{print}\NormalTok{(}\StringTok{"Superset?"}\NormalTok{, a }\OperatorTok{\textgreater{}=}\NormalTok{ \{}\DecValTok{2}\NormalTok{, }\DecValTok{3}\NormalTok{\})     }\CommentTok{\# True}
\end{Highlighting}
\end{Shaded}

\subsubsection{Why it Matters}\label{why-it-matters-26}

Set operations allow you to quickly solve problems like finding common
elements, removing duplicates, or checking membership across
collections. They map directly to real-world logic such as ``all
users,'' ``users in both groups,'' or ``items missing from one list.''

\subsubsection{Try It Yourself}\label{try-it-yourself-26}

\begin{enumerate}
\def\labelenumi{\arabic{enumi}.}
\tightlist
\item
  Make two sets of numbers: \texttt{\{1,\ 2,\ 3,\ 4\}} and
  \texttt{\{3,\ 4,\ 5,\ 6\}}. Find their union, intersection, and
  difference.
\item
  Create a set of vowels and a set of letters in the word
  \texttt{"python"}. Find the intersection to see which vowels appear.
\item
  Check if \texttt{\{1,\ 2\}} is a subset of \texttt{\{1,\ 2,\ 3,\ 4\}}.
\item
  Try symmetric difference between \texttt{\{a,\ b,\ c\}} and
  \texttt{\{b,\ c,\ d\}}.
\end{enumerate}

\subsection{27. Dictionaries (creation \&
basics)}\label{dictionaries-creation-basics}

A dictionary in Python is a collection of key--value pairs. Instead of
accessing items by index like lists, you access them by their keys. This
makes dictionaries very powerful for storing and retrieving data when
you want to associate labels with values.

\subsubsection{Deep Dive}\label{deep-dive-27}

Creating Dictionaries You create a dictionary using curly braces
\texttt{\{\}} with keys and values separated by colons:

\begin{Shaded}
\begin{Highlighting}[]
\NormalTok{person }\OperatorTok{=}\NormalTok{ \{}\StringTok{"name"}\NormalTok{: }\StringTok{"Alice"}\NormalTok{, }\StringTok{"age"}\NormalTok{: }\DecValTok{25}\NormalTok{, }\StringTok{"city"}\NormalTok{: }\StringTok{"Paris"}\NormalTok{\}}
\end{Highlighting}
\end{Shaded}

Accessing Values You get values by their keys, not by position:

\begin{Shaded}
\begin{Highlighting}[]
\BuiltInTok{print}\NormalTok{(person[}\StringTok{"name"}\NormalTok{])   }\CommentTok{\# "Alice"}
\BuiltInTok{print}\NormalTok{(person[}\StringTok{"age"}\NormalTok{])    }\CommentTok{\# 25}
\end{Highlighting}
\end{Shaded}

Adding and Updating Dictionaries are mutable---you can add new
key--value pairs or update existing ones:

\begin{Shaded}
\begin{Highlighting}[]
\NormalTok{person[}\StringTok{"job"}\NormalTok{] }\OperatorTok{=} \StringTok{"Engineer"}
\NormalTok{person[}\StringTok{"age"}\NormalTok{] }\OperatorTok{=} \DecValTok{26}
\end{Highlighting}
\end{Shaded}

Keys Must Be Unique If you repeat a key, the latest value will overwrite
the earlier one:

\begin{Shaded}
\begin{Highlighting}[]
\NormalTok{data }\OperatorTok{=}\NormalTok{ \{}\StringTok{"a"}\NormalTok{: }\DecValTok{1}\NormalTok{, }\StringTok{"a"}\NormalTok{: }\DecValTok{2}\NormalTok{\}}
\BuiltInTok{print}\NormalTok{(data)   }\CommentTok{\# \{"a": 2\}}
\end{Highlighting}
\end{Shaded}

Dictionary Keys and Values

\begin{itemize}
\tightlist
\item
  Keys must be immutable types (strings, numbers, tuples).
\item
  Values can be any type: strings, numbers, lists, or even other
  dictionaries.
\end{itemize}

Empty Dictionary You can start with an empty dictionary:

\begin{Shaded}
\begin{Highlighting}[]
\NormalTok{empty }\OperatorTok{=}\NormalTok{ \{\}}
\end{Highlighting}
\end{Shaded}

Quick Summary Table

\begin{longtable}[]{@{}
  >{\raggedright\arraybackslash}p{(\linewidth - 4\tabcolsep) * \real{0.2174}}
  >{\raggedright\arraybackslash}p{(\linewidth - 4\tabcolsep) * \real{0.5109}}
  >{\raggedright\arraybackslash}p{(\linewidth - 4\tabcolsep) * \real{0.2717}}@{}}
\toprule\noalign{}
\begin{minipage}[b]{\linewidth}\raggedright
Operation
\end{minipage} & \begin{minipage}[b]{\linewidth}\raggedright
Example
\end{minipage} & \begin{minipage}[b]{\linewidth}\raggedright
Result
\end{minipage} \\
\midrule\noalign{}
\endhead
\bottomrule\noalign{}
\endlastfoot
Create dictionary & \texttt{\{"a":\ 1,\ "b":\ 2\}} &
\texttt{\{\textquotesingle{}a\textquotesingle{}:\ 1,\ \textquotesingle{}b\textquotesingle{}:\ 2\}} \\
Access by key & \texttt{person{[}"name"{]}} & \texttt{"Alice"} \\
Add / update & \texttt{person{[}"age"{]}\ =\ 30} & changes value for
\texttt{"age"} \\
Empty dictionary & \texttt{\{\}} & \texttt{\{\}} \\
Mixed values allowed &
\texttt{\{"id":\ 1,\ "tags":\ {[}"x",\ "y"{]},\ "active":\ True\}} &
valid dictionary \\
\end{longtable}

\subsubsection{Tiny Code}\label{tiny-code-27}

\begin{Shaded}
\begin{Highlighting}[]
\NormalTok{car }\OperatorTok{=}\NormalTok{ \{}\StringTok{"brand"}\NormalTok{: }\StringTok{"Toyota"}\NormalTok{, }\StringTok{"model"}\NormalTok{: }\StringTok{"Corolla"}\NormalTok{, }\StringTok{"year"}\NormalTok{: }\DecValTok{2020}\NormalTok{\}}

\BuiltInTok{print}\NormalTok{(car[}\StringTok{"brand"}\NormalTok{])      }\CommentTok{\# Toyota}
\NormalTok{car[}\StringTok{"year"}\NormalTok{] }\OperatorTok{=} \DecValTok{2021}       \CommentTok{\# update value}
\NormalTok{car[}\StringTok{"color"}\NormalTok{] }\OperatorTok{=} \StringTok{"blue"}    \CommentTok{\# add new key}
\BuiltInTok{print}\NormalTok{(car)}
\end{Highlighting}
\end{Shaded}

\subsubsection{Why it Matters}\label{why-it-matters-27}

Dictionaries give you a natural way to organize and retrieve data by
name instead of position. They are essential for representing structured
data, like database records, configurations, or JSON data from APIs.

\subsubsection{Try It Yourself}\label{try-it-yourself-27}

\begin{enumerate}
\def\labelenumi{\arabic{enumi}.}
\tightlist
\item
  Create a dictionary called \texttt{student} with keys \texttt{"name"},
  \texttt{"age"}, and \texttt{"grade"}.
\item
  Access and print the \texttt{"grade"}.
\item
  Update the \texttt{"age"} to a new number.
\item
  Add a new key \texttt{"passed"} with the value \texttt{True}.
\item
  Print the whole dictionary to see the changes.
\end{enumerate}

\subsection{28. Dictionary Methods}\label{dictionary-methods}

Dictionaries come with built-in methods that make it easy to work with
their keys and values. These methods let you add, remove, and inspect
data in a structured way.

\subsubsection{Deep Dive}\label{deep-dive-28}

Accessing Keys, Values, and Items

\begin{itemize}
\tightlist
\item
  \texttt{dict.keys()} → returns all keys.
\item
  \texttt{dict.values()} → returns all values.
\item
  \texttt{dict.items()} → returns pairs of \texttt{(key,\ value)}.
\end{itemize}

\begin{Shaded}
\begin{Highlighting}[]
\NormalTok{person }\OperatorTok{=}\NormalTok{ \{}\StringTok{"name"}\NormalTok{: }\StringTok{"Alice"}\NormalTok{, }\StringTok{"age"}\NormalTok{: }\DecValTok{25}\NormalTok{\}}

\BuiltInTok{print}\NormalTok{(person.keys())    }\CommentTok{\# dict\_keys([\textquotesingle{}name\textquotesingle{}, \textquotesingle{}age\textquotesingle{}])}
\BuiltInTok{print}\NormalTok{(person.values())  }\CommentTok{\# dict\_values([\textquotesingle{}Alice\textquotesingle{}, 25])}
\BuiltInTok{print}\NormalTok{(person.items())   }\CommentTok{\# dict\_items([(\textquotesingle{}name\textquotesingle{}, \textquotesingle{}Alice\textquotesingle{}), (\textquotesingle{}age\textquotesingle{}, 25)])}
\end{Highlighting}
\end{Shaded}

Adding and Updating

\begin{itemize}
\tightlist
\item
  \texttt{update(other\_dict)} → adds or updates key--value pairs.
\end{itemize}

\begin{Shaded}
\begin{Highlighting}[]
\NormalTok{person.update(\{}\StringTok{"age"}\NormalTok{: }\DecValTok{26}\NormalTok{, }\StringTok{"city"}\NormalTok{: }\StringTok{"Paris"}\NormalTok{\})}
\end{Highlighting}
\end{Shaded}

Removing Items

\begin{itemize}
\tightlist
\item
  \texttt{pop(key)} → removes and returns the value for a key.
\item
  \texttt{popitem()} → removes and returns the last inserted pair.
\item
  \texttt{del\ dict{[}key{]}} → deletes a key--value pair.
\item
  \texttt{clear()} → empties the dictionary.
\end{itemize}

\begin{Shaded}
\begin{Highlighting}[]
\BuiltInTok{print}\NormalTok{(person.pop(}\StringTok{"age"}\NormalTok{))     }\CommentTok{\# 26}
\BuiltInTok{print}\NormalTok{(person.popitem())      }\CommentTok{\# (\textquotesingle{}city\textquotesingle{}, \textquotesingle{}Paris\textquotesingle{})}
\KeywordTok{del}\NormalTok{ person[}\StringTok{"name"}\NormalTok{]           }\CommentTok{\# removes "name"}
\NormalTok{person.clear()               }\CommentTok{\# \{\}}
\end{Highlighting}
\end{Shaded}

Get with Default

\begin{itemize}
\tightlist
\item
  \texttt{get(key,\ default)} → safely gets a value; returns
  \texttt{default} if the key doesn't exist.
\end{itemize}

\begin{Shaded}
\begin{Highlighting}[]
\NormalTok{person }\OperatorTok{=}\NormalTok{ \{}\StringTok{"name"}\NormalTok{: }\StringTok{"Alice"}\NormalTok{\}}
\BuiltInTok{print}\NormalTok{(person.get(}\StringTok{"age"}\NormalTok{, }\StringTok{"Not found"}\NormalTok{))  }\CommentTok{\# "Not found"}
\end{Highlighting}
\end{Shaded}

From Keys

\begin{itemize}
\tightlist
\item
  \texttt{dict.fromkeys(keys,\ value)} → creates a dictionary with given
  keys and default value.
\end{itemize}

\begin{Shaded}
\begin{Highlighting}[]
\NormalTok{keys }\OperatorTok{=}\NormalTok{ [}\StringTok{"a"}\NormalTok{, }\StringTok{"b"}\NormalTok{, }\StringTok{"c"}\NormalTok{]}
\BuiltInTok{print}\NormalTok{(}\BuiltInTok{dict}\NormalTok{.fromkeys(keys, }\DecValTok{0}\NormalTok{))   }\CommentTok{\# \{\textquotesingle{}a\textquotesingle{}: 0, \textquotesingle{}b\textquotesingle{}: 0, \textquotesingle{}c\textquotesingle{}: 0\}}
\end{Highlighting}
\end{Shaded}

Quick Summary Table

\begin{longtable}[]{@{}
  >{\raggedright\arraybackslash}p{(\linewidth - 4\tabcolsep) * \real{0.1644}}
  >{\raggedright\arraybackslash}p{(\linewidth - 4\tabcolsep) * \real{0.4110}}
  >{\raggedright\arraybackslash}p{(\linewidth - 4\tabcolsep) * \real{0.4247}}@{}}
\toprule\noalign{}
\begin{minipage}[b]{\linewidth}\raggedright
Method
\end{minipage} & \begin{minipage}[b]{\linewidth}\raggedright
Purpose
\end{minipage} & \begin{minipage}[b]{\linewidth}\raggedright
Example
\end{minipage} \\
\midrule\noalign{}
\endhead
\bottomrule\noalign{}
\endlastfoot
\texttt{keys()} & Get all keys & \texttt{person.keys()} \\
\texttt{values()} & Get all values & \texttt{person.values()} \\
\texttt{items()} & Get all pairs & \texttt{person.items()} \\
\texttt{update()} & Add/update multiple pairs &
\texttt{person.update(\{"age":\ 26\})} \\
\texttt{pop(key)} & Remove by key, return value &
\texttt{person.pop("name")} \\
\texttt{popitem()} & Remove last inserted pair &
\texttt{person.popitem()} \\
\texttt{get()} & Safe value access with default &
\texttt{person.get("city",\ "Unknown")} \\
\texttt{clear()} & Remove all pairs & \texttt{person.clear()} \\
\texttt{fromkeys()} & Create new dict from keys &
\texttt{dict.fromkeys({[}"x",\ "y"{]},\ 1)} \\
\end{longtable}

\subsubsection{Tiny Code}\label{tiny-code-28}

\begin{Shaded}
\begin{Highlighting}[]
\NormalTok{student }\OperatorTok{=}\NormalTok{ \{}\StringTok{"name"}\NormalTok{: }\StringTok{"Bob"}\NormalTok{, }\StringTok{"age"}\NormalTok{: }\DecValTok{20}\NormalTok{, }\StringTok{"grade"}\NormalTok{: }\StringTok{"A"}\NormalTok{\}}

\BuiltInTok{print}\NormalTok{(student.keys())       }\CommentTok{\# dict\_keys([\textquotesingle{}name\textquotesingle{}, \textquotesingle{}age\textquotesingle{}, \textquotesingle{}grade\textquotesingle{}])}
\BuiltInTok{print}\NormalTok{(student.get(}\StringTok{"city"}\NormalTok{, }\StringTok{"N/A"}\NormalTok{))  }\CommentTok{\# N/A}

\NormalTok{student.update(\{}\StringTok{"age"}\NormalTok{: }\DecValTok{21}\NormalTok{\})}
\BuiltInTok{print}\NormalTok{(student)}

\NormalTok{student.pop(}\StringTok{"grade"}\NormalTok{)}
\BuiltInTok{print}\NormalTok{(student)}
\end{Highlighting}
\end{Shaded}

\subsubsection{Why it Matters}\label{why-it-matters-28}

Dictionary methods let you manipulate structured data efficiently.
Whether you're cleaning data, merging information, or safely handling
missing values, these methods are essential for working with real-world
datasets and configurations.

\subsubsection{Try It Yourself}\label{try-it-yourself-28}

\begin{enumerate}
\def\labelenumi{\arabic{enumi}.}
\tightlist
\item
  Create a dictionary \texttt{book} with \texttt{"title"},
  \texttt{"author"}, and \texttt{"year"}.
\item
  Use \texttt{keys()}, \texttt{values()}, and \texttt{items()} to
  inspect it.
\item
  Update the \texttt{"year"} to the current year using
  \texttt{update()}.
\item
  Use \texttt{get()} to safely access a missing \texttt{"publisher"} key
  with a default value.
\item
  Clear the dictionary with \texttt{clear()}.
\end{enumerate}

\subsection{30. Nested Structures}\label{nested-structures}

A nested structure means putting one data structure inside another---for
example, a list of lists, a dictionary containing lists, or even a list
of dictionaries. Nested structures are common when representing more
complex, real-world data.

\subsubsection{Deep Dive}\label{deep-dive-29}

Lists Inside Lists You can create multi-dimensional lists:

\begin{Shaded}
\begin{Highlighting}[]
\NormalTok{matrix }\OperatorTok{=}\NormalTok{ [}
\NormalTok{    [}\DecValTok{1}\NormalTok{, }\DecValTok{2}\NormalTok{, }\DecValTok{3}\NormalTok{],}
\NormalTok{    [}\DecValTok{4}\NormalTok{, }\DecValTok{5}\NormalTok{, }\DecValTok{6}\NormalTok{],}
\NormalTok{    [}\DecValTok{7}\NormalTok{, }\DecValTok{8}\NormalTok{, }\DecValTok{9}\NormalTok{]}
\NormalTok{]}
\BuiltInTok{print}\NormalTok{(matrix[}\DecValTok{0}\NormalTok{][}\DecValTok{1}\NormalTok{])   }\CommentTok{\# 2}
\end{Highlighting}
\end{Shaded}

Dictionaries with Lists Values in a dictionary can be lists:

\begin{Shaded}
\begin{Highlighting}[]
\NormalTok{student }\OperatorTok{=}\NormalTok{ \{}
    \StringTok{"name"}\NormalTok{: }\StringTok{"Alice"}\NormalTok{,}
    \StringTok{"grades"}\NormalTok{: [}\DecValTok{85}\NormalTok{, }\DecValTok{90}\NormalTok{, }\DecValTok{92}\NormalTok{]}
\NormalTok{\}}
\BuiltInTok{print}\NormalTok{(student[}\StringTok{"grades"}\NormalTok{][}\DecValTok{1}\NormalTok{])   }\CommentTok{\# 90}
\end{Highlighting}
\end{Shaded}

Lists of Dictionaries A list can contain multiple dictionaries, useful
for structured records:

\begin{Shaded}
\begin{Highlighting}[]
\NormalTok{people }\OperatorTok{=}\NormalTok{ [}
\NormalTok{    \{}\StringTok{"name"}\NormalTok{: }\StringTok{"Alice"}\NormalTok{, }\StringTok{"age"}\NormalTok{: }\DecValTok{25}\NormalTok{\},}
\NormalTok{    \{}\StringTok{"name"}\NormalTok{: }\StringTok{"Bob"}\NormalTok{, }\StringTok{"age"}\NormalTok{: }\DecValTok{30}\NormalTok{\}}
\NormalTok{]}
\BuiltInTok{print}\NormalTok{(people[}\DecValTok{1}\NormalTok{][}\StringTok{"name"}\NormalTok{])   }\CommentTok{\# Bob}
\end{Highlighting}
\end{Shaded}

Dictionaries of Dictionaries Dictionaries can be nested, too:

\begin{Shaded}
\begin{Highlighting}[]
\NormalTok{users }\OperatorTok{=}\NormalTok{ \{}
    \StringTok{"alice"}\NormalTok{: \{}\StringTok{"age"}\NormalTok{: }\DecValTok{25}\NormalTok{, }\StringTok{"city"}\NormalTok{: }\StringTok{"Paris"}\NormalTok{\},}
    \StringTok{"bob"}\NormalTok{: \{}\StringTok{"age"}\NormalTok{: }\DecValTok{30}\NormalTok{, }\StringTok{"city"}\NormalTok{: }\StringTok{"London"}\NormalTok{\}}
\NormalTok{\}}
\BuiltInTok{print}\NormalTok{(users[}\StringTok{"bob"}\NormalTok{][}\StringTok{"city"}\NormalTok{])   }\CommentTok{\# London}
\end{Highlighting}
\end{Shaded}

Iteration Through Nested Structures You can use loops inside loops to
navigate deeper levels:

\begin{Shaded}
\begin{Highlighting}[]
\ControlFlowTok{for}\NormalTok{ row }\KeywordTok{in}\NormalTok{ matrix:}
    \ControlFlowTok{for}\NormalTok{ val }\KeywordTok{in}\NormalTok{ row:}
        \BuiltInTok{print}\NormalTok{(val, end}\OperatorTok{=}\StringTok{" "}\NormalTok{)}
\end{Highlighting}
\end{Shaded}

Quick Summary Table

\begin{longtable}[]{@{}
  >{\raggedright\arraybackslash}p{(\linewidth - 4\tabcolsep) * \real{0.2222}}
  >{\raggedright\arraybackslash}p{(\linewidth - 4\tabcolsep) * \real{0.4444}}
  >{\raggedright\arraybackslash}p{(\linewidth - 4\tabcolsep) * \real{0.3333}}@{}}
\toprule\noalign{}
\begin{minipage}[b]{\linewidth}\raggedright
Nested Type
\end{minipage} & \begin{minipage}[b]{\linewidth}\raggedright
Example
\end{minipage} & \begin{minipage}[b]{\linewidth}\raggedright
Access Example
\end{minipage} \\
\midrule\noalign{}
\endhead
\bottomrule\noalign{}
\endlastfoot
List of lists & \texttt{{[}{[}1,2{]},{[}3,4{]}{]}} &
\texttt{x{[}0{]}{[}1{]}\ →\ 2} \\
Dict with list & \texttt{\{"scores":{[}10,20{]}\}} &
\texttt{d{[}"scores"{]}{[}0{]}\ →\ 10} \\
List of dicts & \texttt{{[}\{"n":"a"\},\{"n":"b"\}{]}} &
\texttt{lst{[}1{]}{[}"n"{]}\ →\ "b"} \\
Dict of dicts & \texttt{\{"a":\{"x":1\},\ "b":\{"x":2\}\}} &
\texttt{d{[}"b"{]}{[}"x"{]}\ →\ 2} \\
\end{longtable}

\subsubsection{Tiny Code}\label{tiny-code-29}

\begin{Shaded}
\begin{Highlighting}[]
\NormalTok{classrooms }\OperatorTok{=}\NormalTok{ \{}
    \StringTok{"A"}\NormalTok{: [}\StringTok{"Alice"}\NormalTok{, }\StringTok{"Bob"}\NormalTok{],}
    \StringTok{"B"}\NormalTok{: [}\StringTok{"Charlie"}\NormalTok{, }\StringTok{"Diana"}\NormalTok{]}
\NormalTok{\}}

\ControlFlowTok{for}\NormalTok{ room, students }\KeywordTok{in}\NormalTok{ classrooms.items():}
    \BuiltInTok{print}\NormalTok{(}\StringTok{"Room:"}\NormalTok{, room)}
    \ControlFlowTok{for}\NormalTok{ student }\KeywordTok{in}\NormalTok{ students:}
        \BuiltInTok{print}\NormalTok{(}\StringTok{"{-}"}\NormalTok{, student)}
\end{Highlighting}
\end{Shaded}

\subsubsection{Why it Matters}\label{why-it-matters-29}

Real-world data is rarely flat---it's often hierarchical or structured
in layers (like JSON from APIs, database rows with embedded fields, or
spreadsheets). Nested structures let you represent and work with this
complexity directly in Python.

\subsubsection{Try It Yourself}\label{try-it-yourself-29}

\begin{enumerate}
\def\labelenumi{\arabic{enumi}.}
\tightlist
\item
  Create a list of lists to represent a 3×3 grid and print the center
  value.
\item
  Make a dictionary with a key \texttt{"friends"} pointing to a list of
  three names. Print the second name.
\item
  Create a list of dictionaries, each with \texttt{"title"} and
  \texttt{"year"}, for your favorite movies. Print the title of the last
  one.
\item
  Build a dictionary of dictionaries representing two countries with
  their capital cities, then print one capital.
\end{enumerate}

\section{Chapter 4. Functions}\label{chapter-4.-functions}

\subsection{\texorpdfstring{31. Defining a Function
(\texttt{def})}{31. Defining a Function (def)}}\label{defining-a-function-def}

A function is a reusable block of code that performs a specific task.
Functions let you avoid repetition, organize your code, and make
programs easier to understand. In Python, you define a function using
the \texttt{def} keyword.

\subsubsection{Deep Dive}\label{deep-dive-30}

Basic Function Definition

\begin{Shaded}
\begin{Highlighting}[]
\KeywordTok{def}\NormalTok{ greet():}
    \BuiltInTok{print}\NormalTok{(}\StringTok{"Hello!"}\NormalTok{)}
\end{Highlighting}
\end{Shaded}

Calling \texttt{greet()} runs the code inside.

Functions with Parameters You can pass data into functions using
parameters:

\begin{Shaded}
\begin{Highlighting}[]
\KeywordTok{def}\NormalTok{ greet(name):}
    \BuiltInTok{print}\NormalTok{(}\StringTok{"Hello,"}\NormalTok{, name)}

\NormalTok{greet(}\StringTok{"Alice"}\NormalTok{)   }\CommentTok{\# Hello, Alice}
\end{Highlighting}
\end{Shaded}

Return Values Functions can return data with \texttt{return}:

\begin{Shaded}
\begin{Highlighting}[]
\KeywordTok{def}\NormalTok{ add(a, b):}
    \ControlFlowTok{return}\NormalTok{ a }\OperatorTok{+}\NormalTok{ b}

\NormalTok{result }\OperatorTok{=}\NormalTok{ add(}\DecValTok{3}\NormalTok{, }\DecValTok{4}\NormalTok{)}
\BuiltInTok{print}\NormalTok{(result)   }\CommentTok{\# 7}
\end{Highlighting}
\end{Shaded}

Default Behavior

\begin{itemize}
\tightlist
\item
  If a function doesn't explicitly \texttt{return}, it returns
  \texttt{None}.
\item
  Functions can be defined before or after other code, but must be
  defined before they are called.
\end{itemize}

Why Use Functions?

\begin{itemize}
\tightlist
\item
  Reusability: write once, use many times.
\item
  Readability: group code into meaningful chunks.
\item
  Maintainability: easier to test and fix.
\end{itemize}

Quick Summary Table

\begin{longtable}[]{@{}
  >{\raggedright\arraybackslash}p{(\linewidth - 4\tabcolsep) * \real{0.2381}}
  >{\raggedright\arraybackslash}p{(\linewidth - 4\tabcolsep) * \real{0.3968}}
  >{\raggedright\arraybackslash}p{(\linewidth - 4\tabcolsep) * \real{0.3651}}@{}}
\toprule\noalign{}
\begin{minipage}[b]{\linewidth}\raggedright
Feature
\end{minipage} & \begin{minipage}[b]{\linewidth}\raggedright
Example
\end{minipage} & \begin{minipage}[b]{\linewidth}\raggedright
Notes
\end{minipage} \\
\midrule\noalign{}
\endhead
\bottomrule\noalign{}
\endlastfoot
Define function & \texttt{def\ f():} & Code block indented \\
Call function & \texttt{f()} & Executes the block \\
With parameter & \texttt{def\ f(x):} & Pass value when calling \\
With return & \texttt{def\ f(x):\ return\ x+1} & Gives back a value \\
Implicit return & function without \texttt{return} & Returns
\texttt{None} \\
\end{longtable}

\subsubsection{Tiny Code}\label{tiny-code-30}

\begin{Shaded}
\begin{Highlighting}[]
\KeywordTok{def}\NormalTok{ square(n):}
    \ControlFlowTok{return}\NormalTok{ n }\OperatorTok{*}\NormalTok{ n}

\BuiltInTok{print}\NormalTok{(square(}\DecValTok{5}\NormalTok{))   }\CommentTok{\# 25}

\KeywordTok{def}\NormalTok{ welcome(name):}
    \BuiltInTok{print}\NormalTok{(}\StringTok{"Welcome,"}\NormalTok{, name)}

\NormalTok{welcome(}\StringTok{"Bob"}\NormalTok{)     }\CommentTok{\# Welcome, Bob}
\end{Highlighting}
\end{Shaded}

\subsubsection{Why it Matters}\label{why-it-matters-30}

Functions are the building blocks of programs. They let you break down
complex problems into smaller pieces, reuse code efficiently, and make
your programs easier to maintain and understand.

\subsubsection{Try It Yourself}\label{try-it-yourself-30}

\begin{enumerate}
\def\labelenumi{\arabic{enumi}.}
\tightlist
\item
  Write a function \texttt{hello()} that prints
  \texttt{"Hello,\ Python!"}.
\item
  Write a function \texttt{double(x)} that returns twice the number
  given.
\item
  Define a function \texttt{say\_name(name)} that prints
  \texttt{"My\ name\ is\ ..."} with the input name.
\item
  Call your functions multiple times to see the benefits of reuse.
\end{enumerate}

\subsection{32. Function Arguments}\label{function-arguments}

Functions can take arguments (also called parameters) so you can pass
information into them. Arguments make functions flexible because they
can work with different inputs instead of being hardcoded.

\subsubsection{Deep Dive}\label{deep-dive-31}

Positional Arguments The most common type---values are matched to
parameters in order.

\begin{Shaded}
\begin{Highlighting}[]
\KeywordTok{def}\NormalTok{ greet(name, age):}
    \BuiltInTok{print}\NormalTok{(}\StringTok{"Hello,"}\NormalTok{, name, }\StringTok{"you are"}\NormalTok{, age, }\StringTok{"years old"}\NormalTok{)}

\NormalTok{greet(}\StringTok{"Alice"}\NormalTok{, }\DecValTok{25}\NormalTok{)}
\end{Highlighting}
\end{Shaded}

Keyword Arguments You can pass values by naming the parameters. This
makes the call clearer and order doesn't matter.

\begin{Shaded}
\begin{Highlighting}[]
\NormalTok{greet(age}\OperatorTok{=}\DecValTok{30}\NormalTok{, name}\OperatorTok{=}\StringTok{"Bob"}\NormalTok{)}
\end{Highlighting}
\end{Shaded}

Default Arguments You can give parameters default values, making them
optional when calling the function.

\begin{Shaded}
\begin{Highlighting}[]
\KeywordTok{def}\NormalTok{ greet(name, age}\OperatorTok{=}\DecValTok{18}\NormalTok{):}
    \BuiltInTok{print}\NormalTok{(}\StringTok{"Hello,"}\NormalTok{, name, }\StringTok{"you are"}\NormalTok{, age)}

\NormalTok{greet(}\StringTok{"Charlie"}\NormalTok{)      }\CommentTok{\# uses default age = 18}
\NormalTok{greet(}\StringTok{"Diana"}\NormalTok{, }\DecValTok{22}\NormalTok{)    }\CommentTok{\# overrides default}
\end{Highlighting}
\end{Shaded}

Mixing Arguments When mixing, positional arguments come first, then
keyword arguments.

\begin{Shaded}
\begin{Highlighting}[]
\KeywordTok{def}\NormalTok{ student(name, grade}\OperatorTok{=}\StringTok{"A"}\NormalTok{):}
    \BuiltInTok{print}\NormalTok{(name, }\StringTok{"got grade"}\NormalTok{, grade)}

\NormalTok{student(}\StringTok{"Eva"}\NormalTok{)            }\CommentTok{\# Eva got grade A}
\NormalTok{student(}\StringTok{"Frank"}\NormalTok{, grade}\OperatorTok{=}\StringTok{"B"}\NormalTok{)}
\end{Highlighting}
\end{Shaded}

Wrong Usage Causes Errors

\begin{Shaded}
\begin{Highlighting}[]
\NormalTok{greet(}\DecValTok{25}\NormalTok{, }\StringTok{"Alice"}\NormalTok{)   }\CommentTok{\# order matters for positional}
\end{Highlighting}
\end{Shaded}

Quick Summary Table

\begin{longtable}[]{@{}
  >{\raggedright\arraybackslash}p{(\linewidth - 4\tabcolsep) * \real{0.1688}}
  >{\raggedright\arraybackslash}p{(\linewidth - 4\tabcolsep) * \real{0.4416}}
  >{\raggedright\arraybackslash}p{(\linewidth - 4\tabcolsep) * \real{0.3896}}@{}}
\toprule\noalign{}
\begin{minipage}[b]{\linewidth}\raggedright
Type
\end{minipage} & \begin{minipage}[b]{\linewidth}\raggedright
Example call
\end{minipage} & \begin{minipage}[b]{\linewidth}\raggedright
Notes
\end{minipage} \\
\midrule\noalign{}
\endhead
\bottomrule\noalign{}
\endlastfoot
Positional & \texttt{f(1,\ 2)} & Order matters \\
Keyword & \texttt{f(b=2,\ a=1)} & Order doesn't matter \\
Default value & \texttt{f(1)} when defined as \texttt{f(a,\ b=2)} & Uses
default if missing \\
Mixed & \texttt{f(1,\ b=3)} & Positional first, keyword next \\
\end{longtable}

\subsubsection{Tiny Code}\label{tiny-code-31}

\begin{Shaded}
\begin{Highlighting}[]
\KeywordTok{def}\NormalTok{ introduce(name, country}\OperatorTok{=}\StringTok{"Unknown"}\NormalTok{):}
    \BuiltInTok{print}\NormalTok{(}\StringTok{"I am"}\NormalTok{, name, }\StringTok{"from"}\NormalTok{, country)}

\NormalTok{introduce(}\StringTok{"Alice"}\NormalTok{)                 }\CommentTok{\# I am Alice from Unknown}
\NormalTok{introduce(}\StringTok{"Bob"}\NormalTok{, }\StringTok{"France"}\NormalTok{)         }\CommentTok{\# I am Bob from France}
\NormalTok{introduce(name}\OperatorTok{=}\StringTok{"Charlie"}\NormalTok{, country}\OperatorTok{=}\StringTok{"Japan"}\NormalTok{)}
\end{Highlighting}
\end{Shaded}

\subsubsection{Why it Matters}\label{why-it-matters-31}

Arguments let you write one function that works in many situations.
Instead of duplicating code, you can pass in different values and reuse
the same function. This is one of the core ideas of programming.

\subsubsection{Try It Yourself}\label{try-it-yourself-31}

\begin{enumerate}
\def\labelenumi{\arabic{enumi}.}
\tightlist
\item
  Write a function \texttt{add(a,\ b)} that prints the sum of two
  numbers.
\item
  Call it with both positional (\texttt{add(3,\ 4)}) and keyword
  (\texttt{add(b=4,\ a=3)}) arguments.
\item
  Create a function \texttt{greet(name="Friend")} that has a default
  value for \texttt{name}. Call it with and without providing the
  argument.
\item
  Write a function \texttt{power(base,\ exponent=2)} that returns
  \texttt{base} raised to \texttt{exponent}. Call it with one and two
  arguments.
\end{enumerate}

\subsection{33. Default \& Keyword
Arguments}\label{default-keyword-arguments}

Python functions can define default values for parameters and accept
keyword arguments when called. These features make functions flexible
and easier to use by reducing how much you need to type and improving
readability.

\subsubsection{Deep Dive}\label{deep-dive-32}

Default Arguments When defining a function, you can give a parameter a
default value. If the caller doesn't provide it, Python uses the
default.

\begin{Shaded}
\begin{Highlighting}[]
\KeywordTok{def}\NormalTok{ greet(name}\OperatorTok{=}\StringTok{"Friend"}\NormalTok{):}
    \BuiltInTok{print}\NormalTok{(}\StringTok{"Hello,"}\NormalTok{, name)}

\NormalTok{greet()              }\CommentTok{\# Hello, Friend}
\NormalTok{greet(}\StringTok{"Alice"}\NormalTok{)       }\CommentTok{\# Hello, Alice}
\end{Highlighting}
\end{Shaded}

Multiple Defaults You can set defaults for more than one parameter.

\begin{Shaded}
\begin{Highlighting}[]
\KeywordTok{def} \ExtensionTok{connect}\NormalTok{(host}\OperatorTok{=}\StringTok{"localhost"}\NormalTok{, port}\OperatorTok{=}\DecValTok{8080}\NormalTok{):}
    \BuiltInTok{print}\NormalTok{(}\StringTok{"Connecting to"}\NormalTok{, host, }\StringTok{"on port"}\NormalTok{, port)}

\ExtensionTok{connect}\NormalTok{()                      }\CommentTok{\# localhost, 8080}
\ExtensionTok{connect}\NormalTok{(}\StringTok{"example.com"}\NormalTok{)         }\CommentTok{\# example.com, 8080}
\ExtensionTok{connect}\NormalTok{(port}\OperatorTok{=}\DecValTok{5000}\NormalTok{)             }\CommentTok{\# localhost, 5000}
\end{Highlighting}
\end{Shaded}

Keyword Arguments When calling a function, you can use parameter names.
This makes it clear what each value means, and order doesn't matter.

\begin{Shaded}
\begin{Highlighting}[]
\KeywordTok{def}\NormalTok{ introduce(name, age):}
    \BuiltInTok{print}\NormalTok{(name, }\StringTok{"is"}\NormalTok{, age, }\StringTok{"years old"}\NormalTok{)}

\NormalTok{introduce(age}\OperatorTok{=}\DecValTok{30}\NormalTok{, name}\OperatorTok{=}\StringTok{"Bob"}\NormalTok{)  }\CommentTok{\# Bob is 30 years old}
\end{Highlighting}
\end{Shaded}

Mixing Positional and Keyword Arguments You can mix both, but positional
arguments must come first.

\begin{Shaded}
\begin{Highlighting}[]
\KeywordTok{def}\NormalTok{ describe(animal, sound}\OperatorTok{=}\StringTok{"unknown"}\NormalTok{):}
    \BuiltInTok{print}\NormalTok{(animal, }\StringTok{"goes"}\NormalTok{, sound)}

\NormalTok{describe(}\StringTok{"Dog"}\NormalTok{)                      }\CommentTok{\# Dog goes unknown}
\NormalTok{describe(}\StringTok{"Cat"}\NormalTok{, sound}\OperatorTok{=}\StringTok{"Meow"}\NormalTok{)        }\CommentTok{\# Cat goes Meow}
\end{Highlighting}
\end{Shaded}

Important Rule Default arguments are evaluated once, when the function
is defined. Be careful with mutable defaults like lists or
dictionaries---they can persist changes between calls.

\begin{Shaded}
\begin{Highlighting}[]
\KeywordTok{def}\NormalTok{ add\_item(item, container}\OperatorTok{=}\NormalTok{[]):}
\NormalTok{    container.append(item)}
    \ControlFlowTok{return}\NormalTok{ container}

\BuiltInTok{print}\NormalTok{(add\_item(}\DecValTok{1}\NormalTok{))   }\CommentTok{\# [1]}
\BuiltInTok{print}\NormalTok{(add\_item(}\DecValTok{2}\NormalTok{))   }\CommentTok{\# [1, 2]  ← reused same list!}
\end{Highlighting}
\end{Shaded}

The safe way is:

\begin{Shaded}
\begin{Highlighting}[]
\KeywordTok{def}\NormalTok{ add\_item(item, container}\OperatorTok{=}\VariableTok{None}\NormalTok{):}
    \ControlFlowTok{if}\NormalTok{ container }\KeywordTok{is} \VariableTok{None}\NormalTok{:}
\NormalTok{        container }\OperatorTok{=}\NormalTok{ []}
\NormalTok{    container.append(item)}
    \ControlFlowTok{return}\NormalTok{ container}
\end{Highlighting}
\end{Shaded}

Quick Summary Table

\begin{longtable}[]{@{}
  >{\raggedright\arraybackslash}p{(\linewidth - 4\tabcolsep) * \real{0.3676}}
  >{\raggedright\arraybackslash}p{(\linewidth - 4\tabcolsep) * \real{0.2206}}
  >{\raggedright\arraybackslash}p{(\linewidth - 4\tabcolsep) * \real{0.4118}}@{}}
\toprule\noalign{}
\begin{minipage}[b]{\linewidth}\raggedright
Feature
\end{minipage} & \begin{minipage}[b]{\linewidth}\raggedright
Example
\end{minipage} & \begin{minipage}[b]{\linewidth}\raggedright
Benefit
\end{minipage} \\
\midrule\noalign{}
\endhead
\bottomrule\noalign{}
\endlastfoot
Default parameter & \texttt{def\ f(x=10)} & Optional arguments \\
Keyword argument call & \texttt{f(y=2,\ x=1)} & Clear meaning,
order-free \\
Mixing positional+keyword & \texttt{f(1,\ y=2)} & Flexible calls \\
Mutable default trap & \texttt{def\ f(lst={[}{]})} & Avoid with
\texttt{None} as default \\
\end{longtable}

\subsubsection{Tiny Code}\label{tiny-code-32}

\begin{Shaded}
\begin{Highlighting}[]
\KeywordTok{def}\NormalTok{ greet(name}\OperatorTok{=}\StringTok{"Guest"}\NormalTok{, lang}\OperatorTok{=}\StringTok{"en"}\NormalTok{):}
    \ControlFlowTok{if}\NormalTok{ lang }\OperatorTok{==} \StringTok{"en"}\NormalTok{:}
        \BuiltInTok{print}\NormalTok{(}\StringTok{"Hello,"}\NormalTok{, name)}
    \ControlFlowTok{elif}\NormalTok{ lang }\OperatorTok{==} \StringTok{"fr"}\NormalTok{:}
        \BuiltInTok{print}\NormalTok{(}\StringTok{"Bonjour,"}\NormalTok{, name)}
    \ControlFlowTok{else}\NormalTok{:}
        \BuiltInTok{print}\NormalTok{(}\StringTok{"Hi,"}\NormalTok{, name)}

\NormalTok{greet()}
\NormalTok{greet(}\StringTok{"Alice"}\NormalTok{)}
\NormalTok{greet(}\StringTok{"Bob"}\NormalTok{, lang}\OperatorTok{=}\StringTok{"fr"}\NormalTok{)}
\end{Highlighting}
\end{Shaded}

\subsubsection{Why it Matters}\label{why-it-matters-32}

Default and keyword arguments make functions more user-friendly. They
reduce repetitive code, prevent errors from missing values, and improve
readability when functions have many parameters.

\subsubsection{Try It Yourself}\label{try-it-yourself-32}

\begin{enumerate}
\def\labelenumi{\arabic{enumi}.}
\tightlist
\item
  Write a function \texttt{multiply(a,\ b=2)} that returns
  \texttt{a\ *\ b}. Call it with one argument and with two.
\item
  Create a function \texttt{profile(name,\ age=18,\ city="Unknown")} and
  call it using keyword arguments in any order.
\item
  Test the mutable default trap by defining a function with
  \texttt{list={[}{]}}. See how it behaves after multiple calls.
\item
  Rewrite it using \texttt{None} as the default and verify the issue is
  fixed.
\end{enumerate}

\subsection{34. Return Values}\label{return-values}

Functions don't just perform actions---they can also send results back
using the \texttt{return} statement. This makes functions powerful,
because you can store their output, use it in calculations, or pass it
into other functions.

\subsubsection{Deep Dive}\label{deep-dive-33}

Basic Return

\begin{Shaded}
\begin{Highlighting}[]
\KeywordTok{def}\NormalTok{ add(a, b):}
    \ControlFlowTok{return}\NormalTok{ a }\OperatorTok{+}\NormalTok{ b}

\NormalTok{result }\OperatorTok{=}\NormalTok{ add(}\DecValTok{3}\NormalTok{, }\DecValTok{4}\NormalTok{)}
\BuiltInTok{print}\NormalTok{(result)   }\CommentTok{\# 7}
\end{Highlighting}
\end{Shaded}

When Python hits \texttt{return}, the function stops and sends the value
back.

Returning Multiple Values Python functions can return more than one
value by returning a tuple:

\begin{Shaded}
\begin{Highlighting}[]
\KeywordTok{def}\NormalTok{ get\_stats(numbers):}
    \ControlFlowTok{return} \BuiltInTok{min}\NormalTok{(numbers), }\BuiltInTok{max}\NormalTok{(numbers), }\BuiltInTok{sum}\NormalTok{(numbers) }\OperatorTok{/} \BuiltInTok{len}\NormalTok{(numbers)}

\NormalTok{low, high, avg }\OperatorTok{=}\NormalTok{ get\_stats([}\DecValTok{10}\NormalTok{, }\DecValTok{20}\NormalTok{, }\DecValTok{30}\NormalTok{])}
\BuiltInTok{print}\NormalTok{(low, high, avg)   }\CommentTok{\# 10 30 20.0}
\end{Highlighting}
\end{Shaded}

No Return = \texttt{None} If a function doesn't have a \texttt{return},
it automatically returns \texttt{None}.

\begin{Shaded}
\begin{Highlighting}[]
\KeywordTok{def}\NormalTok{ say\_hello():}
    \BuiltInTok{print}\NormalTok{(}\StringTok{"Hello"}\NormalTok{)}

\NormalTok{result }\OperatorTok{=}\NormalTok{ say\_hello()}
\BuiltInTok{print}\NormalTok{(result)   }\CommentTok{\# None}
\end{Highlighting}
\end{Shaded}

Return vs Print

\begin{itemize}
\tightlist
\item
  \texttt{print()} shows something on the screen.
\item
  \texttt{return} gives a value back to the program.
\end{itemize}

\begin{Shaded}
\begin{Highlighting}[]
\KeywordTok{def}\NormalTok{ square(x):}
    \ControlFlowTok{return}\NormalTok{ x }\OperatorTok{*}\NormalTok{ x}

\BuiltInTok{print}\NormalTok{(square(}\DecValTok{5}\NormalTok{))   }\CommentTok{\# 25 (returned value printed)}
\end{Highlighting}
\end{Shaded}

Without \texttt{return}, you can't reuse the result later.

Early Return You can use \texttt{return} to exit a function early.

\begin{Shaded}
\begin{Highlighting}[]
\KeywordTok{def}\NormalTok{ safe\_divide(a, b):}
    \ControlFlowTok{if}\NormalTok{ b }\OperatorTok{==} \DecValTok{0}\NormalTok{:}
        \ControlFlowTok{return} \StringTok{"Cannot divide by zero"}
    \ControlFlowTok{return}\NormalTok{ a }\OperatorTok{/}\NormalTok{ b}
\end{Highlighting}
\end{Shaded}

Quick Summary Table

\begin{longtable}[]{@{}
  >{\raggedright\arraybackslash}p{(\linewidth - 4\tabcolsep) * \real{0.2000}}
  >{\raggedright\arraybackslash}p{(\linewidth - 4\tabcolsep) * \real{0.5200}}
  >{\raggedright\arraybackslash}p{(\linewidth - 4\tabcolsep) * \real{0.2800}}@{}}
\toprule\noalign{}
\begin{minipage}[b]{\linewidth}\raggedright
Behavior
\end{minipage} & \begin{minipage}[b]{\linewidth}\raggedright
Example
\end{minipage} & \begin{minipage}[b]{\linewidth}\raggedright
Result
\end{minipage} \\
\midrule\noalign{}
\endhead
\bottomrule\noalign{}
\endlastfoot
Single return & \texttt{return\ x\ +\ y} & one value \\
Multiple return & \texttt{return\ a,\ b} & tuple of values \\
No return & no \texttt{return} & \texttt{None} \\
Return vs print & \texttt{return} gives data, \texttt{print} shows data
& difference in purpose \\
\end{longtable}

\subsubsection{Tiny Code}\label{tiny-code-33}

\begin{Shaded}
\begin{Highlighting}[]
\KeywordTok{def}\NormalTok{ cube(n):}
    \ControlFlowTok{return}\NormalTok{ n  }\DecValTok{3}

\KeywordTok{def}\NormalTok{ min\_max(nums):}
    \ControlFlowTok{return} \BuiltInTok{min}\NormalTok{(nums), }\BuiltInTok{max}\NormalTok{(nums)}

\BuiltInTok{print}\NormalTok{(cube(}\DecValTok{4}\NormalTok{))             }\CommentTok{\# 64}
\NormalTok{low, high }\OperatorTok{=}\NormalTok{ min\_max([}\DecValTok{3}\NormalTok{, }\DecValTok{7}\NormalTok{, }\DecValTok{2}\NormalTok{, }\DecValTok{9}\NormalTok{])}
\BuiltInTok{print}\NormalTok{(low, high)           }\CommentTok{\# 2 9}
\end{Highlighting}
\end{Shaded}

\subsubsection{Why it Matters}\label{why-it-matters-33}

Return values make functions reusable building blocks. Instead of just
displaying results, functions can calculate and hand back values,
letting you compose larger programs from smaller pieces.

\subsubsection{Try It Yourself}\label{try-it-yourself-33}

\begin{enumerate}
\def\labelenumi{\arabic{enumi}.}
\tightlist
\item
  Write a function \texttt{square(n)} that returns the square of a
  number.
\item
  Create a function \texttt{divide(a,\ b)} that returns the result, but
  if \texttt{b} is 0, return \texttt{"Error"}.
\item
  Write a function \texttt{circle\_area(radius)} that returns the area
  using \texttt{3.14\ *\ r\ *\ r}.
\item
  Make a function that returns both the smallest and largest number from
  a list.
\end{enumerate}

\subsection{35. Variable Scope (local vs
global)}\label{variable-scope-local-vs-global}

In Python, scope refers to where a variable can be accessed in your
code. Variables created inside a function exist only there, while
variables created outside are available globally. Understanding scope
helps avoid bugs and keeps code organized.

\subsubsection{Deep Dive}\label{deep-dive-34}

Local Variables A variable created inside a function is local to that
function. It only exists while the function runs.

\begin{Shaded}
\begin{Highlighting}[]
\KeywordTok{def}\NormalTok{ greet():}
\NormalTok{    message }\OperatorTok{=} \StringTok{"Hello"}   \CommentTok{\# local variable}
    \BuiltInTok{print}\NormalTok{(message)}

\NormalTok{greet()}
\CommentTok{\# print(message) ❌ Error: message not defined}
\end{Highlighting}
\end{Shaded}

Global Variables A variable created outside functions is global and can
be used anywhere.

\begin{Shaded}
\begin{Highlighting}[]
\NormalTok{name }\OperatorTok{=} \StringTok{"Alice"}   \CommentTok{\# global variable}

\KeywordTok{def}\NormalTok{ say\_name():}
    \BuiltInTok{print}\NormalTok{(}\StringTok{"My name is"}\NormalTok{, name)}

\NormalTok{say\_name()       }\CommentTok{\# works fine}
\end{Highlighting}
\end{Shaded}

Local vs Global Priority If a local variable has the same name as a
global one, Python uses the local one inside the function.

\begin{Shaded}
\begin{Highlighting}[]
\NormalTok{x }\OperatorTok{=} \DecValTok{10}   \CommentTok{\# global}

\KeywordTok{def}\NormalTok{ show():}
\NormalTok{    x }\OperatorTok{=} \DecValTok{5}   \CommentTok{\# local}
    \BuiltInTok{print}\NormalTok{(x)}

\NormalTok{show()      }\CommentTok{\# 5}
\BuiltInTok{print}\NormalTok{(x)    }\CommentTok{\# 10}
\end{Highlighting}
\end{Shaded}

Using \texttt{global} Keyword If you want to modify a global variable
inside a function, use \texttt{global}.

\begin{Shaded}
\begin{Highlighting}[]
\NormalTok{count }\OperatorTok{=} \DecValTok{0}

\KeywordTok{def}\NormalTok{ increase():}
    \KeywordTok{global}\NormalTok{ count}
\NormalTok{    count }\OperatorTok{+=} \DecValTok{1}

\NormalTok{increase()}
\BuiltInTok{print}\NormalTok{(count)   }\CommentTok{\# 1}
\end{Highlighting}
\end{Shaded}

Best Practice

\begin{itemize}
\tightlist
\item
  Use local variables whenever possible---they are safer and easier to
  manage.
\item
  Avoid modifying global variables inside functions unless absolutely
  necessary.
\end{itemize}

Quick Summary Table

\begin{longtable}[]{@{}
  >{\raggedright\arraybackslash}p{(\linewidth - 4\tabcolsep) * \real{0.1912}}
  >{\raggedright\arraybackslash}p{(\linewidth - 4\tabcolsep) * \real{0.3529}}
  >{\raggedright\arraybackslash}p{(\linewidth - 4\tabcolsep) * \real{0.4559}}@{}}
\toprule\noalign{}
\begin{minipage}[b]{\linewidth}\raggedright
Variable Type
\end{minipage} & \begin{minipage}[b]{\linewidth}\raggedright
Defined Where
\end{minipage} & \begin{minipage}[b]{\linewidth}\raggedright
Accessible Where
\end{minipage} \\
\midrule\noalign{}
\endhead
\bottomrule\noalign{}
\endlastfoot
Local & Inside a function & Only inside that function \\
Global & Outside functions & Anywhere in the program \\
Shadowing & Local overrides global & Local used inside function \\
\texttt{global} & Marks variable as global & Allows modification in
function \\
\end{longtable}

\subsubsection{Tiny Code}\label{tiny-code-34}

\begin{Shaded}
\begin{Highlighting}[]
\NormalTok{x }\OperatorTok{=} \DecValTok{100}   \CommentTok{\# global}

\KeywordTok{def}\NormalTok{ test():}
\NormalTok{    x }\OperatorTok{=} \DecValTok{50}   \CommentTok{\# local}
    \BuiltInTok{print}\NormalTok{(}\StringTok{"Inside function:"}\NormalTok{, x)}

\NormalTok{test()}
\BuiltInTok{print}\NormalTok{(}\StringTok{"Outside function:"}\NormalTok{, x)}
\end{Highlighting}
\end{Shaded}

\subsubsection{Why it Matters}\label{why-it-matters-34}

Scope controls variable visibility and prevents accidental overwriting
of values. By understanding local vs global variables, you can write
cleaner, more reliable code that avoids confusing bugs.

\subsubsection{Try It Yourself}\label{try-it-yourself-34}

\begin{enumerate}
\def\labelenumi{\arabic{enumi}.}
\tightlist
\item
  Create a global variable \texttt{city\ =\ "Paris"} and write a
  function that prints it.
\item
  Define a function with a local variable \texttt{city\ =\ "London"} and
  see which value prints inside vs outside.
\item
  Make a counter using a global variable and a function that increases
  it with the \texttt{global} keyword.
\item
  Write two functions that each define their own local variable with the
  same name, and confirm they don't affect each other.
\end{enumerate}

\subsection{\texorpdfstring{36. \texttt{*args} and
\texttt{kwargs}}{36. *args and kwargs}}\label{args-and-kwargs}

In Python, functions can accept a flexible number of arguments using
\texttt{*args} and \texttt{kwargs}. These let you handle situations
where you don't know in advance how many inputs the user will provide.

\subsubsection{Deep Dive}\label{deep-dive-35}

\texttt{*args} → Variable Positional Arguments

\begin{itemize}
\tightlist
\item
  Collects extra positional arguments into a tuple.
\end{itemize}

\begin{Shaded}
\begin{Highlighting}[]
\KeywordTok{def}\NormalTok{ add\_all(}\OperatorTok{*}\NormalTok{args):}
    \BuiltInTok{print}\NormalTok{(args)}

\NormalTok{add\_all(}\DecValTok{1}\NormalTok{, }\DecValTok{2}\NormalTok{, }\DecValTok{3}\NormalTok{)   }\CommentTok{\# (1, 2, 3)}
\end{Highlighting}
\end{Shaded}

You can loop through \texttt{args} to process them:

\begin{Shaded}
\begin{Highlighting}[]
\KeywordTok{def}\NormalTok{ add\_all(}\OperatorTok{*}\NormalTok{args):}
    \ControlFlowTok{return} \BuiltInTok{sum}\NormalTok{(args)}

\BuiltInTok{print}\NormalTok{(add\_all(}\DecValTok{1}\NormalTok{, }\DecValTok{2}\NormalTok{, }\DecValTok{3}\NormalTok{, }\DecValTok{4}\NormalTok{))   }\CommentTok{\# 10}
\end{Highlighting}
\end{Shaded}

\texttt{kwargs} → Variable Keyword Arguments

\begin{itemize}
\tightlist
\item
  Collects extra keyword arguments into a dictionary.
\end{itemize}

\begin{Shaded}
\begin{Highlighting}[]
\KeywordTok{def}\NormalTok{ show\_info(kwargs):}
    \BuiltInTok{print}\NormalTok{(kwargs)}

\NormalTok{show\_info(name}\OperatorTok{=}\StringTok{"Alice"}\NormalTok{, age}\OperatorTok{=}\DecValTok{25}\NormalTok{)}
\CommentTok{\# \{\textquotesingle{}name\textquotesingle{}: \textquotesingle{}Alice\textquotesingle{}, \textquotesingle{}age\textquotesingle{}: 25\}}
\end{Highlighting}
\end{Shaded}

You can access values like a normal dictionary:

\begin{Shaded}
\begin{Highlighting}[]
\KeywordTok{def}\NormalTok{ show\_info(kwargs):}
    \ControlFlowTok{for}\NormalTok{ key, value }\KeywordTok{in}\NormalTok{ kwargs.items():}
        \BuiltInTok{print}\NormalTok{(key, }\StringTok{"="}\NormalTok{, value)}

\NormalTok{show\_info(city}\OperatorTok{=}\StringTok{"Paris"}\NormalTok{, country}\OperatorTok{=}\StringTok{"France"}\NormalTok{)}
\end{Highlighting}
\end{Shaded}

Combining \texttt{*args} and \texttt{kwargs} You can use both in the
same function, but \texttt{*args} must come before \texttt{kwargs}.

\begin{Shaded}
\begin{Highlighting}[]
\KeywordTok{def}\NormalTok{ demo(a, }\OperatorTok{*}\NormalTok{args, kwargs):}
    \BuiltInTok{print}\NormalTok{(}\StringTok{"a:"}\NormalTok{, a)}
    \BuiltInTok{print}\NormalTok{(}\StringTok{"args:"}\NormalTok{, args)}
    \BuiltInTok{print}\NormalTok{(}\StringTok{"kwargs:"}\NormalTok{, kwargs)}

\NormalTok{demo(}\DecValTok{1}\NormalTok{, }\DecValTok{2}\NormalTok{, }\DecValTok{3}\NormalTok{, x}\OperatorTok{=}\DecValTok{10}\NormalTok{, y}\OperatorTok{=}\DecValTok{20}\NormalTok{)}
\CommentTok{\# a: 1}
\CommentTok{\# args: (2, 3)}
\CommentTok{\# kwargs: \{\textquotesingle{}x\textquotesingle{}: 10, \textquotesingle{}y\textquotesingle{}: 20\}}
\end{Highlighting}
\end{Shaded}

Unpacking with \texttt{*} and
\texttt{You\ can\ also\ use\ \textasciigrave{}*\textasciigrave{}\ and}
to unpack lists/tuples and dictionaries into arguments.

\begin{Shaded}
\begin{Highlighting}[]
\NormalTok{nums }\OperatorTok{=}\NormalTok{ [}\DecValTok{1}\NormalTok{, }\DecValTok{2}\NormalTok{, }\DecValTok{3}\NormalTok{]}
\BuiltInTok{print}\NormalTok{(add\_all(}\OperatorTok{*}\NormalTok{nums))   }\CommentTok{\# 6}

\NormalTok{options }\OperatorTok{=}\NormalTok{ \{}\StringTok{"city"}\NormalTok{: }\StringTok{"Tokyo"}\NormalTok{, }\StringTok{"year"}\NormalTok{: }\DecValTok{2025}\NormalTok{\}}
\NormalTok{show\_info(options)}
\end{Highlighting}
\end{Shaded}

Quick Summary Table

\begin{longtable}[]{@{}
  >{\raggedright\arraybackslash}p{(\linewidth - 6\tabcolsep) * \real{0.2029}}
  >{\raggedright\arraybackslash}p{(\linewidth - 6\tabcolsep) * \real{0.1884}}
  >{\raggedright\arraybackslash}p{(\linewidth - 6\tabcolsep) * \real{0.2029}}
  >{\raggedright\arraybackslash}p{(\linewidth - 6\tabcolsep) * \real{0.4058}}@{}}
\toprule\noalign{}
\begin{minipage}[b]{\linewidth}\raggedright
Feature
\end{minipage} & \begin{minipage}[b]{\linewidth}\raggedright
Collects Into
\end{minipage} & \begin{minipage}[b]{\linewidth}\raggedright
Example Call
\end{minipage} & \begin{minipage}[b]{\linewidth}\raggedright
Example Result
\end{minipage} \\
\midrule\noalign{}
\endhead
\bottomrule\noalign{}
\endlastfoot
\texttt{*args} & Tuple & \texttt{f(1,2,3)} & \texttt{(1,2,3)} \\
\texttt{kwargs} & Dictionary & \texttt{f(a=1,\ b=2)} &
\texttt{\{\textquotesingle{}a\textquotesingle{}:1,\textquotesingle{}b\textquotesingle{}:2\}} \\
Both combined & args + kwargs & \texttt{f(1,2,\ x=10)} &
\texttt{args=(2,),\ kwargs=\{\textquotesingle{}x\textquotesingle{}:10\}} \\
Unpacking \texttt{*} & Splits list & \texttt{f(*{[}1,2{]})} & like
\texttt{f(1,2)} \\
Unpacking
`\texttt{\textbar{}\ Splits\ dict\ \ \ \textbar{}}f(\{`a':1\})\texttt{\textbar{}\ like}f(a=1)`
& & & \\
\end{longtable}

\subsubsection{Tiny Code}\label{tiny-code-35}

\begin{Shaded}
\begin{Highlighting}[]
\KeywordTok{def}\NormalTok{ greet(}\OperatorTok{*}\NormalTok{names, options):}
    \ControlFlowTok{for}\NormalTok{ name }\KeywordTok{in}\NormalTok{ names:}
        \BuiltInTok{print}\NormalTok{(}\StringTok{"Hello,"}\NormalTok{, name)}
    \ControlFlowTok{if} \StringTok{"lang"} \KeywordTok{in}\NormalTok{ options:}
        \BuiltInTok{print}\NormalTok{(}\StringTok{"Language:"}\NormalTok{, options[}\StringTok{"lang"}\NormalTok{])}

\NormalTok{greet(}\StringTok{"Alice"}\NormalTok{, }\StringTok{"Bob"}\NormalTok{, lang}\OperatorTok{=}\StringTok{"English"}\NormalTok{)}
\end{Highlighting}
\end{Shaded}

\subsubsection{Why it Matters}\label{why-it-matters-35}

\texttt{*args} and \texttt{kwargs} make functions more flexible and
reusable. They let you handle unknown numbers of inputs, write cleaner
APIs, and pass around configurations easily.

\subsubsection{Try It Yourself}\label{try-it-yourself-35}

\begin{enumerate}
\def\labelenumi{\arabic{enumi}.}
\tightlist
\item
  Write a function \texttt{multiply\_all(*nums)} that multiplies any
  number of values.
\item
  Create a function \texttt{print\_info(data)} that prints each
  key--value pair.
\item
  Combine them: \texttt{f(x,\ *args,\ kwargs)} and test with mixed
  inputs.
\item
  Experiment with unpacking a list into \texttt{*args} and a dictionary
  into \texttt{kwargs}.
\end{enumerate}

\subsection{37. Lambda Functions}\label{lambda-functions}

A lambda function is a small, anonymous function defined with the
keyword \texttt{lambda}. Unlike normal functions defined with
\texttt{def}, lambda functions are written in a single line and don't
need a name unless you assign them to a variable. They're often used for
quick, throwaway functions.

\subsubsection{Deep Dive}\label{deep-dive-36}

Basic Syntax

\begin{Shaded}
\begin{Highlighting}[]
\KeywordTok{lambda}\NormalTok{ arguments: expression}
\end{Highlighting}
\end{Shaded}

\begin{itemize}
\tightlist
\item
  \texttt{arguments} → input parameters.
\item
  \texttt{expression} → a single expression that is evaluated and
  returned.
\end{itemize}

Example:

\begin{Shaded}
\begin{Highlighting}[]
\NormalTok{square }\OperatorTok{=} \KeywordTok{lambda}\NormalTok{ x: x }\OperatorTok{*}\NormalTok{ x}
\BuiltInTok{print}\NormalTok{(square(}\DecValTok{5}\NormalTok{))   }\CommentTok{\# 25}
\end{Highlighting}
\end{Shaded}

Multiple Arguments

\begin{Shaded}
\begin{Highlighting}[]
\NormalTok{add }\OperatorTok{=} \KeywordTok{lambda}\NormalTok{ a, b: a }\OperatorTok{+}\NormalTok{ b}
\BuiltInTok{print}\NormalTok{(add(}\DecValTok{3}\NormalTok{, }\DecValTok{4}\NormalTok{))   }\CommentTok{\# 7}
\end{Highlighting}
\end{Shaded}

No Arguments

\begin{Shaded}
\begin{Highlighting}[]
\NormalTok{hello }\OperatorTok{=} \KeywordTok{lambda}\NormalTok{: }\StringTok{"Hello!"}
\BuiltInTok{print}\NormalTok{(hello())     }\CommentTok{\# Hello!}
\end{Highlighting}
\end{Shaded}

Use with Built-in Functions Lambdas are often used with \texttt{map()},
\texttt{filter()}, and \texttt{sorted()}.

\begin{itemize}
\tightlist
\item
  With \texttt{map()} to apply a function to all items:
\end{itemize}

\begin{Shaded}
\begin{Highlighting}[]
\NormalTok{nums }\OperatorTok{=}\NormalTok{ [}\DecValTok{1}\NormalTok{, }\DecValTok{2}\NormalTok{, }\DecValTok{3}\NormalTok{, }\DecValTok{4}\NormalTok{]}
\NormalTok{squares }\OperatorTok{=} \BuiltInTok{list}\NormalTok{(}\BuiltInTok{map}\NormalTok{(}\KeywordTok{lambda}\NormalTok{ x: x }\OperatorTok{*}\NormalTok{ x, nums))}
\BuiltInTok{print}\NormalTok{(squares)   }\CommentTok{\# [1, 4, 9, 16]}
\end{Highlighting}
\end{Shaded}

\begin{itemize}
\tightlist
\item
  With \texttt{filter()} to keep items that match a condition:
\end{itemize}

\begin{Shaded}
\begin{Highlighting}[]
\NormalTok{evens }\OperatorTok{=} \BuiltInTok{list}\NormalTok{(}\BuiltInTok{filter}\NormalTok{(}\KeywordTok{lambda}\NormalTok{ x: x }\OperatorTok{\%} \DecValTok{2} \OperatorTok{==} \DecValTok{0}\NormalTok{, nums))}
\BuiltInTok{print}\NormalTok{(evens)   }\CommentTok{\# [2, 4]}
\end{Highlighting}
\end{Shaded}

\begin{itemize}
\tightlist
\item
  With \texttt{sorted()} to customize sorting:
\end{itemize}

\begin{Shaded}
\begin{Highlighting}[]
\NormalTok{words }\OperatorTok{=}\NormalTok{ [}\StringTok{"banana"}\NormalTok{, }\StringTok{"apple"}\NormalTok{, }\StringTok{"cherry"}\NormalTok{]}
\NormalTok{words.sort(key}\OperatorTok{=}\KeywordTok{lambda}\NormalTok{ w: }\BuiltInTok{len}\NormalTok{(w))}
\BuiltInTok{print}\NormalTok{(words)   }\CommentTok{\# [\textquotesingle{}apple\textquotesingle{}, \textquotesingle{}banana\textquotesingle{}, \textquotesingle{}cherry\textquotesingle{}]}
\end{Highlighting}
\end{Shaded}

Limitations

\begin{itemize}
\tightlist
\item
  Only one expression (no multiple lines).
\item
  Can't contain statements like \texttt{print}, \texttt{return}, or
  loops (though you can call functions inside).
\item
  Best for short, simple tasks.
\end{itemize}

Quick Summary Table

\begin{longtable}[]{@{}
  >{\raggedright\arraybackslash}p{(\linewidth - 4\tabcolsep) * \real{0.2174}}
  >{\raggedright\arraybackslash}p{(\linewidth - 4\tabcolsep) * \real{0.5217}}
  >{\raggedright\arraybackslash}p{(\linewidth - 4\tabcolsep) * \real{0.2609}}@{}}
\toprule\noalign{}
\begin{minipage}[b]{\linewidth}\raggedright
Feature
\end{minipage} & \begin{minipage}[b]{\linewidth}\raggedright
Example
\end{minipage} & \begin{minipage}[b]{\linewidth}\raggedright
Output
\end{minipage} \\
\midrule\noalign{}
\endhead
\bottomrule\noalign{}
\endlastfoot
Single argument & \texttt{lambda\ x:\ x\ +\ 1} & Adds 1 to x \\
Multiple args & \texttt{lambda\ a,\ b:\ a\ *\ b} & Multiplies a and b \\
No args & \texttt{lambda:\ "hi"} & Returns ``hi'' \\
With \texttt{map()} & \texttt{map(lambda\ x:\ x*x,\ {[}1,2{]})} & {[}1,
4{]} \\
With \texttt{filter()} &
\texttt{filter(lambda\ x:\ x\textgreater{}2,\ {[}1,2,3{]})} & {[}3{]} \\
With \texttt{sorted()} & \texttt{sorted(words,\ key=lambda\ w:len(w))} &
Sorted by length \\
\end{longtable}

\subsubsection{Tiny Code}\label{tiny-code-36}

\begin{Shaded}
\begin{Highlighting}[]
\NormalTok{nums }\OperatorTok{=}\NormalTok{ [}\DecValTok{5}\NormalTok{, }\DecValTok{10}\NormalTok{, }\DecValTok{15}\NormalTok{]}

\CommentTok{\# Double numbers using lambda + map}
\NormalTok{doubles }\OperatorTok{=} \BuiltInTok{list}\NormalTok{(}\BuiltInTok{map}\NormalTok{(}\KeywordTok{lambda}\NormalTok{ n: n }\OperatorTok{*} \DecValTok{2}\NormalTok{, nums))}
\BuiltInTok{print}\NormalTok{(doubles)   }\CommentTok{\# [10, 20, 30]}

\CommentTok{\# Filter numbers greater than 7}
\NormalTok{greater }\OperatorTok{=} \BuiltInTok{list}\NormalTok{(}\BuiltInTok{filter}\NormalTok{(}\KeywordTok{lambda}\NormalTok{ n: n }\OperatorTok{\textgreater{}} \DecValTok{7}\NormalTok{, nums))}
\BuiltInTok{print}\NormalTok{(greater)   }\CommentTok{\# [10, 15]}
\end{Highlighting}
\end{Shaded}

\subsubsection{Why it Matters}\label{why-it-matters-36}

Lambda functions let you write short, inline functions without
cluttering your code. They're especially handy for quick data
transformations, sorting, and filtering when defining a full function
would be unnecessary.

\subsubsection{Try It Yourself}\label{try-it-yourself-36}

\begin{enumerate}
\def\labelenumi{\arabic{enumi}.}
\tightlist
\item
  Write a lambda function that adds 10 to a number.
\item
  Use a lambda with \texttt{filter()} to keep only odd numbers from a
  list.
\item
  Sort a list of names by their last letter using \texttt{sorted()} with
  a lambda key.
\item
  Use \texttt{map()} with a lambda to convert a list of Celsius
  temperatures into Fahrenheit.
\end{enumerate}

\subsection{38. Docstrings}\label{docstrings}

A docstring (documentation string) is a special string placed inside
functions, classes, or modules to explain what they do. Unlike comments,
docstrings are stored at runtime and can be accessed with tools like
\texttt{help()}. They are a key part of writing clean, professional
Python code.

\subsubsection{Deep Dive}\label{deep-dive-37}

Basic Function Docstring Docstrings are written using triple quotes
(\texttt{"""\ ...\ """} or
\texttt{\textquotesingle{}\textquotesingle{}\textquotesingle{}\ ...\ \textquotesingle{}\textquotesingle{}\textquotesingle{}})
right below the function definition:

\begin{Shaded}
\begin{Highlighting}[]
\KeywordTok{def}\NormalTok{ greet(name):}
    \CommentTok{"""Return a greeting message for the given name."""}
    \ControlFlowTok{return} \StringTok{"Hello, "} \OperatorTok{+}\NormalTok{ name}
\end{Highlighting}
\end{Shaded}

Accessing Docstrings You can retrieve the docstring with:

\begin{Shaded}
\begin{Highlighting}[]
\BuiltInTok{print}\NormalTok{(greet.\_\_doc\_\_)}
\BuiltInTok{help}\NormalTok{(greet)}
\end{Highlighting}
\end{Shaded}

Multi-Line Docstrings For more complex functions, use multiple lines:

\begin{Shaded}
\begin{Highlighting}[]
\KeywordTok{def}\NormalTok{ add(a, b):}
    \CommentTok{"""}
\CommentTok{    Add two numbers and return the result.}

\CommentTok{    Parameters:}
\CommentTok{        a (int or float): First number.}
\CommentTok{        b (int or float): Second number.}

\CommentTok{    Returns:}
\CommentTok{        int or float: The sum of a and b.}
\CommentTok{    """}
    \ControlFlowTok{return}\NormalTok{ a }\OperatorTok{+}\NormalTok{ b}
\end{Highlighting}
\end{Shaded}

Docstrings for Classes and Modules

\begin{itemize}
\tightlist
\item
  For classes:
\end{itemize}

\begin{Shaded}
\begin{Highlighting}[]
\KeywordTok{class}\NormalTok{ Person:}
    \CommentTok{"""A simple class representing a person."""}
    \KeywordTok{def} \FunctionTok{\_\_init\_\_}\NormalTok{(}\VariableTok{self}\NormalTok{, name):}
        \VariableTok{self}\NormalTok{.name }\OperatorTok{=}\NormalTok{ name}
\end{Highlighting}
\end{Shaded}

\begin{itemize}
\tightlist
\item
  For modules (at the very top of a file):
\end{itemize}

\begin{Shaded}
\begin{Highlighting}[]
\CommentTok{"""}
\CommentTok{This module provides math helper functions}
\CommentTok{like factorial and Fibonacci.}
\CommentTok{"""}
\end{Highlighting}
\end{Shaded}

PEP 257 Conventions Python has conventions for docstrings:

\begin{enumerate}
\def\labelenumi{\arabic{enumi}.}
\tightlist
\item
  Start with a short summary in one line.
\item
  Leave a blank line after the summary if you add more detail.
\item
  Use triple quotes even for one-liners.
\end{enumerate}

Quick Summary Table

\begin{longtable}[]{@{}lll@{}}
\toprule\noalign{}
Where Used & Example Placement & Purpose \\
\midrule\noalign{}
\endhead
\bottomrule\noalign{}
\endlastfoot
Function & Inside function body & Explain what it does/returns \\
Class & Inside class definition & Describe the class purpose \\
Module & At top of file & Overview of the whole module \\
Accessing & \texttt{obj.\_\_doc\_\_}, \texttt{help()} & See
documentation \\
\end{longtable}

\subsubsection{Tiny Code}\label{tiny-code-37}

\begin{Shaded}
\begin{Highlighting}[]
\KeywordTok{def}\NormalTok{ factorial(n):}
    \CommentTok{"""Calculate the factorial of n using recursion."""}
    \ControlFlowTok{return} \DecValTok{1} \ControlFlowTok{if}\NormalTok{ n }\OperatorTok{==} \DecValTok{0} \ControlFlowTok{else}\NormalTok{ n }\OperatorTok{*}\NormalTok{ factorial(n }\OperatorTok{{-}} \DecValTok{1}\NormalTok{)}

\BuiltInTok{print}\NormalTok{(factorial.\_\_doc\_\_)}
\end{Highlighting}
\end{Shaded}

\subsubsection{Why it Matters}\label{why-it-matters-37}

Docstrings turn your code into self-documenting programs. They help
others (and your future self) understand how functions, classes, and
modules should be used without reading all the code. Tools like Sphinx
and IDEs also use docstrings to generate documentation automatically.

\subsubsection{Try It Yourself}\label{try-it-yourself-37}

\begin{enumerate}
\def\labelenumi{\arabic{enumi}.}
\tightlist
\item
  Write a function \texttt{square(n)} with a one-line docstring
  explaining what it does.
\item
  Create a function \texttt{divide(a,\ b)} with a multi-line docstring
  that explains parameters and return value.
\item
  Add a class \texttt{Car} with a docstring describing its purpose.
\item
  Use \texttt{help()} on your function or class to see the docstring
  displayed.
\end{enumerate}

\subsection{39. Recursive Functions}\label{recursive-functions}

A recursive function is a function that calls itself in order to solve a
problem. Recursion is useful when a problem can be broken down into
smaller, similar subproblems---like calculating factorials, traversing
trees, or solving puzzles.

\subsubsection{Deep Dive}\label{deep-dive-38}

Basic Structure A recursive function always has two parts:

\begin{enumerate}
\def\labelenumi{\arabic{enumi}.}
\tightlist
\item
  Base case → the condition that stops the recursion.
\item
  Recursive case → the function calls itself with a smaller/simpler
  problem.
\end{enumerate}

\begin{Shaded}
\begin{Highlighting}[]
\KeywordTok{def}\NormalTok{ countdown(n):}
    \ControlFlowTok{if}\NormalTok{ n }\OperatorTok{==} \DecValTok{0}\NormalTok{:             }\CommentTok{\# base case}
        \BuiltInTok{print}\NormalTok{(}\StringTok{"Done!"}\NormalTok{)}
    \ControlFlowTok{else}\NormalTok{:}
        \BuiltInTok{print}\NormalTok{(n)}
\NormalTok{        countdown(n }\OperatorTok{{-}} \DecValTok{1}\NormalTok{)   }\CommentTok{\# recursive case}
\end{Highlighting}
\end{Shaded}

Example 1: Factorial The factorial of \texttt{n} is
\texttt{n\ *\ (n-1)\ *\ (n-2)\ *\ ...\ *\ 1}.

\begin{Shaded}
\begin{Highlighting}[]
\KeywordTok{def}\NormalTok{ factorial(n):}
    \ControlFlowTok{if}\NormalTok{ n }\OperatorTok{==} \DecValTok{0}\NormalTok{:     }\CommentTok{\# base case}
        \ControlFlowTok{return} \DecValTok{1}
    \ControlFlowTok{return}\NormalTok{ n }\OperatorTok{*}\NormalTok{ factorial(n }\OperatorTok{{-}} \DecValTok{1}\NormalTok{)   }\CommentTok{\# recursive case}

\BuiltInTok{print}\NormalTok{(factorial(}\DecValTok{5}\NormalTok{))   }\CommentTok{\# 120}
\end{Highlighting}
\end{Shaded}

Example 2: Fibonacci Sequence Each Fibonacci number is the sum of the
previous two.

\begin{Shaded}
\begin{Highlighting}[]
\KeywordTok{def}\NormalTok{ fib(n):}
    \ControlFlowTok{if}\NormalTok{ n }\OperatorTok{\textless{}=} \DecValTok{1}\NormalTok{:   }\CommentTok{\# base case}
        \ControlFlowTok{return}\NormalTok{ n}
    \ControlFlowTok{return}\NormalTok{ fib(n }\OperatorTok{{-}} \DecValTok{1}\NormalTok{) }\OperatorTok{+}\NormalTok{ fib(n }\OperatorTok{{-}} \DecValTok{2}\NormalTok{)}

\BuiltInTok{print}\NormalTok{(fib(}\DecValTok{6}\NormalTok{))   }\CommentTok{\# 8}
\end{Highlighting}
\end{Shaded}

Potential Issues

\begin{itemize}
\tightlist
\item
  Infinite recursion: forgetting a base case causes the function to call
  itself forever, leading to an error (\texttt{RecursionError}).
\item
  Performance: recursion can be slower and use more memory than loops
  for large inputs.
\end{itemize}

Quick Summary Table

\begin{longtable}[]{@{}
  >{\raggedright\arraybackslash}p{(\linewidth - 4\tabcolsep) * \real{0.2169}}
  >{\raggedright\arraybackslash}p{(\linewidth - 4\tabcolsep) * \real{0.4819}}
  >{\raggedright\arraybackslash}p{(\linewidth - 4\tabcolsep) * \real{0.3012}}@{}}
\toprule\noalign{}
\begin{minipage}[b]{\linewidth}\raggedright
Term
\end{minipage} & \begin{minipage}[b]{\linewidth}\raggedright
Meaning
\end{minipage} & \begin{minipage}[b]{\linewidth}\raggedright
Example
\end{minipage} \\
\midrule\noalign{}
\endhead
\bottomrule\noalign{}
\endlastfoot
Base case & Condition that stops recursion &
\texttt{if\ n\ ==\ 0:\ return\ 1} \\
Recursive case & Function calls itself with smaller input &
\texttt{return\ n\ *\ f(n-1)} \\
Infinite recursion & Missing/incorrect base case & Error: never ends \\
Use cases & Factorial, Fibonacci, tree traversal & Many algorithmic
problems \\
\end{longtable}

\subsubsection{Tiny Code}\label{tiny-code-38}

\begin{Shaded}
\begin{Highlighting}[]
\KeywordTok{def}\NormalTok{ sum\_list(numbers):}
    \ControlFlowTok{if} \KeywordTok{not}\NormalTok{ numbers:       }\CommentTok{\# base case}
        \ControlFlowTok{return} \DecValTok{0}
    \ControlFlowTok{return}\NormalTok{ numbers[}\DecValTok{0}\NormalTok{] }\OperatorTok{+}\NormalTok{ sum\_list(numbers[}\DecValTok{1}\NormalTok{:])  }\CommentTok{\# recursive case}

\BuiltInTok{print}\NormalTok{(sum\_list([}\DecValTok{1}\NormalTok{, }\DecValTok{2}\NormalTok{, }\DecValTok{3}\NormalTok{, }\DecValTok{4}\NormalTok{]))   }\CommentTok{\# 10}
\end{Highlighting}
\end{Shaded}

\subsubsection{Why it Matters}\label{why-it-matters-38}

Recursive functions let you write elegant, natural solutions to problems
that involve repetition with smaller pieces---like mathematical
sequences, hierarchical data, or divide-and-conquer algorithms.

\subsubsection{Try It Yourself}\label{try-it-yourself-38}

\begin{enumerate}
\def\labelenumi{\arabic{enumi}.}
\tightlist
\item
  Write a recursive function \texttt{countdown(n)} that prints numbers
  down to 0.
\item
  Create a recursive function \texttt{factorial(n)} and test it with
  \texttt{n=5}.
\item
  Write a recursive function \texttt{fib(n)} to compute Fibonacci
  numbers.
\item
  Challenge: Write a recursive function that calculates the sum of all
  numbers in a list.
\end{enumerate}

\subsection{40. Higher-Order Functions}\label{higher-order-functions}

A higher-order function is a function that either takes another function
as an argument, returns a function, or both. This makes Python very
powerful for writing flexible and reusable code.

\subsubsection{Deep Dive}\label{deep-dive-39}

Functions as Arguments Since functions are objects in Python, you can
pass them around like variables.

\begin{Shaded}
\begin{Highlighting}[]
\KeywordTok{def}\NormalTok{ apply\_twice(func, x):}
    \ControlFlowTok{return}\NormalTok{ func(func(x))}

\KeywordTok{def}\NormalTok{ square(n):}
    \ControlFlowTok{return}\NormalTok{ n }\OperatorTok{*}\NormalTok{ n}

\BuiltInTok{print}\NormalTok{(apply\_twice(square, }\DecValTok{2}\NormalTok{))   }\CommentTok{\# 16}
\end{Highlighting}
\end{Shaded}

Functions Returning Functions A function can also create and return
another function.

\begin{Shaded}
\begin{Highlighting}[]
\KeywordTok{def}\NormalTok{ make\_multiplier(n):}
    \KeywordTok{def}\NormalTok{ multiplier(x):}
        \ControlFlowTok{return}\NormalTok{ x }\OperatorTok{*}\NormalTok{ n}
    \ControlFlowTok{return}\NormalTok{ multiplier}

\NormalTok{double }\OperatorTok{=}\NormalTok{ make\_multiplier(}\DecValTok{2}\NormalTok{)}
\BuiltInTok{print}\NormalTok{(double(}\DecValTok{5}\NormalTok{))   }\CommentTok{\# 10}
\end{Highlighting}
\end{Shaded}

Built-in Higher-Order Functions Python provides many built-in
higher-order functions:

\begin{itemize}
\tightlist
\item
  \texttt{map(func,\ iterable)} → applies a function to each item.
\end{itemize}

\begin{Shaded}
\begin{Highlighting}[]
\NormalTok{nums }\OperatorTok{=}\NormalTok{ [}\DecValTok{1}\NormalTok{, }\DecValTok{2}\NormalTok{, }\DecValTok{3}\NormalTok{]}
\NormalTok{squares }\OperatorTok{=} \BuiltInTok{list}\NormalTok{(}\BuiltInTok{map}\NormalTok{(}\KeywordTok{lambda}\NormalTok{ x: x }\OperatorTok{*}\NormalTok{ x, nums))}
\BuiltInTok{print}\NormalTok{(squares)   }\CommentTok{\# [1, 4, 9]}
\end{Highlighting}
\end{Shaded}

\begin{itemize}
\tightlist
\item
  \texttt{filter(func,\ iterable)} → keeps only items where the function
  returns \texttt{True}.
\end{itemize}

\begin{Shaded}
\begin{Highlighting}[]
\NormalTok{evens }\OperatorTok{=} \BuiltInTok{list}\NormalTok{(}\BuiltInTok{filter}\NormalTok{(}\KeywordTok{lambda}\NormalTok{ x: x }\OperatorTok{\%} \DecValTok{2} \OperatorTok{==} \DecValTok{0}\NormalTok{, nums))}
\BuiltInTok{print}\NormalTok{(evens)   }\CommentTok{\# [2]}
\end{Highlighting}
\end{Shaded}

\begin{itemize}
\tightlist
\item
  \texttt{sorted(iterable,\ key=func)} → sorts by a custom key.
\end{itemize}

\begin{Shaded}
\begin{Highlighting}[]
\NormalTok{words }\OperatorTok{=}\NormalTok{ [}\StringTok{"banana"}\NormalTok{, }\StringTok{"apple"}\NormalTok{, }\StringTok{"cherry"}\NormalTok{]}
\BuiltInTok{print}\NormalTok{(}\BuiltInTok{sorted}\NormalTok{(words, key}\OperatorTok{=}\BuiltInTok{len}\NormalTok{))   }\CommentTok{\# [\textquotesingle{}apple\textquotesingle{}, \textquotesingle{}banana\textquotesingle{}, \textquotesingle{}cherry\textquotesingle{}]}
\end{Highlighting}
\end{Shaded}

\begin{itemize}
\tightlist
\item
  \texttt{reduce(func,\ iterable)} from \texttt{functools} → applies a
  rolling computation.
\end{itemize}

\begin{Shaded}
\begin{Highlighting}[]
\ImportTok{from}\NormalTok{ functools }\ImportTok{import} \BuiltInTok{reduce}
\NormalTok{product }\OperatorTok{=} \BuiltInTok{reduce}\NormalTok{(}\KeywordTok{lambda}\NormalTok{ a, b: a }\OperatorTok{*}\NormalTok{ b, [}\DecValTok{1}\NormalTok{, }\DecValTok{2}\NormalTok{, }\DecValTok{3}\NormalTok{, }\DecValTok{4}\NormalTok{])}
\BuiltInTok{print}\NormalTok{(product)   }\CommentTok{\# 24}
\end{Highlighting}
\end{Shaded}

Why Use Higher-Order Functions?

\begin{itemize}
\tightlist
\item
  They allow abstraction: write logic once and reuse it.
\item
  They make code shorter and cleaner.
\item
  They are the foundation of functional programming.
\end{itemize}

Quick Summary Table

\begin{longtable}[]{@{}
  >{\raggedright\arraybackslash}p{(\linewidth - 4\tabcolsep) * \real{0.2892}}
  >{\raggedright\arraybackslash}p{(\linewidth - 4\tabcolsep) * \real{0.3735}}
  >{\raggedright\arraybackslash}p{(\linewidth - 4\tabcolsep) * \real{0.3373}}@{}}
\toprule\noalign{}
\begin{minipage}[b]{\linewidth}\raggedright
Feature
\end{minipage} & \begin{minipage}[b]{\linewidth}\raggedright
Example
\end{minipage} & \begin{minipage}[b]{\linewidth}\raggedright
Purpose
\end{minipage} \\
\midrule\noalign{}
\endhead
\bottomrule\noalign{}
\endlastfoot
Function as argument & \texttt{apply\_twice(square,\ 2)} & Pass function
in \\
Function as return value & \texttt{make\_multiplier(3)} & Generate new
function \\
\texttt{map()} & \texttt{map(lambda\ x:x+1,\ {[}1,2{]})} & Apply
function to items \\
\texttt{filter()} &
\texttt{filter(lambda\ x:x\textgreater{}2,\ {[}1,2,3{]})} & Keep items
meeting condition \\
\texttt{sorted(...,\ key=func)} & \texttt{sorted(words,\ key=len)} &
Custom sorting \\
\texttt{reduce()} & \texttt{reduce(lambda\ a,b:a*b,\ nums)} & Accumulate
values \\
\end{longtable}

\subsubsection{Tiny Code}\label{tiny-code-39}

\begin{Shaded}
\begin{Highlighting}[]
\KeywordTok{def}\NormalTok{ shout(text):}
    \ControlFlowTok{return}\NormalTok{ text.upper()}

\KeywordTok{def}\NormalTok{ whisper(text):}
    \ControlFlowTok{return}\NormalTok{ text.lower()}

\KeywordTok{def}\NormalTok{ speak(func, message):}
    \BuiltInTok{print}\NormalTok{(func(message))}

\NormalTok{speak(shout, }\StringTok{"Hello"}\NormalTok{)}
\NormalTok{speak(whisper, }\StringTok{"Hello"}\NormalTok{)}
\end{Highlighting}
\end{Shaded}

\subsubsection{Why it Matters}\label{why-it-matters-39}

Higher-order functions let you treat behavior as data. Instead of
hardcoding actions, you can pass in functions to customize behavior.
This leads to more flexible, reusable, and expressive programs.

\subsubsection{Try It Yourself}\label{try-it-yourself-39}

\begin{enumerate}
\def\labelenumi{\arabic{enumi}.}
\tightlist
\item
  Write a function \texttt{apply(func,\ values)} that applies
  \texttt{func} to every item in \texttt{values} (like your own
  \texttt{map}).
\item
  Use \texttt{filter()} with a lambda to keep only numbers greater than
  10 from a list.
\item
  Write a \texttt{make\_adder(n)} function that returns a new function
  adding \texttt{n} to its input.
\item
  Use \texttt{reduce()} to calculate the sum of a list of numbers.
\end{enumerate}

\section{Chapter 5. Modules and
Packages}\label{chapter-5.-modules-and-packages}

\subsection{41. Importing Modules}\label{importing-modules}

A module in Python is a file containing Python code (functions, classes,
variables) that you can reuse in other programs. Importing a module lets
you use its code without rewriting it.

\subsubsection{Deep Dive}\label{deep-dive-40}

Basic Import Use the \texttt{import} keyword followed by the module
name:

\begin{Shaded}
\begin{Highlighting}[]
\ImportTok{import}\NormalTok{ math}

\BuiltInTok{print}\NormalTok{(math.sqrt(}\DecValTok{16}\NormalTok{))   }\CommentTok{\# 4.0}
\end{Highlighting}
\end{Shaded}

Here, \texttt{math} is a built-in module that provides mathematical
functions.

Importing Multiple Modules You can import more than one module in one
line:

\begin{Shaded}
\begin{Highlighting}[]
\ImportTok{import}\NormalTok{ math, random}

\BuiltInTok{print}\NormalTok{(random.randint(}\DecValTok{1}\NormalTok{, }\DecValTok{6}\NormalTok{))   }\CommentTok{\# random number between 1 and 6}
\end{Highlighting}
\end{Shaded}

Accessing Module Contents To use something from a module, write
\texttt{module\_name.item}.

\begin{Shaded}
\begin{Highlighting}[]
\BuiltInTok{print}\NormalTok{(math.pi)      }\CommentTok{\# 3.14159...}
\BuiltInTok{print}\NormalTok{(math.factorial(}\DecValTok{5}\NormalTok{))   }\CommentTok{\# 120}
\end{Highlighting}
\end{Shaded}

Import Once Only A module is loaded once per program run, even if
imported multiple times.

Where Python Looks for Modules

\begin{enumerate}
\def\labelenumi{\arabic{enumi}.}
\tightlist
\item
  The current working directory.
\item
  Installed packages (like built-ins).
\item
  Paths defined in \texttt{sys.path}.
\end{enumerate}

You can check where modules are loaded from:

\begin{Shaded}
\begin{Highlighting}[]
\ImportTok{import}\NormalTok{ sys}
\BuiltInTok{print}\NormalTok{(sys.path)}
\end{Highlighting}
\end{Shaded}

Quick Summary Table

\begin{longtable}[]{@{}ll@{}}
\toprule\noalign{}
Statement & Meaning \\
\midrule\noalign{}
\endhead
\bottomrule\noalign{}
\endlastfoot
\texttt{import\ math} & Import the whole module \\
\texttt{math.sqrt(25)} & Access function using
\texttt{module.function} \\
\texttt{import\ a,\ b} & Import multiple modules at once \\
\texttt{sys.path} & Shows module search paths \\
\end{longtable}

\subsubsection{Tiny Code}\label{tiny-code-40}

\begin{Shaded}
\begin{Highlighting}[]
\ImportTok{import}\NormalTok{ math}

\NormalTok{radius }\OperatorTok{=} \DecValTok{3}
\NormalTok{area }\OperatorTok{=}\NormalTok{ math.pi }\OperatorTok{*}\NormalTok{ (radius  }\DecValTok{2}\NormalTok{)}
\BuiltInTok{print}\NormalTok{(}\StringTok{"Circle area:"}\NormalTok{, area)}
\end{Highlighting}
\end{Shaded}

\subsubsection{Why it Matters}\label{why-it-matters-40}

Modules let you reuse existing solutions instead of reinventing the
wheel. With imports, you can access thousands of built-in and
third-party libraries that extend Python's power for math, networking,
data science, and more.

\subsubsection{Try It Yourself}\label{try-it-yourself-40}

\begin{enumerate}
\def\labelenumi{\arabic{enumi}.}
\tightlist
\item
  Import the \texttt{math} module and calculate the square root of 49.
\item
  Import the \texttt{random} module and generate a random integer
  between 1 and 100.
\item
  Use \texttt{math.pi} to compute the area of a circle with radius 10.
\item
  Print out the list of paths from \texttt{sys.path} and check where
  Python looks for modules.
\end{enumerate}

\subsection{\texorpdfstring{42. Built-in Modules (\texttt{math},
\texttt{random})}{42. Built-in Modules (math, random)}}\label{built-in-modules-math-random}

Python comes with many built-in modules that provide ready-to-use
functionality. Two of the most commonly used are \texttt{math} (for
mathematical operations) and \texttt{random} (for random number
generation).

\subsubsection{Deep Dive}\label{deep-dive-41}

The \texttt{math} Module Provides advanced mathematical functions.

Commonly used functions and constants:

\begin{Shaded}
\begin{Highlighting}[]
\ImportTok{import}\NormalTok{ math}

\BuiltInTok{print}\NormalTok{(math.sqrt(}\DecValTok{25}\NormalTok{))     }\CommentTok{\# 5.0}
\BuiltInTok{print}\NormalTok{(math.}\BuiltInTok{pow}\NormalTok{(}\DecValTok{2}\NormalTok{, }\DecValTok{3}\NormalTok{))    }\CommentTok{\# 8.0}
\BuiltInTok{print}\NormalTok{(math.factorial(}\DecValTok{5}\NormalTok{)) }\CommentTok{\# 120}
\BuiltInTok{print}\NormalTok{(math.pi)           }\CommentTok{\# 3.141592653589793}
\BuiltInTok{print}\NormalTok{(math.e)            }\CommentTok{\# 2.718281828459045}
\end{Highlighting}
\end{Shaded}

Other useful functions:

\begin{itemize}
\tightlist
\item
  \texttt{math.ceil(x)} → round up.
\item
  \texttt{math.floor(x)} → round down.
\item
  \texttt{math.log(x,\ base)} → logarithm.
\item
  \texttt{math.sin(x)}, \texttt{math.cos(x)} → trigonometry.
\end{itemize}

The \texttt{random} Module Used for randomness in numbers, selections,
and shuffling.

Examples:

\begin{Shaded}
\begin{Highlighting}[]
\ImportTok{import}\NormalTok{ random}

\BuiltInTok{print}\NormalTok{(random.random())        }\CommentTok{\# random float [0, 1)}
\BuiltInTok{print}\NormalTok{(random.randint(}\DecValTok{1}\NormalTok{, }\DecValTok{6}\NormalTok{))   }\CommentTok{\# random integer between 1 and 6}
\BuiltInTok{print}\NormalTok{(random.choice([}\StringTok{"red"}\NormalTok{, }\StringTok{"blue"}\NormalTok{, }\StringTok{"green"}\NormalTok{]))  }\CommentTok{\# random choice}
\end{Highlighting}
\end{Shaded}

Other useful functions:

\begin{itemize}
\tightlist
\item
  \texttt{random.shuffle(list)} → shuffle a list in place.
\item
  \texttt{random.uniform(a,\ b)} → random float between \texttt{a} and
  \texttt{b}.
\item
  \texttt{random.sample(population,\ k)} → pick \texttt{k} unique items.
\end{itemize}

Quick Summary Table

\begin{longtable}[]{@{}llll@{}}
\toprule\noalign{}
Module & Function & Example & Result \\
\midrule\noalign{}
\endhead
\bottomrule\noalign{}
\endlastfoot
math & \texttt{math.sqrt(16)} & square root & \texttt{4.0} \\
math & \texttt{math.ceil(2.3)} & round up & \texttt{3} \\
math & \texttt{math.pi} & constant π & \texttt{3.14159...} \\
random & \texttt{random.random()} & float 0--1 & e.g.~\texttt{0.732} \\
random & \texttt{random.randint(1,10)} & random int & between 1 and
10 \\
random & \texttt{random.choice(seq)} & random element & one from list \\
random & \texttt{random.shuffle(seq)} & shuffle list & reorders in
place \\
\end{longtable}

\subsubsection{Tiny Code}\label{tiny-code-41}

\begin{Shaded}
\begin{Highlighting}[]
\ImportTok{import}\NormalTok{ math, random}

\CommentTok{\# math example}
\BuiltInTok{print}\NormalTok{(}\StringTok{"Cos(0):"}\NormalTok{, math.cos(}\DecValTok{0}\NormalTok{))}

\CommentTok{\# random example}
\NormalTok{colors }\OperatorTok{=}\NormalTok{ [}\StringTok{"red"}\NormalTok{, }\StringTok{"green"}\NormalTok{, }\StringTok{"blue"}\NormalTok{]}
\NormalTok{random.shuffle(colors)}
\BuiltInTok{print}\NormalTok{(}\StringTok{"Shuffled colors:"}\NormalTok{, colors)}
\end{Highlighting}
\end{Shaded}

\subsubsection{Why it Matters}\label{why-it-matters-41}

Built-in modules like \texttt{math} and \texttt{random} save you from
writing code from scratch. They provide reliable, optimized tools for
tasks you'll use frequently, from calculating areas to simulating dice
rolls.

\subsubsection{Try It Yourself}\label{try-it-yourself-41}

\begin{enumerate}
\def\labelenumi{\arabic{enumi}.}
\tightlist
\item
  Use \texttt{math.factorial(6)} to calculate \texttt{6!}.
\item
  Generate a random float between 5 and 10 using
  \texttt{random.uniform()}.
\item
  Create a list of 5 numbers, shuffle it, and print the result.
\item
  Use \texttt{random.sample(range(1,\ 50),\ 6)} to simulate lottery
  numbers.
\end{enumerate}

\subsection{\texorpdfstring{43. Aliasing Imports
(\texttt{import\ ...\ as\ ...})}{43. Aliasing Imports (import ... as ...)}}\label{aliasing-imports-import-...-as-...}

Sometimes module names are long, or you want a shorter name for
convenience. Python allows you to alias a module (or part of it) using
\texttt{as}. This doesn't change the module---it just gives it a
nickname in your code.

\subsubsection{Deep Dive}\label{deep-dive-42}

Basic Aliasing

\begin{Shaded}
\begin{Highlighting}[]
\ImportTok{import}\NormalTok{ math }\ImportTok{as}\NormalTok{ m}

\BuiltInTok{print}\NormalTok{(m.sqrt(}\DecValTok{16}\NormalTok{))   }\CommentTok{\# 4.0}
\BuiltInTok{print}\NormalTok{(m.pi)         }\CommentTok{\# 3.14159...}
\end{Highlighting}
\end{Shaded}

Here, instead of typing \texttt{math} every time, you can use
\texttt{m}.

Aliasing Specific Functions You can alias a single function too:

\begin{Shaded}
\begin{Highlighting}[]
\ImportTok{from}\NormalTok{ math }\ImportTok{import}\NormalTok{ factorial }\ImportTok{as}\NormalTok{ fact}

\BuiltInTok{print}\NormalTok{(fact(}\DecValTok{5}\NormalTok{))   }\CommentTok{\# 120}
\end{Highlighting}
\end{Shaded}

Common Conventions Some libraries have standard aliases that are widely
used in the Python community:

\begin{itemize}
\tightlist
\item
  \texttt{import\ numpy\ as\ np}
\item
  \texttt{import\ pandas\ as\ pd}
\item
  \texttt{import\ matplotlib.pyplot\ as\ plt}
\end{itemize}

These conventions make code more readable because most developers
recognize them instantly.

Why Use Aliases?

\begin{enumerate}
\def\labelenumi{\arabic{enumi}.}
\tightlist
\item
  Shorter code → no need to write long names.
\item
  Avoid conflicts → if two modules have the same function name, aliasing
  prevents confusion.
\item
  Readability → follow community conventions.
\end{enumerate}

Quick Summary Table

\begin{longtable}[]{@{}ll@{}}
\toprule\noalign{}
Statement & Meaning \\
\midrule\noalign{}
\endhead
\bottomrule\noalign{}
\endlastfoot
\texttt{import\ module\ as\ alias} & give module a short name \\
\texttt{from\ module\ import\ f\ as\ alias} & give function a short
name \\
\texttt{import\ numpy\ as\ np} & community standard alias \\
\end{longtable}

\subsubsection{Tiny Code}\label{tiny-code-42}

\begin{Shaded}
\begin{Highlighting}[]
\ImportTok{import}\NormalTok{ random }\ImportTok{as}\NormalTok{ r}

\BuiltInTok{print}\NormalTok{(r.randint(}\DecValTok{1}\NormalTok{, }\DecValTok{10}\NormalTok{))}

\ImportTok{from}\NormalTok{ math }\ImportTok{import}\NormalTok{ sqrt }\ImportTok{as}\NormalTok{ root}
\BuiltInTok{print}\NormalTok{(root(}\DecValTok{81}\NormalTok{))   }\CommentTok{\# 9.0}
\end{Highlighting}
\end{Shaded}

\subsubsection{Why it Matters}\label{why-it-matters-42}

Aliasing helps keep code neat, prevents naming conflicts, and improves
readability---especially when using popular libraries with well-known
abbreviations.

\subsubsection{Try It Yourself}\label{try-it-yourself-42}

\begin{enumerate}
\def\labelenumi{\arabic{enumi}.}
\tightlist
\item
  Import the \texttt{math} module as \texttt{m} and compute
  \texttt{m.sin(0)}.
\item
  Import \texttt{random.randint} as \texttt{dice} and use it to simulate
  rolling a dice.
\item
  Import \texttt{math.log} as \texttt{logarithm} and compute
  \texttt{logarithm(100,\ 10)}.
\item
  Think about why \texttt{import\ pandas\ as\ pd} is preferred in
  community codebases.
\end{enumerate}

\subsection{44. Importing Specific
Functions}\label{importing-specific-functions}

Instead of importing an entire module, you can import only the functions
or variables you need. This makes code shorter and sometimes clearer.

\subsubsection{Deep Dive}\label{deep-dive-43}

Basic Syntax

\begin{Shaded}
\begin{Highlighting}[]
\ImportTok{from}\NormalTok{ math }\ImportTok{import}\NormalTok{ sqrt, pi}

\BuiltInTok{print}\NormalTok{(sqrt(}\DecValTok{25}\NormalTok{))   }\CommentTok{\# 5.0}
\BuiltInTok{print}\NormalTok{(pi)         }\CommentTok{\# 3.14159...}
\end{Highlighting}
\end{Shaded}

Here, we can use \texttt{sqrt} and \texttt{pi} directly without
prefixing them with \texttt{math.}.

Import with Aliases You can also alias imported items:

\begin{Shaded}
\begin{Highlighting}[]
\ImportTok{from}\NormalTok{ math }\ImportTok{import}\NormalTok{ factorial }\ImportTok{as}\NormalTok{ fact}

\BuiltInTok{print}\NormalTok{(fact(}\DecValTok{5}\NormalTok{))   }\CommentTok{\# 120}
\end{Highlighting}
\end{Shaded}

Importing Everything (Not Recommended) Using \texttt{*} imports all
names from a module:

\begin{Shaded}
\begin{Highlighting}[]
\ImportTok{from}\NormalTok{ math }\ImportTok{import} \OperatorTok{*}
\BuiltInTok{print}\NormalTok{(sin(}\DecValTok{0}\NormalTok{))   }\CommentTok{\# 0.0}
\end{Highlighting}
\end{Shaded}

This works, but it's discouraged because:

\begin{enumerate}
\def\labelenumi{\arabic{enumi}.}
\tightlist
\item
  It clutters your namespace with too many names.
\item
  You might overwrite existing variables/functions by accident.
\end{enumerate}

When to Import Specific Functions

\begin{itemize}
\tightlist
\item
  When you only need a small part of a large module.
\item
  When you want shorter code without repeating the module name.
\item
  When clarity matters more than knowing the source module.
\end{itemize}

Quick Summary Table

\begin{longtable}[]{@{}ll@{}}
\toprule\noalign{}
Statement & Meaning \\
\midrule\noalign{}
\endhead
\bottomrule\noalign{}
\endlastfoot
\texttt{from\ math\ import\ sqrt} & Import only \texttt{sqrt} \\
\texttt{from\ math\ import\ sqrt,\ pi} & Import multiple names \\
\texttt{from\ math\ import\ factorial\ as\ f} & Import with alias \\
\texttt{from\ math\ import\ *} & Import all (not recommended) \\
\end{longtable}

\subsubsection{Tiny Code}\label{tiny-code-43}

\begin{Shaded}
\begin{Highlighting}[]
\ImportTok{from}\NormalTok{ random }\ImportTok{import}\NormalTok{ choice, randint}

\NormalTok{colors }\OperatorTok{=}\NormalTok{ [}\StringTok{"red"}\NormalTok{, }\StringTok{"green"}\NormalTok{, }\StringTok{"blue"}\NormalTok{]}
\BuiltInTok{print}\NormalTok{(choice(colors))       }\CommentTok{\# random color}
\BuiltInTok{print}\NormalTok{(randint(}\DecValTok{1}\NormalTok{, }\DecValTok{6}\NormalTok{))        }\CommentTok{\# random number 1–6}
\end{Highlighting}
\end{Shaded}

\subsubsection{Why it Matters}\label{why-it-matters-43}

Importing specific functions makes code more concise and sometimes
faster to read. It's especially useful when you're using only a few
tools from a module instead of the whole thing.

\subsubsection{Try It Yourself}\label{try-it-yourself-43}

\begin{enumerate}
\def\labelenumi{\arabic{enumi}.}
\tightlist
\item
  Import only \texttt{sqrt} and \texttt{pow} from \texttt{math} and use
  them to calculate \texttt{sqrt(16)} and \texttt{2\^{}5}.
\item
  Import \texttt{randint} from \texttt{random} and simulate rolling two
  dice.
\item
  Import \texttt{pi} from \texttt{math} and compute the circumference of
  a circle with radius 7.
\item
  Try using \texttt{from\ math\ import\ *}---then explain why this could
  cause confusion in larger programs.
\end{enumerate}

\subsection{\texorpdfstring{45. \texttt{dir()} and
\texttt{help()}}{45. dir() and help()}}\label{dir-and-help}

Python provides built-in functions like \texttt{dir()} and
\texttt{help()} to let you explore modules, objects, and their available
functionality. These are extremely useful when you're learning or
working with unfamiliar code.

\subsubsection{Deep Dive}\label{deep-dive-44}

\texttt{dir()} → List Attributes \texttt{dir(object)} returns a list of
all attributes (functions, variables, classes) that an object has.

Example with a module:

\begin{Shaded}
\begin{Highlighting}[]
\ImportTok{import}\NormalTok{ math}
\BuiltInTok{print}\NormalTok{(}\BuiltInTok{dir}\NormalTok{(math))}
\end{Highlighting}
\end{Shaded}

This will show a list like:

\begin{verbatim}
['acos', 'asin', 'atan', 'ceil', 'cos', 'e', 'pi', 'sqrt', ...]
\end{verbatim}

Example with a list:

\begin{Shaded}
\begin{Highlighting}[]
\NormalTok{nums }\OperatorTok{=}\NormalTok{ [}\DecValTok{1}\NormalTok{, }\DecValTok{2}\NormalTok{, }\DecValTok{3}\NormalTok{]}
\BuiltInTok{print}\NormalTok{(}\BuiltInTok{dir}\NormalTok{(nums))}
\end{Highlighting}
\end{Shaded}

This shows available list methods such as \texttt{append},
\texttt{extend}, \texttt{sort}.

\texttt{help()} → Documentation \texttt{help(object)} gives a detailed
explanation, including docstrings, arguments, and usage.

Example with a module:

\begin{Shaded}
\begin{Highlighting}[]
\ImportTok{import}\NormalTok{ random}
\BuiltInTok{help}\NormalTok{(random.randint)}
\end{Highlighting}
\end{Shaded}

This will display documentation:

\begin{verbatim}
randint(a, b)
    Return a random integer N such that a <= N <= b.
\end{verbatim}

Combining Both

\begin{enumerate}
\def\labelenumi{\arabic{enumi}.}
\tightlist
\item
  Use \texttt{dir()} to discover what functions exist.
\item
  Use \texttt{help()} to learn how a specific one works.
\end{enumerate}

Quick Summary Table

\begin{longtable}[]{@{}lll@{}}
\toprule\noalign{}
Function & Purpose & Example \\
\midrule\noalign{}
\endhead
\bottomrule\noalign{}
\endlastfoot
\texttt{dir(obj)} & Lists all attributes/methods & \texttt{dir(math)} \\
\texttt{help(obj)} & Shows documentation of an object &
\texttt{help(str.upper)} \\
\end{longtable}

\subsubsection{Tiny Code}\label{tiny-code-44}

\begin{Shaded}
\begin{Highlighting}[]
\ImportTok{import}\NormalTok{ math}

\BuiltInTok{print}\NormalTok{(}\StringTok{"Attributes in math:"}\NormalTok{, }\BuiltInTok{dir}\NormalTok{(math)[:}\DecValTok{5}\NormalTok{])   }\CommentTok{\# show first 5 only}
\BuiltInTok{help}\NormalTok{(math.sqrt)   }\CommentTok{\# show docstring for sqrt}
\end{Highlighting}
\end{Shaded}

\subsubsection{Why it Matters}\label{why-it-matters-44}

Instead of searching online every time, you can use \texttt{dir()} and
\texttt{help()} inside Python itself. This makes learning, debugging,
and exploring modules much faster.

\subsubsection{Try It Yourself}\label{try-it-yourself-44}

\begin{enumerate}
\def\labelenumi{\arabic{enumi}.}
\tightlist
\item
  Use \texttt{dir(str)} to see what methods strings have.
\item
  Pick one (like \texttt{.split}) and call \texttt{help(str.split)}.
\item
  Import the \texttt{random} module and run \texttt{dir(random)}---see
  how many functions it provides.
\item
  Use \texttt{help(random.choice)} to understand how it works.
\end{enumerate}

\subsection{46. Creating Your Own
Module}\label{creating-your-own-module}

A module is just a Python file that you can reuse in other programs. By
creating your own module, you can organize code into separate files,
making projects easier to maintain and share.

\subsubsection{Deep Dive}\label{deep-dive-45}

Step 1: Write a Module Any \texttt{.py} file can act as a module.
Example --- create a file called \texttt{mymath.py}:

\begin{Shaded}
\begin{Highlighting}[]
\CommentTok{\# mymath.py}
\KeywordTok{def}\NormalTok{ add(a, b):}
    \ControlFlowTok{return}\NormalTok{ a }\OperatorTok{+}\NormalTok{ b}

\KeywordTok{def}\NormalTok{ multiply(a, b):}
    \ControlFlowTok{return}\NormalTok{ a }\OperatorTok{*}\NormalTok{ b}
\end{Highlighting}
\end{Shaded}

Step 2: Import the Module In another Python file (or interactive shell):

\begin{Shaded}
\begin{Highlighting}[]
\ImportTok{import}\NormalTok{ mymath}

\BuiltInTok{print}\NormalTok{(mymath.add(}\DecValTok{2}\NormalTok{, }\DecValTok{3}\NormalTok{))       }\CommentTok{\# 5}
\BuiltInTok{print}\NormalTok{(mymath.multiply(}\DecValTok{4}\NormalTok{, }\DecValTok{5}\NormalTok{))  }\CommentTok{\# 20}
\end{Highlighting}
\end{Shaded}

Step 3: Import Specific Functions

\begin{Shaded}
\begin{Highlighting}[]
\ImportTok{from}\NormalTok{ mymath }\ImportTok{import}\NormalTok{ add}

\BuiltInTok{print}\NormalTok{(add(}\DecValTok{10}\NormalTok{, }\DecValTok{20}\NormalTok{))   }\CommentTok{\# 30}
\end{Highlighting}
\end{Shaded}

Step 4: Module Location Python looks for modules in the current folder
first, then in installed libraries (\texttt{sys.path}). If your module
is in the same directory, you can import it directly.

Special Variable: \texttt{\_\_name\_\_} Inside every module, Python sets
a special variable \texttt{\_\_name\_\_}.

\begin{itemize}
\tightlist
\item
  If the module is run directly:
  \texttt{\_\_name\_\_\ ==\ "\_\_main\_\_"}.
\item
  If the module is imported: \texttt{\_\_name\_\_\ ==\ "module\_name"}.
\end{itemize}

This lets you write code that runs only when the file is executed, not
when it's imported.

\begin{Shaded}
\begin{Highlighting}[]
\CommentTok{\# mymath.py}
\KeywordTok{def}\NormalTok{ add(a, b):}
    \ControlFlowTok{return}\NormalTok{ a }\OperatorTok{+}\NormalTok{ b}

\ControlFlowTok{if} \VariableTok{\_\_name\_\_} \OperatorTok{==} \StringTok{"\_\_main\_\_"}\NormalTok{:}
    \BuiltInTok{print}\NormalTok{(}\StringTok{"Testing add:"}\NormalTok{, add(}\DecValTok{2}\NormalTok{, }\DecValTok{3}\NormalTok{))}
\end{Highlighting}
\end{Shaded}

Quick Summary Table

\begin{longtable}[]{@{}ll@{}}
\toprule\noalign{}
Step & Example \\
\midrule\noalign{}
\endhead
\bottomrule\noalign{}
\endlastfoot
Create file & \texttt{mymath.py} \\
Import whole module & \texttt{import\ mymath} \\
Import specific function & \texttt{from\ mymath\ import\ add} \\
Check module search path & \texttt{import\ sys;\ print(sys.path)} \\
Run directly check &
\texttt{if\ \_\_name\_\_\ ==\ "\_\_main\_\_":\ ...} \\
\end{longtable}

\subsubsection{Tiny Code}\label{tiny-code-45}

\begin{Shaded}
\begin{Highlighting}[]
\CommentTok{\# File: greetings.py}
\KeywordTok{def}\NormalTok{ hello(name):}
    \ControlFlowTok{return} \SpecialStringTok{f"Hello, }\SpecialCharTok{\{}\NormalTok{name}\SpecialCharTok{\}}\SpecialStringTok{!"}

\CommentTok{\# File: main.py}
\ImportTok{import}\NormalTok{ greetings}
\BuiltInTok{print}\NormalTok{(greetings.hello(}\StringTok{"Alice"}\NormalTok{))}
\end{Highlighting}
\end{Shaded}

\subsubsection{Why it Matters}\label{why-it-matters-45}

Creating your own modules lets you structure larger projects, reuse code
across different scripts, and share your work with others. It's the
foundation for building Python packages and libraries.

\subsubsection{Try It Yourself}\label{try-it-yourself-45}

\begin{enumerate}
\def\labelenumi{\arabic{enumi}.}
\tightlist
\item
  Create a file \texttt{calculator.py} with functions \texttt{add},
  \texttt{subtract}, \texttt{multiply}, and \texttt{divide}.
\item
  Import it in a separate file and test each function.
\item
  Add a test block using \texttt{if\ \_\_name\_\_\ ==\ "\_\_main\_\_":}
  that runs some examples when executed directly.
\item
  Create another module (e.g., \texttt{greetings.py}) and practice
  importing both in a single script.
\end{enumerate}

\subsection{47. Understanding Packages}\label{understanding-packages}

A package is a way to organize related modules into a directory. Unlike
a single module (a \texttt{.py} file), a package is a folder that
contains an extra file called \texttt{\_\_init\_\_.py}. This tells
Python to treat the folder as a package.

\subsubsection{Deep Dive}\label{deep-dive-46}

Basic Structure

\begin{verbatim}
mypackage/
    __init__.py
    math_utils.py
    string_utils.py
\end{verbatim}

\begin{itemize}
\tightlist
\item
  \texttt{\_\_init\_\_.py} → can be empty, or it can define what gets
  imported when the package is used.
\item
  \texttt{math\_utils.py} and \texttt{string\_utils.py} → normal Python
  modules.
\end{itemize}

Importing from a Package

\begin{Shaded}
\begin{Highlighting}[]
\ImportTok{import}\NormalTok{ mypackage.math\_utils}

\BuiltInTok{print}\NormalTok{(mypackage.math\_utils.add(}\DecValTok{2}\NormalTok{, }\DecValTok{3}\NormalTok{))}
\end{Highlighting}
\end{Shaded}

Using \texttt{from\ ...\ import\ ...}

\begin{Shaded}
\begin{Highlighting}[]
\ImportTok{from}\NormalTok{ mypackage }\ImportTok{import}\NormalTok{ string\_utils}
\BuiltInTok{print}\NormalTok{(string\_utils.reverse(}\StringTok{"hello"}\NormalTok{))}
\end{Highlighting}
\end{Shaded}

Importing Functions Directly

\begin{Shaded}
\begin{Highlighting}[]
\ImportTok{from}\NormalTok{ mypackage.math\_utils }\ImportTok{import}\NormalTok{ add}
\BuiltInTok{print}\NormalTok{(add(}\DecValTok{5}\NormalTok{, }\DecValTok{6}\NormalTok{))}
\end{Highlighting}
\end{Shaded}

\texttt{\_\_init\_\_.py} Role If \texttt{\_\_init\_\_.py} includes
imports, you can simplify usage:

\begin{Shaded}
\begin{Highlighting}[]
\CommentTok{\# mypackage/\_\_init\_\_.py}
\ImportTok{from}\NormalTok{ .math\_utils }\ImportTok{import}\NormalTok{ add}
\ImportTok{from}\NormalTok{ .string\_utils }\ImportTok{import}\NormalTok{ reverse}
\end{Highlighting}
\end{Shaded}

Now you can do:

\begin{Shaded}
\begin{Highlighting}[]
\ImportTok{from}\NormalTok{ mypackage }\ImportTok{import}\NormalTok{ add, reverse}
\end{Highlighting}
\end{Shaded}

Nested Packages Packages can contain sub-packages:

\begin{verbatim}
mypackage/
    __init__.py
    utils/
        __init__.py
        file_utils.py
\end{verbatim}

Access with:

\begin{Shaded}
\begin{Highlighting}[]
\ImportTok{import}\NormalTok{ mypackage.utils.file\_utils}
\end{Highlighting}
\end{Shaded}

Quick Summary Table

\begin{longtable}[]{@{}ll@{}}
\toprule\noalign{}
Term & Meaning \\
\midrule\noalign{}
\endhead
\bottomrule\noalign{}
\endlastfoot
Module & Single \texttt{.py} file \\
Package & Directory with \texttt{\_\_init\_\_.py} + modules \\
Sub-package & Package inside another package \\
Import & \texttt{import\ mypackage.module} \\
Simplify import & Define exports in \texttt{\_\_init\_\_.py} \\
\end{longtable}

\subsubsection{Tiny Code}\label{tiny-code-46}

\begin{verbatim}
mypackage/
    __init__.py
    greetings.py
\end{verbatim}

\begin{Shaded}
\begin{Highlighting}[]
\CommentTok{\# greetings.py}
\KeywordTok{def}\NormalTok{ hello(name):}
    \ControlFlowTok{return} \SpecialStringTok{f"Hello, }\SpecialCharTok{\{}\NormalTok{name}\SpecialCharTok{\}}\SpecialStringTok{!"}

\CommentTok{\# main.py}
\ImportTok{from}\NormalTok{ mypackage }\ImportTok{import}\NormalTok{ greetings}
\BuiltInTok{print}\NormalTok{(greetings.hello(}\StringTok{"Alice"}\NormalTok{))}
\end{Highlighting}
\end{Shaded}

\subsubsection{Why it Matters}\label{why-it-matters-46}

Packages make it easy to organize large projects into smaller, logical
parts. They allow you to group related modules together, keep code
clean, and make it reusable for others.

\subsubsection{Try It Yourself}\label{try-it-yourself-46}

\begin{enumerate}
\def\labelenumi{\arabic{enumi}.}
\tightlist
\item
  Create a folder \texttt{shapes/} with \texttt{\_\_init\_\_.py} and a
  module \texttt{circle.py} that has \texttt{area(r)}.
\item
  Import \texttt{circle} in another file and test the function.
\item
  Add another module \texttt{square.py} with \texttt{area(s)} and import
  both.
\item
  Modify \texttt{\_\_init\_\_.py} so you can do
  \texttt{from\ shapes\ import\ area} for both circle and square.
\end{enumerate}

\subsection{\texorpdfstring{48. Using \texttt{pip} to Install
Packages}{48. Using pip to Install Packages}}\label{using-pip-to-install-packages}

While Python's standard library is powerful, you'll often need
third-party packages. Python uses pip (Python Package Installer) to
download and manage these packages from the Python Package Index (PyPI).

\subsubsection{Deep Dive}\label{deep-dive-47}

Check if \texttt{pip} is Installed Most modern Python versions include
it by default. You can check with:

\begin{Shaded}
\begin{Highlighting}[]
\ExtensionTok{pip} \AttributeTok{{-}{-}version}
\end{Highlighting}
\end{Shaded}

Installing a Package

\begin{Shaded}
\begin{Highlighting}[]
\ExtensionTok{pip}\NormalTok{ install requests}
\end{Highlighting}
\end{Shaded}

This downloads and installs the popular \texttt{requests} library for
making HTTP requests.

Using the Installed Package

\begin{Shaded}
\begin{Highlighting}[]
\ImportTok{import}\NormalTok{ requests}

\NormalTok{response }\OperatorTok{=}\NormalTok{ requests.get(}\StringTok{"https://api.github.com"}\NormalTok{)}
\BuiltInTok{print}\NormalTok{(response.status\_code)   }\CommentTok{\# 200}
\end{Highlighting}
\end{Shaded}

Upgrading a Package

\begin{Shaded}
\begin{Highlighting}[]
\ExtensionTok{pip}\NormalTok{ install }\AttributeTok{{-}{-}upgrade}\NormalTok{ requests}
\end{Highlighting}
\end{Shaded}

Uninstalling a Package

\begin{Shaded}
\begin{Highlighting}[]
\ExtensionTok{pip}\NormalTok{ uninstall requests}
\end{Highlighting}
\end{Shaded}

Listing Installed Packages

\begin{Shaded}
\begin{Highlighting}[]
\ExtensionTok{pip}\NormalTok{ list}
\end{Highlighting}
\end{Shaded}

Search for Packages

\begin{Shaded}
\begin{Highlighting}[]
\ExtensionTok{pip}\NormalTok{ search numpy}
\end{Highlighting}
\end{Shaded}

Requirements File You can save dependencies in a file
(\texttt{requirements.txt}) so others can install them easily:

\begin{Shaded}
\begin{Highlighting}[]
\NormalTok{requests==2.31.0}
\NormalTok{numpy\textgreater{}=1.25}
\end{Highlighting}
\end{Shaded}

Install everything at once:

\begin{Shaded}
\begin{Highlighting}[]
\ExtensionTok{pip}\NormalTok{ install }\AttributeTok{{-}r}\NormalTok{ requirements.txt}
\end{Highlighting}
\end{Shaded}

Quick Summary Table

\begin{longtable}[]{@{}ll@{}}
\toprule\noalign{}
Command & Purpose \\
\midrule\noalign{}
\endhead
\bottomrule\noalign{}
\endlastfoot
\texttt{pip\ install\ package} & Install a package \\
\texttt{pip\ install\ -\/-upgrade\ package} & Update a package \\
\texttt{pip\ uninstall\ package} & Remove a package \\
\texttt{pip\ list} & Show installed packages \\
\texttt{pip\ freeze\ \textgreater{}\ requirements.txt} & Save current
dependencies \\
\texttt{pip\ install\ -r\ requirements.txt} & Install from requirements
file \\
\end{longtable}

\subsubsection{Tiny Code}\label{tiny-code-47}

\begin{Shaded}
\begin{Highlighting}[]
\ImportTok{import}\NormalTok{ numpy }\ImportTok{as}\NormalTok{ np}

\NormalTok{arr }\OperatorTok{=}\NormalTok{ np.array([}\DecValTok{1}\NormalTok{, }\DecValTok{2}\NormalTok{, }\DecValTok{3}\NormalTok{])}
\BuiltInTok{print}\NormalTok{(}\StringTok{"Array:"}\NormalTok{, arr)}
\end{Highlighting}
\end{Shaded}

\subsubsection{Why it Matters}\label{why-it-matters-47}

\texttt{pip} opens the door to Python's massive ecosystem. Whether you
need data analysis (\texttt{pandas}), machine learning
(\texttt{scikit-learn}), or web frameworks (\texttt{Flask},
\texttt{Django}), you can install them in seconds and start building.

\subsubsection{Try It Yourself}\label{try-it-yourself-47}

\begin{enumerate}
\def\labelenumi{\arabic{enumi}.}
\tightlist
\item
  Run \texttt{pip\ list} to see what's already installed.
\item
  Install the \texttt{requests} package and use it to fetch a webpage.
\item
  Install \texttt{pandas} and create a simple DataFrame.
\item
  Export your current environment with
  \texttt{pip\ freeze\ \textgreater{}\ requirements.txt} and share it
  with a friend.
\end{enumerate}

\subsection{49. Virtual Environments}\label{virtual-environments}

A virtual environment is a self-contained directory that holds a
specific Python version and its installed packages. It allows you to
isolate dependencies for different projects so they don't conflict with
each other.

\subsubsection{Deep Dive}\label{deep-dive-48}

Why Virtual Environments?

\begin{itemize}
\tightlist
\item
  Different projects may need different versions of the same library.
\item
  Prevents conflicts between global and project-specific packages.
\item
  Keeps your system Python clean.
\end{itemize}

Creating a Virtual Environment Use the built-in \texttt{venv} module:

\begin{Shaded}
\begin{Highlighting}[]
\ExtensionTok{python} \AttributeTok{{-}m}\NormalTok{ venv myenv}
\end{Highlighting}
\end{Shaded}

This creates a folder \texttt{myenv/} with its own Python interpreter
and libraries.

Activating the Environment

\begin{itemize}
\tightlist
\item
  On Windows:
\end{itemize}

\begin{Shaded}
\begin{Highlighting}[]
\ExtensionTok{myenv\textbackslash{}Scripts\textbackslash{}activate}
\end{Highlighting}
\end{Shaded}

\begin{itemize}
\tightlist
\item
  On Mac/Linux:
\end{itemize}

\begin{Shaded}
\begin{Highlighting}[]
\BuiltInTok{source}\NormalTok{ myenv/bin/activate}
\end{Highlighting}
\end{Shaded}

You'll see \texttt{(myenv)} appear in your terminal prompt, showing it's
active.

Installing Packages Inside Once activated, use \texttt{pip}
normally---it only affects this environment:

\begin{Shaded}
\begin{Highlighting}[]
\ExtensionTok{pip}\NormalTok{ install requests}
\end{Highlighting}
\end{Shaded}

Deactivating the Environment

\begin{Shaded}
\begin{Highlighting}[]
\ExtensionTok{deactivate}
\end{Highlighting}
\end{Shaded}

This returns you to the system Python.

Removing the Environment Just delete the folder \texttt{myenv/}---it's
safe.

Quick Summary Table

\begin{longtable}[]{@{}ll@{}}
\toprule\noalign{}
Command & Purpose \\
\midrule\noalign{}
\endhead
\bottomrule\noalign{}
\endlastfoot
\texttt{python\ -m\ venv\ myenv} & Create a virtual environment \\
\texttt{source\ myenv/bin/activate} & Activate (Mac/Linux) \\
\texttt{myenv\textbackslash{}Scripts\textbackslash{}activate} & Activate
(Windows) \\
\texttt{pip\ install\ package} & Install inside environment \\
\texttt{deactivate} & Exit environment \\
\end{longtable}

\subsubsection{Tiny Code}\label{tiny-code-48}

\begin{Shaded}
\begin{Highlighting}[]
\CommentTok{\# Create and activate environment}
\ExtensionTok{python} \AttributeTok{{-}m}\NormalTok{ venv env\_demo}
\BuiltInTok{source}\NormalTok{ env\_demo/bin/activate   }\CommentTok{\# Linux/Mac}

\ExtensionTok{pip}\NormalTok{ install numpy}
\ExtensionTok{python} \AttributeTok{{-}c} \StringTok{"import numpy; print(numpy.\_\_version\_\_)"}
\end{Highlighting}
\end{Shaded}

\subsubsection{Why it Matters}\label{why-it-matters-48}

Virtual environments are essential for professional Python development.
They ensure each project has the right dependencies and prevent ``it
works on my machine'' problems.

\subsubsection{Try It Yourself}\label{try-it-yourself-48}

\begin{enumerate}
\def\labelenumi{\arabic{enumi}.}
\tightlist
\item
  Create a new virtual environment called \texttt{project\_env}.
\item
  Activate it and install \texttt{pandas}.
\item
  Verify by importing \texttt{pandas} in Python.
\item
  Deactivate, then delete the folder to remove the environment.
\end{enumerate}

\subsection{50. Popular Third-Party Packages
(Overview)}\label{popular-third-party-packages-overview}

Beyond the Python standard library, the community has built thousands of
powerful third-party packages available through PyPI (Python Package
Index). These extend Python's capabilities for web development, data
analysis, machine learning, automation, and more.

\subsubsection{Deep Dive}\label{deep-dive-49}

Web Development

\begin{itemize}
\tightlist
\item
  Flask → lightweight framework for web apps.
\item
  Django → full-featured framework for large projects.
\end{itemize}

\begin{Shaded}
\begin{Highlighting}[]
\ImportTok{from}\NormalTok{ flask }\ImportTok{import}\NormalTok{ Flask}
\NormalTok{app }\OperatorTok{=}\NormalTok{ Flask(}\VariableTok{\_\_name\_\_}\NormalTok{)}

\AttributeTok{@app.route}\NormalTok{(}\StringTok{"/"}\NormalTok{)}
\KeywordTok{def}\NormalTok{ home():}
    \ControlFlowTok{return} \StringTok{"Hello, Flask!"}
\end{Highlighting}
\end{Shaded}

Data Science \& Analysis

\begin{itemize}
\tightlist
\item
  NumPy → arrays and fast math operations.
\item
  Pandas → dataframes for data analysis.
\item
  Matplotlib / Seaborn → visualization and charts.
\end{itemize}

\begin{Shaded}
\begin{Highlighting}[]
\ImportTok{import}\NormalTok{ pandas }\ImportTok{as}\NormalTok{ pd}

\NormalTok{data }\OperatorTok{=}\NormalTok{ \{}\StringTok{"Name"}\NormalTok{: [}\StringTok{"Alice"}\NormalTok{, }\StringTok{"Bob"}\NormalTok{], }\StringTok{"Age"}\NormalTok{: [}\DecValTok{25}\NormalTok{, }\DecValTok{30}\NormalTok{]\}}
\NormalTok{df }\OperatorTok{=}\NormalTok{ pd.DataFrame(data)}
\BuiltInTok{print}\NormalTok{(df)}
\end{Highlighting}
\end{Shaded}

Machine Learning \& AI

\begin{itemize}
\tightlist
\item
  scikit-learn → machine learning algorithms.
\item
  TensorFlow / PyTorch → deep learning libraries.
\end{itemize}

\begin{Shaded}
\begin{Highlighting}[]
\ImportTok{from}\NormalTok{ sklearn.linear\_model }\ImportTok{import}\NormalTok{ LinearRegression}
\NormalTok{model }\OperatorTok{=}\NormalTok{ LinearRegression()}
\end{Highlighting}
\end{Shaded}

Networking \& APIs

\begin{itemize}
\tightlist
\item
  Requests → simple HTTP requests.
\item
  FastAPI → modern web APIs with async support.
\end{itemize}

Automation \& Scripting

\begin{itemize}
\tightlist
\item
  BeautifulSoup → web scraping.
\item
  openpyxl → Excel file automation.
\item
  schedule → lightweight task scheduler.
\end{itemize}

Why Use Third-Party Packages?

\begin{itemize}
\tightlist
\item
  Save time → no need to reinvent the wheel.
\item
  Tested \& optimized → reliable, community-supported.
\item
  Ecosystem → Python's real power comes from these packages.
\end{itemize}

Quick Summary Table

\begin{longtable}[]{@{}
  >{\raggedright\arraybackslash}p{(\linewidth - 4\tabcolsep) * \real{0.1951}}
  >{\raggedright\arraybackslash}p{(\linewidth - 4\tabcolsep) * \real{0.4146}}
  >{\raggedright\arraybackslash}p{(\linewidth - 4\tabcolsep) * \real{0.3902}}@{}}
\toprule\noalign{}
\begin{minipage}[b]{\linewidth}\raggedright
Area
\end{minipage} & \begin{minipage}[b]{\linewidth}\raggedright
Popular Packages
\end{minipage} & \begin{minipage}[b]{\linewidth}\raggedright
Use Case
\end{minipage} \\
\midrule\noalign{}
\endhead
\bottomrule\noalign{}
\endlastfoot
Web Development & Flask, Django, FastAPI & Build websites \& APIs \\
Data Analysis & NumPy, Pandas, Matplotlib, Seaborn & Process \&
visualize data \\
Machine Learning & scikit-learn, TensorFlow, PyTorch & ML \& deep
learning \\
Automation & Requests, BeautifulSoup, openpyxl & HTTP, scraping, Excel
automation \\
\end{longtable}

\subsubsection{Tiny Code}\label{tiny-code-49}

\begin{Shaded}
\begin{Highlighting}[]
\ImportTok{import}\NormalTok{ requests}

\NormalTok{response }\OperatorTok{=}\NormalTok{ requests.get(}\StringTok{"https://api.github.com"}\NormalTok{)}
\BuiltInTok{print}\NormalTok{(}\StringTok{"Status:"}\NormalTok{, response.status\_code)}
\end{Highlighting}
\end{Shaded}

\subsubsection{Why it Matters}\label{why-it-matters-49}

Third-party packages are what make Python one of the most popular
languages today. Whether you want to build websites, analyze data, or
train AI models, there's a package ready to help you.

\subsubsection{Try It Yourself}\label{try-it-yourself-49}

\begin{enumerate}
\def\labelenumi{\arabic{enumi}.}
\tightlist
\item
  Use \texttt{pip\ install\ requests} and fetch data from any website.
\item
  Install \texttt{pandas} and create a small table of data.
\item
  Install \texttt{matplotlib} and draw a simple line chart.
\item
  Explore PyPI (\url{https://pypi.org}) and find a package that
  interests you.
\end{enumerate}

\section{Chapter 6. File Handling}\label{chapter-6.-file-handling}

\subsection{\texorpdfstring{51. Opening Files
(\texttt{open})}{51. Opening Files (open)}}\label{opening-files-open}

Working with files is a core part of programming. Python's built-in
\texttt{open()} function lets you read from and write to files easily.

\subsubsection{Deep Dive}\label{deep-dive-50}

Basic Syntax

\begin{Shaded}
\begin{Highlighting}[]
\BuiltInTok{file} \OperatorTok{=} \BuiltInTok{open}\NormalTok{(}\StringTok{"example.txt"}\NormalTok{, }\StringTok{"mode"}\NormalTok{)}
\end{Highlighting}
\end{Shaded}

\begin{itemize}
\tightlist
\item
  \texttt{"example.txt"} → the file name (with path if needed).
\item
  \texttt{"mode"} → tells Python how to open the file.
\end{itemize}

Common modes:

\begin{itemize}
\tightlist
\item
  \texttt{"r"} → read (default).
\item
  \texttt{"w"} → write (creates/overwrites file).
\item
  \texttt{"a"} → append (adds to file).
\item
  \texttt{"b"} → binary mode (e.g., images).
\item
  \texttt{"r+"} → read and write.
\end{itemize}

Example: Opening for Reading

\begin{Shaded}
\begin{Highlighting}[]
\BuiltInTok{file} \OperatorTok{=} \BuiltInTok{open}\NormalTok{(}\StringTok{"example.txt"}\NormalTok{, }\StringTok{"r"}\NormalTok{)}
\NormalTok{content }\OperatorTok{=} \BuiltInTok{file}\NormalTok{.read()}
\BuiltInTok{print}\NormalTok{(content)}
\BuiltInTok{file}\NormalTok{.close()}
\end{Highlighting}
\end{Shaded}

Example: Opening for Writing

\begin{Shaded}
\begin{Highlighting}[]
\BuiltInTok{file} \OperatorTok{=} \BuiltInTok{open}\NormalTok{(}\StringTok{"new.txt"}\NormalTok{, }\StringTok{"w"}\NormalTok{)}
\BuiltInTok{file}\NormalTok{.write(}\StringTok{"Hello, Python!}\CharTok{\textbackslash{}n}\StringTok{"}\NormalTok{)}
\BuiltInTok{file}\NormalTok{.close()}
\end{Highlighting}
\end{Shaded}

File Closing Always close files after use with \texttt{file.close()}.

\begin{itemize}
\tightlist
\item
  This frees system resources.
\item
  Ensures data is written properly.
\end{itemize}

Error Handling If the file doesn't exist in \texttt{"r"} mode, Python
raises an error:

\begin{Shaded}
\begin{Highlighting}[]
\BuiltInTok{open}\NormalTok{(}\StringTok{"missing.txt"}\NormalTok{, }\StringTok{"r"}\NormalTok{)  }\CommentTok{\# FileNotFoundError}
\end{Highlighting}
\end{Shaded}

Quick Summary Table

\begin{longtable}[]{@{}lll@{}}
\toprule\noalign{}
Mode & Meaning & Example \\
\midrule\noalign{}
\endhead
\bottomrule\noalign{}
\endlastfoot
\texttt{"r"} & Read (default) & \texttt{open("f.txt",\ "r")} \\
\texttt{"w"} & Write (overwrite) & \texttt{open("f.txt",\ "w")} \\
\texttt{"a"} & Append & \texttt{open("f.txt",\ "a")} \\
\texttt{"b"} & Binary & \texttt{open("img.png",\ "rb")} \\
\texttt{"r+"} & Read + Write & \texttt{open("f.txt",\ "r+")} \\
\end{longtable}

\subsubsection{Tiny Code}\label{tiny-code-50}

\begin{Shaded}
\begin{Highlighting}[]
\CommentTok{\# Write a file}
\NormalTok{f }\OperatorTok{=} \BuiltInTok{open}\NormalTok{(}\StringTok{"hello.txt"}\NormalTok{, }\StringTok{"w"}\NormalTok{)}
\NormalTok{f.write(}\StringTok{"Hello, world!"}\NormalTok{)}
\NormalTok{f.close()}

\CommentTok{\# Read the file}
\NormalTok{f }\OperatorTok{=} \BuiltInTok{open}\NormalTok{(}\StringTok{"hello.txt"}\NormalTok{, }\StringTok{"r"}\NormalTok{)}
\BuiltInTok{print}\NormalTok{(f.read())}
\NormalTok{f.close()}
\end{Highlighting}
\end{Shaded}

\subsubsection{Why it Matters}\label{why-it-matters-50}

Files let you store information permanently. Whether saving logs,
configurations, or datasets, file handling is essential for almost every
real-world Python project.

\subsubsection{Try It Yourself}\label{try-it-yourself-50}

\begin{enumerate}
\def\labelenumi{\arabic{enumi}.}
\tightlist
\item
  Create a file \texttt{notes.txt} and write three lines of text into
  it.
\item
  Reopen the file in \texttt{"r"} mode and print the contents.
\item
  Open the same file in \texttt{"a"} mode and add another line.
\item
  Try opening a non-existent file in \texttt{"r"} mode and see the
  error.
\end{enumerate}

\subsection{52. Reading Files}\label{reading-files}

Once you open a file in read mode, you can extract its contents in
different ways depending on your needs: the whole file, line by line, or
into a list.

\subsubsection{Deep Dive}\label{deep-dive-51}

Read the Entire File

\begin{Shaded}
\begin{Highlighting}[]
\NormalTok{f }\OperatorTok{=} \BuiltInTok{open}\NormalTok{(}\StringTok{"notes.txt"}\NormalTok{, }\StringTok{"r"}\NormalTok{)}
\NormalTok{content }\OperatorTok{=}\NormalTok{ f.read()}
\BuiltInTok{print}\NormalTok{(content)}
\NormalTok{f.close()}
\end{Highlighting}
\end{Shaded}

\begin{itemize}
\tightlist
\item
  \texttt{f.read()} → returns the whole file as a single string.
\end{itemize}

Read One Line at a Time

\begin{Shaded}
\begin{Highlighting}[]
\NormalTok{f }\OperatorTok{=} \BuiltInTok{open}\NormalTok{(}\StringTok{"notes.txt"}\NormalTok{, }\StringTok{"r"}\NormalTok{)}
\NormalTok{line1 }\OperatorTok{=}\NormalTok{ f.readline()}
\NormalTok{line2 }\OperatorTok{=}\NormalTok{ f.readline()}
\BuiltInTok{print}\NormalTok{(line1, line2)}
\NormalTok{f.close()}
\end{Highlighting}
\end{Shaded}

\begin{itemize}
\tightlist
\item
  Each call to \texttt{readline()} gets the next line (including the
  \texttt{\textbackslash{}n}).
\end{itemize}

Read All Lines into a List

\begin{Shaded}
\begin{Highlighting}[]
\NormalTok{f }\OperatorTok{=} \BuiltInTok{open}\NormalTok{(}\StringTok{"notes.txt"}\NormalTok{, }\StringTok{"r"}\NormalTok{)}
\NormalTok{lines }\OperatorTok{=}\NormalTok{ f.readlines()}
\BuiltInTok{print}\NormalTok{(lines)}
\NormalTok{f.close()}
\end{Highlighting}
\end{Shaded}

\begin{itemize}
\tightlist
\item
  \texttt{f.readlines()} returns a list where each element is one line.
\end{itemize}

Iterating Over a File The most common and memory-friendly way:

\begin{Shaded}
\begin{Highlighting}[]
\NormalTok{f }\OperatorTok{=} \BuiltInTok{open}\NormalTok{(}\StringTok{"notes.txt"}\NormalTok{, }\StringTok{"r"}\NormalTok{)}
\ControlFlowTok{for}\NormalTok{ line }\KeywordTok{in}\NormalTok{ f:}
    \BuiltInTok{print}\NormalTok{(line.strip())}
\NormalTok{f.close()}
\end{Highlighting}
\end{Shaded}

\begin{itemize}
\tightlist
\item
  This reads one line at a time, great for large files.
\end{itemize}

Quick Summary Table

\begin{longtable}[]{@{}
  >{\raggedright\arraybackslash}p{(\linewidth - 4\tabcolsep) * \real{0.2113}}
  >{\raggedright\arraybackslash}p{(\linewidth - 4\tabcolsep) * \real{0.4648}}
  >{\raggedright\arraybackslash}p{(\linewidth - 4\tabcolsep) * \real{0.3239}}@{}}
\toprule\noalign{}
\begin{minipage}[b]{\linewidth}\raggedright
Method
\end{minipage} & \begin{minipage}[b]{\linewidth}\raggedright
What it Does
\end{minipage} & \begin{minipage}[b]{\linewidth}\raggedright
Example
\end{minipage} \\
\midrule\noalign{}
\endhead
\bottomrule\noalign{}
\endlastfoot
\texttt{f.read()} & Reads whole file as a string &
\texttt{content\ =\ f.read()} \\
\texttt{f.readline()} & Reads the next line &
\texttt{line\ =\ f.readline()} \\
\texttt{f.readlines()} & Reads all lines into a list &
\texttt{lines\ =\ f.readlines()} \\
\texttt{for\ line\ in\ f} & Iterates line by line (efficient) &
\texttt{for\ l\ in\ f:\ print(l)} \\
\end{longtable}

\subsubsection{Tiny Code}\label{tiny-code-51}

\begin{Shaded}
\begin{Highlighting}[]
\ControlFlowTok{with} \BuiltInTok{open}\NormalTok{(}\StringTok{"notes.txt"}\NormalTok{, }\StringTok{"r"}\NormalTok{) }\ImportTok{as}\NormalTok{ f:}
    \ControlFlowTok{for}\NormalTok{ line }\KeywordTok{in}\NormalTok{ f:}
        \BuiltInTok{print}\NormalTok{(}\StringTok{"Line:"}\NormalTok{, line.strip())}
\end{Highlighting}
\end{Shaded}

\subsubsection{Why it Matters}\label{why-it-matters-51}

Reading files is fundamental to processing data. Whether you're
analyzing logs, reading configurations, or loading datasets,
understanding the different read methods helps you handle small and
large files efficiently.

\subsubsection{Try It Yourself}\label{try-it-yourself-51}

\begin{enumerate}
\def\labelenumi{\arabic{enumi}.}
\tightlist
\item
  Write three lines into \texttt{data.txt}.
\item
  Read the entire file at once with \texttt{f.read()}.
\item
  Use \texttt{f.readline()} twice to print the first two lines
  separately.
\item
  Use a loop to print each line from the file without extra spaces.
\end{enumerate}

\subsection{53. Writing Files}\label{writing-files}

Python lets you write text to files using the \texttt{write()} and
\texttt{writelines()} methods. This is useful for saving logs, results,
or any output that needs to be stored permanently.

\subsubsection{Deep Dive}\label{deep-dive-52}

Write Text with \texttt{write()} Opening a file in \texttt{"w"} mode
will overwrite it if it already exists, or create it if it doesn't.

\begin{Shaded}
\begin{Highlighting}[]
\NormalTok{f }\OperatorTok{=} \BuiltInTok{open}\NormalTok{(}\StringTok{"output.txt"}\NormalTok{, }\StringTok{"w"}\NormalTok{)}
\NormalTok{f.write(}\StringTok{"Hello, world!}\CharTok{\textbackslash{}n}\StringTok{"}\NormalTok{)}
\NormalTok{f.write(}\StringTok{"This is a new line.}\CharTok{\textbackslash{}n}\StringTok{"}\NormalTok{)}
\NormalTok{f.close()}
\end{Highlighting}
\end{Shaded}

Append Mode (\texttt{"a"}) To keep existing content and add to the end:

\begin{Shaded}
\begin{Highlighting}[]
\NormalTok{f }\OperatorTok{=} \BuiltInTok{open}\NormalTok{(}\StringTok{"output.txt"}\NormalTok{, }\StringTok{"a"}\NormalTok{)}
\NormalTok{f.write(}\StringTok{"Adding more text here.}\CharTok{\textbackslash{}n}\StringTok{"}\NormalTok{)}
\NormalTok{f.close()}
\end{Highlighting}
\end{Shaded}

Write Multiple Lines with \texttt{writelines()}

\begin{Shaded}
\begin{Highlighting}[]
\NormalTok{lines }\OperatorTok{=}\NormalTok{ [}\StringTok{"Line 1}\CharTok{\textbackslash{}n}\StringTok{"}\NormalTok{, }\StringTok{"Line 2}\CharTok{\textbackslash{}n}\StringTok{"}\NormalTok{, }\StringTok{"Line 3}\CharTok{\textbackslash{}n}\StringTok{"}\NormalTok{]}

\NormalTok{f }\OperatorTok{=} \BuiltInTok{open}\NormalTok{(}\StringTok{"multi.txt"}\NormalTok{, }\StringTok{"w"}\NormalTok{)}
\NormalTok{f.writelines(lines)}
\NormalTok{f.close()}
\end{Highlighting}
\end{Shaded}

⚠️ Note: \texttt{writelines()} does not add newlines automatically---you
must include \texttt{\textbackslash{}n} yourself.

Best Practice with \texttt{with} Automatically closes the file after
writing:

\begin{Shaded}
\begin{Highlighting}[]
\ControlFlowTok{with} \BuiltInTok{open}\NormalTok{(}\StringTok{"log.txt"}\NormalTok{, }\StringTok{"w"}\NormalTok{) }\ImportTok{as}\NormalTok{ f:}
\NormalTok{    f.write(}\StringTok{"Log entry 1}\CharTok{\textbackslash{}n}\StringTok{"}\NormalTok{)}
\NormalTok{    f.write(}\StringTok{"Log entry 2}\CharTok{\textbackslash{}n}\StringTok{"}\NormalTok{)}
\end{Highlighting}
\end{Shaded}

Quick Summary Table

\begin{longtable}[]{@{}lll@{}}
\toprule\noalign{}
Mode & Behavior & Example \\
\midrule\noalign{}
\endhead
\bottomrule\noalign{}
\endlastfoot
\texttt{"w"} & Write (overwrite existing file) &
\texttt{open("f.txt",\ "w")} \\
\texttt{"a"} & Append (keep existing, add more) &
\texttt{open("f.txt",\ "a")} \\
\texttt{"x"} & Create (error if file exists) &
\texttt{open("f.txt",\ "x")} \\
\end{longtable}

\subsubsection{Tiny Code}\label{tiny-code-52}

\begin{Shaded}
\begin{Highlighting}[]
\ControlFlowTok{with} \BuiltInTok{open}\NormalTok{(}\StringTok{"diary.txt"}\NormalTok{, }\StringTok{"w"}\NormalTok{) }\ImportTok{as}\NormalTok{ f:}
\NormalTok{    f.write(}\StringTok{"Day 1: Learned Python file writing.}\CharTok{\textbackslash{}n}\StringTok{"}\NormalTok{)}
\NormalTok{    f.write(}\StringTok{"Day 2: Feeling confident!}\CharTok{\textbackslash{}n}\StringTok{"}\NormalTok{)}
\end{Highlighting}
\end{Shaded}

\subsubsection{Why it Matters}\label{why-it-matters-52}

Being able to write files is crucial for persisting data beyond program
execution. Logs, reports, exported data, and notes all rely on writing
to files.

\subsubsection{Try It Yourself}\label{try-it-yourself-52}

\begin{enumerate}
\def\labelenumi{\arabic{enumi}.}
\tightlist
\item
  Create a file \texttt{journal.txt} and write three lines about your
  day.
\item
  Open the file again in \texttt{"a"} mode and add two more lines.
\item
  Use \texttt{writelines()} to add a list of tasks into
  \texttt{tasks.txt}.
\item
  Reopen and read back the contents to confirm everything was saved.
\end{enumerate}

\subsection{\texorpdfstring{54. File Modes (\texttt{r}, \texttt{w},
\texttt{a},
\texttt{b})}{54. File Modes (r, w, a, b)}}\label{file-modes-r-w-a-b}

When opening files in Python with \texttt{open()}, the mode determines
how the file is accessed---read, write, append, or binary. Understanding
modes is essential to avoid overwriting or corrupting files.

\subsubsection{Deep Dive}\label{deep-dive-53}

Text Modes (default)

\begin{itemize}
\tightlist
\item
  \texttt{"r"} → Read (default). File must exist.
\item
  \texttt{"w"} → Write. Creates new file or overwrites existing.
\item
  \texttt{"a"} → Append. Adds to the end, keeps existing content.
\item
  \texttt{"x"} → Create. Errors if the file already exists.
\end{itemize}

\begin{Shaded}
\begin{Highlighting}[]
\BuiltInTok{open}\NormalTok{(}\StringTok{"notes.txt"}\NormalTok{, }\StringTok{"r"}\NormalTok{)  }\CommentTok{\# read}
\BuiltInTok{open}\NormalTok{(}\StringTok{"notes.txt"}\NormalTok{, }\StringTok{"w"}\NormalTok{)  }\CommentTok{\# write (erase contents!)}
\BuiltInTok{open}\NormalTok{(}\StringTok{"notes.txt"}\NormalTok{, }\StringTok{"a"}\NormalTok{)  }\CommentTok{\# append}
\BuiltInTok{open}\NormalTok{(}\StringTok{"newfile.txt"}\NormalTok{, }\StringTok{"x"}\NormalTok{)}\CommentTok{\# create only if not exists}
\end{Highlighting}
\end{Shaded}

Binary Modes Add \texttt{"b"} to handle non-text files (images, audio,
executables).

\begin{itemize}
\tightlist
\item
  \texttt{"rb"} → read binary.
\item
  \texttt{"wb"} → write binary.
\item
  \texttt{"ab"} → append binary.
\end{itemize}

\begin{Shaded}
\begin{Highlighting}[]
\CommentTok{\# Reading an image}
\ControlFlowTok{with} \BuiltInTok{open}\NormalTok{(}\StringTok{"photo.jpg"}\NormalTok{, }\StringTok{"rb"}\NormalTok{) }\ImportTok{as}\NormalTok{ f:}
\NormalTok{    data }\OperatorTok{=}\NormalTok{ f.read()}

\CommentTok{\# Writing binary}
\ControlFlowTok{with} \BuiltInTok{open}\NormalTok{(}\StringTok{"copy.jpg"}\NormalTok{, }\StringTok{"wb"}\NormalTok{) }\ImportTok{as}\NormalTok{ f:}
\NormalTok{    f.write(data)}
\end{Highlighting}
\end{Shaded}

Combining Modes You can mix read/write with \texttt{"+"}:

\begin{itemize}
\tightlist
\item
  \texttt{"r+"} → read \& write (file must exist).
\item
  \texttt{"w+"} → write \& read (overwrites or creates).
\item
  \texttt{"a+"} → append \& read.
\end{itemize}

\begin{Shaded}
\begin{Highlighting}[]
\ControlFlowTok{with} \BuiltInTok{open}\NormalTok{(}\StringTok{"data.txt"}\NormalTok{, }\StringTok{"r+"}\NormalTok{) }\ImportTok{as}\NormalTok{ f:}
\NormalTok{    content }\OperatorTok{=}\NormalTok{ f.read()}
\NormalTok{    f.write(}\StringTok{"}\CharTok{\textbackslash{}n}\StringTok{Extra line"}\NormalTok{)}
\end{Highlighting}
\end{Shaded}

Quick Summary Table

\begin{longtable}[]{@{}lll@{}}
\toprule\noalign{}
Mode & Description & Notes \\
\midrule\noalign{}
\endhead
\bottomrule\noalign{}
\endlastfoot
\texttt{"r"} & Read (default) & File must exist \\
\texttt{"w"} & Write & Overwrites file \\
\texttt{"a"} & Append & Adds at end of file \\
\texttt{"x"} & Create new & Error if file exists \\
\texttt{"b"} & Binary & Add to handle non-text data \\
\texttt{"r+"} & Read + Write & No overwrite, must exist \\
\texttt{"w+"} & Write + Read & Overwrites existing file \\
\texttt{"a+"} & Append + Read & File pointer at end \\
\end{longtable}

\subsubsection{Tiny Code}\label{tiny-code-53}

\begin{Shaded}
\begin{Highlighting}[]
\CommentTok{\# Write + read}
\ControlFlowTok{with} \BuiltInTok{open}\NormalTok{(}\StringTok{"sample.txt"}\NormalTok{, }\StringTok{"w+"}\NormalTok{) }\ImportTok{as}\NormalTok{ f:}
\NormalTok{    f.write(}\StringTok{"Hello!}\CharTok{\textbackslash{}n}\StringTok{"}\NormalTok{)}
\NormalTok{    f.seek(}\DecValTok{0}\NormalTok{)}
    \BuiltInTok{print}\NormalTok{(f.read())}
\end{Highlighting}
\end{Shaded}

\subsubsection{Why it Matters}\label{why-it-matters-53}

Choosing the right mode ensures you don't lose data accidentally (like
\texttt{"w"} erasing files) and allows you to correctly handle binary
files like images or PDFs.

\subsubsection{Try It Yourself}\label{try-it-yourself-53}

\begin{enumerate}
\def\labelenumi{\arabic{enumi}.}
\tightlist
\item
  Open a file in \texttt{"w"} mode and write two lines. Reopen it in
  \texttt{"r"} mode and confirm old content was overwritten.
\item
  Open the same file in \texttt{"a"} mode and add another line.
\item
  Try using \texttt{"x"} mode to create a new file. Run it twice and
  observe the error on the second run.
\item
  Copy an image using \texttt{"rb"} and \texttt{"wb"}.
\end{enumerate}

\subsection{55. Closing Files}\label{closing-files}

When you open a file in Python, the system allocates resources to manage
it. To free these resources and ensure all data is written properly, you
must close the file once you're done.

\subsubsection{Deep Dive}\label{deep-dive-54}

Manual Closing with \texttt{close()}

\begin{Shaded}
\begin{Highlighting}[]
\NormalTok{f }\OperatorTok{=} \BuiltInTok{open}\NormalTok{(}\StringTok{"notes.txt"}\NormalTok{, }\StringTok{"w"}\NormalTok{)}
\NormalTok{f.write(}\StringTok{"Hello, file!"}\NormalTok{)}
\NormalTok{f.close()}
\end{Highlighting}
\end{Shaded}

\begin{itemize}
\tightlist
\item
  \texttt{close()} ensures data is flushed from memory to disk.
\item
  If you forget, data may not be saved properly.
\end{itemize}

Checking if a File is Closed

\begin{Shaded}
\begin{Highlighting}[]
\NormalTok{f }\OperatorTok{=} \BuiltInTok{open}\NormalTok{(}\StringTok{"notes.txt"}\NormalTok{, }\StringTok{"r"}\NormalTok{)}
\BuiltInTok{print}\NormalTok{(f.closed)   }\CommentTok{\# False}
\NormalTok{f.close()}
\BuiltInTok{print}\NormalTok{(f.closed)   }\CommentTok{\# True}
\end{Highlighting}
\end{Shaded}

Best Practice: \texttt{with} Statement Instead of manually calling
\texttt{close()}, use \texttt{with}. It automatically closes the file,
even if an error occurs.

\begin{Shaded}
\begin{Highlighting}[]
\ControlFlowTok{with} \BuiltInTok{open}\NormalTok{(}\StringTok{"notes.txt"}\NormalTok{, }\StringTok{"r"}\NormalTok{) }\ImportTok{as}\NormalTok{ f:}
\NormalTok{    content }\OperatorTok{=}\NormalTok{ f.read()}
\BuiltInTok{print}\NormalTok{(f.closed)   }\CommentTok{\# True}
\end{Highlighting}
\end{Shaded}

Flushing Without Closing If you want to save changes but keep the file
open:

\begin{Shaded}
\begin{Highlighting}[]
\NormalTok{f }\OperatorTok{=} \BuiltInTok{open}\NormalTok{(}\StringTok{"data.txt"}\NormalTok{, }\StringTok{"w"}\NormalTok{)}
\NormalTok{f.write(}\StringTok{"Line 1}\CharTok{\textbackslash{}n}\StringTok{"}\NormalTok{)}
\NormalTok{f.flush()     }\CommentTok{\# forces write to disk}
\CommentTok{\# file still open}
\NormalTok{f.close()}
\end{Highlighting}
\end{Shaded}

What Happens if You Don't Close?

\begin{itemize}
\tightlist
\item
  Data might not be saved (especially in write mode).
\item
  Too many open files can exhaust system resources.
\item
  On some systems, files stay locked until closed.
\end{itemize}

Quick Summary Table

\begin{longtable}[]{@{}ll@{}}
\toprule\noalign{}
Method & Behavior \\
\midrule\noalign{}
\endhead
\bottomrule\noalign{}
\endlastfoot
\texttt{f.close()} & Manually closes the file \\
\texttt{f.closed} & Check if file is closed \\
\texttt{f.flush()} & Force save data without closing \\
\texttt{with\ open()} & Automatically closes after block \\
\end{longtable}

\subsubsection{Tiny Code}\label{tiny-code-54}

\begin{Shaded}
\begin{Highlighting}[]
\ControlFlowTok{with} \BuiltInTok{open}\NormalTok{(}\StringTok{"log.txt"}\NormalTok{, }\StringTok{"w"}\NormalTok{) }\ImportTok{as}\NormalTok{ f:}
\NormalTok{    f.write(}\StringTok{"Session started.}\CharTok{\textbackslash{}n}\StringTok{"}\NormalTok{)}

\BuiltInTok{print}\NormalTok{(}\StringTok{"Closed?"}\NormalTok{, f.closed)  }\CommentTok{\# True}
\end{Highlighting}
\end{Shaded}

\subsubsection{Why it Matters}\label{why-it-matters-54}

Closing files ensures data safety and efficient resource usage.
Forgetting to close files can lead to bugs, data loss, or locked files.
The \texttt{with} statement makes it almost impossible to forget.

\subsubsection{Try It Yourself}\label{try-it-yourself-54}

\begin{enumerate}
\def\labelenumi{\arabic{enumi}.}
\tightlist
\item
  Open a file in write mode, write some text, and check
  \texttt{f.closed} before and after calling \texttt{close()}.
\item
  Use \texttt{with\ open()} to write two lines and verify that the file
  is closed outside the block.
\item
  Experiment with \texttt{f.flush()}---write text, flush, then write
  more before closing.
\item
  Try opening many files in a loop without closing them, then observe
  system warnings/errors.
\end{enumerate}

\subsection{\texorpdfstring{56. Using \texttt{with} Context
Manager}{56. Using with Context Manager}}\label{using-with-context-manager}

The \texttt{with} statement in Python provides a clean and safe way to
work with files. It automatically takes care of opening and closing the
file, even if errors occur while processing.

\subsubsection{Deep Dive}\label{deep-dive-55}

Basic Usage

\begin{Shaded}
\begin{Highlighting}[]
\ControlFlowTok{with} \BuiltInTok{open}\NormalTok{(}\StringTok{"notes.txt"}\NormalTok{, }\StringTok{"r"}\NormalTok{) }\ImportTok{as}\NormalTok{ f:}
\NormalTok{    content }\OperatorTok{=}\NormalTok{ f.read()}
\BuiltInTok{print}\NormalTok{(}\StringTok{"File closed?"}\NormalTok{, f.closed)  }\CommentTok{\# True}
\end{Highlighting}
\end{Shaded}

\begin{itemize}
\tightlist
\item
  The file is automatically closed after the \texttt{with} block.
\item
  You don't need to call \texttt{f.close()} manually.
\end{itemize}

Writing with \texttt{with}

\begin{Shaded}
\begin{Highlighting}[]
\ControlFlowTok{with} \BuiltInTok{open}\NormalTok{(}\StringTok{"output.txt"}\NormalTok{, }\StringTok{"w"}\NormalTok{) }\ImportTok{as}\NormalTok{ f:}
\NormalTok{    f.write(}\StringTok{"Hello, Python!}\CharTok{\textbackslash{}n}\StringTok{"}\NormalTok{)}
\NormalTok{    f.write(}\StringTok{"Writing with context manager.}\CharTok{\textbackslash{}n}\StringTok{"}\NormalTok{)}
\end{Highlighting}
\end{Shaded}

The file is saved and closed as soon as the block ends.

Why Use \texttt{with}?

\begin{enumerate}
\def\labelenumi{\arabic{enumi}.}
\tightlist
\item
  Ensures proper cleanup (file is closed automatically).
\item
  Handles exceptions safely.
\item
  Makes code cleaner and shorter.
\end{enumerate}

Multiple Files with One \texttt{with} You can work with multiple files
in a single \texttt{with} statement:

\begin{Shaded}
\begin{Highlighting}[]
\ControlFlowTok{with} \BuiltInTok{open}\NormalTok{(}\StringTok{"input.txt"}\NormalTok{, }\StringTok{"r"}\NormalTok{) }\ImportTok{as}\NormalTok{ infile, }\BuiltInTok{open}\NormalTok{(}\StringTok{"copy.txt"}\NormalTok{, }\StringTok{"w"}\NormalTok{) }\ImportTok{as}\NormalTok{ outfile:}
    \ControlFlowTok{for}\NormalTok{ line }\KeywordTok{in}\NormalTok{ infile:}
\NormalTok{        outfile.write(line)}
\end{Highlighting}
\end{Shaded}

Custom Context Managers The \texttt{with} statement isn't just for
files---it works with anything that supports the context manager
protocol (\texttt{\_\_enter\_\_} and \texttt{\_\_exit\_\_}).

Example:

\begin{Shaded}
\begin{Highlighting}[]
\KeywordTok{class}\NormalTok{ MyResource:}
    \KeywordTok{def} \FunctionTok{\_\_enter\_\_}\NormalTok{(}\VariableTok{self}\NormalTok{):}
        \BuiltInTok{print}\NormalTok{(}\StringTok{"Resource acquired"}\NormalTok{)}
        \ControlFlowTok{return} \VariableTok{self}
    \KeywordTok{def} \FunctionTok{\_\_exit\_\_}\NormalTok{(}\VariableTok{self}\NormalTok{, exc\_type, exc\_value, traceback):}
        \BuiltInTok{print}\NormalTok{(}\StringTok{"Resource released"}\NormalTok{)}

\ControlFlowTok{with}\NormalTok{ MyResource():}
    \BuiltInTok{print}\NormalTok{(}\StringTok{"Using resource"}\NormalTok{)}
\end{Highlighting}
\end{Shaded}

Quick Summary Table

\begin{longtable}[]{@{}ll@{}}
\toprule\noalign{}
Feature & Example \\
\midrule\noalign{}
\endhead
\bottomrule\noalign{}
\endlastfoot
Auto-close file & \texttt{with\ open("f.txt")\ as\ f:} \\
Write file & \texttt{with\ open("f.txt","w")\ as\ f:\ f.write("x")} \\
Multiple files &
\texttt{with\ open("a.txt")\ as\ a,\ open("b.txt")\ as\ b:} \\
Custom manager & Define \texttt{\_\_enter\_\_}, \texttt{\_\_exit\_\_} \\
\end{longtable}

\subsubsection{Tiny Code}\label{tiny-code-55}

\begin{Shaded}
\begin{Highlighting}[]
\ControlFlowTok{with} \BuiltInTok{open}\NormalTok{(}\StringTok{"data.txt"}\NormalTok{, }\StringTok{"w"}\NormalTok{) }\ImportTok{as}\NormalTok{ f:}
\NormalTok{    f.write(}\StringTok{"Line 1}\CharTok{\textbackslash{}n}\StringTok{"}\NormalTok{)}
\NormalTok{    f.write(}\StringTok{"Line 2}\CharTok{\textbackslash{}n}\StringTok{"}\NormalTok{)}

\BuiltInTok{print}\NormalTok{(}\StringTok{"Closed?"}\NormalTok{, f.closed)  }\CommentTok{\# True}
\end{Highlighting}
\end{Shaded}

\subsubsection{Why it Matters}\label{why-it-matters-55}

The \texttt{with} statement is the best practice for file handling in
Python. It makes code safer, shorter, and more reliable by guaranteeing
cleanup.

\subsubsection{Try It Yourself}\label{try-it-yourself-55}

\begin{enumerate}
\def\labelenumi{\arabic{enumi}.}
\tightlist
\item
  Use \texttt{with\ open("log.txt",\ "w")} to write three lines. Confirm
  the file is closed afterwards.
\item
  Copy the contents of one file into another using a \texttt{with}
  block.
\item
  Experiment by raising an error inside a \texttt{with} block---notice
  the file is still closed.
\item
  Create a simple class with \texttt{\_\_enter\_\_} and
  \texttt{\_\_exit\_\_} to practice writing your own context manager.
\end{enumerate}

\subsection{57. Working with CSV Files}\label{working-with-csv-files}

CSV (Comma-Separated Values) files are widely used for storing tabular
data like spreadsheets or databases. Python's built-in \texttt{csv}
module makes it easy to read and write CSV files.

\subsubsection{Deep Dive}\label{deep-dive-56}

Reading a CSV File

\begin{Shaded}
\begin{Highlighting}[]
\ImportTok{import}\NormalTok{ csv}

\ControlFlowTok{with} \BuiltInTok{open}\NormalTok{(}\StringTok{"data.csv"}\NormalTok{, }\StringTok{"r"}\NormalTok{) }\ImportTok{as}\NormalTok{ f:}
\NormalTok{    reader }\OperatorTok{=}\NormalTok{ csv.reader(f)}
    \ControlFlowTok{for}\NormalTok{ row }\KeywordTok{in}\NormalTok{ reader:}
        \BuiltInTok{print}\NormalTok{(row)}
\end{Highlighting}
\end{Shaded}

\begin{itemize}
\tightlist
\item
  \texttt{csv.reader} → reads file line by line, splitting values by
  commas.
\item
  Each row is returned as a list of strings.
\end{itemize}

Writing to a CSV File

\begin{Shaded}
\begin{Highlighting}[]
\ImportTok{import}\NormalTok{ csv}

\NormalTok{rows }\OperatorTok{=}\NormalTok{ [}
\NormalTok{    [}\StringTok{"Name"}\NormalTok{, }\StringTok{"Age"}\NormalTok{],}
\NormalTok{    [}\StringTok{"Alice"}\NormalTok{, }\DecValTok{25}\NormalTok{],}
\NormalTok{    [}\StringTok{"Bob"}\NormalTok{, }\DecValTok{30}\NormalTok{]}
\NormalTok{]}

\ControlFlowTok{with} \BuiltInTok{open}\NormalTok{(}\StringTok{"people.csv"}\NormalTok{, }\StringTok{"w"}\NormalTok{, newline}\OperatorTok{=}\StringTok{""}\NormalTok{) }\ImportTok{as}\NormalTok{ f:}
\NormalTok{    writer }\OperatorTok{=}\NormalTok{ csv.writer(f)}
\NormalTok{    writer.writerows(rows)}
\end{Highlighting}
\end{Shaded}

\begin{itemize}
\tightlist
\item
  \texttt{writerow()} → writes a single row.
\item
  \texttt{writerows()} → writes multiple rows.
\item
  \texttt{newline=""} avoids blank lines on Windows.
\end{itemize}

Using Dictionaries with CSV Instead of working with lists, you can use
DictReader and DictWriter.

\begin{Shaded}
\begin{Highlighting}[]
\ImportTok{import}\NormalTok{ csv}

\CommentTok{\# Writing}
\ControlFlowTok{with} \BuiltInTok{open}\NormalTok{(}\StringTok{"people.csv"}\NormalTok{, }\StringTok{"w"}\NormalTok{, newline}\OperatorTok{=}\StringTok{""}\NormalTok{) }\ImportTok{as}\NormalTok{ f:}
\NormalTok{    fieldnames }\OperatorTok{=}\NormalTok{ [}\StringTok{"Name"}\NormalTok{, }\StringTok{"Age"}\NormalTok{]}
\NormalTok{    writer }\OperatorTok{=}\NormalTok{ csv.DictWriter(f, fieldnames}\OperatorTok{=}\NormalTok{fieldnames)}
\NormalTok{    writer.writeheader()}
\NormalTok{    writer.writerow(\{}\StringTok{"Name"}\NormalTok{: }\StringTok{"Charlie"}\NormalTok{, }\StringTok{"Age"}\NormalTok{: }\DecValTok{35}\NormalTok{\})}

\CommentTok{\# Reading}
\ControlFlowTok{with} \BuiltInTok{open}\NormalTok{(}\StringTok{"people.csv"}\NormalTok{, }\StringTok{"r"}\NormalTok{) }\ImportTok{as}\NormalTok{ f:}
\NormalTok{    reader }\OperatorTok{=}\NormalTok{ csv.DictReader(f)}
    \ControlFlowTok{for}\NormalTok{ row }\KeywordTok{in}\NormalTok{ reader:}
        \BuiltInTok{print}\NormalTok{(row[}\StringTok{"Name"}\NormalTok{], row[}\StringTok{"Age"}\NormalTok{])}
\end{Highlighting}
\end{Shaded}

Quick Summary Table

\begin{longtable}[]{@{}ll@{}}
\toprule\noalign{}
Class/Function & Purpose \\
\midrule\noalign{}
\endhead
\bottomrule\noalign{}
\endlastfoot
\texttt{csv.reader} & Reads CSV into lists \\
\texttt{csv.writer} & Writes CSV from lists \\
\texttt{csv.DictReader} & Reads CSV into dictionaries \\
\texttt{csv.DictWriter} & Writes CSV from dictionaries \\
\end{longtable}

\subsubsection{Tiny Code}\label{tiny-code-56}

\begin{Shaded}
\begin{Highlighting}[]
\ImportTok{import}\NormalTok{ csv}

\ControlFlowTok{with} \BuiltInTok{open}\NormalTok{(}\StringTok{"scores.csv"}\NormalTok{, }\StringTok{"w"}\NormalTok{, newline}\OperatorTok{=}\StringTok{""}\NormalTok{) }\ImportTok{as}\NormalTok{ f:}
\NormalTok{    writer }\OperatorTok{=}\NormalTok{ csv.writer(f)}
\NormalTok{    writer.writerow([}\StringTok{"Name"}\NormalTok{, }\StringTok{"Score"}\NormalTok{])}
\NormalTok{    writer.writerow([}\StringTok{"Alice"}\NormalTok{, }\DecValTok{90}\NormalTok{])}
\NormalTok{    writer.writerow([}\StringTok{"Bob"}\NormalTok{, }\DecValTok{85}\NormalTok{])}

\ControlFlowTok{with} \BuiltInTok{open}\NormalTok{(}\StringTok{"scores.csv"}\NormalTok{, }\StringTok{"r"}\NormalTok{) }\ImportTok{as}\NormalTok{ f:}
\NormalTok{    reader }\OperatorTok{=}\NormalTok{ csv.reader(f)}
    \ControlFlowTok{for}\NormalTok{ row }\KeywordTok{in}\NormalTok{ reader:}
        \BuiltInTok{print}\NormalTok{(row)}
\end{Highlighting}
\end{Shaded}

\subsubsection{Why it Matters}\label{why-it-matters-56}

CSV is the most common format for sharing data between systems. By
mastering the \texttt{csv} module, you can process spreadsheets, export
reports, and integrate with databases or analytics tools.

\subsubsection{Try It Yourself}\label{try-it-yourself-56}

\begin{enumerate}
\def\labelenumi{\arabic{enumi}.}
\tightlist
\item
  Create a file \texttt{students.csv} with three rows
  (\texttt{Name,\ Age}).
\item
  Write Python code to read and print all rows.
\item
  Use \texttt{DictWriter} to add a new student to the file.
\item
  Use \texttt{DictReader} to print only the \texttt{Name} column.
\end{enumerate}

\subsection{58. Working with JSON Files}\label{working-with-json-files}

JSON (JavaScript Object Notation) is a lightweight data format often
used for APIs, configs, and data exchange. Python has a built-in
\texttt{json} module that makes it easy to read and write JSON files.

\subsubsection{Deep Dive}\label{deep-dive-57}

Importing the Module

\begin{Shaded}
\begin{Highlighting}[]
\ImportTok{import}\NormalTok{ json}
\end{Highlighting}
\end{Shaded}

Writing JSON to a File

\begin{Shaded}
\begin{Highlighting}[]
\ImportTok{import}\NormalTok{ json}

\NormalTok{data }\OperatorTok{=}\NormalTok{ \{}
    \StringTok{"name"}\NormalTok{: }\StringTok{"Alice"}\NormalTok{,}
    \StringTok{"age"}\NormalTok{: }\DecValTok{25}\NormalTok{,}
    \StringTok{"languages"}\NormalTok{: [}\StringTok{"Python"}\NormalTok{, }\StringTok{"JavaScript"}\NormalTok{]}
\NormalTok{\}}

\ControlFlowTok{with} \BuiltInTok{open}\NormalTok{(}\StringTok{"data.json"}\NormalTok{, }\StringTok{"w"}\NormalTok{) }\ImportTok{as}\NormalTok{ f:}
\NormalTok{    json.dump(data, f)}
\end{Highlighting}
\end{Shaded}

\begin{itemize}
\tightlist
\item
  \texttt{json.dump(obj,\ file)} → saves Python object as JSON.
\item
  Automatically converts dicts, lists, strings, numbers, booleans.
\end{itemize}

Reading JSON from a File

\begin{Shaded}
\begin{Highlighting}[]
\ControlFlowTok{with} \BuiltInTok{open}\NormalTok{(}\StringTok{"data.json"}\NormalTok{, }\StringTok{"r"}\NormalTok{) }\ImportTok{as}\NormalTok{ f:}
\NormalTok{    loaded }\OperatorTok{=}\NormalTok{ json.load(f)}

\BuiltInTok{print}\NormalTok{(loaded[}\StringTok{"name"}\NormalTok{])   }\CommentTok{\# Alice}
\BuiltInTok{print}\NormalTok{(loaded[}\StringTok{"languages"}\NormalTok{])  }\CommentTok{\# [\textquotesingle{}Python\textquotesingle{}, \textquotesingle{}JavaScript\textquotesingle{}]}
\end{Highlighting}
\end{Shaded}

Convert Between JSON and String

\begin{itemize}
\tightlist
\item
  \texttt{json.dumps(obj)} → convert Python object → JSON string.
\item
  \texttt{json.loads(str)} → convert JSON string → Python object.
\end{itemize}

\begin{Shaded}
\begin{Highlighting}[]
\NormalTok{s }\OperatorTok{=}\NormalTok{ json.dumps(data)}
\BuiltInTok{print}\NormalTok{(s)   }\CommentTok{\# \textquotesingle{}\{"name": "Alice", "age": 25, ...\}\textquotesingle{}}

\NormalTok{obj }\OperatorTok{=}\NormalTok{ json.loads(s)}
\BuiltInTok{print}\NormalTok{(obj[}\StringTok{"age"}\NormalTok{])   }\CommentTok{\# 25}
\end{Highlighting}
\end{Shaded}

Pretty Printing JSON

\begin{Shaded}
\begin{Highlighting}[]
\BuiltInTok{print}\NormalTok{(json.dumps(data, indent}\OperatorTok{=}\DecValTok{4}\NormalTok{))}
\end{Highlighting}
\end{Shaded}

Quick Summary Table

\begin{longtable}[]{@{}ll@{}}
\toprule\noalign{}
Function & Purpose \\
\midrule\noalign{}
\endhead
\bottomrule\noalign{}
\endlastfoot
\texttt{json.dump(obj,f)} & Write JSON to a file \\
\texttt{json.load(f)} & Read JSON from a file \\
\texttt{json.dumps(obj)} & Convert object to JSON string \\
\texttt{json.loads(str)} & Convert JSON string to Python object \\
\end{longtable}

\subsubsection{Tiny Code}\label{tiny-code-57}

\begin{Shaded}
\begin{Highlighting}[]
\ImportTok{import}\NormalTok{ json}

\NormalTok{user }\OperatorTok{=}\NormalTok{ \{}\StringTok{"id"}\NormalTok{: }\DecValTok{1}\NormalTok{, }\StringTok{"active"}\NormalTok{: }\VariableTok{True}\NormalTok{, }\StringTok{"roles"}\NormalTok{: [}\StringTok{"admin"}\NormalTok{, }\StringTok{"editor"}\NormalTok{]\}}

\ControlFlowTok{with} \BuiltInTok{open}\NormalTok{(}\StringTok{"user.json"}\NormalTok{, }\StringTok{"w"}\NormalTok{) }\ImportTok{as}\NormalTok{ f:}
\NormalTok{    json.dump(user, f, indent}\OperatorTok{=}\DecValTok{2}\NormalTok{)}

\ControlFlowTok{with} \BuiltInTok{open}\NormalTok{(}\StringTok{"user.json"}\NormalTok{, }\StringTok{"r"}\NormalTok{) }\ImportTok{as}\NormalTok{ f:}
    \BuiltInTok{print}\NormalTok{(json.load(f))}
\end{Highlighting}
\end{Shaded}

\subsubsection{Why it Matters}\label{why-it-matters-57}

JSON is the universal format for modern applications---from web APIs to
configuration files. By mastering Python's \texttt{json} module, you can
easily communicate with APIs, save structured data, and exchange
information with other systems.

\subsubsection{Try It Yourself}\label{try-it-yourself-57}

\begin{enumerate}
\def\labelenumi{\arabic{enumi}.}
\tightlist
\item
  Create a dictionary with your name, age, and hobbies, then save it to
  \texttt{me.json}.
\item
  Reopen \texttt{me.json} and print the hobbies.
\item
  Use \texttt{json.dumps()} to print the same dictionary as a formatted
  JSON string.
\item
  Convert a JSON string back into a Python dictionary using
  \texttt{json.loads()}.
\end{enumerate}

\subsection{59. File Exceptions}\label{file-exceptions}

When working with files, many things can go wrong: the file might not
exist, permissions might be missing, or the disk might be full. Python
uses exceptions to handle these errors safely.

\subsubsection{Deep Dive}\label{deep-dive-58}

Common File Exceptions

\begin{itemize}
\tightlist
\item
  \texttt{FileNotFoundError} → trying to open a non-existent file.
\item
  \texttt{PermissionError} → trying to open/write without permission.
\item
  \texttt{IsADirectoryError} → opening a directory instead of a file.
\item
  \texttt{IOError} / \texttt{OSError} → general input/output errors
  (disk, encoding).
\end{itemize}

Handling File Exceptions

\begin{Shaded}
\begin{Highlighting}[]
\ControlFlowTok{try}\NormalTok{:}
\NormalTok{    f }\OperatorTok{=} \BuiltInTok{open}\NormalTok{(}\StringTok{"missing.txt"}\NormalTok{, }\StringTok{"r"}\NormalTok{)}
\NormalTok{    content }\OperatorTok{=}\NormalTok{ f.read()}
\NormalTok{    f.close()}
\ControlFlowTok{except} \PreprocessorTok{FileNotFoundError}\NormalTok{:}
    \BuiltInTok{print}\NormalTok{(}\StringTok{"The file does not exist."}\NormalTok{)}
\end{Highlighting}
\end{Shaded}

Catching Multiple Exceptions

\begin{Shaded}
\begin{Highlighting}[]
\ControlFlowTok{try}\NormalTok{:}
\NormalTok{    f }\OperatorTok{=} \BuiltInTok{open}\NormalTok{(}\StringTok{"/protected/data.txt"}\NormalTok{, }\StringTok{"r"}\NormalTok{)}
\ControlFlowTok{except}\NormalTok{ (}\PreprocessorTok{FileNotFoundError}\NormalTok{, }\PreprocessorTok{PermissionError}\NormalTok{) }\ImportTok{as}\NormalTok{ e:}
    \BuiltInTok{print}\NormalTok{(}\StringTok{"Error:"}\NormalTok{, e)}
\end{Highlighting}
\end{Shaded}

Using \texttt{finally} for Cleanup

\begin{Shaded}
\begin{Highlighting}[]
\ControlFlowTok{try}\NormalTok{:}
\NormalTok{    f }\OperatorTok{=} \BuiltInTok{open}\NormalTok{(}\StringTok{"data.txt"}\NormalTok{, }\StringTok{"r"}\NormalTok{)}
    \BuiltInTok{print}\NormalTok{(f.read())}
\ControlFlowTok{finally}\NormalTok{:}
\NormalTok{    f.close()   }\CommentTok{\# ensures file closes even on error}
\end{Highlighting}
\end{Shaded}

Safer with \texttt{with} The \texttt{with} statement avoids many of
these issues automatically, but exceptions can still happen when
opening:

\begin{Shaded}
\begin{Highlighting}[]
\ControlFlowTok{try}\NormalTok{:}
    \ControlFlowTok{with} \BuiltInTok{open}\NormalTok{(}\StringTok{"notes.txt"}\NormalTok{, }\StringTok{"r"}\NormalTok{) }\ImportTok{as}\NormalTok{ f:}
        \BuiltInTok{print}\NormalTok{(f.read())}
\ControlFlowTok{except} \PreprocessorTok{FileNotFoundError}\NormalTok{:}
    \BuiltInTok{print}\NormalTok{(}\StringTok{"File not found!"}\NormalTok{)}
\end{Highlighting}
\end{Shaded}

Quick Summary Table

\begin{longtable}[]{@{}ll@{}}
\toprule\noalign{}
Exception & Cause \\
\midrule\noalign{}
\endhead
\bottomrule\noalign{}
\endlastfoot
\texttt{FileNotFoundError} & File does not exist \\
\texttt{PermissionError} & No permission to access file \\
\texttt{IsADirectoryError} & Tried to open a directory as a file \\
\texttt{IOError} / \texttt{OSError} & General input/output failure \\
\end{longtable}

\subsubsection{Tiny Code}\label{tiny-code-58}

\begin{Shaded}
\begin{Highlighting}[]
\NormalTok{filename }\OperatorTok{=} \StringTok{"example.txt"}

\ControlFlowTok{try}\NormalTok{:}
    \ControlFlowTok{with} \BuiltInTok{open}\NormalTok{(filename, }\StringTok{"r"}\NormalTok{) }\ImportTok{as}\NormalTok{ f:}
        \BuiltInTok{print}\NormalTok{(f.read())}
\ControlFlowTok{except} \PreprocessorTok{FileNotFoundError}\NormalTok{:}
    \BuiltInTok{print}\NormalTok{(}\SpecialStringTok{f"Error: }\SpecialCharTok{\{}\NormalTok{filename}\SpecialCharTok{\}}\SpecialStringTok{ was not found."}\NormalTok{)}
\end{Highlighting}
\end{Shaded}

\subsubsection{Why it Matters}\label{why-it-matters-58}

Errors in file handling are inevitable. Exception handling makes your
programs robust, user-friendly, and prevents crashes when dealing with
unpredictable files and systems.

\subsubsection{Try It Yourself}\label{try-it-yourself-58}

\begin{enumerate}
\def\labelenumi{\arabic{enumi}.}
\tightlist
\item
  Try opening a file that doesn't exist, catch the
  \texttt{FileNotFoundError}, and print a custom message.
\item
  Write code that catches both \texttt{FileNotFoundError} and
  \texttt{PermissionError}.
\item
  Use \texttt{finally} to always print \texttt{"Done"} after attempting
  to open a file.
\item
  Combine \texttt{with\ open()} and \texttt{try...except} to safely read
  a file only if it exists.
\end{enumerate}

\subsection{\texorpdfstring{60. Paths \& Directories (\texttt{os},
\texttt{pathlib})}{60. Paths \& Directories (os, pathlib)}}\label{paths-directories-os-pathlib}

Working with files often means dealing with paths and directories.
Python provides two main tools for this: the older \texttt{os} module
and the modern \texttt{pathlib} module.

\subsubsection{Deep Dive}\label{deep-dive-59}

Getting Current Working Directory

\begin{Shaded}
\begin{Highlighting}[]
\ImportTok{import}\NormalTok{ os}
\BuiltInTok{print}\NormalTok{(os.getcwd())   }\CommentTok{\# shows current directory}
\end{Highlighting}
\end{Shaded}

With \texttt{pathlib}:

\begin{Shaded}
\begin{Highlighting}[]
\ImportTok{from}\NormalTok{ pathlib }\ImportTok{import}\NormalTok{ Path}
\BuiltInTok{print}\NormalTok{(Path.cwd())}
\end{Highlighting}
\end{Shaded}

Changing Directory

\begin{Shaded}
\begin{Highlighting}[]
\NormalTok{os.chdir(}\StringTok{"/tmp"}\NormalTok{)}
\end{Highlighting}
\end{Shaded}

Listing Files in a Directory

\begin{Shaded}
\begin{Highlighting}[]
\BuiltInTok{print}\NormalTok{(os.listdir(}\StringTok{"."}\NormalTok{))   }\CommentTok{\# list all files/folders}
\end{Highlighting}
\end{Shaded}

With \texttt{pathlib}:

\begin{Shaded}
\begin{Highlighting}[]
\NormalTok{p }\OperatorTok{=}\NormalTok{ Path(}\StringTok{"."}\NormalTok{)}
\ControlFlowTok{for} \BuiltInTok{file} \KeywordTok{in}\NormalTok{ p.iterdir():}
    \BuiltInTok{print}\NormalTok{(}\BuiltInTok{file}\NormalTok{)}
\end{Highlighting}
\end{Shaded}

Joining Paths Instead of manually adding slashes, use:

\begin{Shaded}
\begin{Highlighting}[]
\NormalTok{os.path.join(}\StringTok{"folder"}\NormalTok{, }\StringTok{"file.txt"}\NormalTok{)   }\CommentTok{\# "folder/file.txt"}
\end{Highlighting}
\end{Shaded}

With \texttt{pathlib}:

\begin{Shaded}
\begin{Highlighting}[]
\NormalTok{Path(}\StringTok{"folder"}\NormalTok{) }\OperatorTok{/} \StringTok{"file.txt"}
\end{Highlighting}
\end{Shaded}

Checking File/Folder Existence

\begin{Shaded}
\begin{Highlighting}[]
\NormalTok{os.path.exists(}\StringTok{"notes.txt"}\NormalTok{)   }\CommentTok{\# True/False}
\end{Highlighting}
\end{Shaded}

With \texttt{pathlib}:

\begin{Shaded}
\begin{Highlighting}[]
\NormalTok{p }\OperatorTok{=}\NormalTok{ Path(}\StringTok{"notes.txt"}\NormalTok{)}
\BuiltInTok{print}\NormalTok{(p.exists())}
\BuiltInTok{print}\NormalTok{(p.is\_file())}
\BuiltInTok{print}\NormalTok{(p.is\_dir())}
\end{Highlighting}
\end{Shaded}

Creating Directories

\begin{Shaded}
\begin{Highlighting}[]
\NormalTok{os.mkdir(}\StringTok{"newfolder"}\NormalTok{)}
\end{Highlighting}
\end{Shaded}

With parents:

\begin{Shaded}
\begin{Highlighting}[]
\NormalTok{Path(}\StringTok{"a/b/c"}\NormalTok{).mkdir(parents}\OperatorTok{=}\VariableTok{True}\NormalTok{, exist\_ok}\OperatorTok{=}\VariableTok{True}\NormalTok{)}
\end{Highlighting}
\end{Shaded}

Removing Files and Folders

\begin{Shaded}
\begin{Highlighting}[]
\NormalTok{os.remove(}\StringTok{"file.txt"}\NormalTok{)      }\CommentTok{\# delete file}
\NormalTok{os.rmdir(}\StringTok{"empty\_folder"}\NormalTok{)   }\CommentTok{\# remove empty folder}
\end{Highlighting}
\end{Shaded}

With \texttt{pathlib}:

\begin{Shaded}
\begin{Highlighting}[]
\NormalTok{Path(}\StringTok{"file.txt"}\NormalTok{).unlink()}
\end{Highlighting}
\end{Shaded}

Quick Summary Table

\begin{longtable}[]{@{}lll@{}}
\toprule\noalign{}
Action & \texttt{os} Example & \texttt{pathlib} Example \\
\midrule\noalign{}
\endhead
\bottomrule\noalign{}
\endlastfoot
Current dir & \texttt{os.getcwd()} & \texttt{Path.cwd()} \\
List dir & \texttt{os.listdir(".")} & \texttt{Path(".").iterdir()} \\
Join paths & \texttt{os.path.join("a","b")} &
\texttt{Path("a")\ /\ "b"} \\
Exists? & \texttt{os.path.exists("f.txt")} &
\texttt{Path("f.txt").exists()} \\
Make dir & \texttt{os.mkdir("new")} & \texttt{Path("new").mkdir()} \\
Remove file & \texttt{os.remove("f.txt")} &
\texttt{Path("f.txt").unlink()} \\
\end{longtable}

\subsubsection{Tiny Code}\label{tiny-code-59}

\begin{Shaded}
\begin{Highlighting}[]
\ImportTok{from}\NormalTok{ pathlib }\ImportTok{import}\NormalTok{ Path}

\NormalTok{p }\OperatorTok{=}\NormalTok{ Path(}\StringTok{"demo\_folder"}\NormalTok{)}
\NormalTok{p.mkdir(exist\_ok}\OperatorTok{=}\VariableTok{True}\NormalTok{)}

\BuiltInTok{file} \OperatorTok{=}\NormalTok{ p }\OperatorTok{/} \StringTok{"hello.txt"}
\BuiltInTok{file}\NormalTok{.write\_text(}\StringTok{"Hello, pathlib!"}\NormalTok{)}

\BuiltInTok{print}\NormalTok{(}\BuiltInTok{file}\NormalTok{.read\_text())}
\end{Highlighting}
\end{Shaded}

\subsubsection{Why it Matters}\label{why-it-matters-59}

Paths and directories are essential for any project involving files.
\texttt{pathlib} provides a modern, object-oriented approach, while
\texttt{os} ensures backward compatibility with older code. Knowing both
makes you flexible.

\subsubsection{Try It Yourself}\label{try-it-yourself-59}

\begin{enumerate}
\def\labelenumi{\arabic{enumi}.}
\tightlist
\item
  Print your current working directory with both \texttt{os} and
  \texttt{pathlib}.
\item
  Create a folder called \texttt{projects} and inside it, a file
  \texttt{readme.txt} with some text.
\item
  List all files inside \texttt{projects}.
\item
  Write a script that checks if \texttt{archive/} exists, and if not,
  creates it.
\end{enumerate}

\section{Chapter 7. Object-Oriented
Python}\label{chapter-7.-object-oriented-python}

\subsection{61. Classes \& Objects}\label{classes-objects}

Python is an object-oriented programming (OOP) language. A class is like
a blueprint for creating objects, and an object is an instance of that
class. Classes define the structure (attributes) and behavior (methods)
of objects.

\subsubsection{Deep Dive}\label{deep-dive-60}

Defining a Class

\begin{Shaded}
\begin{Highlighting}[]
\KeywordTok{class}\NormalTok{ Person:}
    \ControlFlowTok{pass}
\end{Highlighting}
\end{Shaded}

This defines a new class called \texttt{Person}.

Creating an Object (Instance)

\begin{Shaded}
\begin{Highlighting}[]
\NormalTok{p1 }\OperatorTok{=}\NormalTok{ Person()}
\BuiltInTok{print}\NormalTok{(}\BuiltInTok{type}\NormalTok{(p1))   }\CommentTok{\# \textless{}class \textquotesingle{}\_\_main\_\_.Person\textquotesingle{}\textgreater{}}
\end{Highlighting}
\end{Shaded}

Here, \texttt{p1} is an object of type \texttt{Person}.

Adding Attributes

\begin{Shaded}
\begin{Highlighting}[]
\KeywordTok{class}\NormalTok{ Person:}
    \KeywordTok{def} \FunctionTok{\_\_init\_\_}\NormalTok{(}\VariableTok{self}\NormalTok{, name, age):}
        \VariableTok{self}\NormalTok{.name }\OperatorTok{=}\NormalTok{ name    }\CommentTok{\# attribute}
        \VariableTok{self}\NormalTok{.age }\OperatorTok{=}\NormalTok{ age}

\NormalTok{p1 }\OperatorTok{=}\NormalTok{ Person(}\StringTok{"Alice"}\NormalTok{, }\DecValTok{25}\NormalTok{)}
\BuiltInTok{print}\NormalTok{(p1.name, p1.age)   }\CommentTok{\# Alice 25}
\end{Highlighting}
\end{Shaded}

\begin{itemize}
\tightlist
\item
  \texttt{\_\_init\_\_} → constructor method, runs when creating an
  object.
\item
  \texttt{self} → refers to the current object.
\end{itemize}

Adding Methods

\begin{Shaded}
\begin{Highlighting}[]
\KeywordTok{class}\NormalTok{ Person:}
    \KeywordTok{def} \FunctionTok{\_\_init\_\_}\NormalTok{(}\VariableTok{self}\NormalTok{, name):}
        \VariableTok{self}\NormalTok{.name }\OperatorTok{=}\NormalTok{ name}

    \KeywordTok{def}\NormalTok{ greet(}\VariableTok{self}\NormalTok{):}
        \ControlFlowTok{return} \SpecialStringTok{f"Hello, my name is }\SpecialCharTok{\{}\VariableTok{self}\SpecialCharTok{.}\NormalTok{name}\SpecialCharTok{\}}\SpecialStringTok{."}

\NormalTok{p1 }\OperatorTok{=}\NormalTok{ Person(}\StringTok{"Bob"}\NormalTok{)}
\BuiltInTok{print}\NormalTok{(p1.greet())   }\CommentTok{\# Hello, my name is Bob.}
\end{Highlighting}
\end{Shaded}

A method is just a function inside a class that operates on its objects.

Quick Summary Table

\begin{longtable}[]{@{}
  >{\raggedright\arraybackslash}p{(\linewidth - 4\tabcolsep) * \real{0.1333}}
  >{\raggedright\arraybackslash}p{(\linewidth - 4\tabcolsep) * \real{0.5600}}
  >{\raggedright\arraybackslash}p{(\linewidth - 4\tabcolsep) * \real{0.3067}}@{}}
\toprule\noalign{}
\begin{minipage}[b]{\linewidth}\raggedright
Concept
\end{minipage} & \begin{minipage}[b]{\linewidth}\raggedright
Definition
\end{minipage} & \begin{minipage}[b]{\linewidth}\raggedright
Example
\end{minipage} \\
\midrule\noalign{}
\endhead
\bottomrule\noalign{}
\endlastfoot
Class & Blueprint for objects & \texttt{class\ Car:\ ...} \\
Object & Instance of a class & \texttt{c1\ =\ Car()} \\
Attributes & Data stored in objects & \texttt{self.name},
\texttt{self.age} \\
Methods & Functions inside a class & \texttt{def\ drive(self):\ ...} \\
\texttt{\_\_init\_\_} & Constructor, called when object is created &
\texttt{def\ \_\_init\_\_(...)} \\
\texttt{self} & Refers to the current instance &
\texttt{self.name\ =\ name} \\
\end{longtable}

\subsubsection{Tiny Code}\label{tiny-code-60}

\begin{Shaded}
\begin{Highlighting}[]
\KeywordTok{class}\NormalTok{ Dog:}
    \KeywordTok{def} \FunctionTok{\_\_init\_\_}\NormalTok{(}\VariableTok{self}\NormalTok{, name, breed):}
        \VariableTok{self}\NormalTok{.name }\OperatorTok{=}\NormalTok{ name}
        \VariableTok{self}\NormalTok{.breed }\OperatorTok{=}\NormalTok{ breed}
    
    \KeywordTok{def}\NormalTok{ bark(}\VariableTok{self}\NormalTok{):}
        \ControlFlowTok{return} \SpecialStringTok{f"}\SpecialCharTok{\{}\VariableTok{self}\SpecialCharTok{.}\NormalTok{name}\SpecialCharTok{\}}\SpecialStringTok{ says Woof!"}

\NormalTok{d1 }\OperatorTok{=}\NormalTok{ Dog(}\StringTok{"Max"}\NormalTok{, }\StringTok{"Labrador"}\NormalTok{)}
\BuiltInTok{print}\NormalTok{(d1.bark())}
\end{Highlighting}
\end{Shaded}

\subsubsection{Why it Matters}\label{why-it-matters-60}

Classes and objects are the foundation of OOP. They let you model
real-world things (like cars, users, or bank accounts) in code, organize
functionality, and build scalable applications.

\subsubsection{Try It Yourself}\label{try-it-yourself-60}

\begin{enumerate}
\def\labelenumi{\arabic{enumi}.}
\tightlist
\item
  Create a \texttt{Car} class with attributes \texttt{brand} and
  \texttt{year}.
\item
  Add a method \texttt{drive()} that prints
  \texttt{"The\ car\ is\ driving"}.
\item
  Make two different \texttt{Car} objects and call their
  \texttt{drive()} method.
\item
  Add another method that prints the car's brand and year.
\end{enumerate}

\subsection{62. Attributes \& Methods}\label{attributes-methods}

In Python classes, attributes are variables that belong to objects, and
methods are functions that belong to objects. Together, they define what
an object has (data) and what it does (behavior).

\subsubsection{Deep Dive}\label{deep-dive-61}

Attributes (Object Data) Attributes store information about an object.

\begin{Shaded}
\begin{Highlighting}[]
\KeywordTok{class}\NormalTok{ Car:}
    \KeywordTok{def} \FunctionTok{\_\_init\_\_}\NormalTok{(}\VariableTok{self}\NormalTok{, brand, year):}
        \VariableTok{self}\NormalTok{.brand }\OperatorTok{=}\NormalTok{ brand}
        \VariableTok{self}\NormalTok{.year }\OperatorTok{=}\NormalTok{ year}

\NormalTok{c1 }\OperatorTok{=}\NormalTok{ Car(}\StringTok{"Toyota"}\NormalTok{, }\DecValTok{2020}\NormalTok{)}
\BuiltInTok{print}\NormalTok{(c1.brand)   }\CommentTok{\# Toyota}
\BuiltInTok{print}\NormalTok{(c1.year)    }\CommentTok{\# 2020}
\end{Highlighting}
\end{Shaded}

Here, \texttt{brand} and \texttt{year} are attributes of the
\texttt{Car} object.

Instance Methods (Object Behavior) Methods define actions an object can
perform.

\begin{Shaded}
\begin{Highlighting}[]
\KeywordTok{class}\NormalTok{ Car:}
    \KeywordTok{def} \FunctionTok{\_\_init\_\_}\NormalTok{(}\VariableTok{self}\NormalTok{, brand, year):}
        \VariableTok{self}\NormalTok{.brand }\OperatorTok{=}\NormalTok{ brand}
        \VariableTok{self}\NormalTok{.year }\OperatorTok{=}\NormalTok{ year}
    
    \KeywordTok{def}\NormalTok{ drive(}\VariableTok{self}\NormalTok{):}
        \ControlFlowTok{return} \SpecialStringTok{f"}\SpecialCharTok{\{}\VariableTok{self}\SpecialCharTok{.}\NormalTok{brand}\SpecialCharTok{\}}\SpecialStringTok{ is driving."}

\NormalTok{c1 }\OperatorTok{=}\NormalTok{ Car(}\StringTok{"Honda"}\NormalTok{, }\DecValTok{2019}\NormalTok{)}
\BuiltInTok{print}\NormalTok{(c1.drive())   }\CommentTok{\# Honda is driving.}
\end{Highlighting}
\end{Shaded}

\begin{itemize}
\tightlist
\item
  \texttt{self} allows the method to access the object's attributes.
\end{itemize}

Updating Attributes Attributes can be changed dynamically:

\begin{Shaded}
\begin{Highlighting}[]
\NormalTok{c1.year }\OperatorTok{=} \DecValTok{2022}
\BuiltInTok{print}\NormalTok{(c1.year)   }\CommentTok{\# 2022}
\end{Highlighting}
\end{Shaded}

Adding New Attributes at Runtime

\begin{Shaded}
\begin{Highlighting}[]
\NormalTok{c1.color }\OperatorTok{=} \StringTok{"red"}
\BuiltInTok{print}\NormalTok{(c1.color)   }\CommentTok{\# red}
\end{Highlighting}
\end{Shaded}

(But it's better to define attributes in \texttt{\_\_init\_\_} for
consistency.)

Class Attributes vs Instance Attributes

\begin{itemize}
\tightlist
\item
  Instance attribute → unique to each object.
\item
  Class attribute → shared by all objects of the class.
\end{itemize}

\begin{Shaded}
\begin{Highlighting}[]
\KeywordTok{class}\NormalTok{ Dog:}
\NormalTok{    species }\OperatorTok{=} \StringTok{"Canis lupus familiaris"}   \CommentTok{\# class attribute}
    \KeywordTok{def} \FunctionTok{\_\_init\_\_}\NormalTok{(}\VariableTok{self}\NormalTok{, name):}
        \VariableTok{self}\NormalTok{.name }\OperatorTok{=}\NormalTok{ name                 }\CommentTok{\# instance attribute}

\NormalTok{d1 }\OperatorTok{=}\NormalTok{ Dog(}\StringTok{"Buddy"}\NormalTok{)}
\NormalTok{d2 }\OperatorTok{=}\NormalTok{ Dog(}\StringTok{"Charlie"}\NormalTok{)}
\BuiltInTok{print}\NormalTok{(d1.species, d2.species)   }\CommentTok{\# same for all}
\BuiltInTok{print}\NormalTok{(d1.name, d2.name)         }\CommentTok{\# unique per dog}
\end{Highlighting}
\end{Shaded}

Quick Summary Table

\begin{longtable}[]{@{}
  >{\raggedright\arraybackslash}p{(\linewidth - 4\tabcolsep) * \real{0.2250}}
  >{\raggedright\arraybackslash}p{(\linewidth - 4\tabcolsep) * \real{0.4625}}
  >{\raggedright\arraybackslash}p{(\linewidth - 4\tabcolsep) * \real{0.3125}}@{}}
\toprule\noalign{}
\begin{minipage}[b]{\linewidth}\raggedright
Term
\end{minipage} & \begin{minipage}[b]{\linewidth}\raggedright
Meaning
\end{minipage} & \begin{minipage}[b]{\linewidth}\raggedright
Example
\end{minipage} \\
\midrule\noalign{}
\endhead
\bottomrule\noalign{}
\endlastfoot
Instance attribute & Data unique to each object & \texttt{self.brand},
\texttt{self.year} \\
Class attribute & Shared across all objects &
\texttt{species\ =\ ...} \\
Method & Function inside a class & \texttt{def\ drive(self)} \\
\texttt{self} & Refers to the current object instance &
\texttt{self.name\ =\ name} \\
\end{longtable}

\subsubsection{Tiny Code}\label{tiny-code-61}

\begin{Shaded}
\begin{Highlighting}[]
\KeywordTok{class}\NormalTok{ Student:}
\NormalTok{    school }\OperatorTok{=} \StringTok{"Python Academy"}   \CommentTok{\# class attribute}
    
    \KeywordTok{def} \FunctionTok{\_\_init\_\_}\NormalTok{(}\VariableTok{self}\NormalTok{, name, grade):}
        \VariableTok{self}\NormalTok{.name }\OperatorTok{=}\NormalTok{ name}
        \VariableTok{self}\NormalTok{.grade }\OperatorTok{=}\NormalTok{ grade      }\CommentTok{\# instance attribute}
    
    \KeywordTok{def}\NormalTok{ introduce(}\VariableTok{self}\NormalTok{):}
        \ControlFlowTok{return} \SpecialStringTok{f"I am }\SpecialCharTok{\{}\VariableTok{self}\SpecialCharTok{.}\NormalTok{name}\SpecialCharTok{\}}\SpecialStringTok{, grade }\SpecialCharTok{\{}\VariableTok{self}\SpecialCharTok{.}\NormalTok{grade}\SpecialCharTok{\}}\SpecialStringTok{."}

\NormalTok{s1 }\OperatorTok{=}\NormalTok{ Student(}\StringTok{"Alice"}\NormalTok{, }\StringTok{"A"}\NormalTok{)}
\NormalTok{s2 }\OperatorTok{=}\NormalTok{ Student(}\StringTok{"Bob"}\NormalTok{, }\StringTok{"B"}\NormalTok{)}

\BuiltInTok{print}\NormalTok{(s1.introduce())}
\BuiltInTok{print}\NormalTok{(s2.introduce())}
\BuiltInTok{print}\NormalTok{(}\StringTok{"School:"}\NormalTok{, s1.school)}
\end{Highlighting}
\end{Shaded}

\subsubsection{Why it Matters}\label{why-it-matters-61}

Attributes and methods are the building blocks of object-oriented
programming. Attributes give objects state, while methods give them
behavior. Together, they let you model real-world entities in code.

\subsubsection{Try It Yourself}\label{try-it-yourself-61}

\begin{enumerate}
\def\labelenumi{\arabic{enumi}.}
\tightlist
\item
  Define a \texttt{Book} class with attributes \texttt{title} and
  \texttt{author}.
\item
  Add a method \texttt{describe()} that prints
  \texttt{"Title\ by\ Author"}.
\item
  Create two \texttt{Book} objects with different details and call
  \texttt{describe()} on both.
\item
  Add a class attribute \texttt{library\ =\ "City\ Library"} and print
  it from both objects.
\end{enumerate}

\subsection{\texorpdfstring{63. \texttt{\_\_init\_\_}
Constructor}{63. \_\_init\_\_ Constructor}}\label{init__-constructor}

In Python, the \texttt{\_\_init\_\_} method is a special method that
runs automatically when you create a new object. It's often called the
constructor because it initializes (sets up) the object's attributes.

\subsubsection{Deep Dive}\label{deep-dive-62}

Basic Example

\begin{Shaded}
\begin{Highlighting}[]
\KeywordTok{class}\NormalTok{ Person:}
    \KeywordTok{def} \FunctionTok{\_\_init\_\_}\NormalTok{(}\VariableTok{self}\NormalTok{, name, age):}
        \VariableTok{self}\NormalTok{.name }\OperatorTok{=}\NormalTok{ name}
        \VariableTok{self}\NormalTok{.age }\OperatorTok{=}\NormalTok{ age}

\NormalTok{p1 }\OperatorTok{=}\NormalTok{ Person(}\StringTok{"Alice"}\NormalTok{, }\DecValTok{25}\NormalTok{)}
\BuiltInTok{print}\NormalTok{(p1.name, p1.age)   }\CommentTok{\# Alice 25}
\end{Highlighting}
\end{Shaded}

\begin{itemize}
\tightlist
\item
  \texttt{\_\_init\_\_} is called right after an object is created.
\item
  \texttt{self} refers to the new object being initialized.
\end{itemize}

Default Values You can give parameters default values:

\begin{Shaded}
\begin{Highlighting}[]
\KeywordTok{class}\NormalTok{ Person:}
    \KeywordTok{def} \FunctionTok{\_\_init\_\_}\NormalTok{(}\VariableTok{self}\NormalTok{, name}\OperatorTok{=}\StringTok{"Unknown"}\NormalTok{, age}\OperatorTok{=}\DecValTok{0}\NormalTok{):}
        \VariableTok{self}\NormalTok{.name }\OperatorTok{=}\NormalTok{ name}
        \VariableTok{self}\NormalTok{.age }\OperatorTok{=}\NormalTok{ age}

\NormalTok{p1 }\OperatorTok{=}\NormalTok{ Person()}
\BuiltInTok{print}\NormalTok{(p1.name, p1.age)   }\CommentTok{\# Unknown 0}
\end{Highlighting}
\end{Shaded}

Constructor with Logic You can add checks or calculations during
initialization:

\begin{Shaded}
\begin{Highlighting}[]
\KeywordTok{class}\NormalTok{ Rectangle:}
    \KeywordTok{def} \FunctionTok{\_\_init\_\_}\NormalTok{(}\VariableTok{self}\NormalTok{, width, height):}
        \VariableTok{self}\NormalTok{.width }\OperatorTok{=}\NormalTok{ width}
        \VariableTok{self}\NormalTok{.height }\OperatorTok{=}\NormalTok{ height}
        \VariableTok{self}\NormalTok{.area }\OperatorTok{=}\NormalTok{ width }\OperatorTok{*}\NormalTok{ height   }\CommentTok{\# auto{-}calculate}

\NormalTok{r }\OperatorTok{=}\NormalTok{ Rectangle(}\DecValTok{4}\NormalTok{, }\DecValTok{5}\NormalTok{)}
\BuiltInTok{print}\NormalTok{(r.area)   }\CommentTok{\# 20}
\end{Highlighting}
\end{Shaded}

Multiple Objects, Independent Attributes Each object gets its own copy
of instance attributes:

\begin{Shaded}
\begin{Highlighting}[]
\NormalTok{p1 }\OperatorTok{=}\NormalTok{ Person(}\StringTok{"Alice"}\NormalTok{, }\DecValTok{25}\NormalTok{)}
\NormalTok{p2 }\OperatorTok{=}\NormalTok{ Person(}\StringTok{"Bob"}\NormalTok{, }\DecValTok{30}\NormalTok{)}

\BuiltInTok{print}\NormalTok{(p1.name)   }\CommentTok{\# Alice}
\BuiltInTok{print}\NormalTok{(p2.name)   }\CommentTok{\# Bob}
\end{Highlighting}
\end{Shaded}

Quick Summary Table

\begin{longtable}[]{@{}
  >{\raggedright\arraybackslash}p{(\linewidth - 4\tabcolsep) * \real{0.1970}}
  >{\raggedright\arraybackslash}p{(\linewidth - 4\tabcolsep) * \real{0.3939}}
  >{\raggedright\arraybackslash}p{(\linewidth - 4\tabcolsep) * \real{0.4091}}@{}}
\toprule\noalign{}
\begin{minipage}[b]{\linewidth}\raggedright
Feature
\end{minipage} & \begin{minipage}[b]{\linewidth}\raggedright
Example
\end{minipage} & \begin{minipage}[b]{\linewidth}\raggedright
Purpose
\end{minipage} \\
\midrule\noalign{}
\endhead
\bottomrule\noalign{}
\endlastfoot
Define init & \texttt{def\ \_\_init\_\_(self,\ ...):} & Runs on object
creation \\
Assign values & \texttt{self.attr\ =\ value} & Stores attributes in
object \\
Defaults & \texttt{def\ \_\_init\_\_(self,\ x=0)} & Optional
parameters \\
With logic & Compute or validate values & Setup object cleanly \\
\end{longtable}

\subsubsection{Tiny Code}\label{tiny-code-62}

\begin{Shaded}
\begin{Highlighting}[]
\KeywordTok{class}\NormalTok{ Dog:}
    \KeywordTok{def} \FunctionTok{\_\_init\_\_}\NormalTok{(}\VariableTok{self}\NormalTok{, name, breed}\OperatorTok{=}\StringTok{"Unknown"}\NormalTok{):}
        \VariableTok{self}\NormalTok{.name }\OperatorTok{=}\NormalTok{ name}
        \VariableTok{self}\NormalTok{.breed }\OperatorTok{=}\NormalTok{ breed}

\NormalTok{d1 }\OperatorTok{=}\NormalTok{ Dog(}\StringTok{"Max"}\NormalTok{, }\StringTok{"Beagle"}\NormalTok{)}
\NormalTok{d2 }\OperatorTok{=}\NormalTok{ Dog(}\StringTok{"Charlie"}\NormalTok{)}

\BuiltInTok{print}\NormalTok{(d1.name, d1.breed)}
\BuiltInTok{print}\NormalTok{(d2.name, d2.breed)}
\end{Highlighting}
\end{Shaded}

\subsubsection{Why it Matters}\label{why-it-matters-62}

The \texttt{\_\_init\_\_} constructor ensures every object starts in a
well-defined state. Without it, you'd have to manually assign attributes
after creating objects, which is error-prone and messy.

\subsubsection{Try It Yourself}\label{try-it-yourself-62}

\begin{enumerate}
\def\labelenumi{\arabic{enumi}.}
\tightlist
\item
  Create a \texttt{Car} class with attributes \texttt{brand},
  \texttt{model}, and \texttt{year} set in \texttt{\_\_init\_\_}.
\item
  Add a method \texttt{info()} that prints
  \texttt{"Brand\ Model\ (Year)"}.
\item
  Give \texttt{year} a default value if not provided.
\item
  Create two \texttt{Car} objects---one with all values, one with just
  brand and model---and call \texttt{info()} on both.
\end{enumerate}

\subsection{64. Instance vs Class
Variables}\label{instance-vs-class-variables}

In Python classes, variables can belong either to a specific object
(instance variables) or to the class itself (class variables). Knowing
the difference is key to writing predictable, reusable code.

\subsubsection{Deep Dive}\label{deep-dive-63}

Instance Variables

\begin{itemize}
\tightlist
\item
  Defined inside \texttt{\_\_init\_\_} using \texttt{self}.
\item
  Each object gets its own copy.
\end{itemize}

\begin{Shaded}
\begin{Highlighting}[]
\KeywordTok{class}\NormalTok{ Dog:}
    \KeywordTok{def} \FunctionTok{\_\_init\_\_}\NormalTok{(}\VariableTok{self}\NormalTok{, name):}
        \VariableTok{self}\NormalTok{.name }\OperatorTok{=}\NormalTok{ name    }\CommentTok{\# instance variable}

\NormalTok{d1 }\OperatorTok{=}\NormalTok{ Dog(}\StringTok{"Buddy"}\NormalTok{)}
\NormalTok{d2 }\OperatorTok{=}\NormalTok{ Dog(}\StringTok{"Charlie"}\NormalTok{)}

\BuiltInTok{print}\NormalTok{(d1.name)   }\CommentTok{\# Buddy}
\BuiltInTok{print}\NormalTok{(d2.name)   }\CommentTok{\# Charlie}
\end{Highlighting}
\end{Shaded}

Each dog has its own \texttt{name}.

Class Variables

\begin{itemize}
\tightlist
\item
  Shared across all objects of the class.
\item
  Defined directly inside the class, outside methods.
\end{itemize}

\begin{Shaded}
\begin{Highlighting}[]
\KeywordTok{class}\NormalTok{ Dog:}
\NormalTok{    species }\OperatorTok{=} \StringTok{"Canis lupus familiaris"}   \CommentTok{\# class variable}

    \KeywordTok{def} \FunctionTok{\_\_init\_\_}\NormalTok{(}\VariableTok{self}\NormalTok{, name):}
        \VariableTok{self}\NormalTok{.name }\OperatorTok{=}\NormalTok{ name}

\NormalTok{d1 }\OperatorTok{=}\NormalTok{ Dog(}\StringTok{"Buddy"}\NormalTok{)}
\NormalTok{d2 }\OperatorTok{=}\NormalTok{ Dog(}\StringTok{"Charlie"}\NormalTok{)}

\BuiltInTok{print}\NormalTok{(d1.species)   }\CommentTok{\# Canis lupus familiaris}
\BuiltInTok{print}\NormalTok{(d2.species)   }\CommentTok{\# Canis lupus familiaris}
\end{Highlighting}
\end{Shaded}

Changing it affects all instances:

\begin{Shaded}
\begin{Highlighting}[]
\NormalTok{Dog.species }\OperatorTok{=} \StringTok{"Dog"}
\BuiltInTok{print}\NormalTok{(d1.species, d2.species)  }\CommentTok{\# Dog Dog}
\end{Highlighting}
\end{Shaded}

Overriding Class Variables per Instance You can assign a new value to a
class variable on a specific object, but then it becomes an instance
variable for that object only:

\begin{Shaded}
\begin{Highlighting}[]
\NormalTok{d1.species }\OperatorTok{=} \StringTok{"Wolf"}   \CommentTok{\# overrides for d1 only}
\BuiltInTok{print}\NormalTok{(d1.species)     }\CommentTok{\# Wolf}
\BuiltInTok{print}\NormalTok{(d2.species)     }\CommentTok{\# Dog}
\end{Highlighting}
\end{Shaded}

Quick Summary Table

\begin{longtable}[]{@{}
  >{\raggedright\arraybackslash}p{(\linewidth - 6\tabcolsep) * \real{0.1857}}
  >{\raggedright\arraybackslash}p{(\linewidth - 6\tabcolsep) * \real{0.4000}}
  >{\raggedright\arraybackslash}p{(\linewidth - 6\tabcolsep) * \real{0.1571}}
  >{\raggedright\arraybackslash}p{(\linewidth - 6\tabcolsep) * \real{0.2571}}@{}}
\toprule\noalign{}
\begin{minipage}[b]{\linewidth}\raggedright
Variable Type
\end{minipage} & \begin{minipage}[b]{\linewidth}\raggedright
Defined Where
\end{minipage} & \begin{minipage}[b]{\linewidth}\raggedright
Belongs To
\end{minipage} & \begin{minipage}[b]{\linewidth}\raggedright
Example
\end{minipage} \\
\midrule\noalign{}
\endhead
\bottomrule\noalign{}
\endlastfoot
Instance & Inside \texttt{\_\_init\_\_} via \texttt{self} & Each object
& \texttt{self.name\ =\ name} \\
Class & Inside class body & The class & \texttt{species\ =\ "Dog"} \\
\end{longtable}

\subsubsection{Tiny Code}\label{tiny-code-63}

\begin{Shaded}
\begin{Highlighting}[]
\KeywordTok{class}\NormalTok{ Student:}
\NormalTok{    school }\OperatorTok{=} \StringTok{"Python Academy"}   \CommentTok{\# class variable}
    
    \KeywordTok{def} \FunctionTok{\_\_init\_\_}\NormalTok{(}\VariableTok{self}\NormalTok{, name):}
        \VariableTok{self}\NormalTok{.name }\OperatorTok{=}\NormalTok{ name        }\CommentTok{\# instance variable}

\NormalTok{s1 }\OperatorTok{=}\NormalTok{ Student(}\StringTok{"Alice"}\NormalTok{)}
\NormalTok{s2 }\OperatorTok{=}\NormalTok{ Student(}\StringTok{"Bob"}\NormalTok{)}

\BuiltInTok{print}\NormalTok{(s1.name, }\StringTok{"{-}"}\NormalTok{, s1.school)}
\BuiltInTok{print}\NormalTok{(s2.name, }\StringTok{"{-}"}\NormalTok{, s2.school)}

\NormalTok{Student.school }\OperatorTok{=} \StringTok{"Code Academy"}
\BuiltInTok{print}\NormalTok{(s1.school, s2.school)}
\end{Highlighting}
\end{Shaded}

\subsubsection{Why it Matters}\label{why-it-matters-63}

\begin{itemize}
\tightlist
\item
  Use instance variables for data unique to each object.
\item
  Use class variables for properties shared across all objects. Mixing
  them up can cause bugs, so it's important to understand the
  difference.
\end{itemize}

\subsubsection{Try It Yourself}\label{try-it-yourself-63}

\begin{enumerate}
\def\labelenumi{\arabic{enumi}.}
\tightlist
\item
  Create a \texttt{Car} class with a class variable
  \texttt{wheels\ =\ 4}.
\item
  Add an instance variable \texttt{brand} inside \texttt{\_\_init\_\_}.
\item
  Make two cars with different brands, and confirm they both show 4
  wheels.
\item
  Change \texttt{Car.wheels\ =\ 6} and check how it affects both
  objects.
\end{enumerate}

\subsection{65. Inheritance Basics}\label{inheritance-basics}

Inheritance allows one class to take on the attributes and methods of
another. This promotes code reuse and models real-world relationships
(e.g., a \texttt{Dog} is an \texttt{Animal}).

\subsubsection{Deep Dive}\label{deep-dive-64}

Parent and Child Classes

\begin{Shaded}
\begin{Highlighting}[]
\KeywordTok{class}\NormalTok{ Animal:}
    \KeywordTok{def} \FunctionTok{\_\_init\_\_}\NormalTok{(}\VariableTok{self}\NormalTok{, name):}
        \VariableTok{self}\NormalTok{.name }\OperatorTok{=}\NormalTok{ name}
    
    \KeywordTok{def}\NormalTok{ speak(}\VariableTok{self}\NormalTok{):}
        \ControlFlowTok{return} \SpecialStringTok{f"}\SpecialCharTok{\{}\VariableTok{self}\SpecialCharTok{.}\NormalTok{name}\SpecialCharTok{\}}\SpecialStringTok{ makes a sound."}

\KeywordTok{class}\NormalTok{ Dog(Animal):   }\CommentTok{\# Dog inherits from Animal}
    \KeywordTok{def}\NormalTok{ bark(}\VariableTok{self}\NormalTok{):}
        \ControlFlowTok{return} \SpecialStringTok{f"}\SpecialCharTok{\{}\VariableTok{self}\SpecialCharTok{.}\NormalTok{name}\SpecialCharTok{\}}\SpecialStringTok{ says Woof!"}
\end{Highlighting}
\end{Shaded}

\begin{Shaded}
\begin{Highlighting}[]
\NormalTok{a }\OperatorTok{=}\NormalTok{ Animal(}\StringTok{"Generic"}\NormalTok{)}
\BuiltInTok{print}\NormalTok{(a.speak())     }\CommentTok{\# Generic makes a sound.}

\NormalTok{d }\OperatorTok{=}\NormalTok{ Dog(}\StringTok{"Buddy"}\NormalTok{)}
\BuiltInTok{print}\NormalTok{(d.speak())     }\CommentTok{\# Buddy makes a sound. (inherited)}
\BuiltInTok{print}\NormalTok{(d.bark())      }\CommentTok{\# Buddy says Woof! (own method)}
\end{Highlighting}
\end{Shaded}

The \texttt{super()} Function \texttt{super()} lets the child class call
methods from the parent class.

\begin{Shaded}
\begin{Highlighting}[]
\KeywordTok{class}\NormalTok{ Animal:}
    \KeywordTok{def} \FunctionTok{\_\_init\_\_}\NormalTok{(}\VariableTok{self}\NormalTok{, name):}
        \VariableTok{self}\NormalTok{.name }\OperatorTok{=}\NormalTok{ name}

\KeywordTok{class}\NormalTok{ Cat(Animal):}
    \KeywordTok{def} \FunctionTok{\_\_init\_\_}\NormalTok{(}\VariableTok{self}\NormalTok{, name, color):}
        \BuiltInTok{super}\NormalTok{().}\FunctionTok{\_\_init\_\_}\NormalTok{(name)   }\CommentTok{\# call parent constructor}
        \VariableTok{self}\NormalTok{.color }\OperatorTok{=}\NormalTok{ color}
\end{Highlighting}
\end{Shaded}

\begin{Shaded}
\begin{Highlighting}[]
\NormalTok{c }\OperatorTok{=}\NormalTok{ Cat(}\StringTok{"Luna"}\NormalTok{, }\StringTok{"Gray"}\NormalTok{)}
\BuiltInTok{print}\NormalTok{(c.name, c.color)   }\CommentTok{\# Luna Gray}
\end{Highlighting}
\end{Shaded}

Overriding Methods A child can redefine methods from the parent:

\begin{Shaded}
\begin{Highlighting}[]
\KeywordTok{class}\NormalTok{ Animal:}
    \KeywordTok{def}\NormalTok{ speak(}\VariableTok{self}\NormalTok{):}
        \ControlFlowTok{return} \StringTok{"Some sound"}

\KeywordTok{class}\NormalTok{ Dog(Animal):}
    \KeywordTok{def}\NormalTok{ speak(}\VariableTok{self}\NormalTok{):}
        \ControlFlowTok{return} \StringTok{"Woof!"}

\BuiltInTok{print}\NormalTok{(Dog().speak())   }\CommentTok{\# Woof!}
\end{Highlighting}
\end{Shaded}

Inheritance Hierarchy

\begin{itemize}
\tightlist
\item
  A class can inherit from another class.
\item
  You can create chains (e.g., \texttt{A\ →\ B\ →\ C}).
\item
  Python supports multiple inheritance (covered later).
\end{itemize}

Quick Summary Table

\begin{longtable}[]{@{}
  >{\raggedright\arraybackslash}p{(\linewidth - 4\tabcolsep) * \real{0.1622}}
  >{\raggedright\arraybackslash}p{(\linewidth - 4\tabcolsep) * \real{0.5270}}
  >{\raggedright\arraybackslash}p{(\linewidth - 4\tabcolsep) * \real{0.3108}}@{}}
\toprule\noalign{}
\begin{minipage}[b]{\linewidth}\raggedright
Concept
\end{minipage} & \begin{minipage}[b]{\linewidth}\raggedright
Meaning
\end{minipage} & \begin{minipage}[b]{\linewidth}\raggedright
Example
\end{minipage} \\
\midrule\noalign{}
\endhead
\bottomrule\noalign{}
\endlastfoot
Parent class & Base class being inherited from &
\texttt{class\ Animal:} \\
Child class & Derived class that inherits from parent &
\texttt{class\ Dog(Animal):} \\
Inheritance & Child gets parent's attributes/methods & \texttt{Dog} uses
\texttt{speak()} \\
\texttt{super()} & Call parent methods inside child &
\texttt{super().\_\_init\_\_(...)} \\
Overriding & Redefining a parent method in the child &
\texttt{def\ speak(self):\ ...} \\
\end{longtable}

\subsubsection{Tiny Code}\label{tiny-code-64}

\begin{Shaded}
\begin{Highlighting}[]
\KeywordTok{class}\NormalTok{ Vehicle:}
    \KeywordTok{def} \FunctionTok{\_\_init\_\_}\NormalTok{(}\VariableTok{self}\NormalTok{, brand):}
        \VariableTok{self}\NormalTok{.brand }\OperatorTok{=}\NormalTok{ brand}
    \KeywordTok{def}\NormalTok{ drive(}\VariableTok{self}\NormalTok{):}
        \ControlFlowTok{return} \SpecialStringTok{f"}\SpecialCharTok{\{}\VariableTok{self}\SpecialCharTok{.}\NormalTok{brand}\SpecialCharTok{\}}\SpecialStringTok{ is moving."}

\KeywordTok{class}\NormalTok{ Car(Vehicle):}
    \KeywordTok{def}\NormalTok{ drive(}\VariableTok{self}\NormalTok{):}
        \ControlFlowTok{return} \SpecialStringTok{f"}\SpecialCharTok{\{}\VariableTok{self}\SpecialCharTok{.}\NormalTok{brand}\SpecialCharTok{\}}\SpecialStringTok{ is driving on the road."}

\NormalTok{v }\OperatorTok{=}\NormalTok{ Vehicle(}\StringTok{"Generic Vehicle"}\NormalTok{)}
\NormalTok{c }\OperatorTok{=}\NormalTok{ Car(}\StringTok{"Toyota"}\NormalTok{)}

\BuiltInTok{print}\NormalTok{(v.drive())}
\BuiltInTok{print}\NormalTok{(c.drive())}
\end{Highlighting}
\end{Shaded}

\subsubsection{Why it Matters}\label{why-it-matters-64}

Inheritance reduces duplication and makes code more organized. By
building hierarchies, you can model relationships between classes
naturally, reusing and extending existing functionality.

\subsubsection{Try It Yourself}\label{try-it-yourself-64}

\begin{enumerate}
\def\labelenumi{\arabic{enumi}.}
\tightlist
\item
  Create a base class \texttt{Shape} with a method \texttt{area()} that
  returns 0.
\item
  Make a child class \texttt{Circle} that overrides \texttt{area()} to
  compute \texttt{πr²}.
\item
  Create a class \texttt{Square} that overrides \texttt{area()} to
  compute \texttt{side²}.
\item
  Use \texttt{super().\_\_init\_\_()} to pass shared attributes from
  parent to child.
\end{enumerate}

\subsection{66. Method Overriding}\label{method-overriding}

Method overriding happens when a child class defines a method with the
same name as one in its parent class. The child's version replaces
(overrides) the parent's when called on a child object.

\subsubsection{Deep Dive}\label{deep-dive-65}

Basic Example

\begin{Shaded}
\begin{Highlighting}[]
\KeywordTok{class}\NormalTok{ Animal:}
    \KeywordTok{def}\NormalTok{ speak(}\VariableTok{self}\NormalTok{):}
        \ControlFlowTok{return} \StringTok{"Some generic sound"}

\KeywordTok{class}\NormalTok{ Dog(Animal):}
    \KeywordTok{def}\NormalTok{ speak(}\VariableTok{self}\NormalTok{):   }\CommentTok{\# overrides parent method}
        \ControlFlowTok{return} \StringTok{"Woof!"}

\NormalTok{a }\OperatorTok{=}\NormalTok{ Animal()}
\NormalTok{d }\OperatorTok{=}\NormalTok{ Dog()}

\BuiltInTok{print}\NormalTok{(a.speak())   }\CommentTok{\# Some generic sound}
\BuiltInTok{print}\NormalTok{(d.speak())   }\CommentTok{\# Woof!}
\end{Highlighting}
\end{Shaded}

Why Override?

\begin{itemize}
\tightlist
\item
  To provide specialized behavior in a child class.
\item
  Keeps shared structure in the parent but allows customization.
\end{itemize}

Using \texttt{super()} with Overrides You can call the parent's version
inside the override:

\begin{Shaded}
\begin{Highlighting}[]
\KeywordTok{class}\NormalTok{ Vehicle:}
    \KeywordTok{def}\NormalTok{ drive(}\VariableTok{self}\NormalTok{):}
        \ControlFlowTok{return} \StringTok{"The vehicle is moving."}

\KeywordTok{class}\NormalTok{ Car(Vehicle):}
    \KeywordTok{def}\NormalTok{ drive(}\VariableTok{self}\NormalTok{):}
\NormalTok{        parent\_drive }\OperatorTok{=} \BuiltInTok{super}\NormalTok{().drive()}
        \ControlFlowTok{return}\NormalTok{ parent\_drive }\OperatorTok{+} \StringTok{" Specifically, the car is driving."}

\NormalTok{c }\OperatorTok{=}\NormalTok{ Car()}
\BuiltInTok{print}\NormalTok{(c.drive())}
\end{Highlighting}
\end{Shaded}

Partial Overrides You don't always have to replace the entire
method---you can extend it:

\begin{Shaded}
\begin{Highlighting}[]
\KeywordTok{class}\NormalTok{ Logger:}
    \KeywordTok{def}\NormalTok{ log(}\VariableTok{self}\NormalTok{, message):}
        \BuiltInTok{print}\NormalTok{(}\StringTok{"Log:"}\NormalTok{, message)}

\KeywordTok{class}\NormalTok{ TimestampLogger(Logger):}
    \KeywordTok{def}\NormalTok{ log(}\VariableTok{self}\NormalTok{, message):}
        \ImportTok{import}\NormalTok{ datetime}
\NormalTok{        time }\OperatorTok{=}\NormalTok{ datetime.datetime.now()}
        \BuiltInTok{super}\NormalTok{().log(}\SpecialStringTok{f"}\SpecialCharTok{\{}\NormalTok{time}\SpecialCharTok{\}}\SpecialStringTok{ {-} }\SpecialCharTok{\{}\NormalTok{message}\SpecialCharTok{\}}\SpecialStringTok{"}\NormalTok{)}
\end{Highlighting}
\end{Shaded}

Quick Summary Table

\begin{longtable}[]{@{}
  >{\raggedright\arraybackslash}p{(\linewidth - 4\tabcolsep) * \real{0.1354}}
  >{\raggedright\arraybackslash}p{(\linewidth - 4\tabcolsep) * \real{0.3854}}
  >{\raggedright\arraybackslash}p{(\linewidth - 4\tabcolsep) * \real{0.4792}}@{}}
\toprule\noalign{}
\begin{minipage}[b]{\linewidth}\raggedright
Concept
\end{minipage} & \begin{minipage}[b]{\linewidth}\raggedright
Meaning
\end{minipage} & \begin{minipage}[b]{\linewidth}\raggedright
Example
\end{minipage} \\
\midrule\noalign{}
\endhead
\bottomrule\noalign{}
\endlastfoot
Overriding & Redefine method in child class & \texttt{Dog.speak()}
replaces \texttt{Animal.speak()} \\
Specialized & Child provides its own implementation &
\texttt{Car.drive()} different from \texttt{Vehicle.drive()} \\
\texttt{super()} use & Call parent version inside child &
\texttt{super().log(...)} \\
\end{longtable}

\subsubsection{Tiny Code}\label{tiny-code-65}

\begin{Shaded}
\begin{Highlighting}[]
\KeywordTok{class}\NormalTok{ Employee:}
    \KeywordTok{def}\NormalTok{ work(}\VariableTok{self}\NormalTok{):}
        \ControlFlowTok{return} \StringTok{"Employee is working."}

\KeywordTok{class}\NormalTok{ Manager(Employee):}
    \KeywordTok{def}\NormalTok{ work(}\VariableTok{self}\NormalTok{):}
        \ControlFlowTok{return} \StringTok{"Manager is planning and managing."}

\NormalTok{e }\OperatorTok{=}\NormalTok{ Employee()}
\NormalTok{m }\OperatorTok{=}\NormalTok{ Manager()}

\BuiltInTok{print}\NormalTok{(e.work())   }\CommentTok{\# Employee is working.}
\BuiltInTok{print}\NormalTok{(m.work())   }\CommentTok{\# Manager is planning and managing.}
\end{Highlighting}
\end{Shaded}

\subsubsection{Why it Matters}\label{why-it-matters-65}

Method overriding lets subclasses adapt behavior without rewriting
everything from scratch. It's a cornerstone of polymorphism, where
different classes can define the same method name but act differently.

\subsubsection{Try It Yourself}\label{try-it-yourself-65}

\begin{enumerate}
\def\labelenumi{\arabic{enumi}.}
\tightlist
\item
  Create a base class \texttt{Animal} with \texttt{sound()} that returns
  \texttt{"Unknown\ sound"}.
\item
  Make \texttt{Dog} and \texttt{Cat} subclasses that override
  \texttt{sound()} with \texttt{"Woof"} and \texttt{"Meow"}.
\item
  Use a loop to call \texttt{sound()} on both objects and see
  polymorphism in action.
\item
  Extend the base method in one subclass using \texttt{super()} to add
  extra behavior.
\end{enumerate}

\subsection{67. Multiple Inheritance}\label{multiple-inheritance}

Python allows a class to inherit from more than one parent class. This
is called multiple inheritance. It can be powerful but must be used
carefully to avoid confusion.

\subsubsection{Deep Dive}\label{deep-dive-66}

Basic Example

\begin{Shaded}
\begin{Highlighting}[]
\KeywordTok{class}\NormalTok{ Flyer:}
    \KeywordTok{def}\NormalTok{ fly(}\VariableTok{self}\NormalTok{):}
        \ControlFlowTok{return} \StringTok{"I can fly!"}

\KeywordTok{class}\NormalTok{ Swimmer:}
    \KeywordTok{def}\NormalTok{ swim(}\VariableTok{self}\NormalTok{):}
        \ControlFlowTok{return} \StringTok{"I can swim!"}

\KeywordTok{class}\NormalTok{ Duck(Flyer, Swimmer):   }\CommentTok{\# inherits from both}
    \ControlFlowTok{pass}

\NormalTok{d }\OperatorTok{=}\NormalTok{ Duck()}
\BuiltInTok{print}\NormalTok{(d.fly())   }\CommentTok{\# I can fly!}
\BuiltInTok{print}\NormalTok{(d.swim())  }\CommentTok{\# I can swim!}
\end{Highlighting}
\end{Shaded}

Here, \texttt{Duck} inherits methods from both \texttt{Flyer} and
\texttt{Swimmer}.

The Diamond Problem \& MRO If multiple parents have methods with the
same name, Python uses the Method Resolution Order (MRO) to decide which
one to call.

\begin{Shaded}
\begin{Highlighting}[]
\KeywordTok{class}\NormalTok{ A:}
    \KeywordTok{def}\NormalTok{ hello(}\VariableTok{self}\NormalTok{):}
        \ControlFlowTok{return} \StringTok{"Hello from A"}

\KeywordTok{class}\NormalTok{ B(A):}
    \KeywordTok{def}\NormalTok{ hello(}\VariableTok{self}\NormalTok{):}
        \ControlFlowTok{return} \StringTok{"Hello from B"}

\KeywordTok{class}\NormalTok{ C(A):}
    \KeywordTok{def}\NormalTok{ hello(}\VariableTok{self}\NormalTok{):}
        \ControlFlowTok{return} \StringTok{"Hello from C"}

\KeywordTok{class}\NormalTok{ D(B, C):}
    \ControlFlowTok{pass}

\NormalTok{d }\OperatorTok{=}\NormalTok{ D()}
\BuiltInTok{print}\NormalTok{(d.hello())        }\CommentTok{\# Hello from B}
\BuiltInTok{print}\NormalTok{(D.mro())          }\CommentTok{\# [D, B, C, A, object]}
\end{Highlighting}
\end{Shaded}

\begin{itemize}
\tightlist
\item
  Python searches left to right in the inheritance list (\texttt{B}
  before \texttt{C}).
\item
  \texttt{mro()} shows the order.
\end{itemize}

Using \texttt{super()} with Multiple Inheritance \texttt{super()}
respects the MRO, allowing cooperative behavior:

\begin{Shaded}
\begin{Highlighting}[]
\KeywordTok{class}\NormalTok{ A:}
    \KeywordTok{def}\NormalTok{ action(}\VariableTok{self}\NormalTok{):}
        \BuiltInTok{print}\NormalTok{(}\StringTok{"A action"}\NormalTok{)}

\KeywordTok{class}\NormalTok{ B(A):}
    \KeywordTok{def}\NormalTok{ action(}\VariableTok{self}\NormalTok{):}
        \BuiltInTok{super}\NormalTok{().action()}
        \BuiltInTok{print}\NormalTok{(}\StringTok{"B action"}\NormalTok{)}

\KeywordTok{class}\NormalTok{ C(A):}
    \KeywordTok{def}\NormalTok{ action(}\VariableTok{self}\NormalTok{):}
        \BuiltInTok{super}\NormalTok{().action()}
        \BuiltInTok{print}\NormalTok{(}\StringTok{"C action"}\NormalTok{)}

\KeywordTok{class}\NormalTok{ D(B, C):}
    \KeywordTok{def}\NormalTok{ action(}\VariableTok{self}\NormalTok{):}
        \BuiltInTok{super}\NormalTok{().action()}
        \BuiltInTok{print}\NormalTok{(}\StringTok{"D action"}\NormalTok{)}

\NormalTok{d }\OperatorTok{=}\NormalTok{ D()}
\NormalTok{d.action()}
\end{Highlighting}
\end{Shaded}

Output:

\begin{verbatim}
A action
C action
B action
D action
\end{verbatim}

Quick Summary Table

\begin{longtable}[]{@{}ll@{}}
\toprule\noalign{}
Concept & Meaning \\
\midrule\noalign{}
\endhead
\bottomrule\noalign{}
\endlastfoot
Multiple inheritance & Class inherits from more than one parent \\
MRO & Defines search order for methods/attributes \\
Diamond problem & Ambiguity when same method exists in parents \\
\texttt{super()} in MRO & Ensures cooperative method calls \\
\end{longtable}

\subsubsection{Tiny Code}\label{tiny-code-66}

\begin{Shaded}
\begin{Highlighting}[]
\KeywordTok{class}\NormalTok{ Writer:}
    \KeywordTok{def}\NormalTok{ write(}\VariableTok{self}\NormalTok{):}
        \ControlFlowTok{return} \StringTok{"Writing..."}

\KeywordTok{class}\NormalTok{ Reader:}
    \KeywordTok{def}\NormalTok{ read(}\VariableTok{self}\NormalTok{):}
        \ControlFlowTok{return} \StringTok{"Reading..."}

\KeywordTok{class}\NormalTok{ Author(Writer, Reader):}
    \ControlFlowTok{pass}

\NormalTok{a }\OperatorTok{=}\NormalTok{ Author()}
\BuiltInTok{print}\NormalTok{(a.write())}
\BuiltInTok{print}\NormalTok{(a.read())}
\end{Highlighting}
\end{Shaded}

\subsubsection{Why it Matters}\label{why-it-matters-66}

Multiple inheritance allows you to combine behaviors from different
classes, making code flexible and modular. But without understanding
MRO, it can introduce bugs and unexpected results.

\subsubsection{Try It Yourself}\label{try-it-yourself-66}

\begin{enumerate}
\def\labelenumi{\arabic{enumi}.}
\tightlist
\item
  Create two classes \texttt{Walker} and \texttt{Runner}, each with a
  method.
\item
  Create a class \texttt{Athlete} that inherits from both and test all
  methods.
\item
  Add the same method \texttt{train()} in both parents and see which one
  \texttt{Athlete} uses.
\item
  Use \texttt{ClassName.mro()} to confirm the method resolution order.
\end{enumerate}

\subsection{68. Encapsulation \& Private
Members}\label{encapsulation-private-members}

Encapsulation is the principle of restricting direct access to some
parts of an object, protecting its internal state. In Python, this is
done through naming conventions rather than strict enforcement.

\subsubsection{Deep Dive}\label{deep-dive-67}

Public Members

\begin{itemize}
\tightlist
\item
  Accessible from anywhere.
\item
  Default in Python.
\end{itemize}

\begin{Shaded}
\begin{Highlighting}[]
\KeywordTok{class}\NormalTok{ Person:}
    \KeywordTok{def} \FunctionTok{\_\_init\_\_}\NormalTok{(}\VariableTok{self}\NormalTok{, name):}
        \VariableTok{self}\NormalTok{.name }\OperatorTok{=}\NormalTok{ name   }\CommentTok{\# public attribute}

\NormalTok{p }\OperatorTok{=}\NormalTok{ Person(}\StringTok{"Alice"}\NormalTok{)}
\BuiltInTok{print}\NormalTok{(p.name)   }\CommentTok{\# Alice}
\end{Highlighting}
\end{Shaded}

Protected Members (\texttt{\_var})

\begin{itemize}
\tightlist
\item
  Indicated with a single underscore.
\item
  Treated as ``internal use only'', but still accessible.
\end{itemize}

\begin{Shaded}
\begin{Highlighting}[]
\KeywordTok{class}\NormalTok{ Person:}
    \KeywordTok{def} \FunctionTok{\_\_init\_\_}\NormalTok{(}\VariableTok{self}\NormalTok{, name):}
        \VariableTok{self}\NormalTok{.\_secret }\OperatorTok{=} \StringTok{"hidden"}

\NormalTok{p }\OperatorTok{=}\NormalTok{ Person(}\StringTok{"Alice"}\NormalTok{)}
\BuiltInTok{print}\NormalTok{(p.\_secret)   }\CommentTok{\# possible, but discouraged}
\end{Highlighting}
\end{Shaded}

Private Members (\texttt{\_\_var})

\begin{itemize}
\tightlist
\item
  Indicated with double underscores.
\item
  Name-mangled to prevent accidental access.
\end{itemize}

\begin{Shaded}
\begin{Highlighting}[]
\KeywordTok{class}\NormalTok{ BankAccount:}
    \KeywordTok{def} \FunctionTok{\_\_init\_\_}\NormalTok{(}\VariableTok{self}\NormalTok{, balance):}
        \VariableTok{self}\NormalTok{.\_\_balance }\OperatorTok{=}\NormalTok{ balance   }\CommentTok{\# private}

    \KeywordTok{def}\NormalTok{ deposit(}\VariableTok{self}\NormalTok{, amount):}
        \VariableTok{self}\NormalTok{.\_\_balance }\OperatorTok{+=}\NormalTok{ amount}

    \KeywordTok{def}\NormalTok{ get\_balance(}\VariableTok{self}\NormalTok{):}
        \ControlFlowTok{return} \VariableTok{self}\NormalTok{.\_\_balance}

\NormalTok{acc }\OperatorTok{=}\NormalTok{ BankAccount(}\DecValTok{100}\NormalTok{)}
\NormalTok{acc.deposit(}\DecValTok{50}\NormalTok{)}
\BuiltInTok{print}\NormalTok{(acc.get\_balance())   }\CommentTok{\# 150}
\end{Highlighting}
\end{Shaded}

Trying to access directly:

\begin{Shaded}
\begin{Highlighting}[]
\BuiltInTok{print}\NormalTok{(acc.\_\_balance)   }\CommentTok{\# AttributeError}
\BuiltInTok{print}\NormalTok{(acc.\_BankAccount\_\_balance)   }\CommentTok{\# works (name{-}mangled)}
\end{Highlighting}
\end{Shaded}

Why Encapsulation?

\begin{enumerate}
\def\labelenumi{\arabic{enumi}.}
\tightlist
\item
  Prevent accidental modification of sensitive data.
\item
  Provide controlled access via methods (getters/setters).
\item
  Separate internal logic from public API.
\end{enumerate}

Quick Summary Table

\begin{longtable}[]{@{}lll@{}}
\toprule\noalign{}
Convention & Syntax & Access Level \\
\midrule\noalign{}
\endhead
\bottomrule\noalign{}
\endlastfoot
Public & \texttt{var} & Free to access \\
Protected & \texttt{\_var} & Internal use only \\
Private & \texttt{\_\_var} & Strongly restricted (name-mangled) \\
\end{longtable}

\subsubsection{Tiny Code}\label{tiny-code-67}

\begin{Shaded}
\begin{Highlighting}[]
\KeywordTok{class}\NormalTok{ Student:}
    \KeywordTok{def} \FunctionTok{\_\_init\_\_}\NormalTok{(}\VariableTok{self}\NormalTok{, name, grade):}
        \VariableTok{self}\NormalTok{.name }\OperatorTok{=}\NormalTok{ name            }\CommentTok{\# public}
        \VariableTok{self}\NormalTok{.\_grade }\OperatorTok{=}\NormalTok{ grade         }\CommentTok{\# protected}
        \VariableTok{self}\NormalTok{.\_\_id }\OperatorTok{=} \DecValTok{12345}           \CommentTok{\# private}
    
    \KeywordTok{def}\NormalTok{ get\_id(}\VariableTok{self}\NormalTok{):}
        \ControlFlowTok{return} \VariableTok{self}\NormalTok{.\_\_id}

\NormalTok{s }\OperatorTok{=}\NormalTok{ Student(}\StringTok{"Bob"}\NormalTok{, }\StringTok{"A"}\NormalTok{)}
\BuiltInTok{print}\NormalTok{(s.name)       }\CommentTok{\# Public}
\BuiltInTok{print}\NormalTok{(s.\_grade)     }\CommentTok{\# Accessible but discouraged}
\BuiltInTok{print}\NormalTok{(s.get\_id())   }\CommentTok{\# Safe access}
\end{Highlighting}
\end{Shaded}

\subsubsection{Why it Matters}\label{why-it-matters-67}

Encapsulation protects the integrity of your objects. By controlling
access, you reduce bugs and make your code safer and more maintainable.

\subsubsection{Try It Yourself}\label{try-it-yourself-67}

\begin{enumerate}
\def\labelenumi{\arabic{enumi}.}
\tightlist
\item
  Create a \texttt{BankAccount} class with a private
  \texttt{\_\_balance}.
\item
  Add \texttt{deposit()} and \texttt{withdraw()} methods that safely
  modify it.
\item
  Add a method \texttt{get\_balance()} to return the balance.
\item
  Try accessing \texttt{\_\_balance} directly and observe the error.
\end{enumerate}

\subsection{\texorpdfstring{69. Special Methods (\texttt{\_\_str\_\_},
\texttt{\_\_len\_\_},
etc.)}{69. Special Methods (\_\_str\_\_, \_\_len\_\_, etc.)}}\label{special-methods-__str__-__len__-etc.}

Python classes can define special methods (also called \emph{dunder
methods}, because they have double underscores). These let objects
behave like built-in types and integrate smoothly with Python features.

\subsubsection{Deep Dive}\label{deep-dive-68}

\texttt{\_\_str\_\_} → String Representation Defines what
\texttt{print(obj)} shows.

\begin{Shaded}
\begin{Highlighting}[]
\KeywordTok{class}\NormalTok{ Person:}
    \KeywordTok{def} \FunctionTok{\_\_init\_\_}\NormalTok{(}\VariableTok{self}\NormalTok{, name, age):}
        \VariableTok{self}\NormalTok{.name }\OperatorTok{=}\NormalTok{ name}
        \VariableTok{self}\NormalTok{.age }\OperatorTok{=}\NormalTok{ age}
    \KeywordTok{def} \FunctionTok{\_\_str\_\_}\NormalTok{(}\VariableTok{self}\NormalTok{):}
        \ControlFlowTok{return} \SpecialStringTok{f"}\SpecialCharTok{\{}\VariableTok{self}\SpecialCharTok{.}\NormalTok{name}\SpecialCharTok{\}}\SpecialStringTok{, }\SpecialCharTok{\{}\VariableTok{self}\SpecialCharTok{.}\NormalTok{age}\SpecialCharTok{\}}\SpecialStringTok{ years old"}

\NormalTok{p }\OperatorTok{=}\NormalTok{ Person(}\StringTok{"Alice"}\NormalTok{, }\DecValTok{25}\NormalTok{)}
\BuiltInTok{print}\NormalTok{(p)   }\CommentTok{\# Alice, 25 years old}
\end{Highlighting}
\end{Shaded}

\texttt{\_\_repr\_\_} → Developer-Friendly Representation Used in
debugging and interactive shells.

\begin{Shaded}
\begin{Highlighting}[]
\KeywordTok{class}\NormalTok{ Person:}
    \KeywordTok{def} \FunctionTok{\_\_repr\_\_}\NormalTok{(}\VariableTok{self}\NormalTok{):}
        \ControlFlowTok{return} \SpecialStringTok{f"Person(name=\textquotesingle{}}\SpecialCharTok{\{}\VariableTok{self}\SpecialCharTok{.}\NormalTok{name}\SpecialCharTok{\}}\SpecialStringTok{\textquotesingle{}, age=}\SpecialCharTok{\{}\VariableTok{self}\SpecialCharTok{.}\NormalTok{age}\SpecialCharTok{\}}\SpecialStringTok{)"}
\end{Highlighting}
\end{Shaded}

\texttt{\_\_len\_\_} → Length Lets your object work with
\texttt{len(obj)}.

\begin{Shaded}
\begin{Highlighting}[]
\KeywordTok{class}\NormalTok{ Team:}
    \KeywordTok{def} \FunctionTok{\_\_init\_\_}\NormalTok{(}\VariableTok{self}\NormalTok{, members):}
        \VariableTok{self}\NormalTok{.members }\OperatorTok{=}\NormalTok{ members}
    \KeywordTok{def} \FunctionTok{\_\_len\_\_}\NormalTok{(}\VariableTok{self}\NormalTok{):}
        \ControlFlowTok{return} \BuiltInTok{len}\NormalTok{(}\VariableTok{self}\NormalTok{.members)}

\NormalTok{t }\OperatorTok{=}\NormalTok{ Team([}\StringTok{"Alice"}\NormalTok{, }\StringTok{"Bob"}\NormalTok{])}
\BuiltInTok{print}\NormalTok{(}\BuiltInTok{len}\NormalTok{(t))   }\CommentTok{\# 2}
\end{Highlighting}
\end{Shaded}

\texttt{\_\_getitem\_\_} and \texttt{\_\_setitem\_\_} → Indexing Make
objects behave like lists/dicts.

\begin{Shaded}
\begin{Highlighting}[]
\KeywordTok{class}\NormalTok{ Notebook:}
    \KeywordTok{def} \FunctionTok{\_\_init\_\_}\NormalTok{(}\VariableTok{self}\NormalTok{):}
        \VariableTok{self}\NormalTok{.notes }\OperatorTok{=}\NormalTok{ \{\}}
    \KeywordTok{def} \FunctionTok{\_\_getitem\_\_}\NormalTok{(}\VariableTok{self}\NormalTok{, key):}
        \ControlFlowTok{return} \VariableTok{self}\NormalTok{.notes[key]}
    \KeywordTok{def} \FunctionTok{\_\_setitem\_\_}\NormalTok{(}\VariableTok{self}\NormalTok{, key, value):}
        \VariableTok{self}\NormalTok{.notes[key] }\OperatorTok{=}\NormalTok{ value}

\NormalTok{n }\OperatorTok{=}\NormalTok{ Notebook()}
\NormalTok{n[}\StringTok{"day1"}\NormalTok{] }\OperatorTok{=} \StringTok{"Learn Python"}
\BuiltInTok{print}\NormalTok{(n[}\StringTok{"day1"}\NormalTok{])   }\CommentTok{\# Learn Python}
\end{Highlighting}
\end{Shaded}

Other Useful Special Methods

\begin{itemize}
\tightlist
\item
  \texttt{\_\_eq\_\_} → equality (\texttt{==})
\item
  \texttt{\_\_lt\_\_} → less than (\texttt{\textless{}})
\item
  \texttt{\_\_add\_\_} → addition (\texttt{+})
\item
  \texttt{\_\_call\_\_} → make object callable like a function
\item
  \texttt{\_\_iter\_\_} → make object iterable in \texttt{for} loops
\end{itemize}

Quick Summary Table

\begin{longtable}[]{@{}lll@{}}
\toprule\noalign{}
Method & Purpose & Example Use \\
\midrule\noalign{}
\endhead
\bottomrule\noalign{}
\endlastfoot
\texttt{\_\_str\_\_} & User-friendly string & \texttt{print(obj)} \\
\texttt{\_\_repr\_\_} & Debug/developer string & \texttt{obj} in
console \\
\texttt{\_\_len\_\_} & Length & \texttt{len(obj)} \\
\texttt{\_\_getitem\_\_} & Indexing & \texttt{obj{[}key{]}} \\
\texttt{\_\_setitem\_\_} & Assigning by key &
\texttt{obj{[}key{]}\ =\ value} \\
\texttt{\_\_eq\_\_} & Equality check & \texttt{obj1\ ==\ obj2} \\
\texttt{\_\_add\_\_} & Addition & \texttt{obj1\ +\ obj2} \\
\texttt{\_\_call\_\_} & Callable object & \texttt{obj()} \\
\end{longtable}

\subsubsection{Tiny Code}\label{tiny-code-68}

\begin{Shaded}
\begin{Highlighting}[]
\KeywordTok{class}\NormalTok{ Counter:}
    \KeywordTok{def} \FunctionTok{\_\_init\_\_}\NormalTok{(}\VariableTok{self}\NormalTok{, count}\OperatorTok{=}\DecValTok{0}\NormalTok{):}
        \VariableTok{self}\NormalTok{.count }\OperatorTok{=}\NormalTok{ count}
    
    \KeywordTok{def} \FunctionTok{\_\_str\_\_}\NormalTok{(}\VariableTok{self}\NormalTok{):}
        \ControlFlowTok{return} \SpecialStringTok{f"Counter(}\SpecialCharTok{\{}\VariableTok{self}\SpecialCharTok{.}\NormalTok{count}\SpecialCharTok{\}}\SpecialStringTok{)"}
    
    \KeywordTok{def} \FunctionTok{\_\_add\_\_}\NormalTok{(}\VariableTok{self}\NormalTok{, other):}
        \ControlFlowTok{return}\NormalTok{ Counter(}\VariableTok{self}\NormalTok{.count }\OperatorTok{+}\NormalTok{ other.count)}

\NormalTok{c1 }\OperatorTok{=}\NormalTok{ Counter(}\DecValTok{3}\NormalTok{)}
\NormalTok{c2 }\OperatorTok{=}\NormalTok{ Counter(}\DecValTok{7}\NormalTok{)}
\BuiltInTok{print}\NormalTok{(c1)           }\CommentTok{\# Counter(3)}
\BuiltInTok{print}\NormalTok{(c1 }\OperatorTok{+}\NormalTok{ c2)      }\CommentTok{\# Counter(10)}
\end{Highlighting}
\end{Shaded}

\subsubsection{Why it Matters}\label{why-it-matters-68}

Special methods let you design objects that feel natural to use, just
like built-in types. This makes your classes more powerful, expressive,
and Pythonic.

\subsubsection{Try It Yourself}\label{try-it-yourself-68}

\begin{enumerate}
\def\labelenumi{\arabic{enumi}.}
\tightlist
\item
  Create a \texttt{Book} class with \texttt{title} and \texttt{author},
  and override \texttt{\_\_str\_\_} to print
  \texttt{"Title\ by\ Author"}.
\item
  Add \texttt{\_\_len\_\_} to return the length of the title.
\item
  Implement \texttt{\_\_eq\_\_} to compare two books by title and
  author.
\item
  Implement \texttt{\_\_add\_\_} so that adding two books returns a
  string joining both titles.
\end{enumerate}

\subsection{70. Static \& Class Methods}\label{static-class-methods}

In Python, not all methods need to work with a specific object.
Sometimes they belong to the class itself. Python provides class methods
and static methods for these cases.

\subsubsection{Deep Dive}\label{deep-dive-69}

Instance Method (Default)

\begin{itemize}
\tightlist
\item
  The usual method, works with an instance.
\item
  First parameter is always \texttt{self}.
\end{itemize}

\begin{Shaded}
\begin{Highlighting}[]
\KeywordTok{class}\NormalTok{ Person:}
    \KeywordTok{def}\NormalTok{ greet(}\VariableTok{self}\NormalTok{):}
        \ControlFlowTok{return} \StringTok{"Hello!"}
\end{Highlighting}
\end{Shaded}

Class Method (\texttt{@classmethod})

\begin{itemize}
\tightlist
\item
  Works with the class, not an individual object.
\item
  First parameter is \texttt{cls} (the class).
\item
  Declared with \texttt{@classmethod} decorator.
\end{itemize}

\begin{Shaded}
\begin{Highlighting}[]
\KeywordTok{class}\NormalTok{ Person:}
\NormalTok{    species }\OperatorTok{=} \StringTok{"Homo sapiens"}

    \AttributeTok{@classmethod}
    \KeywordTok{def}\NormalTok{ get\_species(cls):}
        \ControlFlowTok{return}\NormalTok{ cls.species}

\BuiltInTok{print}\NormalTok{(Person.get\_species())  }\CommentTok{\# Homo sapiens}
\end{Highlighting}
\end{Shaded}

Static Method (\texttt{@staticmethod})

\begin{itemize}
\tightlist
\item
  Does not use \texttt{self} or \texttt{cls}.
\item
  A regular function inside a class for logical grouping.
\item
  Declared with \texttt{@staticmethod}.
\end{itemize}

\begin{Shaded}
\begin{Highlighting}[]
\KeywordTok{class}\NormalTok{ MathUtils:}
    \AttributeTok{@staticmethod}
    \KeywordTok{def}\NormalTok{ add(a, b):}
        \ControlFlowTok{return}\NormalTok{ a }\OperatorTok{+}\NormalTok{ b}

\BuiltInTok{print}\NormalTok{(MathUtils.add(}\DecValTok{5}\NormalTok{, }\DecValTok{7}\NormalTok{))   }\CommentTok{\# 12}
\end{Highlighting}
\end{Shaded}

When to Use What

\begin{itemize}
\tightlist
\item
  Instance method → operates on object data.
\item
  Class method → operates on class-level data.
\item
  Static method → utility function logically related to the class.
\end{itemize}

Quick Summary Table

\begin{longtable}[]{@{}llll@{}}
\toprule\noalign{}
Type & First Arg & Accesses & Use Case \\
\midrule\noalign{}
\endhead
\bottomrule\noalign{}
\endlastfoot
Instance Method & \texttt{self} & Object & Work with object
attributes \\
Class Method & \texttt{cls} & Class & Work with class attributes \\
Static Method & None & Nothing & Utility/helper function \\
\end{longtable}

\subsubsection{Tiny Code}\label{tiny-code-69}

\begin{Shaded}
\begin{Highlighting}[]
\KeywordTok{class}\NormalTok{ Temperature:}
    \KeywordTok{def} \FunctionTok{\_\_init\_\_}\NormalTok{(}\VariableTok{self}\NormalTok{, celsius):}
        \VariableTok{self}\NormalTok{.celsius }\OperatorTok{=}\NormalTok{ celsius}

    \AttributeTok{@classmethod}
    \KeywordTok{def}\NormalTok{ from\_fahrenheit(cls, f):}
        \ControlFlowTok{return}\NormalTok{ cls((f }\OperatorTok{{-}} \DecValTok{32}\NormalTok{) }\OperatorTok{*} \DecValTok{5}\OperatorTok{/}\DecValTok{9}\NormalTok{)}

    \AttributeTok{@staticmethod}
    \KeywordTok{def}\NormalTok{ is\_freezing(temp\_c):}
        \ControlFlowTok{return}\NormalTok{ temp\_c }\OperatorTok{\textless{}=} \DecValTok{0}

\NormalTok{t }\OperatorTok{=}\NormalTok{ Temperature.from\_fahrenheit(}\DecValTok{32}\NormalTok{)}
\BuiltInTok{print}\NormalTok{(t.celsius)                  }\CommentTok{\# 0.0}
\BuiltInTok{print}\NormalTok{(Temperature.is\_freezing(}\OperatorTok{{-}}\DecValTok{5}\NormalTok{)) }\CommentTok{\# True}
\end{Highlighting}
\end{Shaded}

\subsubsection{Why it Matters}\label{why-it-matters-69}

Static and class methods give you more flexibility in structuring code.
They help keep related functions together inside classes, even if they
don't act on specific objects.

\subsubsection{Try It Yourself}\label{try-it-yourself-69}

\begin{enumerate}
\def\labelenumi{\arabic{enumi}.}
\tightlist
\item
  Create a \texttt{Circle} class with a class variable
  \texttt{pi\ =\ 3.14}. Add a \texttt{@classmethod} \texttt{get\_pi()}
  that returns it.
\item
  Add a \texttt{@staticmethod} \texttt{area(radius)} that computes
  circle area using \texttt{pi}.
\item
  Create a circle and check both methods.
\item
  Try calling them on both the class and an instance.
\end{enumerate}

\section{Chapter 8. Error Handling and
Exceptions}\label{chapter-8.-error-handling-and-exceptions}

\subsection{71. What Are Exceptions?}\label{what-are-exceptions}

An exception is an error that happens during program execution,
interrupting the normal flow. Unlike syntax errors (which stop code
before running), exceptions occur at runtime and can be handled so the
program doesn't crash.

\subsubsection{Deep Dive}\label{deep-dive-70}

Common Examples of Exceptions

\begin{Shaded}
\begin{Highlighting}[]
\BuiltInTok{print}\NormalTok{(}\DecValTok{10} \OperatorTok{/} \DecValTok{0}\NormalTok{)       }\CommentTok{\# ZeroDivisionError}
\NormalTok{numbers }\OperatorTok{=}\NormalTok{ [}\DecValTok{1}\NormalTok{, }\DecValTok{2}\NormalTok{, }\DecValTok{3}\NormalTok{]}
\BuiltInTok{print}\NormalTok{(numbers[}\DecValTok{5}\NormalTok{])   }\CommentTok{\# IndexError}
\BuiltInTok{int}\NormalTok{(}\StringTok{"hello"}\NormalTok{)        }\CommentTok{\# ValueError}
\BuiltInTok{open}\NormalTok{(}\StringTok{"nofile.txt"}\NormalTok{)  }\CommentTok{\# FileNotFoundError}
\end{Highlighting}
\end{Shaded}

Without handling, these errors stop the program immediately.

Python Exception Hierarchy

\begin{itemize}
\item
  All exceptions inherit from the built-in \texttt{Exception} class.
\item
  Examples:

  \begin{itemize}
  \tightlist
  \item
    \texttt{ValueError} → invalid type of value.
  \item
    \texttt{TypeError} → wrong data type.
  \item
    \texttt{KeyError} → missing dictionary key.
  \item
    \texttt{OSError} → file system-related errors.
  \end{itemize}
\end{itemize}

Difference Between Errors and Exceptions

\begin{itemize}
\tightlist
\item
  Error: general term for something wrong (syntax or runtime).
\item
  Exception: specific type of runtime error that can be caught and
  handled.
\end{itemize}

Quick Summary Table

\begin{longtable}[]{@{}ll@{}}
\toprule\noalign{}
Exception Type & Example Situation \\
\midrule\noalign{}
\endhead
\bottomrule\noalign{}
\endlastfoot
\texttt{ZeroDivisionError} & Dividing by zero \\
\texttt{IndexError} & Accessing list index that doesn't exist \\
\texttt{KeyError} & Accessing missing dict key \\
\texttt{FileNotFoundError} & File does not exist \\
\texttt{ValueError} & Wrong value type \\
\texttt{TypeError} & Wrong operation on data type \\
\end{longtable}

\subsubsection{Tiny Code}\label{tiny-code-70}

\begin{Shaded}
\begin{Highlighting}[]
\ControlFlowTok{try}\NormalTok{:}
\NormalTok{    num }\OperatorTok{=} \BuiltInTok{int}\NormalTok{(}\StringTok{"abc"}\NormalTok{)   }\CommentTok{\# invalid conversion}
\ControlFlowTok{except} \PreprocessorTok{ValueError}\NormalTok{:}
    \BuiltInTok{print}\NormalTok{(}\StringTok{"Oops! That was not a valid number."}\NormalTok{)}
\end{Highlighting}
\end{Shaded}

\subsubsection{Why it Matters}\label{why-it-matters-70}

Exceptions are unavoidable in real-world programs. By understanding
them, you can write code that fails gracefully instead of crashing
unexpectedly.

\subsubsection{Try It Yourself}\label{try-it-yourself-70}

\begin{enumerate}
\def\labelenumi{\arabic{enumi}.}
\tightlist
\item
  Try dividing a number by zero and observe the exception.
\item
  Access an element outside a list's range and note the error.
\item
  Use \texttt{int("abc")} and catch the \texttt{ValueError}.
\item
  Try opening a file that doesn't exist to see a
  \texttt{FileNotFoundError}.
\end{enumerate}

\subsection{\texorpdfstring{72. Common Exceptions (\texttt{ValueError},
\texttt{TypeError},
etc.)}{72. Common Exceptions (ValueError, TypeError, etc.)}}\label{common-exceptions-valueerror-typeerror-etc.}

Python has many built-in exceptions that you will encounter often.
Knowing them helps you quickly identify problems and handle them
gracefully.

\subsubsection{Deep Dive}\label{deep-dive-71}

ValueError Occurs when a function gets the right type of input but an
inappropriate value.

\begin{Shaded}
\begin{Highlighting}[]
\BuiltInTok{int}\NormalTok{(}\StringTok{"hello"}\NormalTok{)    }\CommentTok{\# ValueError}
\end{Highlighting}
\end{Shaded}

TypeError Occurs when an operation or function is applied to an object
of the wrong type.

\begin{Shaded}
\begin{Highlighting}[]
\CommentTok{"5"} \OperatorTok{+} \DecValTok{3}   \CommentTok{\# TypeError: cannot add str and int}
\end{Highlighting}
\end{Shaded}

IndexError Happens when you try to access an index outside the valid
range of a list.

\begin{Shaded}
\begin{Highlighting}[]
\NormalTok{nums }\OperatorTok{=}\NormalTok{ [}\DecValTok{1}\NormalTok{, }\DecValTok{2}\NormalTok{, }\DecValTok{3}\NormalTok{]}
\BuiltInTok{print}\NormalTok{(nums[}\DecValTok{5}\NormalTok{])   }\CommentTok{\# IndexError}
\end{Highlighting}
\end{Shaded}

KeyError Raised when trying to access a dictionary key that doesn't
exist.

\begin{Shaded}
\begin{Highlighting}[]
\NormalTok{person }\OperatorTok{=}\NormalTok{ \{}\StringTok{"name"}\NormalTok{: }\StringTok{"Alice"}\NormalTok{\}}
\BuiltInTok{print}\NormalTok{(person[}\StringTok{"age"}\NormalTok{])   }\CommentTok{\# KeyError}
\end{Highlighting}
\end{Shaded}

FileNotFoundError Occurs when you try to open a file that doesn't exist.

\begin{Shaded}
\begin{Highlighting}[]
\BuiltInTok{open}\NormalTok{(}\StringTok{"missing.txt"}\NormalTok{)   }\CommentTok{\# FileNotFoundError}
\end{Highlighting}
\end{Shaded}

ZeroDivisionError Raised when dividing a number by zero.

\begin{Shaded}
\begin{Highlighting}[]
\DecValTok{10} \OperatorTok{/} \DecValTok{0}   \CommentTok{\# ZeroDivisionError}
\end{Highlighting}
\end{Shaded}

Quick Summary Table

\begin{longtable}[]{@{}ll@{}}
\toprule\noalign{}
Exception & Example Trigger \\
\midrule\noalign{}
\endhead
\bottomrule\noalign{}
\endlastfoot
\texttt{ValueError} & \texttt{int("abc")} \\
\texttt{TypeError} & \texttt{"5"\ +\ 3} \\
\texttt{IndexError} & \texttt{{[}1,2,3{]}{[}10{]}} \\
\texttt{KeyError} & \texttt{\{"a":1\}{[}"b"{]}} \\
\texttt{FileNotFoundError} & \texttt{open("nofile.txt")} \\
\texttt{ZeroDivisionError} & \texttt{1\ /\ 0} \\
\end{longtable}

\subsubsection{Tiny Code}\label{tiny-code-71}

\begin{Shaded}
\begin{Highlighting}[]
\ControlFlowTok{try}\NormalTok{:}
\NormalTok{    nums }\OperatorTok{=}\NormalTok{ [}\DecValTok{1}\NormalTok{, }\DecValTok{2}\NormalTok{, }\DecValTok{3}\NormalTok{]}
    \BuiltInTok{print}\NormalTok{(nums[}\DecValTok{10}\NormalTok{])}
\ControlFlowTok{except} \PreprocessorTok{IndexError}\NormalTok{:}
    \BuiltInTok{print}\NormalTok{(}\StringTok{"Oops! That index doesn\textquotesingle{}t exist."}\NormalTok{)}
\end{Highlighting}
\end{Shaded}

\subsubsection{Why it Matters}\label{why-it-matters-71}

These exceptions are among the most frequent in Python. Understanding
them helps you debug faster and design safer programs by predicting
possible errors.

\subsubsection{Try It Yourself}\label{try-it-yourself-71}

\begin{enumerate}
\def\labelenumi{\arabic{enumi}.}
\tightlist
\item
  Trigger a \texttt{TypeError} by adding a string and a number.
\item
  Create a dictionary and access a non-existent key to raise a
  \texttt{KeyError}.
\item
  Open a file that doesn't exist and catch the
  \texttt{FileNotFoundError}.
\item
  Write code that divides by zero and catch the
  \texttt{ZeroDivisionError}.
\end{enumerate}

\subsection{\texorpdfstring{73. \texttt{try} and \texttt{except}
Blocks}{73. try and except Blocks}}\label{try-and-except-blocks}

Python uses \texttt{try} and \texttt{except} to handle exceptions
gracefully. Instead of crashing, the program jumps to the
\texttt{except} block when an error occurs.

\subsubsection{Deep Dive}\label{deep-dive-72}

Basic Structure

\begin{Shaded}
\begin{Highlighting}[]
\ControlFlowTok{try}\NormalTok{:}
    \CommentTok{\# code that may cause an error}
\NormalTok{    x }\OperatorTok{=} \BuiltInTok{int}\NormalTok{(}\StringTok{"abc"}\NormalTok{)}
\ControlFlowTok{except} \PreprocessorTok{ValueError}\NormalTok{:}
    \BuiltInTok{print}\NormalTok{(}\StringTok{"That was not a number!"}\NormalTok{)}
\end{Highlighting}
\end{Shaded}

\begin{itemize}
\tightlist
\item
  The code inside \texttt{try} is executed.
\item
  If an exception occurs, the matching \texttt{except} block runs.
\item
  If no error happens, the \texttt{except} block is skipped.
\end{itemize}

Catching Different Exceptions You can handle multiple specific errors
separately:

\begin{Shaded}
\begin{Highlighting}[]
\ControlFlowTok{try}\NormalTok{:}
\NormalTok{    result }\OperatorTok{=} \DecValTok{10} \OperatorTok{/} \DecValTok{0}
\ControlFlowTok{except} \PreprocessorTok{ZeroDivisionError}\NormalTok{:}
    \BuiltInTok{print}\NormalTok{(}\StringTok{"You can\textquotesingle{}t divide by zero."}\NormalTok{)}
\ControlFlowTok{except} \PreprocessorTok{ValueError}\NormalTok{:}
    \BuiltInTok{print}\NormalTok{(}\StringTok{"Invalid value."}\NormalTok{)}
\end{Highlighting}
\end{Shaded}

Catching Any Exception

\begin{Shaded}
\begin{Highlighting}[]
\ControlFlowTok{try}\NormalTok{:}
\NormalTok{    f }\OperatorTok{=} \BuiltInTok{open}\NormalTok{(}\StringTok{"nofile.txt"}\NormalTok{)}
\ControlFlowTok{except} \PreprocessorTok{Exception} \ImportTok{as}\NormalTok{ e:}
    \BuiltInTok{print}\NormalTok{(}\StringTok{"Error occurred:"}\NormalTok{, e)}
\end{Highlighting}
\end{Shaded}

⚠️ Be careful---catching all exceptions may hide bugs.

Multiple Statements in \texttt{try} If one statement fails, control
jumps immediately to \texttt{except}, skipping the rest of the
\texttt{try} block.

\begin{Shaded}
\begin{Highlighting}[]
\ControlFlowTok{try}\NormalTok{:}
    \BuiltInTok{print}\NormalTok{(}\StringTok{"Before error"}\NormalTok{)}
\NormalTok{    x }\OperatorTok{=} \DecValTok{5} \OperatorTok{/} \DecValTok{0}
    \BuiltInTok{print}\NormalTok{(}\StringTok{"This won\textquotesingle{}t run"}\NormalTok{)}
\ControlFlowTok{except} \PreprocessorTok{ZeroDivisionError}\NormalTok{:}
    \BuiltInTok{print}\NormalTok{(}\StringTok{"Handled division by zero"}\NormalTok{)}
\end{Highlighting}
\end{Shaded}

Quick Summary Table

\begin{longtable}[]{@{}ll@{}}
\toprule\noalign{}
Keyword & Purpose \\
\midrule\noalign{}
\endhead
\bottomrule\noalign{}
\endlastfoot
\texttt{try} & Wraps code that may cause an error \\
\texttt{except} & Defines how to handle specific exceptions \\
\texttt{as\ e} & Captures the exception object \\
\end{longtable}

\subsubsection{Tiny Code}\label{tiny-code-72}

\begin{Shaded}
\begin{Highlighting}[]
\ControlFlowTok{try}\NormalTok{:}
\NormalTok{    num }\OperatorTok{=} \BuiltInTok{int}\NormalTok{(}\StringTok{"42a"}\NormalTok{)}
    \BuiltInTok{print}\NormalTok{(}\StringTok{"Converted:"}\NormalTok{, num)}
\ControlFlowTok{except} \PreprocessorTok{ValueError} \ImportTok{as}\NormalTok{ e:}
    \BuiltInTok{print}\NormalTok{(}\StringTok{"Error:"}\NormalTok{, e)}
\end{Highlighting}
\end{Shaded}

\subsubsection{Why it Matters}\label{why-it-matters-72}

\texttt{try/except} is the foundation of error handling in Python. It
lets you recover from errors, give helpful messages, and keep your
program running.

\subsubsection{Try It Yourself}\label{try-it-yourself-72}

\begin{enumerate}
\def\labelenumi{\arabic{enumi}.}
\tightlist
\item
  Write code that divides two numbers but catches
  \texttt{ZeroDivisionError}.
\item
  Try converting a string to \texttt{int}, and catch
  \texttt{ValueError}.
\item
  Open a non-existent file and catch \texttt{FileNotFoundError}.
\item
  Use \texttt{except\ Exception\ as\ e} to print the error message.
\end{enumerate}

\subsection{74. Catching Multiple
Exceptions}\label{catching-multiple-exceptions}

Sometimes, different types of errors can occur in the same block of
code. Python allows you to handle multiple exceptions separately or
together.

\subsubsection{Deep Dive}\label{deep-dive-73}

Separate Except Blocks You can write different handlers for each type of
exception:

\begin{Shaded}
\begin{Highlighting}[]
\ControlFlowTok{try}\NormalTok{:}
\NormalTok{    x }\OperatorTok{=} \BuiltInTok{int}\NormalTok{(}\StringTok{"abc"}\NormalTok{)    }\CommentTok{\# may cause ValueError}
\NormalTok{    y }\OperatorTok{=} \DecValTok{10} \OperatorTok{/} \DecValTok{0}        \CommentTok{\# may cause ZeroDivisionError}
\ControlFlowTok{except} \PreprocessorTok{ValueError}\NormalTok{:}
    \BuiltInTok{print}\NormalTok{(}\StringTok{"Invalid conversion to int."}\NormalTok{)}
\ControlFlowTok{except} \PreprocessorTok{ZeroDivisionError}\NormalTok{:}
    \BuiltInTok{print}\NormalTok{(}\StringTok{"Cannot divide by zero."}\NormalTok{)}
\end{Highlighting}
\end{Shaded}

Catching Multiple Exceptions in One Block You can group exceptions in a
tuple:

\begin{Shaded}
\begin{Highlighting}[]
\ControlFlowTok{try}\NormalTok{:}
\NormalTok{    data }\OperatorTok{=}\NormalTok{ [}\DecValTok{1}\NormalTok{, }\DecValTok{2}\NormalTok{, }\DecValTok{3}\NormalTok{]}
    \BuiltInTok{print}\NormalTok{(data[}\DecValTok{5}\NormalTok{])    }\CommentTok{\# IndexError}
\ControlFlowTok{except}\NormalTok{ (}\PreprocessorTok{ValueError}\NormalTok{, }\PreprocessorTok{IndexError}\NormalTok{) }\ImportTok{as}\NormalTok{ e:}
    \BuiltInTok{print}\NormalTok{(}\StringTok{"Caught an error:"}\NormalTok{, e)}
\end{Highlighting}
\end{Shaded}

Generic Catch-All The \texttt{Exception} base class catches everything
derived from it:

\begin{Shaded}
\begin{Highlighting}[]
\ControlFlowTok{try}\NormalTok{:}
\NormalTok{    result }\OperatorTok{=} \DecValTok{10} \OperatorTok{/} \DecValTok{0}
\ControlFlowTok{except} \PreprocessorTok{Exception} \ImportTok{as}\NormalTok{ e:}
    \BuiltInTok{print}\NormalTok{(}\StringTok{"Something went wrong:"}\NormalTok{, e)}
\end{Highlighting}
\end{Shaded}

Order Matters Python matches the first fitting \texttt{except}.

\begin{Shaded}
\begin{Highlighting}[]
\ControlFlowTok{try}\NormalTok{:}
    \DecValTok{10} \OperatorTok{/} \DecValTok{0}
\ControlFlowTok{except} \PreprocessorTok{Exception}\NormalTok{:}
    \BuiltInTok{print}\NormalTok{(}\StringTok{"General error"}\NormalTok{)     }\CommentTok{\# this will run}
\ControlFlowTok{except} \PreprocessorTok{ZeroDivisionError}\NormalTok{:}
    \BuiltInTok{print}\NormalTok{(}\StringTok{"Specific error"}\NormalTok{)    }\CommentTok{\# never reached}
\end{Highlighting}
\end{Shaded}

⚠️ Always put specific exceptions first before generic ones.

Quick Summary Table

\begin{longtable}[]{@{}
  >{\raggedright\arraybackslash}p{(\linewidth - 4\tabcolsep) * \real{0.3412}}
  >{\raggedright\arraybackslash}p{(\linewidth - 4\tabcolsep) * \real{0.2824}}
  >{\raggedright\arraybackslash}p{(\linewidth - 4\tabcolsep) * \real{0.3765}}@{}}
\toprule\noalign{}
\begin{minipage}[b]{\linewidth}\raggedright
Style
\end{minipage} & \begin{minipage}[b]{\linewidth}\raggedright
Example
\end{minipage} & \begin{minipage}[b]{\linewidth}\raggedright
Use Case
\end{minipage} \\
\midrule\noalign{}
\endhead
\bottomrule\noalign{}
\endlastfoot
Separate handlers & \texttt{except\ ValueError:\ ...} & Different
handling per exception \\
Grouped in tuple & \texttt{except\ (A,\ B):\ ...} & Same handling for
multiple types \\
General \texttt{Exception} catch-all &
\texttt{except\ Exception\ as\ e:} & Debugging, fallback handling \\
\end{longtable}

\subsubsection{Tiny Code}\label{tiny-code-73}

\begin{Shaded}
\begin{Highlighting}[]
\ControlFlowTok{try}\NormalTok{:}
\NormalTok{    num }\OperatorTok{=} \BuiltInTok{int}\NormalTok{(}\StringTok{"xyz"}\NormalTok{)}
\NormalTok{    result }\OperatorTok{=} \DecValTok{10} \OperatorTok{/} \DecValTok{0}
\ControlFlowTok{except} \PreprocessorTok{ValueError}\NormalTok{:}
    \BuiltInTok{print}\NormalTok{(}\StringTok{"Conversion failed."}\NormalTok{)}
\ControlFlowTok{except} \PreprocessorTok{ZeroDivisionError}\NormalTok{:}
    \BuiltInTok{print}\NormalTok{(}\StringTok{"Math error: division by zero."}\NormalTok{)}
\end{Highlighting}
\end{Shaded}

\subsubsection{Why it Matters}\label{why-it-matters-73}

Most real-world code must guard against different failure modes. Being
able to catch multiple exceptions lets you handle each case correctly
without stopping the whole program.

\subsubsection{Try It Yourself}\label{try-it-yourself-73}

\begin{enumerate}
\def\labelenumi{\arabic{enumi}.}
\tightlist
\item
  Convert \texttt{"abc"} to an integer and catch \texttt{ValueError}.
\item
  Divide by zero in the same block, and handle
  \texttt{ZeroDivisionError}.
\item
  Use one \texttt{except\ (ValueError,\ ZeroDivisionError)} to handle
  both at once.
\item
  Add a final generic \texttt{except\ Exception\ as\ e:} to print any
  unexpected error.
\end{enumerate}

\subsection{\texorpdfstring{75. \texttt{else} in Exception
Handling}{75. else in Exception Handling}}\label{else-in-exception-handling}

In Python, you can use an \texttt{else} block with
\texttt{try}/\texttt{except}. The \texttt{else} block runs only if no
exception was raised in the \texttt{try} block.

\subsubsection{Deep Dive}\label{deep-dive-74}

Basic Structure

\begin{Shaded}
\begin{Highlighting}[]
\ControlFlowTok{try}\NormalTok{:}
\NormalTok{    x }\OperatorTok{=} \BuiltInTok{int}\NormalTok{(}\StringTok{"42"}\NormalTok{)   }\CommentTok{\# no error here}
\ControlFlowTok{except} \PreprocessorTok{ValueError}\NormalTok{:}
    \BuiltInTok{print}\NormalTok{(}\StringTok{"Conversion failed."}\NormalTok{)}
\ControlFlowTok{else}\NormalTok{:}
    \BuiltInTok{print}\NormalTok{(}\StringTok{"Conversion successful:"}\NormalTok{, x)}
\end{Highlighting}
\end{Shaded}

\begin{itemize}
\tightlist
\item
  If the code in \texttt{try} succeeds, the \texttt{else} block runs.
\item
  If an exception occurs, the \texttt{else} block is skipped.
\end{itemize}

Why Use \texttt{else}?

\begin{itemize}
\tightlist
\item
  Keeps your \texttt{try} block focused only on code that might fail.
\item
  Puts the ``safe'' code in \texttt{else}, separating it clearly.
\end{itemize}

Example:

\begin{Shaded}
\begin{Highlighting}[]
\ControlFlowTok{try}\NormalTok{:}
\NormalTok{    f }\OperatorTok{=} \BuiltInTok{open}\NormalTok{(}\StringTok{"data.txt"}\NormalTok{)}
\ControlFlowTok{except} \PreprocessorTok{FileNotFoundError}\NormalTok{:}
    \BuiltInTok{print}\NormalTok{(}\StringTok{"File not found."}\NormalTok{)}
\ControlFlowTok{else}\NormalTok{:}
    \BuiltInTok{print}\NormalTok{(}\StringTok{"File opened successfully."}\NormalTok{)}
\NormalTok{    f.close()}
\end{Highlighting}
\end{Shaded}

With Multiple Exceptions

\begin{Shaded}
\begin{Highlighting}[]
\ControlFlowTok{try}\NormalTok{:}
\NormalTok{    num }\OperatorTok{=} \BuiltInTok{int}\NormalTok{(}\StringTok{"100"}\NormalTok{)}
\ControlFlowTok{except} \PreprocessorTok{ValueError}\NormalTok{:}
    \BuiltInTok{print}\NormalTok{(}\StringTok{"Invalid number."}\NormalTok{)}
\ControlFlowTok{else}\NormalTok{:}
    \BuiltInTok{print}\NormalTok{(}\StringTok{"Parsed successfully:"}\NormalTok{, num)}
\end{Highlighting}
\end{Shaded}

Quick Summary Table

\begin{longtable}[]{@{}ll@{}}
\toprule\noalign{}
Block & Runs When \\
\midrule\noalign{}
\endhead
\bottomrule\noalign{}
\endlastfoot
\texttt{try} & Always, until error happens \\
\texttt{except} & If an error of specified type occurs \\
\texttt{else} & If no errors happened in \texttt{try} \\
\end{longtable}

\subsubsection{Tiny Code}\label{tiny-code-74}

\begin{Shaded}
\begin{Highlighting}[]
\ControlFlowTok{try}\NormalTok{:}
\NormalTok{    result }\OperatorTok{=} \DecValTok{10} \OperatorTok{/} \DecValTok{2}
\ControlFlowTok{except} \PreprocessorTok{ZeroDivisionError}\NormalTok{:}
    \BuiltInTok{print}\NormalTok{(}\StringTok{"Division failed."}\NormalTok{)}
\ControlFlowTok{else}\NormalTok{:}
    \BuiltInTok{print}\NormalTok{(}\StringTok{"Division successful:"}\NormalTok{, result)}
\end{Highlighting}
\end{Shaded}

\subsubsection{Why it Matters}\label{why-it-matters-74}

Using \texttt{else} makes exception handling cleaner: risky code in
\texttt{try}, error handling in \texttt{except}, and safe follow-up code
in \texttt{else}. This improves readability and reduces mistakes.

\subsubsection{Try It Yourself}\label{try-it-yourself-74}

\begin{enumerate}
\def\labelenumi{\arabic{enumi}.}
\tightlist
\item
  Write code that reads a number from a string with \texttt{int()}. If
  it fails, handle \texttt{ValueError}. If it succeeds, print
  \texttt{"Valid\ number"} in \texttt{else}.
\item
  Try dividing two numbers, catching \texttt{ZeroDivisionError}, and use
  \texttt{else} to print the result if successful.
\item
  Open an existing file in \texttt{try}, handle
  \texttt{FileNotFoundError}, and confirm success in \texttt{else}.
\end{enumerate}

\subsection{\texorpdfstring{76. \texttt{finally}
Block}{76. finally Block}}\label{finally-block}

In Python, the \texttt{finally} block is used with
\texttt{try}/\texttt{except} to guarantee that certain code always runs
--- no matter what happens. This is useful for cleanup tasks like
closing files or releasing resources.

\subsubsection{Deep Dive}\label{deep-dive-75}

Basic Structure

\begin{Shaded}
\begin{Highlighting}[]
\ControlFlowTok{try}\NormalTok{:}
\NormalTok{    x }\OperatorTok{=} \DecValTok{10} \OperatorTok{/} \DecValTok{2}
\ControlFlowTok{except} \PreprocessorTok{ZeroDivisionError}\NormalTok{:}
    \BuiltInTok{print}\NormalTok{(}\StringTok{"Division failed."}\NormalTok{)}
\ControlFlowTok{finally}\NormalTok{:}
    \BuiltInTok{print}\NormalTok{(}\StringTok{"This always runs."}\NormalTok{)}
\end{Highlighting}
\end{Shaded}

\begin{itemize}
\tightlist
\item
  If no error: \texttt{finally} still runs.
\item
  If an error occurs and is caught: \texttt{finally} still runs.
\item
  If an error occurs and is not caught: \texttt{finally} still runs
  before the program crashes.
\end{itemize}

With \texttt{else} and \texttt{finally} Together

\begin{Shaded}
\begin{Highlighting}[]
\ControlFlowTok{try}\NormalTok{:}
\NormalTok{    num }\OperatorTok{=} \BuiltInTok{int}\NormalTok{(}\StringTok{"42"}\NormalTok{)}
\ControlFlowTok{except} \PreprocessorTok{ValueError}\NormalTok{:}
    \BuiltInTok{print}\NormalTok{(}\StringTok{"Invalid number"}\NormalTok{)}
\ControlFlowTok{else}\NormalTok{:}
    \BuiltInTok{print}\NormalTok{(}\StringTok{"Conversion successful:"}\NormalTok{, num)}
\ControlFlowTok{finally}\NormalTok{:}
    \BuiltInTok{print}\NormalTok{(}\StringTok{"Execution finished"}\NormalTok{)}
\end{Highlighting}
\end{Shaded}

Order of execution here:

\begin{enumerate}
\def\labelenumi{\arabic{enumi}.}
\tightlist
\item
  \texttt{try} block
\item
  \texttt{except} (if error) OR \texttt{else} (if no error)
\item
  \texttt{finally} (always)
\end{enumerate}

Practical Example: Closing Files

\begin{Shaded}
\begin{Highlighting}[]
\ControlFlowTok{try}\NormalTok{:}
\NormalTok{    f }\OperatorTok{=} \BuiltInTok{open}\NormalTok{(}\StringTok{"data.txt"}\NormalTok{, }\StringTok{"r"}\NormalTok{)}
\NormalTok{    content }\OperatorTok{=}\NormalTok{ f.read()}
\ControlFlowTok{except} \PreprocessorTok{FileNotFoundError}\NormalTok{:}
    \BuiltInTok{print}\NormalTok{(}\StringTok{"File not found."}\NormalTok{)}
\ControlFlowTok{finally}\NormalTok{:}
    \BuiltInTok{print}\NormalTok{(}\StringTok{"Closing file..."}\NormalTok{)}
    \ControlFlowTok{try}\NormalTok{:}
\NormalTok{        f.close()}
    \ControlFlowTok{except}\NormalTok{:}
        \ControlFlowTok{pass}
\end{Highlighting}
\end{Shaded}

Quick Summary Table

\begin{longtable}[]{@{}ll@{}}
\toprule\noalign{}
Block & Runs When \\
\midrule\noalign{}
\endhead
\bottomrule\noalign{}
\endlastfoot
\texttt{try} & Always, until error happens \\
\texttt{except} & If an error occurs \\
\texttt{else} & If no error occurs \\
\texttt{finally} & Always, regardless of error or success \\
\end{longtable}

\subsubsection{Tiny Code}\label{tiny-code-75}

\begin{Shaded}
\begin{Highlighting}[]
\ControlFlowTok{try}\NormalTok{:}
    \BuiltInTok{print}\NormalTok{(}\StringTok{"Opening file..."}\NormalTok{)}
\NormalTok{    f }\OperatorTok{=} \BuiltInTok{open}\NormalTok{(}\StringTok{"missing.txt"}\NormalTok{)}
\ControlFlowTok{except} \PreprocessorTok{FileNotFoundError}\NormalTok{:}
    \BuiltInTok{print}\NormalTok{(}\StringTok{"Error: File not found."}\NormalTok{)}
\ControlFlowTok{finally}\NormalTok{:}
    \BuiltInTok{print}\NormalTok{(}\StringTok{"Cleanup done."}\NormalTok{)}
\end{Highlighting}
\end{Shaded}

\subsubsection{Why it Matters}\label{why-it-matters-75}

The \texttt{finally} block ensures important cleanup (like closing
files, saving data, disconnecting from databases) always happens ---
even if the program crashes in the middle.

\subsubsection{Try It Yourself}\label{try-it-yourself-75}

\begin{enumerate}
\def\labelenumi{\arabic{enumi}.}
\tightlist
\item
  Write code that divides two numbers with \texttt{try/except}, then add
  a \texttt{finally} block to print \texttt{"End\ of\ operation"}.
\item
  Try opening a file in \texttt{try}, handle \texttt{FileNotFoundError},
  and in \texttt{finally} print \texttt{"Closing\ resources"}.
\item
  Combine \texttt{try}, \texttt{except}, \texttt{else}, and
  \texttt{finally} in one program and observe the execution order.
\end{enumerate}

\subsection{\texorpdfstring{77. Raising Exceptions
(\texttt{raise})}{77. Raising Exceptions (raise)}}\label{raising-exceptions-raise}

Sometimes, instead of waiting for Python to throw an error, you may want
to raise an exception yourself when something unexpected happens. This
is done with the \texttt{raise} keyword.

\subsubsection{Deep Dive}\label{deep-dive-76}

Basic Usage

\begin{Shaded}
\begin{Highlighting}[]
\KeywordTok{def}\NormalTok{ divide(a, b):}
    \ControlFlowTok{if}\NormalTok{ b }\OperatorTok{==} \DecValTok{0}\NormalTok{:}
        \ControlFlowTok{raise} \PreprocessorTok{ZeroDivisionError}\NormalTok{(}\StringTok{"Cannot divide by zero!"}\NormalTok{)}
    \ControlFlowTok{return}\NormalTok{ a }\OperatorTok{/}\NormalTok{ b}

\BuiltInTok{print}\NormalTok{(divide(}\DecValTok{10}\NormalTok{, }\DecValTok{2}\NormalTok{))   }\CommentTok{\# 5.0}
\BuiltInTok{print}\NormalTok{(divide(}\DecValTok{5}\NormalTok{, }\DecValTok{0}\NormalTok{))    }\CommentTok{\# Raises ZeroDivisionError}
\end{Highlighting}
\end{Shaded}

Here, we explicitly raise \texttt{ZeroDivisionError} when dividing by
zero.

Raising Built-in Exceptions You can raise any built-in exception
manually:

\begin{Shaded}
\begin{Highlighting}[]
\NormalTok{age }\OperatorTok{=} \OperatorTok{{-}}\DecValTok{1}
\ControlFlowTok{if}\NormalTok{ age }\OperatorTok{\textless{}} \DecValTok{0}\NormalTok{:}
    \ControlFlowTok{raise} \PreprocessorTok{ValueError}\NormalTok{(}\StringTok{"Age cannot be negative"}\NormalTok{)}
\end{Highlighting}
\end{Shaded}

Raising Custom Messages Exceptions can carry useful error messages:

\begin{Shaded}
\begin{Highlighting}[]
\NormalTok{name }\OperatorTok{=} \StringTok{""}
\ControlFlowTok{if} \KeywordTok{not}\NormalTok{ name:}
    \ControlFlowTok{raise} \PreprocessorTok{Exception}\NormalTok{(}\StringTok{"Name must not be empty"}\NormalTok{)}
\end{Highlighting}
\end{Shaded}

Re-raising Exceptions Sometimes you catch an error but still want to
pass it upward:

\begin{Shaded}
\begin{Highlighting}[]
\ControlFlowTok{try}\NormalTok{:}
\NormalTok{    x }\OperatorTok{=} \BuiltInTok{int}\NormalTok{(}\StringTok{"abc"}\NormalTok{)}
\ControlFlowTok{except} \PreprocessorTok{ValueError} \ImportTok{as}\NormalTok{ e:}
    \BuiltInTok{print}\NormalTok{(}\StringTok{"Caught an error:"}\NormalTok{, e)}
    \ControlFlowTok{raise}   \CommentTok{\# re{-}raises the same exception}
\end{Highlighting}
\end{Shaded}

Quick Summary Table

\begin{longtable}[]{@{}
  >{\raggedright\arraybackslash}p{(\linewidth - 4\tabcolsep) * \real{0.1026}}
  >{\raggedright\arraybackslash}p{(\linewidth - 4\tabcolsep) * \real{0.3718}}
  >{\raggedright\arraybackslash}p{(\linewidth - 4\tabcolsep) * \real{0.5256}}@{}}
\toprule\noalign{}
\begin{minipage}[b]{\linewidth}\raggedright
Keyword
\end{minipage} & \begin{minipage}[b]{\linewidth}\raggedright
Purpose
\end{minipage} & \begin{minipage}[b]{\linewidth}\raggedright
Example
\end{minipage} \\
\midrule\noalign{}
\endhead
\bottomrule\noalign{}
\endlastfoot
\texttt{raise} & Manually throw an exception &
\texttt{raise\ ValueError("Invalid\ input")} \\
Message & Provide details for debugging &
\texttt{raise\ Exception("Something\ went\ wrong")} \\
Re-raise & Pass the error up the stack & \texttt{raise} inside
\texttt{except} \\
\end{longtable}

\subsubsection{Tiny Code}\label{tiny-code-76}

\begin{Shaded}
\begin{Highlighting}[]
\KeywordTok{def}\NormalTok{ check\_age(age):}
    \ControlFlowTok{if}\NormalTok{ age }\OperatorTok{\textless{}} \DecValTok{18}\NormalTok{:}
        \ControlFlowTok{raise} \PreprocessorTok{ValueError}\NormalTok{(}\StringTok{"Must be at least 18 years old."}\NormalTok{)}
    \ControlFlowTok{return} \StringTok{"Access granted."}

\BuiltInTok{print}\NormalTok{(check\_age(}\DecValTok{20}\NormalTok{))   }\CommentTok{\# Access granted}
\BuiltInTok{print}\NormalTok{(check\_age(}\DecValTok{15}\NormalTok{))   }\CommentTok{\# Raises ValueError}
\end{Highlighting}
\end{Shaded}

\subsubsection{Why it Matters}\label{why-it-matters-76}

Raising exceptions gives you control. Instead of letting bad data
silently continue, you can stop execution, show a meaningful error, and
prevent bigger problems later.

\subsubsection{Try It Yourself}\label{try-it-yourself-76}

\begin{enumerate}
\def\labelenumi{\arabic{enumi}.}
\tightlist
\item
  Write a \texttt{withdraw(balance,\ amount)} function. If
  \texttt{amount\ \textgreater{}\ balance}, raise a \texttt{ValueError}.
\item
  Create a \texttt{check\_name(name)} function that raises an exception
  if the string is empty.
\item
  Inside a \texttt{try/except}, catch a \texttt{ValueError} and then
  re-raise it to see the traceback.
\item
  Raise a custom \texttt{Exception("Custom\ error\ message")} and print
  it.
\end{enumerate}

\subsection{78. Creating Custom
Exceptions}\label{creating-custom-exceptions}

In addition to Python's built-in exceptions, you can define your own
custom exceptions to make error handling more meaningful in your
programs.

\subsubsection{Deep Dive}\label{deep-dive-77}

Defining a Custom Exception A custom exception is just a class that
inherits from Python's built-in \texttt{Exception} class.

\begin{Shaded}
\begin{Highlighting}[]
\KeywordTok{class}\NormalTok{ NegativeNumberError(}\PreprocessorTok{Exception}\NormalTok{):}
    \CommentTok{"""Raised when a number is negative."""}
    \ControlFlowTok{pass}
\end{Highlighting}
\end{Shaded}

Using the Custom Exception

\begin{Shaded}
\begin{Highlighting}[]
\KeywordTok{def}\NormalTok{ square\_root(x):}
    \ControlFlowTok{if}\NormalTok{ x }\OperatorTok{\textless{}} \DecValTok{0}\NormalTok{:}
        \ControlFlowTok{raise}\NormalTok{ NegativeNumberError(}\StringTok{"Cannot take square root of negative number"}\NormalTok{)}
    \ControlFlowTok{return}\NormalTok{ x  }\FloatTok{0.5}

\BuiltInTok{print}\NormalTok{(square\_root(}\DecValTok{9}\NormalTok{))   }\CommentTok{\# 3.0}
\BuiltInTok{print}\NormalTok{(square\_root(}\OperatorTok{{-}}\DecValTok{4}\NormalTok{))  }\CommentTok{\# Raises NegativeNumberError}
\end{Highlighting}
\end{Shaded}

Adding Extra Functionality You can extend custom exceptions with
attributes.

\begin{Shaded}
\begin{Highlighting}[]
\KeywordTok{class}\NormalTok{ BalanceError(}\PreprocessorTok{Exception}\NormalTok{):}
    \KeywordTok{def} \FunctionTok{\_\_init\_\_}\NormalTok{(}\VariableTok{self}\NormalTok{, balance, message}\OperatorTok{=}\StringTok{"Insufficient funds"}\NormalTok{):}
        \VariableTok{self}\NormalTok{.balance }\OperatorTok{=}\NormalTok{ balance}
        \VariableTok{self}\NormalTok{.message }\OperatorTok{=}\NormalTok{ message}
        \BuiltInTok{super}\NormalTok{().}\FunctionTok{\_\_init\_\_}\NormalTok{(}\SpecialStringTok{f"}\SpecialCharTok{\{}\NormalTok{message}\SpecialCharTok{\}}\SpecialStringTok{. Balance: }\SpecialCharTok{\{}\NormalTok{balance}\SpecialCharTok{\}}\SpecialStringTok{"}\NormalTok{)}

\KeywordTok{def}\NormalTok{ withdraw(balance, amount):}
    \ControlFlowTok{if}\NormalTok{ amount }\OperatorTok{\textgreater{}}\NormalTok{ balance:}
        \ControlFlowTok{raise}\NormalTok{ BalanceError(balance)}
    \ControlFlowTok{return}\NormalTok{ balance }\OperatorTok{{-}}\NormalTok{ amount}

\NormalTok{withdraw(}\DecValTok{100}\NormalTok{, }\DecValTok{200}\NormalTok{)   }\CommentTok{\# Raises BalanceError}
\end{Highlighting}
\end{Shaded}

Catching Custom Exceptions

\begin{Shaded}
\begin{Highlighting}[]
\ControlFlowTok{try}\NormalTok{:}
\NormalTok{    square\_root(}\OperatorTok{{-}}\DecValTok{1}\NormalTok{)}
\ControlFlowTok{except}\NormalTok{ NegativeNumberError }\ImportTok{as}\NormalTok{ e:}
    \BuiltInTok{print}\NormalTok{(}\StringTok{"Custom error caught:"}\NormalTok{, e)}
\end{Highlighting}
\end{Shaded}

Why Create Custom Exceptions?

\begin{enumerate}
\def\labelenumi{\arabic{enumi}.}
\tightlist
\item
  Make your errors descriptive and domain-specific.
\item
  Easier debugging since you know exactly what went wrong.
\item
  Provide structured error handling in larger projects.
\end{enumerate}

Quick Summary Table

\begin{longtable}[]{@{}ll@{}}
\toprule\noalign{}
Step & Example \\
\midrule\noalign{}
\endhead
\bottomrule\noalign{}
\endlastfoot
Define custom error & \texttt{class\ MyError(Exception):\ ...} \\
Raise it & \texttt{raise\ MyError("Something\ happened")} \\
Catch it & \texttt{except\ MyError\ as\ e:} \\
\end{longtable}

\subsubsection{Tiny Code}\label{tiny-code-77}

\begin{Shaded}
\begin{Highlighting}[]
\KeywordTok{class}\NormalTok{ AgeError(}\PreprocessorTok{Exception}\NormalTok{):}
    \ControlFlowTok{pass}

\KeywordTok{def}\NormalTok{ register(age):}
    \ControlFlowTok{if}\NormalTok{ age }\OperatorTok{\textless{}} \DecValTok{18}\NormalTok{:}
        \ControlFlowTok{raise}\NormalTok{ AgeError(}\StringTok{"Must be 18 or older to register"}\NormalTok{)}
    \ControlFlowTok{return} \StringTok{"Registered!"}

\ControlFlowTok{try}\NormalTok{:}
    \BuiltInTok{print}\NormalTok{(register(}\DecValTok{16}\NormalTok{))}
\ControlFlowTok{except}\NormalTok{ AgeError }\ImportTok{as}\NormalTok{ e:}
    \BuiltInTok{print}\NormalTok{(}\StringTok{"Registration failed:"}\NormalTok{, e)}
\end{Highlighting}
\end{Shaded}

\subsubsection{Why it Matters}\label{why-it-matters-77}

Custom exceptions make your programs more self-explanatory and
professional. Instead of generic errors, you provide meaningful messages
tailored to your application's domain.

\subsubsection{Try It Yourself}\label{try-it-yourself-77}

\begin{enumerate}
\def\labelenumi{\arabic{enumi}.}
\tightlist
\item
  Create a \texttt{PasswordError} class for invalid passwords.
\item
  Write a function \texttt{set\_password(pw)} that raises
  \texttt{PasswordError} if the password is less than 8 characters.
\item
  Create a \texttt{TemperatureError} class and raise it if input
  temperature is below absolute zero (-273°C).
\item
  Catch your custom exception and print the message.
\end{enumerate}

\subsection{\texorpdfstring{79. Assertions
(\texttt{assert})}{79. Assertions (assert)}}\label{assertions-assert}

An assertion is a quick way to test if a condition in your program is
true. If the condition is false, Python raises an
\texttt{AssertionError}. Assertions are often used for debugging and
catching mistakes early.

\subsubsection{Deep Dive}\label{deep-dive-78}

Basic Usage

\begin{Shaded}
\begin{Highlighting}[]
\NormalTok{x }\OperatorTok{=} \DecValTok{5}
\ControlFlowTok{assert}\NormalTok{ x }\OperatorTok{\textgreater{}} \DecValTok{0}    \CommentTok{\# passes, nothing happens}
\ControlFlowTok{assert}\NormalTok{ x }\OperatorTok{\textless{}} \DecValTok{0}    \CommentTok{\# fails, raises AssertionError}
\end{Highlighting}
\end{Shaded}

With a Custom Message

\begin{Shaded}
\begin{Highlighting}[]
\NormalTok{age }\OperatorTok{=} \OperatorTok{{-}}\DecValTok{1}
\ControlFlowTok{assert}\NormalTok{ age }\OperatorTok{\textgreater{}=} \DecValTok{0}\NormalTok{, }\StringTok{"Age cannot be negative"}
\end{Highlighting}
\end{Shaded}

If the condition is false, it raises:

\begin{verbatim}
AssertionError: Age cannot be negative
\end{verbatim}

When to Use Assertions

\begin{itemize}
\tightlist
\item
  To check assumptions during development.
\item
  To catch impossible states in your logic.
\item
  For debugging, not for handling user errors (use exceptions for that).
\end{itemize}

Turning Off Assertions

\begin{itemize}
\tightlist
\item
  Assertions can be disabled when running Python with the \texttt{-O}
  (optimize) flag.
\item
  Example: \texttt{python\ -O\ program.py} → all \texttt{assert}
  statements are skipped.
\end{itemize}

Practical Example

\begin{Shaded}
\begin{Highlighting}[]
\KeywordTok{def}\NormalTok{ divide(a, b):}
    \ControlFlowTok{assert}\NormalTok{ b }\OperatorTok{!=} \DecValTok{0}\NormalTok{, }\StringTok{"Denominator must not be zero"}
    \ControlFlowTok{return}\NormalTok{ a }\OperatorTok{/}\NormalTok{ b}

\BuiltInTok{print}\NormalTok{(divide(}\DecValTok{10}\NormalTok{, }\DecValTok{2}\NormalTok{))   }\CommentTok{\# 5.0}
\BuiltInTok{print}\NormalTok{(divide(}\DecValTok{5}\NormalTok{, }\DecValTok{0}\NormalTok{))    }\CommentTok{\# AssertionError}
\end{Highlighting}
\end{Shaded}

Quick Summary Table

\begin{longtable}[]{@{}ll@{}}
\toprule\noalign{}
Syntax & Behavior \\
\midrule\noalign{}
\endhead
\bottomrule\noalign{}
\endlastfoot
\texttt{assert\ condition} & Raises \texttt{AssertionError} if condition
false \\
\texttt{assert\ condition,\ msg} & Raises with custom message \\
Disabled with \texttt{-O} & Skips all asserts \\
\end{longtable}

\subsubsection{Tiny Code}\label{tiny-code-78}

\begin{Shaded}
\begin{Highlighting}[]
\NormalTok{score }\OperatorTok{=} \DecValTok{95}
\ControlFlowTok{assert} \DecValTok{0} \OperatorTok{\textless{}=}\NormalTok{ score }\OperatorTok{\textless{}=} \DecValTok{100}\NormalTok{, }\StringTok{"Score must be between 0 and 100"}
\BuiltInTok{print}\NormalTok{(}\StringTok{"Score is valid!"}\NormalTok{)}
\end{Highlighting}
\end{Shaded}

\subsubsection{Why it Matters}\label{why-it-matters-78}

Assertions help you detect logic errors early. They make your intentions
clear in code and act as built-in sanity checks during development.

\subsubsection{Try It Yourself}\label{try-it-yourself-78}

\begin{enumerate}
\def\labelenumi{\arabic{enumi}.}
\tightlist
\item
  Write an \texttt{assert} to check that a number is positive.
\item
  Add an assertion in a function to make sure a list isn't empty before
  accessing it.
\item
  Use \texttt{assert} to check that temperature is above -273 (absolute
  zero).
\item
  Run your program with \texttt{python\ -O} and see that assertions are
  skipped.
\end{enumerate}

\subsection{80. Best Practices for Error
Handling}\label{best-practices-for-error-handling}

Good error handling makes your programs reliable, readable, and easier
to maintain. Instead of letting programs crash or hiding bugs, you
should follow certain best practices.

\subsubsection{Deep Dive}\label{deep-dive-79}

\begin{enumerate}
\def\labelenumi{\arabic{enumi}.}
\tightlist
\item
  Be Specific in \texttt{except} Blocks Catch only the exceptions you
  expect, not all of them.
\end{enumerate}

\begin{Shaded}
\begin{Highlighting}[]
\ControlFlowTok{try}\NormalTok{:}
\NormalTok{    num }\OperatorTok{=} \BuiltInTok{int}\NormalTok{(}\StringTok{"abc"}\NormalTok{)}
\ControlFlowTok{except} \PreprocessorTok{ValueError}\NormalTok{:}
    \BuiltInTok{print}\NormalTok{(}\StringTok{"Invalid number!"}\NormalTok{)   }\CommentTok{\# good}
\end{Highlighting}
\end{Shaded}

Avoid:

\begin{Shaded}
\begin{Highlighting}[]
\ControlFlowTok{except}\NormalTok{:}
    \BuiltInTok{print}\NormalTok{(}\StringTok{"Something went wrong"}\NormalTok{)   }\CommentTok{\# too vague}
\end{Highlighting}
\end{Shaded}

\begin{enumerate}
\def\labelenumi{\arabic{enumi}.}
\setcounter{enumi}{1}
\tightlist
\item
  Use \texttt{finally} for Cleanup Always free resources like files,
  network connections, or databases.
\end{enumerate}

\begin{Shaded}
\begin{Highlighting}[]
\ControlFlowTok{try}\NormalTok{:}
\NormalTok{    f }\OperatorTok{=} \BuiltInTok{open}\NormalTok{(}\StringTok{"data.txt"}\NormalTok{)}
\NormalTok{    content }\OperatorTok{=}\NormalTok{ f.read()}
\ControlFlowTok{except} \PreprocessorTok{FileNotFoundError}\NormalTok{:}
    \BuiltInTok{print}\NormalTok{(}\StringTok{"File not found."}\NormalTok{)}
\ControlFlowTok{finally}\NormalTok{:}
\NormalTok{    f.close()}
\end{Highlighting}
\end{Shaded}

\begin{enumerate}
\def\labelenumi{\arabic{enumi}.}
\setcounter{enumi}{2}
\tightlist
\item
  Keep \texttt{try} Blocks Small Put only the risky code inside
  \texttt{try}, not everything.
\end{enumerate}

\begin{Shaded}
\begin{Highlighting}[]
\ControlFlowTok{try}\NormalTok{:}
\NormalTok{    result }\OperatorTok{=} \DecValTok{10} \OperatorTok{/} \DecValTok{0}
\ControlFlowTok{except} \PreprocessorTok{ZeroDivisionError}\NormalTok{:}
    \BuiltInTok{print}\NormalTok{(}\StringTok{"Math error"}\NormalTok{)}
\end{Highlighting}
\end{Shaded}

Better than wrapping the entire function.

\begin{enumerate}
\def\labelenumi{\arabic{enumi}.}
\setcounter{enumi}{3}
\item
  Don't Hide Bugs Catching all exceptions with
  \texttt{except\ Exception} should be a last resort. Otherwise, real
  bugs get hidden.
\item
  Raise Exceptions When Needed Instead of returning special values like
  \texttt{-1}, raise meaningful errors.
\end{enumerate}

\begin{Shaded}
\begin{Highlighting}[]
\KeywordTok{def}\NormalTok{ withdraw(balance, amount):}
    \ControlFlowTok{if}\NormalTok{ amount }\OperatorTok{\textgreater{}}\NormalTok{ balance:}
        \ControlFlowTok{raise} \PreprocessorTok{ValueError}\NormalTok{(}\StringTok{"Insufficient funds"}\NormalTok{)}
    \ControlFlowTok{return}\NormalTok{ balance }\OperatorTok{{-}}\NormalTok{ amount}
\end{Highlighting}
\end{Shaded}

\begin{enumerate}
\def\labelenumi{\arabic{enumi}.}
\setcounter{enumi}{5}
\item
  Create Custom Exceptions for Clarity For domain-specific logic, define
  your own exceptions (e.g., \texttt{PasswordTooShortError}).
\item
  Log Errors Use Python's \texttt{logging} module instead of just
  \texttt{print()}.
\end{enumerate}

\begin{Shaded}
\begin{Highlighting}[]
\ImportTok{import}\NormalTok{ logging}
\NormalTok{logging.error(}\StringTok{"File not found"}\NormalTok{, exc\_info}\OperatorTok{=}\VariableTok{True}\NormalTok{)}
\end{Highlighting}
\end{Shaded}

Quick Summary Table

\begin{longtable}[]{@{}ll@{}}
\toprule\noalign{}
Practice & Why It Matters \\
\midrule\noalign{}
\endhead
\bottomrule\noalign{}
\endlastfoot
Catch specific exceptions & Avoids hiding unrelated bugs \\
Use \texttt{finally} for cleanup & Ensures resources are freed \\
Keep \texttt{try} small & Improves readability \\
Raise exceptions & Signals errors clearly \\
Custom exceptions & Domain-specific clarity \\
Logging over printing & Professional error tracking \\
\end{longtable}

\subsubsection{Tiny Code}\label{tiny-code-79}

\begin{Shaded}
\begin{Highlighting}[]
\KeywordTok{def}\NormalTok{ safe\_divide(a, b):}
    \ControlFlowTok{try}\NormalTok{:}
        \ControlFlowTok{return}\NormalTok{ a }\OperatorTok{/}\NormalTok{ b}
    \ControlFlowTok{except} \PreprocessorTok{ZeroDivisionError}\NormalTok{:}
        \ControlFlowTok{raise} \PreprocessorTok{ValueError}\NormalTok{(}\StringTok{"b must not be zero"}\NormalTok{)}

\BuiltInTok{print}\NormalTok{(safe\_divide(}\DecValTok{10}\NormalTok{, }\DecValTok{2}\NormalTok{))}
\BuiltInTok{print}\NormalTok{(safe\_divide(}\DecValTok{5}\NormalTok{, }\DecValTok{0}\NormalTok{))   }\CommentTok{\# raises ValueError}
\end{Highlighting}
\end{Shaded}

\subsubsection{Why it Matters}\label{why-it-matters-79}

Well-structured error handling prevents small mistakes from becoming big
failures. It keeps your programs predictable, professional, and easier
to debug.

\subsubsection{Try It Yourself}\label{try-it-yourself-79}

\begin{enumerate}
\def\labelenumi{\arabic{enumi}.}
\tightlist
\item
  Write a function \texttt{read\_file(filename)} that catches
  \texttt{FileNotFoundError} and raises a new exception with a clearer
  message.
\item
  Add a \texttt{finally} block to always print
  \texttt{"Operation\ complete"}.
\item
  Try logging an error instead of printing it.
\item
  Refactor a long \texttt{try} block so it only wraps the risky line of
  code.
\end{enumerate}

\section{Chapter 9. Advanced Python
Features}\label{chapter-9.-advanced-python-features}

\subsection{81. List Comprehensions}\label{list-comprehensions}

A list comprehension is a concise way to create lists in Python. It lets
you generate new lists by applying an expression to each item in an
existing sequence (or iterable), often replacing loops with a single
readable line.

\subsubsection{Deep Dive}\label{deep-dive-80}

Basic Syntax

\begin{Shaded}
\begin{Highlighting}[]
\NormalTok{[expression }\ControlFlowTok{for}\NormalTok{ item }\KeywordTok{in}\NormalTok{ iterable]}
\end{Highlighting}
\end{Shaded}

Example:

\begin{Shaded}
\begin{Highlighting}[]
\NormalTok{nums }\OperatorTok{=}\NormalTok{ [}\DecValTok{1}\NormalTok{, }\DecValTok{2}\NormalTok{, }\DecValTok{3}\NormalTok{, }\DecValTok{4}\NormalTok{]}
\NormalTok{squares }\OperatorTok{=}\NormalTok{ [x2 }\ControlFlowTok{for}\NormalTok{ x }\KeywordTok{in}\NormalTok{ nums]}
\BuiltInTok{print}\NormalTok{(squares)   }\CommentTok{\# [1, 4, 9, 16]}
\end{Highlighting}
\end{Shaded}

With a Condition

\begin{Shaded}
\begin{Highlighting}[]
\NormalTok{evens }\OperatorTok{=}\NormalTok{ [x }\ControlFlowTok{for}\NormalTok{ x }\KeywordTok{in} \BuiltInTok{range}\NormalTok{(}\DecValTok{10}\NormalTok{) }\ControlFlowTok{if}\NormalTok{ x }\OperatorTok{\%} \DecValTok{2} \OperatorTok{==} \DecValTok{0}\NormalTok{]}
\BuiltInTok{print}\NormalTok{(evens)   }\CommentTok{\# [0, 2, 4, 6, 8]}
\end{Highlighting}
\end{Shaded}

Nested Loops in Comprehensions

\begin{Shaded}
\begin{Highlighting}[]
\NormalTok{pairs }\OperatorTok{=}\NormalTok{ [(x, y) }\ControlFlowTok{for}\NormalTok{ x }\KeywordTok{in}\NormalTok{ [}\DecValTok{1}\NormalTok{, }\DecValTok{2}\NormalTok{] }\ControlFlowTok{for}\NormalTok{ y }\KeywordTok{in}\NormalTok{ [}\DecValTok{3}\NormalTok{, }\DecValTok{4}\NormalTok{]]}
\BuiltInTok{print}\NormalTok{(pairs)   }\CommentTok{\# [(1, 3), (1, 4), (2, 3), (2, 4)]}
\end{Highlighting}
\end{Shaded}

With Functions

\begin{Shaded}
\begin{Highlighting}[]
\NormalTok{words }\OperatorTok{=}\NormalTok{ [}\StringTok{"hello"}\NormalTok{, }\StringTok{"python"}\NormalTok{, }\StringTok{"world"}\NormalTok{]}
\NormalTok{uppercased }\OperatorTok{=}\NormalTok{ [w.upper() }\ControlFlowTok{for}\NormalTok{ w }\KeywordTok{in}\NormalTok{ words]}
\BuiltInTok{print}\NormalTok{(uppercased)   }\CommentTok{\# [\textquotesingle{}HELLO\textquotesingle{}, \textquotesingle{}PYTHON\textquotesingle{}, \textquotesingle{}WORLD\textquotesingle{}]}
\end{Highlighting}
\end{Shaded}

Replacing Loops Loop version:

\begin{Shaded}
\begin{Highlighting}[]
\NormalTok{squares }\OperatorTok{=}\NormalTok{ []}
\ControlFlowTok{for}\NormalTok{ x }\KeywordTok{in} \BuiltInTok{range}\NormalTok{(}\DecValTok{5}\NormalTok{):}
\NormalTok{    squares.append(x2)}
\end{Highlighting}
\end{Shaded}

Comprehension version:

\begin{Shaded}
\begin{Highlighting}[]
\NormalTok{squares }\OperatorTok{=}\NormalTok{ [x2 }\ControlFlowTok{for}\NormalTok{ x }\KeywordTok{in} \BuiltInTok{range}\NormalTok{(}\DecValTok{5}\NormalTok{)]}
\end{Highlighting}
\end{Shaded}

Quick Summary Table

\begin{longtable}[]{@{}ll@{}}
\toprule\noalign{}
Form & Example \\
\midrule\noalign{}
\endhead
\bottomrule\noalign{}
\endlastfoot
Simple comprehension & \texttt{{[}x*2\ for\ x\ in\ range(5){]}} \\
With condition &
\texttt{{[}x\ for\ x\ in\ range(10)\ if\ x\ \%\ 2\ ==\ 0{]}} \\
Nested loops &
\texttt{{[}(x,y)\ for\ x\ in\ {[}1,2{]}\ for\ y\ in\ {[}3,4{]}{]}} \\
With function & \texttt{{[}f(x)\ for\ x\ in\ items{]}} \\
\end{longtable}

\subsubsection{Tiny Code}\label{tiny-code-80}

\begin{Shaded}
\begin{Highlighting}[]
\NormalTok{nums }\OperatorTok{=}\NormalTok{ [}\DecValTok{1}\NormalTok{, }\DecValTok{2}\NormalTok{, }\DecValTok{3}\NormalTok{, }\DecValTok{4}\NormalTok{, }\DecValTok{5}\NormalTok{]}
\NormalTok{double }\OperatorTok{=}\NormalTok{ [n }\OperatorTok{*} \DecValTok{2} \ControlFlowTok{for}\NormalTok{ n }\KeywordTok{in}\NormalTok{ nums }\ControlFlowTok{if}\NormalTok{ n }\OperatorTok{\%} \DecValTok{2} \OperatorTok{!=} \DecValTok{0}\NormalTok{]}
\BuiltInTok{print}\NormalTok{(double)   }\CommentTok{\# [2, 6, 10]}
\end{Highlighting}
\end{Shaded}

\subsubsection{Why it Matters}\label{why-it-matters-80}

List comprehensions make your code shorter, faster, and easier to read.
They are a hallmark of Pythonic style, turning loops and conditions into
expressive one-liners.

\subsubsection{Try It Yourself}\label{try-it-yourself-80}

\begin{enumerate}
\def\labelenumi{\arabic{enumi}.}
\tightlist
\item
  Create a list of squares from 1 to 10 using a list comprehension.
\item
  Make a list of only the odd numbers between 1 and 20.
\item
  Use a comprehension to extract the first letter of each word in
  \texttt{{[}"apple",\ "banana",\ "cherry"{]}}.
\item
  Build a list of coordinate pairs \texttt{(x,\ y)} for \texttt{x} in
  \texttt{{[}1,2,3{]}} and \texttt{y} in \texttt{{[}4,5{]}}.
\end{enumerate}

\subsection{82. Dictionary
Comprehensions}\label{dictionary-comprehensions}

A dictionary comprehension is a compact way to build dictionaries by
combining expressions and loops into a single line. It works like list
comprehensions but produces key--value pairs instead of list elements.

\subsubsection{Deep Dive}\label{deep-dive-81}

Basic Syntax

\begin{Shaded}
\begin{Highlighting}[]
\NormalTok{\{key\_expression: value\_expression }\ControlFlowTok{for}\NormalTok{ item }\KeywordTok{in}\NormalTok{ iterable\}}
\end{Highlighting}
\end{Shaded}

Example:

\begin{Shaded}
\begin{Highlighting}[]
\NormalTok{nums }\OperatorTok{=}\NormalTok{ [}\DecValTok{1}\NormalTok{, }\DecValTok{2}\NormalTok{, }\DecValTok{3}\NormalTok{, }\DecValTok{4}\NormalTok{]}
\NormalTok{squares }\OperatorTok{=}\NormalTok{ \{x: x2 }\ControlFlowTok{for}\NormalTok{ x }\KeywordTok{in}\NormalTok{ nums\}}
\BuiltInTok{print}\NormalTok{(squares)   }\CommentTok{\# \{1: 1, 2: 4, 3: 9, 4: 16\}}
\end{Highlighting}
\end{Shaded}

With a Condition

\begin{Shaded}
\begin{Highlighting}[]
\NormalTok{even\_squares }\OperatorTok{=}\NormalTok{ \{x: x2 }\ControlFlowTok{for}\NormalTok{ x }\KeywordTok{in} \BuiltInTok{range}\NormalTok{(}\DecValTok{10}\NormalTok{) }\ControlFlowTok{if}\NormalTok{ x }\OperatorTok{\%} \DecValTok{2} \OperatorTok{==} \DecValTok{0}\NormalTok{\}}
\BuiltInTok{print}\NormalTok{(even\_squares)   }\CommentTok{\# \{0: 0, 2: 4, 4: 16, 6: 36, 8: 64\}}
\end{Highlighting}
\end{Shaded}

Swapping Keys and Values

\begin{Shaded}
\begin{Highlighting}[]
\NormalTok{fruit }\OperatorTok{=}\NormalTok{ \{}\StringTok{"a"}\NormalTok{: }\StringTok{"apple"}\NormalTok{, }\StringTok{"b"}\NormalTok{: }\StringTok{"banana"}\NormalTok{, }\StringTok{"c"}\NormalTok{: }\StringTok{"cherry"}\NormalTok{\}}
\NormalTok{swap }\OperatorTok{=}\NormalTok{ \{v: k }\ControlFlowTok{for}\NormalTok{ k, v }\KeywordTok{in}\NormalTok{ fruit.items()\}}
\BuiltInTok{print}\NormalTok{(swap)   }\CommentTok{\# \{\textquotesingle{}apple\textquotesingle{}: \textquotesingle{}a\textquotesingle{}, \textquotesingle{}banana\textquotesingle{}: \textquotesingle{}b\textquotesingle{}, \textquotesingle{}cherry\textquotesingle{}: \textquotesingle{}c\textquotesingle{}\}}
\end{Highlighting}
\end{Shaded}

With Functions

\begin{Shaded}
\begin{Highlighting}[]
\NormalTok{words }\OperatorTok{=}\NormalTok{ [}\StringTok{"hello"}\NormalTok{, }\StringTok{"world"}\NormalTok{, }\StringTok{"python"}\NormalTok{]}
\NormalTok{lengths }\OperatorTok{=}\NormalTok{ \{w: }\BuiltInTok{len}\NormalTok{(w) }\ControlFlowTok{for}\NormalTok{ w }\KeywordTok{in}\NormalTok{ words\}}
\BuiltInTok{print}\NormalTok{(lengths)   }\CommentTok{\# \{\textquotesingle{}hello\textquotesingle{}: 5, \textquotesingle{}world\textquotesingle{}: 5, \textquotesingle{}python\textquotesingle{}: 6\}}
\end{Highlighting}
\end{Shaded}

Nested Loops in Dictionary Comprehensions

\begin{Shaded}
\begin{Highlighting}[]
\NormalTok{pairs }\OperatorTok{=}\NormalTok{ \{(x, y): x}\OperatorTok{*}\NormalTok{y }\ControlFlowTok{for}\NormalTok{ x }\KeywordTok{in}\NormalTok{ [}\DecValTok{1}\NormalTok{, }\DecValTok{2}\NormalTok{] }\ControlFlowTok{for}\NormalTok{ y }\KeywordTok{in}\NormalTok{ [}\DecValTok{3}\NormalTok{, }\DecValTok{4}\NormalTok{]\}}
\BuiltInTok{print}\NormalTok{(pairs)   }\CommentTok{\# \{(1, 3): 3, (1, 4): 4, (2, 3): 6, (2, 4): 8\}}
\end{Highlighting}
\end{Shaded}

Quick Summary Table

\begin{longtable}[]{@{}ll@{}}
\toprule\noalign{}
Form & Example \\
\midrule\noalign{}
\endhead
\bottomrule\noalign{}
\endlastfoot
Basic dict comp & \texttt{\{x:\ x*2\ for\ x\ in\ range(3)\}} \\
With condition &
\texttt{\{x:\ x2\ for\ x\ in\ range(6)\ if\ x\ \%\ 2\ ==\ 0\}} \\
Swap keys and values &
\texttt{\{v:\ k\ for\ k,\ v\ in\ dict.items()\}} \\
Using function & \texttt{\{w:\ len(w)\ for\ w\ in\ words\}} \\
Nested loops & \texttt{\{(x,y):\ x*y\ for\ x\ in\ A\ for\ y\ in\ B\}} \\
\end{longtable}

\subsubsection{Tiny Code}\label{tiny-code-81}

\begin{Shaded}
\begin{Highlighting}[]
\NormalTok{students }\OperatorTok{=}\NormalTok{ [}\StringTok{"Alice"}\NormalTok{, }\StringTok{"Bob"}\NormalTok{, }\StringTok{"Charlie"}\NormalTok{]}
\NormalTok{grades }\OperatorTok{=}\NormalTok{ \{name: }\StringTok{"Pass"} \ControlFlowTok{if} \BuiltInTok{len}\NormalTok{(name) }\OperatorTok{\textless{}=} \DecValTok{4} \ControlFlowTok{else} \StringTok{"Review"} \ControlFlowTok{for}\NormalTok{ name }\KeywordTok{in}\NormalTok{ students\}}
\BuiltInTok{print}\NormalTok{(grades)   }\CommentTok{\# \{\textquotesingle{}Alice\textquotesingle{}: \textquotesingle{}Review\textquotesingle{}, \textquotesingle{}Bob\textquotesingle{}: \textquotesingle{}Pass\textquotesingle{}, \textquotesingle{}Charlie\textquotesingle{}: \textquotesingle{}Review\textquotesingle{}\}}
\end{Highlighting}
\end{Shaded}

\subsubsection{Why it Matters}\label{why-it-matters-81}

Dictionary comprehensions save time and reduce boilerplate when building
mappings from existing data. They make code cleaner, more expressive,
and Pythonic.

\subsubsection{Try It Yourself}\label{try-it-yourself-81}

\begin{enumerate}
\def\labelenumi{\arabic{enumi}.}
\tightlist
\item
  Create a dictionary mapping numbers 1--5 to their cubes.
\item
  Build a dictionary of words and their lengths from
  \texttt{{[}"cat",\ "elephant",\ "dog"{]}}.
\item
  Flip a dictionary \texttt{\{"x":\ 1,\ "y":\ 2\}} so values become
  keys.
\item
  Generate a dictionary mapping \texttt{(x,\ y)} pairs to
  \texttt{x\ +\ y} for \texttt{x} in \texttt{{[}1,2{]}} and \texttt{y}
  in \texttt{{[}3,4{]}}.
\end{enumerate}

\subsection{83. Set Comprehensions}\label{set-comprehensions}

A set comprehension is similar to a list comprehension, but it produces
a set---an unordered collection of unique elements. It's a concise way
to build sets with loops and conditions.

\subsubsection{Deep Dive}\label{deep-dive-82}

Basic Syntax

\begin{Shaded}
\begin{Highlighting}[]
\NormalTok{\{expression }\ControlFlowTok{for}\NormalTok{ item }\KeywordTok{in}\NormalTok{ iterable\}}
\end{Highlighting}
\end{Shaded}

Example:

\begin{Shaded}
\begin{Highlighting}[]
\NormalTok{nums }\OperatorTok{=}\NormalTok{ [}\DecValTok{1}\NormalTok{, }\DecValTok{2}\NormalTok{, }\DecValTok{2}\NormalTok{, }\DecValTok{3}\NormalTok{, }\DecValTok{4}\NormalTok{, }\DecValTok{4}\NormalTok{]}
\NormalTok{unique\_squares }\OperatorTok{=}\NormalTok{ \{x2 }\ControlFlowTok{for}\NormalTok{ x }\KeywordTok{in}\NormalTok{ nums\}}
\BuiltInTok{print}\NormalTok{(unique\_squares)   }\CommentTok{\# \{16, 1, 4, 9\}}
\end{Highlighting}
\end{Shaded}

With a Condition

\begin{Shaded}
\begin{Highlighting}[]
\NormalTok{evens }\OperatorTok{=}\NormalTok{ \{x }\ControlFlowTok{for}\NormalTok{ x }\KeywordTok{in} \BuiltInTok{range}\NormalTok{(}\DecValTok{10}\NormalTok{) }\ControlFlowTok{if}\NormalTok{ x }\OperatorTok{\%} \DecValTok{2} \OperatorTok{==} \DecValTok{0}\NormalTok{\}}
\BuiltInTok{print}\NormalTok{(evens)   }\CommentTok{\# \{0, 2, 4, 6, 8\}}
\end{Highlighting}
\end{Shaded}

From a String

\begin{Shaded}
\begin{Highlighting}[]
\NormalTok{letters }\OperatorTok{=}\NormalTok{ \{ch }\ControlFlowTok{for}\NormalTok{ ch }\KeywordTok{in} \StringTok{"banana"}\NormalTok{\}}
\BuiltInTok{print}\NormalTok{(letters)   }\CommentTok{\# \{\textquotesingle{}a\textquotesingle{}, \textquotesingle{}b\textquotesingle{}, \textquotesingle{}n\textquotesingle{}\}}
\end{Highlighting}
\end{Shaded}

With Functions

\begin{Shaded}
\begin{Highlighting}[]
\NormalTok{words }\OperatorTok{=}\NormalTok{ [}\StringTok{"hello"}\NormalTok{, }\StringTok{"world"}\NormalTok{, }\StringTok{"python"}\NormalTok{]}
\NormalTok{lengths }\OperatorTok{=}\NormalTok{ \{}\BuiltInTok{len}\NormalTok{(w) }\ControlFlowTok{for}\NormalTok{ w }\KeywordTok{in}\NormalTok{ words\}}
\BuiltInTok{print}\NormalTok{(lengths)   }\CommentTok{\# \{5, 6\}}
\end{Highlighting}
\end{Shaded}

Nested Loops in Set Comprehensions

\begin{Shaded}
\begin{Highlighting}[]
\NormalTok{pairs }\OperatorTok{=}\NormalTok{ \{(x, y) }\ControlFlowTok{for}\NormalTok{ x }\KeywordTok{in}\NormalTok{ [}\DecValTok{1}\NormalTok{, }\DecValTok{2}\NormalTok{] }\ControlFlowTok{for}\NormalTok{ y }\KeywordTok{in}\NormalTok{ [}\DecValTok{3}\NormalTok{, }\DecValTok{4}\NormalTok{]\}}
\BuiltInTok{print}\NormalTok{(pairs)   }\CommentTok{\# \{(1, 3), (1, 4), (2, 3), (2, 4)\}}
\end{Highlighting}
\end{Shaded}

Quick Summary Table

\begin{longtable}[]{@{}ll@{}}
\toprule\noalign{}
Form & Example \\
\midrule\noalign{}
\endhead
\bottomrule\noalign{}
\endlastfoot
Simple set comp & \texttt{\{x*2\ for\ x\ in\ range(5)\}} \\
With condition &
\texttt{\{x\ for\ x\ in\ range(10)\ if\ x\ \%\ 2\ ==\ 0\}} \\
From string & \texttt{\{ch\ for\ ch\ in\ "banana"\}} \\
With function & \texttt{\{len(w)\ for\ w\ in\ words\}} \\
Nested loops & \texttt{\{(x,y)\ for\ x\ in\ A\ for\ y\ in\ B\}} \\
\end{longtable}

\subsubsection{Tiny Code}\label{tiny-code-82}

\begin{Shaded}
\begin{Highlighting}[]
\NormalTok{nums }\OperatorTok{=}\NormalTok{ [}\DecValTok{1}\NormalTok{, }\DecValTok{2}\NormalTok{, }\DecValTok{3}\NormalTok{, }\DecValTok{2}\NormalTok{, }\DecValTok{1}\NormalTok{, }\DecValTok{4}\NormalTok{]}
\NormalTok{squares }\OperatorTok{=}\NormalTok{ \{n2 }\ControlFlowTok{for}\NormalTok{ n }\KeywordTok{in}\NormalTok{ nums }\ControlFlowTok{if}\NormalTok{ n }\OperatorTok{\%} \DecValTok{2} \OperatorTok{!=} \DecValTok{0}\NormalTok{\}}
\BuiltInTok{print}\NormalTok{(squares)   }\CommentTok{\# \{1, 9\}}
\end{Highlighting}
\end{Shaded}

\subsubsection{Why it Matters}\label{why-it-matters-82}

Set comprehensions provide a quick way to eliminate duplicates and apply
transformations at the same time. They're useful for data cleaning,
filtering, and fast membership checks.

\subsubsection{Try It Yourself}\label{try-it-yourself-82}

\begin{enumerate}
\def\labelenumi{\arabic{enumi}.}
\tightlist
\item
  Create a set of squares from 1--10.
\item
  Build a set of all vowels in the word \texttt{"programming"}.
\item
  Make a set of numbers between 1--20 that are divisible by 3.
\item
  Generate a set of \texttt{(x,\ y)} pairs where \texttt{x} in
  \texttt{{[}1,2,3{]}} and \texttt{y} in \texttt{{[}4,5{]}}.
\end{enumerate}

\subsection{\texorpdfstring{84. Generators
(\texttt{yield})}{84. Generators (yield)}}\label{generators-yield}

A generator is a special type of function that lets you produce a
sequence of values lazily, one at a time, using the \texttt{yield}
keyword. Unlike regular functions, generators don't return everything at
once---they pause and resume.

\subsubsection{Deep Dive}\label{deep-dive-83}

Basic Generator

\begin{Shaded}
\begin{Highlighting}[]
\KeywordTok{def}\NormalTok{ count\_up\_to(n):}
\NormalTok{    i }\OperatorTok{=} \DecValTok{1}
    \ControlFlowTok{while}\NormalTok{ i }\OperatorTok{\textless{}=}\NormalTok{ n:}
        \ControlFlowTok{yield}\NormalTok{ i}
\NormalTok{        i }\OperatorTok{+=} \DecValTok{1}

\ControlFlowTok{for}\NormalTok{ num }\KeywordTok{in}\NormalTok{ count\_up\_to(}\DecValTok{5}\NormalTok{):}
    \BuiltInTok{print}\NormalTok{(num)}
\end{Highlighting}
\end{Shaded}

Output:

\begin{verbatim}
1
2
3
4
5
\end{verbatim}

Difference Between \texttt{return} and \texttt{yield}

\begin{itemize}
\tightlist
\item
  \texttt{return} → ends the function and gives a single value.
\item
  \texttt{yield} → pauses the function, remembers its state, and
  continues next time.
\end{itemize}

Using Generators with \texttt{next()}

\begin{Shaded}
\begin{Highlighting}[]
\NormalTok{gen }\OperatorTok{=}\NormalTok{ count\_up\_to(}\DecValTok{3}\NormalTok{)}
\BuiltInTok{print}\NormalTok{(}\BuiltInTok{next}\NormalTok{(gen))  }\CommentTok{\# 1}
\BuiltInTok{print}\NormalTok{(}\BuiltInTok{next}\NormalTok{(gen))  }\CommentTok{\# 2}
\BuiltInTok{print}\NormalTok{(}\BuiltInTok{next}\NormalTok{(gen))  }\CommentTok{\# 3}
\end{Highlighting}
\end{Shaded}

Infinite Generators Generators can produce endless sequences:

\begin{Shaded}
\begin{Highlighting}[]
\KeywordTok{def}\NormalTok{ even\_numbers():}
\NormalTok{    n }\OperatorTok{=} \DecValTok{0}
    \ControlFlowTok{while} \VariableTok{True}\NormalTok{:}
        \ControlFlowTok{yield}\NormalTok{ n}
\NormalTok{        n }\OperatorTok{+=} \DecValTok{2}

\NormalTok{gen }\OperatorTok{=}\NormalTok{ even\_numbers()}
\ControlFlowTok{for}\NormalTok{ \_ }\KeywordTok{in} \BuiltInTok{range}\NormalTok{(}\DecValTok{5}\NormalTok{):}
    \BuiltInTok{print}\NormalTok{(}\BuiltInTok{next}\NormalTok{(gen))   }\CommentTok{\# 0 2 4 6 8}
\end{Highlighting}
\end{Shaded}

Generator Expressions Like list comprehensions but with parentheses:

\begin{Shaded}
\begin{Highlighting}[]
\NormalTok{squares }\OperatorTok{=}\NormalTok{ (x2 }\ControlFlowTok{for}\NormalTok{ x }\KeywordTok{in} \BuiltInTok{range}\NormalTok{(}\DecValTok{5}\NormalTok{))}
\ControlFlowTok{for}\NormalTok{ s }\KeywordTok{in}\NormalTok{ squares:}
    \BuiltInTok{print}\NormalTok{(s)}
\end{Highlighting}
\end{Shaded}

Quick Summary Table

\begin{longtable}[]{@{}
  >{\raggedright\arraybackslash}p{(\linewidth - 4\tabcolsep) * \real{0.2500}}
  >{\raggedright\arraybackslash}p{(\linewidth - 4\tabcolsep) * \real{0.3611}}
  >{\raggedright\arraybackslash}p{(\linewidth - 4\tabcolsep) * \real{0.3889}}@{}}
\toprule\noalign{}
\begin{minipage}[b]{\linewidth}\raggedright
Feature
\end{minipage} & \begin{minipage}[b]{\linewidth}\raggedright
Example
\end{minipage} & \begin{minipage}[b]{\linewidth}\raggedright
Behavior
\end{minipage} \\
\midrule\noalign{}
\endhead
\bottomrule\noalign{}
\endlastfoot
\texttt{yield} keyword & \texttt{yield\ x} & Produces one value at a
time \\
Pause \& resume & Uses \texttt{next()} & Continues from last state \\
Generator function & \texttt{def\ f():\ yield\ ...} & Creates a
generator \\
Generator expr & \texttt{(x2\ for\ x\ in\ range(5))} & Compact generator
syntax \\
\end{longtable}

\subsubsection{Tiny Code}\label{tiny-code-83}

\begin{Shaded}
\begin{Highlighting}[]
\KeywordTok{def}\NormalTok{ fibonacci(limit):}
\NormalTok{    a, b }\OperatorTok{=} \DecValTok{0}\NormalTok{, }\DecValTok{1}
    \ControlFlowTok{while}\NormalTok{ a }\OperatorTok{\textless{}=}\NormalTok{ limit:}
        \ControlFlowTok{yield}\NormalTok{ a}
\NormalTok{        a, b }\OperatorTok{=}\NormalTok{ b, a }\OperatorTok{+}\NormalTok{ b}

\ControlFlowTok{for}\NormalTok{ num }\KeywordTok{in}\NormalTok{ fibonacci(}\DecValTok{20}\NormalTok{):}
    \BuiltInTok{print}\NormalTok{(num)}
\end{Highlighting}
\end{Shaded}

\subsubsection{Why it Matters}\label{why-it-matters-83}

Generators are memory-efficient because they don't build the whole list
in memory. They're ideal for large datasets, streams of data, or
infinite sequences.

\subsubsection{Try It Yourself}\label{try-it-yourself-83}

\begin{enumerate}
\def\labelenumi{\arabic{enumi}.}
\tightlist
\item
  Write a generator \texttt{countdown(n)} that yields numbers from
  \texttt{n} down to \texttt{1}.
\item
  Make a generator that yields only odd numbers up to 15.
\item
  Create a generator expression for cubes of numbers 1--5.
\item
  Modify the Fibonacci generator to stop after producing 10 numbers.
\end{enumerate}

\subsection{85. Iterators}\label{iterators}

An iterator is an object that represents a stream of data. It returns
items one at a time when you call \texttt{next()} on it, and it
remembers its position between calls. Iterators are the foundation of
loops, comprehensions, and generators in Python.

\subsubsection{Deep Dive}\label{deep-dive-84}

Iterator Protocol An object is an iterator if it implements two methods:

\begin{itemize}
\tightlist
\item
  \texttt{\_\_iter\_\_()} → returns the iterator object itself.
\item
  \texttt{\_\_next\_\_()} → returns the next value, or raises
  \texttt{StopIteration} when done.
\end{itemize}

Built-in Iterators

\begin{Shaded}
\begin{Highlighting}[]
\NormalTok{nums }\OperatorTok{=}\NormalTok{ [}\DecValTok{1}\NormalTok{, }\DecValTok{2}\NormalTok{, }\DecValTok{3}\NormalTok{]}
\NormalTok{it }\OperatorTok{=} \BuiltInTok{iter}\NormalTok{(nums)   }\CommentTok{\# get iterator}

\BuiltInTok{print}\NormalTok{(}\BuiltInTok{next}\NormalTok{(it))   }\CommentTok{\# 1}
\BuiltInTok{print}\NormalTok{(}\BuiltInTok{next}\NormalTok{(it))   }\CommentTok{\# 2}
\BuiltInTok{print}\NormalTok{(}\BuiltInTok{next}\NormalTok{(it))   }\CommentTok{\# 3}
\CommentTok{\# next(it) now raises StopIteration}
\end{Highlighting}
\end{Shaded}

For Loops Use Iterators Under the Hood

\begin{Shaded}
\begin{Highlighting}[]
\ControlFlowTok{for}\NormalTok{ n }\KeywordTok{in}\NormalTok{ [}\DecValTok{1}\NormalTok{, }\DecValTok{2}\NormalTok{, }\DecValTok{3}\NormalTok{]:}
    \BuiltInTok{print}\NormalTok{(n)}
\end{Highlighting}
\end{Shaded}

is equivalent to:

\begin{Shaded}
\begin{Highlighting}[]
\NormalTok{nums }\OperatorTok{=}\NormalTok{ [}\DecValTok{1}\NormalTok{, }\DecValTok{2}\NormalTok{, }\DecValTok{3}\NormalTok{]}
\NormalTok{it }\OperatorTok{=} \BuiltInTok{iter}\NormalTok{(nums)}
\ControlFlowTok{while} \VariableTok{True}\NormalTok{:}
    \ControlFlowTok{try}\NormalTok{:}
        \BuiltInTok{print}\NormalTok{(}\BuiltInTok{next}\NormalTok{(it))}
    \ControlFlowTok{except} \PreprocessorTok{StopIteration}\NormalTok{:}
        \ControlFlowTok{break}
\end{Highlighting}
\end{Shaded}

Custom Iterator You can build your own iterator by defining
\texttt{\_\_iter\_\_} and \texttt{\_\_next\_\_}:

\begin{Shaded}
\begin{Highlighting}[]
\KeywordTok{class}\NormalTok{ CountDown:}
    \KeywordTok{def} \FunctionTok{\_\_init\_\_}\NormalTok{(}\VariableTok{self}\NormalTok{, start):}
        \VariableTok{self}\NormalTok{.current }\OperatorTok{=}\NormalTok{ start}

    \KeywordTok{def} \FunctionTok{\_\_iter\_\_}\NormalTok{(}\VariableTok{self}\NormalTok{):}
        \ControlFlowTok{return} \VariableTok{self}

    \KeywordTok{def} \FunctionTok{\_\_next\_\_}\NormalTok{(}\VariableTok{self}\NormalTok{):}
        \ControlFlowTok{if} \VariableTok{self}\NormalTok{.current }\OperatorTok{\textless{}=} \DecValTok{0}\NormalTok{:}
            \ControlFlowTok{raise} \PreprocessorTok{StopIteration}
        \VariableTok{self}\NormalTok{.current }\OperatorTok{{-}=} \DecValTok{1}
        \ControlFlowTok{return} \VariableTok{self}\NormalTok{.current }\OperatorTok{+} \DecValTok{1}

\ControlFlowTok{for}\NormalTok{ num }\KeywordTok{in}\NormalTok{ CountDown(}\DecValTok{5}\NormalTok{):}
    \BuiltInTok{print}\NormalTok{(num)}
\end{Highlighting}
\end{Shaded}

Output:

\begin{verbatim}
5
4
3
2
1
\end{verbatim}

Quick Summary Table

\begin{longtable}[]{@{}
  >{\raggedright\arraybackslash}p{(\linewidth - 4\tabcolsep) * \real{0.1912}}
  >{\raggedright\arraybackslash}p{(\linewidth - 4\tabcolsep) * \real{0.4265}}
  >{\raggedright\arraybackslash}p{(\linewidth - 4\tabcolsep) * \real{0.3824}}@{}}
\toprule\noalign{}
\begin{minipage}[b]{\linewidth}\raggedright
Concept
\end{minipage} & \begin{minipage}[b]{\linewidth}\raggedright
Example
\end{minipage} & \begin{minipage}[b]{\linewidth}\raggedright
Purpose
\end{minipage} \\
\midrule\noalign{}
\endhead
\bottomrule\noalign{}
\endlastfoot
\texttt{iter(obj)} & \texttt{it\ =\ iter({[}1,2,3{]})} & Get iterator
from iterable \\
\texttt{next(it)} & \texttt{next(it)} & Get next value \\
StopIteration & Exception when done & Signals end of iteration \\
Custom & Define \texttt{\_\_iter\_\_}, \texttt{\_\_next\_\_} & Create
your own sequence \\
\end{longtable}

\subsubsection{Tiny Code}\label{tiny-code-84}

\begin{Shaded}
\begin{Highlighting}[]
\NormalTok{nums }\OperatorTok{=}\NormalTok{ [}\DecValTok{10}\NormalTok{, }\DecValTok{20}\NormalTok{, }\DecValTok{30}\NormalTok{]}
\NormalTok{it }\OperatorTok{=} \BuiltInTok{iter}\NormalTok{(nums)}

\BuiltInTok{print}\NormalTok{(}\BuiltInTok{next}\NormalTok{(it))  }\CommentTok{\# 10}
\BuiltInTok{print}\NormalTok{(}\BuiltInTok{next}\NormalTok{(it))  }\CommentTok{\# 20}
\BuiltInTok{print}\NormalTok{(}\BuiltInTok{next}\NormalTok{(it))  }\CommentTok{\# 30}
\end{Highlighting}
\end{Shaded}

\subsubsection{Why it Matters}\label{why-it-matters-84}

Understanding iterators explains how loops, generators, and
comprehensions actually work in Python. Iterators allow Python to handle
large datasets efficiently, consuming one item at a time.

\subsubsection{Try It Yourself}\label{try-it-yourself-84}

\begin{enumerate}
\def\labelenumi{\arabic{enumi}.}
\tightlist
\item
  Use \texttt{iter()} and \texttt{next()} on a string like
  \texttt{"hello"} to get characters one by one.
\item
  Build a simple custom iterator that counts from 1 to 5.
\item
  Write a \texttt{for} loop manually using \texttt{while\ True} and
  \texttt{next()} with \texttt{StopIteration}.
\item
  Create a custom iterator \texttt{EvenNumbers(n)} that yields even
  numbers up to \texttt{n}.
\end{enumerate}

\subsection{86. Decorators}\label{decorators}

A decorator is a special function that takes another function as input,
adds extra behavior to it, and returns a new function. In Python,
decorators are often used for logging, authentication, caching, and
more.

\subsubsection{Deep Dive}\label{deep-dive-85}

Basic Decorator

\begin{Shaded}
\begin{Highlighting}[]
\KeywordTok{def}\NormalTok{ my\_decorator(func):}
    \KeywordTok{def}\NormalTok{ wrapper():}
        \BuiltInTok{print}\NormalTok{(}\StringTok{"Before function runs"}\NormalTok{)}
\NormalTok{        func()}
        \BuiltInTok{print}\NormalTok{(}\StringTok{"After function runs"}\NormalTok{)}
    \ControlFlowTok{return}\NormalTok{ wrapper}

\AttributeTok{@my\_decorator}
\KeywordTok{def}\NormalTok{ say\_hello():}
    \BuiltInTok{print}\NormalTok{(}\StringTok{"Hello!"}\NormalTok{)}

\NormalTok{say\_hello()}
\end{Highlighting}
\end{Shaded}

Output:

\begin{verbatim}
Before function runs
Hello!
After function runs
\end{verbatim}

\begin{itemize}
\tightlist
\item
  \texttt{@my\_decorator} is shorthand for
  \texttt{say\_hello\ =\ my\_decorator(say\_hello)}.
\end{itemize}

Decorators with Arguments

\begin{Shaded}
\begin{Highlighting}[]
\KeywordTok{def}\NormalTok{ repeat(func):}
    \KeywordTok{def}\NormalTok{ wrapper():}
        \ControlFlowTok{for}\NormalTok{ \_ }\KeywordTok{in} \BuiltInTok{range}\NormalTok{(}\DecValTok{3}\NormalTok{):}
\NormalTok{            func()}
    \ControlFlowTok{return}\NormalTok{ wrapper}

\AttributeTok{@repeat}
\KeywordTok{def}\NormalTok{ greet():}
    \BuiltInTok{print}\NormalTok{(}\StringTok{"Hi!"}\NormalTok{)}

\NormalTok{greet()}
\end{Highlighting}
\end{Shaded}

Output:

\begin{verbatim}
Hi!
Hi!
Hi!
\end{verbatim}

Passing Arguments to Wrapped Function

\begin{Shaded}
\begin{Highlighting}[]
\KeywordTok{def}\NormalTok{ log\_args(func):}
    \KeywordTok{def}\NormalTok{ wrapper(}\OperatorTok{*}\NormalTok{args, kwargs):}
        \BuiltInTok{print}\NormalTok{(}\StringTok{"Arguments:"}\NormalTok{, args, kwargs)}
        \ControlFlowTok{return}\NormalTok{ func(}\OperatorTok{*}\NormalTok{args, kwargs)}
    \ControlFlowTok{return}\NormalTok{ wrapper}

\AttributeTok{@log\_args}
\KeywordTok{def}\NormalTok{ add(a, b):}
    \ControlFlowTok{return}\NormalTok{ a }\OperatorTok{+}\NormalTok{ b}

\BuiltInTok{print}\NormalTok{(add(}\DecValTok{3}\NormalTok{, }\DecValTok{5}\NormalTok{))}
\end{Highlighting}
\end{Shaded}

Using \texttt{functools.wraps} Without it, the decorated function loses
its original name and docstring.

\begin{Shaded}
\begin{Highlighting}[]
\ImportTok{from}\NormalTok{ functools }\ImportTok{import}\NormalTok{ wraps}

\KeywordTok{def}\NormalTok{ decorator(func):}
    \AttributeTok{@wraps}\NormalTok{(func)}
    \KeywordTok{def}\NormalTok{ wrapper(}\OperatorTok{*}\NormalTok{args, kwargs):}
        \ControlFlowTok{return}\NormalTok{ func(}\OperatorTok{*}\NormalTok{args, kwargs)}
    \ControlFlowTok{return}\NormalTok{ wrapper}
\end{Highlighting}
\end{Shaded}

Quick Summary Table

\begin{longtable}[]{@{}
  >{\raggedright\arraybackslash}p{(\linewidth - 4\tabcolsep) * \real{0.2317}}
  >{\raggedright\arraybackslash}p{(\linewidth - 4\tabcolsep) * \real{0.3537}}
  >{\raggedright\arraybackslash}p{(\linewidth - 4\tabcolsep) * \real{0.4146}}@{}}
\toprule\noalign{}
\begin{minipage}[b]{\linewidth}\raggedright
Feature
\end{minipage} & \begin{minipage}[b]{\linewidth}\raggedright
Example
\end{minipage} & \begin{minipage}[b]{\linewidth}\raggedright
Purpose
\end{minipage} \\
\midrule\noalign{}
\endhead
\bottomrule\noalign{}
\endlastfoot
Basic decorator & \texttt{@my\_decorator} & Add behavior before/after
function \\
With args & \texttt{def\ wrapper(*args,kwargs)} & Works with any
function signature \\
Multiple decorators & \texttt{@d1} + \texttt{@d2} & Stacks behaviors \\
\texttt{functools.wraps} & \texttt{@wraps(func)} & Preserve metadata \\
\end{longtable}

\subsubsection{Tiny Code}\label{tiny-code-85}

\begin{Shaded}
\begin{Highlighting}[]
\KeywordTok{def}\NormalTok{ uppercase(func):}
    \KeywordTok{def}\NormalTok{ wrapper():}
\NormalTok{        result }\OperatorTok{=}\NormalTok{ func()}
        \ControlFlowTok{return}\NormalTok{ result.upper()}
    \ControlFlowTok{return}\NormalTok{ wrapper}

\AttributeTok{@uppercase}
\KeywordTok{def}\NormalTok{ message():}
    \ControlFlowTok{return} \StringTok{"hello world"}

\BuiltInTok{print}\NormalTok{(message())   }\CommentTok{\# HELLO WORLD}
\end{Highlighting}
\end{Shaded}

\subsubsection{Why it Matters}\label{why-it-matters-85}

Decorators are a powerful way to separate what a function does from how
it's used. They make code reusable, clean, and Pythonic.

\subsubsection{Try It Yourself}\label{try-it-yourself-85}

\begin{enumerate}
\def\labelenumi{\arabic{enumi}.}
\tightlist
\item
  Write a decorator \texttt{@timer} that prints how long a function
  takes to run.
\item
  Create a decorator \texttt{@authenticate} that prints
  \texttt{"Access\ denied"} unless a variable
  \texttt{user\_logged\_in\ =\ True}.
\item
  Combine two decorators on the same function and observe the order of
  execution.
\item
  Use \texttt{functools.wraps} to keep the function's original
  \texttt{\_\_name\_\_}.
\end{enumerate}

\subsection{87. Context Managers
(Custom)}\label{context-managers-custom}

A context manager is a Python construct that properly manages resources,
like opening and closing files. You usually use it with the
\texttt{with} statement. While Python has built-in context managers
(like \texttt{open}), you can also create your own.

\subsubsection{Deep Dive}\label{deep-dive-86}

Using \texttt{with} Built-in

\begin{Shaded}
\begin{Highlighting}[]
\ControlFlowTok{with} \BuiltInTok{open}\NormalTok{(}\StringTok{"data.txt"}\NormalTok{, }\StringTok{"r"}\NormalTok{) }\ImportTok{as}\NormalTok{ f:}
\NormalTok{    content }\OperatorTok{=}\NormalTok{ f.read()}
\end{Highlighting}
\end{Shaded}

Here, \texttt{open} is a context manager: it opens the file, then
automatically closes it when done.

Creating a Custom Context Manager with a Class To make your own, define
\texttt{\_\_enter\_\_} and \texttt{\_\_exit\_\_}.

\begin{Shaded}
\begin{Highlighting}[]
\KeywordTok{class}\NormalTok{ MyResource:}
    \KeywordTok{def} \FunctionTok{\_\_enter\_\_}\NormalTok{(}\VariableTok{self}\NormalTok{):}
        \BuiltInTok{print}\NormalTok{(}\StringTok{"Resource acquired"}\NormalTok{)}
        \ControlFlowTok{return} \VariableTok{self}
    
    \KeywordTok{def} \FunctionTok{\_\_exit\_\_}\NormalTok{(}\VariableTok{self}\NormalTok{, exc\_type, exc\_value, traceback):}
        \BuiltInTok{print}\NormalTok{(}\StringTok{"Resource released"}\NormalTok{)}

\ControlFlowTok{with}\NormalTok{ MyResource() }\ImportTok{as}\NormalTok{ r:}
    \BuiltInTok{print}\NormalTok{(}\StringTok{"Using resource"}\NormalTok{)}
\end{Highlighting}
\end{Shaded}

Output:

\begin{verbatim}
Resource acquired
Using resource
Resource released
\end{verbatim}

Handling Errors in \texttt{\_\_exit\_\_} \texttt{\_\_exit\_\_} can
suppress exceptions if it returns \texttt{True}.

\begin{Shaded}
\begin{Highlighting}[]
\KeywordTok{class}\NormalTok{ SafeDivide:}
    \KeywordTok{def} \FunctionTok{\_\_enter\_\_}\NormalTok{(}\VariableTok{self}\NormalTok{):}
        \ControlFlowTok{return} \VariableTok{self}
    
    \KeywordTok{def} \FunctionTok{\_\_exit\_\_}\NormalTok{(}\VariableTok{self}\NormalTok{, exc\_type, exc\_value, traceback):}
        \ControlFlowTok{return} \VariableTok{True}   \CommentTok{\# suppress error}

\ControlFlowTok{with}\NormalTok{ SafeDivide():}
    \BuiltInTok{print}\NormalTok{(}\DecValTok{10} \OperatorTok{/} \DecValTok{0}\NormalTok{)   }\CommentTok{\# No crash!}
\end{Highlighting}
\end{Shaded}

Creating a Context Manager with \texttt{contextlib}

\begin{Shaded}
\begin{Highlighting}[]
\ImportTok{from}\NormalTok{ contextlib }\ImportTok{import}\NormalTok{ contextmanager}

\AttributeTok{@contextmanager}
\KeywordTok{def}\NormalTok{ managed\_resource():}
    \BuiltInTok{print}\NormalTok{(}\StringTok{"Start"}\NormalTok{)}
    \ControlFlowTok{yield}
    \BuiltInTok{print}\NormalTok{(}\StringTok{"End"}\NormalTok{)}

\ControlFlowTok{with}\NormalTok{ managed\_resource():}
    \BuiltInTok{print}\NormalTok{(}\StringTok{"Inside block"}\NormalTok{)}
\end{Highlighting}
\end{Shaded}

Output:

\begin{verbatim}
Start
Inside block
End
\end{verbatim}

Quick Summary Table

\begin{longtable}[]{@{}
  >{\raggedright\arraybackslash}p{(\linewidth - 4\tabcolsep) * \real{0.2073}}
  >{\raggedright\arraybackslash}p{(\linewidth - 4\tabcolsep) * \real{0.4268}}
  >{\raggedright\arraybackslash}p{(\linewidth - 4\tabcolsep) * \real{0.3659}}@{}}
\toprule\noalign{}
\begin{minipage}[b]{\linewidth}\raggedright
Method
\end{minipage} & \begin{minipage}[b]{\linewidth}\raggedright
How it Works
\end{minipage} & \begin{minipage}[b]{\linewidth}\raggedright
Example
\end{minipage} \\
\midrule\noalign{}
\endhead
\bottomrule\noalign{}
\endlastfoot
Class-based & Define \texttt{\_\_enter\_\_} and \texttt{\_\_exit\_\_} &
\texttt{with\ MyClass():\ ...} \\
Function-based & Use \texttt{@contextmanager} decorator &
\texttt{with\ managed\_resource():\ ...} \\
Built-in examples & \texttt{open}, \texttt{threading.Lock},
\texttt{sqlite3} & \texttt{with\ open("f.txt")\ as\ f:} \\
\end{longtable}

\subsubsection{Tiny Code}\label{tiny-code-86}

\begin{Shaded}
\begin{Highlighting}[]
\ImportTok{from}\NormalTok{ contextlib }\ImportTok{import}\NormalTok{ contextmanager}

\AttributeTok{@contextmanager}
\KeywordTok{def}\NormalTok{ open\_upper(filename):}
\NormalTok{    f }\OperatorTok{=} \BuiltInTok{open}\NormalTok{(filename, }\StringTok{"r"}\NormalTok{)}
    \ControlFlowTok{try}\NormalTok{:}
        \ControlFlowTok{yield}\NormalTok{ (line.upper() }\ControlFlowTok{for}\NormalTok{ line }\KeywordTok{in}\NormalTok{ f)}
    \ControlFlowTok{finally}\NormalTok{:}
\NormalTok{        f.close()}

\ControlFlowTok{with}\NormalTok{ open\_upper(}\StringTok{"data.txt"}\NormalTok{) }\ImportTok{as}\NormalTok{ lines:}
    \ControlFlowTok{for}\NormalTok{ line }\KeywordTok{in}\NormalTok{ lines:}
        \BuiltInTok{print}\NormalTok{(line)}
\end{Highlighting}
\end{Shaded}

\subsubsection{Why it Matters}\label{why-it-matters-86}

Custom context managers let you manage setup and cleanup tasks
automatically. They make code safer, reduce errors, and ensure resources
are always released properly.

\subsubsection{Try It Yourself}\label{try-it-yourself-86}

\begin{enumerate}
\def\labelenumi{\arabic{enumi}.}
\tightlist
\item
  Write a context manager class that prints \texttt{"Start"} when
  entering and \texttt{"End"} when exiting.
\item
  Create one that temporarily changes the working directory and restores
  it afterwards.
\item
  Use \texttt{@contextmanager} to make a timer context that prints how
  long the block took.
\item
  Build a safe database connection context that opens, yields, then
  closes automatically.
\end{enumerate}

\subsection{\texorpdfstring{88. \texttt{with} and Resource
Management}{88. with and Resource Management}}\label{with-and-resource-management}

The \texttt{with} statement in Python is a shortcut for using context
managers. It ensures resources (like files, network connections, or
locks) are acquired and released properly, even if errors occur.

\subsubsection{Deep Dive}\label{deep-dive-87}

File Handling with \texttt{with}

\begin{Shaded}
\begin{Highlighting}[]
\ControlFlowTok{with} \BuiltInTok{open}\NormalTok{(}\StringTok{"notes.txt"}\NormalTok{, }\StringTok{"w"}\NormalTok{) }\ImportTok{as}\NormalTok{ f:}
\NormalTok{    f.write(}\StringTok{"Hello, Python!"}\NormalTok{)}
\end{Highlighting}
\end{Shaded}

\begin{itemize}
\tightlist
\item
  File opens automatically.
\item
  File closes automatically after the block, even if an error happens.
\end{itemize}

Multiple Resources in One \texttt{with}

\begin{Shaded}
\begin{Highlighting}[]
\ControlFlowTok{with} \BuiltInTok{open}\NormalTok{(}\StringTok{"input.txt"}\NormalTok{, }\StringTok{"r"}\NormalTok{) }\ImportTok{as}\NormalTok{ infile, }\BuiltInTok{open}\NormalTok{(}\StringTok{"output.txt"}\NormalTok{, }\StringTok{"w"}\NormalTok{) }\ImportTok{as}\NormalTok{ outfile:}
    \ControlFlowTok{for}\NormalTok{ line }\KeywordTok{in}\NormalTok{ infile:}
\NormalTok{        outfile.write(line.upper())}
\end{Highlighting}
\end{Shaded}

Both files are managed safely within the same \texttt{with} statement.

Using \texttt{with} for Locks (Threading Example)

\begin{Shaded}
\begin{Highlighting}[]
\ImportTok{import}\NormalTok{ threading}

\NormalTok{lock }\OperatorTok{=}\NormalTok{ threading.Lock()}
\ControlFlowTok{with}\NormalTok{ lock:}
    \CommentTok{\# critical section}
    \BuiltInTok{print}\NormalTok{(}\StringTok{"Safe access"}\NormalTok{)}
\end{Highlighting}
\end{Shaded}

The lock is automatically acquired and released.

Database Connections Some libraries provide context managers for
connections.

\begin{Shaded}
\begin{Highlighting}[]
\ImportTok{import}\NormalTok{ sqlite3}

\ControlFlowTok{with}\NormalTok{ sqlite3.}\ExtensionTok{connect}\NormalTok{(}\StringTok{"example.db"}\NormalTok{) }\ImportTok{as}\NormalTok{ conn:}
\NormalTok{    cursor }\OperatorTok{=}\NormalTok{ conn.cursor()}
\NormalTok{    cursor.execute(}\StringTok{"CREATE TABLE IF NOT EXISTS users(id INTEGER)"}\NormalTok{)}
\end{Highlighting}
\end{Shaded}

Connection commits and closes automatically at the end.

Custom Resource Management Any class with \texttt{\_\_enter\_\_} and
\texttt{\_\_exit\_\_} can be used in a \texttt{with} block.

\begin{Shaded}
\begin{Highlighting}[]
\KeywordTok{class}\NormalTok{ Resource:}
    \KeywordTok{def} \FunctionTok{\_\_enter\_\_}\NormalTok{(}\VariableTok{self}\NormalTok{):}
        \BuiltInTok{print}\NormalTok{(}\StringTok{"Acquired resource"}\NormalTok{)}
        \ControlFlowTok{return} \VariableTok{self}
    \KeywordTok{def} \FunctionTok{\_\_exit\_\_}\NormalTok{(}\VariableTok{self}\NormalTok{, exc\_type, exc\_val, exc\_tb):}
        \BuiltInTok{print}\NormalTok{(}\StringTok{"Released resource"}\NormalTok{)}

\ControlFlowTok{with}\NormalTok{ Resource():}
    \BuiltInTok{print}\NormalTok{(}\StringTok{"Using resource"}\NormalTok{)}
\end{Highlighting}
\end{Shaded}

Output:

\begin{verbatim}
Acquired resource
Using resource
Released resource
\end{verbatim}

Quick Summary Table

\begin{longtable}[]{@{}
  >{\raggedright\arraybackslash}p{(\linewidth - 4\tabcolsep) * \real{0.2568}}
  >{\raggedright\arraybackslash}p{(\linewidth - 4\tabcolsep) * \real{0.4865}}
  >{\raggedright\arraybackslash}p{(\linewidth - 4\tabcolsep) * \real{0.2568}}@{}}
\toprule\noalign{}
\begin{minipage}[b]{\linewidth}\raggedright
Resource Type
\end{minipage} & \begin{minipage}[b]{\linewidth}\raggedright
Example with \texttt{with}
\end{minipage} & \begin{minipage}[b]{\linewidth}\raggedright
Benefit
\end{minipage} \\
\midrule\noalign{}
\endhead
\bottomrule\noalign{}
\endlastfoot
File & \texttt{with\ open("file.txt")\ as\ f:} & Auto-close file \\
Thread lock & \texttt{with\ lock:} & Auto-release lock \\
Database connection & \texttt{with\ sqlite3.connect(...)\ as\ conn:} &
Auto-commit \& close \\
Custom resource & \texttt{with\ MyResource():\ ...} & Custom cleanup \\
\end{longtable}

\subsubsection{Tiny Code}\label{tiny-code-87}

\begin{Shaded}
\begin{Highlighting}[]
\ControlFlowTok{with} \BuiltInTok{open}\NormalTok{(}\StringTok{"demo.txt"}\NormalTok{, }\StringTok{"w"}\NormalTok{) }\ImportTok{as}\NormalTok{ f:}
\NormalTok{    f.write(}\StringTok{"Resource managed with \textquotesingle{}with\textquotesingle{}"}\NormalTok{)}
\end{Highlighting}
\end{Shaded}

\subsubsection{Why it Matters}\label{why-it-matters-87}

Resource management is crucial to avoid memory leaks, file corruption,
or dangling connections. The \texttt{with} statement makes code safer,
cleaner, and more professional.

\subsubsection{Try It Yourself}\label{try-it-yourself-87}

\begin{enumerate}
\def\labelenumi{\arabic{enumi}.}
\tightlist
\item
  Write a \texttt{with\ open("data.txt",\ "r")} block that prints each
  line.
\item
  Use \texttt{with} to copy one file into another.
\item
  Create a threading lock and use it with \texttt{with} in a simple
  program.
\item
  Write a custom class with \texttt{\_\_enter\_\_} and
  \texttt{\_\_exit\_\_} that logs when it starts and stops.
\end{enumerate}

\subsection{\texorpdfstring{89. Modules \texttt{itertools} \&
\texttt{functools}}{89. Modules itertools \& functools}}\label{modules-itertools-functools}

Python provides \texttt{itertools} and \texttt{functools} as standard
libraries to work with iterators and functional programming tools. They
let you process data efficiently and write more expressive code.

\subsubsection{Deep Dive}\label{deep-dive-88}

\texttt{itertools} -- Tools for Iteration

\begin{itemize}
\tightlist
\item
  Infinite Iterators
\end{itemize}

\begin{Shaded}
\begin{Highlighting}[]
\ImportTok{import}\NormalTok{ itertools}

\NormalTok{counter }\OperatorTok{=}\NormalTok{ itertools.count(start}\OperatorTok{=}\DecValTok{1}\NormalTok{, step}\OperatorTok{=}\DecValTok{2}\NormalTok{)}
\BuiltInTok{print}\NormalTok{(}\BuiltInTok{next}\NormalTok{(counter))  }\CommentTok{\# 1}
\BuiltInTok{print}\NormalTok{(}\BuiltInTok{next}\NormalTok{(counter))  }\CommentTok{\# 3}
\end{Highlighting}
\end{Shaded}

\begin{itemize}
\tightlist
\item
  Cycling and Repeating
\end{itemize}

\begin{Shaded}
\begin{Highlighting}[]
\NormalTok{colors }\OperatorTok{=}\NormalTok{ itertools.cycle([}\StringTok{"red"}\NormalTok{, }\StringTok{"green"}\NormalTok{, }\StringTok{"blue"}\NormalTok{])}
\BuiltInTok{print}\NormalTok{(}\BuiltInTok{next}\NormalTok{(colors))  }\CommentTok{\# red}
\BuiltInTok{print}\NormalTok{(}\BuiltInTok{next}\NormalTok{(colors))  }\CommentTok{\# green}

\NormalTok{repeat\_hello }\OperatorTok{=}\NormalTok{ itertools.repeat(}\StringTok{"hello"}\NormalTok{, }\DecValTok{3}\NormalTok{)}
\BuiltInTok{print}\NormalTok{(}\BuiltInTok{list}\NormalTok{(repeat\_hello))  }\CommentTok{\# [\textquotesingle{}hello\textquotesingle{}, \textquotesingle{}hello\textquotesingle{}, \textquotesingle{}hello\textquotesingle{}]}
\end{Highlighting}
\end{Shaded}

\begin{itemize}
\tightlist
\item
  Combinatorics
\end{itemize}

\begin{Shaded}
\begin{Highlighting}[]
\ImportTok{from}\NormalTok{ itertools }\ImportTok{import}\NormalTok{ permutations, combinations}

\BuiltInTok{print}\NormalTok{(}\BuiltInTok{list}\NormalTok{(permutations([}\DecValTok{1}\NormalTok{, }\DecValTok{2}\NormalTok{, }\DecValTok{3}\NormalTok{], }\DecValTok{2}\NormalTok{)))}
\CommentTok{\# [(1, 2), (1, 3), (2, 1), (2, 3), (3, 1), (3, 2)]}

\BuiltInTok{print}\NormalTok{(}\BuiltInTok{list}\NormalTok{(combinations([}\DecValTok{1}\NormalTok{, }\DecValTok{2}\NormalTok{, }\DecValTok{3}\NormalTok{], }\DecValTok{2}\NormalTok{)))}
\CommentTok{\# [(1, 2), (1, 3), (2, 3)]}
\end{Highlighting}
\end{Shaded}

\begin{itemize}
\tightlist
\item
  Chaining Iterables
\end{itemize}

\begin{Shaded}
\begin{Highlighting}[]
\ImportTok{from}\NormalTok{ itertools }\ImportTok{import}\NormalTok{ chain}
\BuiltInTok{print}\NormalTok{(}\BuiltInTok{list}\NormalTok{(chain(}\StringTok{"ABC"}\NormalTok{, }\StringTok{"123"}\NormalTok{)))  }\CommentTok{\# [\textquotesingle{}A\textquotesingle{},\textquotesingle{}B\textquotesingle{},\textquotesingle{}C\textquotesingle{},\textquotesingle{}1\textquotesingle{},\textquotesingle{}2\textquotesingle{},\textquotesingle{}3\textquotesingle{}]}
\end{Highlighting}
\end{Shaded}

\texttt{functools} -- Tools for Functions

\begin{itemize}
\tightlist
\item
  \texttt{reduce} → apply a function cumulatively.
\end{itemize}

\begin{Shaded}
\begin{Highlighting}[]
\ImportTok{from}\NormalTok{ functools }\ImportTok{import} \BuiltInTok{reduce}

\NormalTok{nums }\OperatorTok{=}\NormalTok{ [}\DecValTok{1}\NormalTok{, }\DecValTok{2}\NormalTok{, }\DecValTok{3}\NormalTok{, }\DecValTok{4}\NormalTok{]}
\NormalTok{product }\OperatorTok{=} \BuiltInTok{reduce}\NormalTok{(}\KeywordTok{lambda}\NormalTok{ a, b: a }\OperatorTok{*}\NormalTok{ b, nums)}
\BuiltInTok{print}\NormalTok{(product)  }\CommentTok{\# 24}
\end{Highlighting}
\end{Shaded}

\begin{itemize}
\tightlist
\item
  \texttt{lru\_cache} → memoize function results.
\end{itemize}

\begin{Shaded}
\begin{Highlighting}[]
\ImportTok{from}\NormalTok{ functools }\ImportTok{import}\NormalTok{ lru\_cache}

\AttributeTok{@lru\_cache}\NormalTok{(maxsize}\OperatorTok{=}\VariableTok{None}\NormalTok{)}
\KeywordTok{def}\NormalTok{ fib(n):}
    \ControlFlowTok{if}\NormalTok{ n }\OperatorTok{\textless{}} \DecValTok{2}\NormalTok{:}
        \ControlFlowTok{return}\NormalTok{ n}
    \ControlFlowTok{return}\NormalTok{ fib(n}\OperatorTok{{-}}\DecValTok{1}\NormalTok{) }\OperatorTok{+}\NormalTok{ fib(n}\OperatorTok{{-}}\DecValTok{2}\NormalTok{)}

\BuiltInTok{print}\NormalTok{(fib(}\DecValTok{30}\NormalTok{))  }\CommentTok{\# fast due to caching}
\end{Highlighting}
\end{Shaded}

\begin{itemize}
\tightlist
\item
  \texttt{partial} → fix some arguments of a function.
\end{itemize}

\begin{Shaded}
\begin{Highlighting}[]
\ImportTok{from}\NormalTok{ functools }\ImportTok{import}\NormalTok{ partial}

\KeywordTok{def}\NormalTok{ power(base, exponent):}
    \ControlFlowTok{return}\NormalTok{ base  exponent}

\NormalTok{square }\OperatorTok{=}\NormalTok{ partial(power, exponent}\OperatorTok{=}\DecValTok{2}\NormalTok{)}
\BuiltInTok{print}\NormalTok{(square(}\DecValTok{5}\NormalTok{))  }\CommentTok{\# 25}
\end{Highlighting}
\end{Shaded}

Quick Summary Table

\begin{longtable}[]{@{}
  >{\raggedright\arraybackslash}p{(\linewidth - 6\tabcolsep) * \real{0.1125}}
  >{\raggedright\arraybackslash}p{(\linewidth - 6\tabcolsep) * \real{0.1750}}
  >{\raggedright\arraybackslash}p{(\linewidth - 6\tabcolsep) * \real{0.4250}}
  >{\raggedright\arraybackslash}p{(\linewidth - 6\tabcolsep) * \real{0.2875}}@{}}
\toprule\noalign{}
\begin{minipage}[b]{\linewidth}\raggedright
Module
\end{minipage} & \begin{minipage}[b]{\linewidth}\raggedright
Function
\end{minipage} & \begin{minipage}[b]{\linewidth}\raggedright
Example
\end{minipage} & \begin{minipage}[b]{\linewidth}\raggedright
Purpose
\end{minipage} \\
\midrule\noalign{}
\endhead
\bottomrule\noalign{}
\endlastfoot
itertools & \texttt{count} & \texttt{count(1,2)} & Infinite counter \\
itertools & \texttt{cycle} &
\texttt{cycle({[}\textquotesingle{}A\textquotesingle{},\textquotesingle{}B\textquotesingle{}{]})}
& Repeat sequence forever \\
itertools & \texttt{permutations} & \texttt{permutations({[}1,2,3{]},2)}
& All orderings \\
itertools & \texttt{combinations} & \texttt{combinations({[}1,2,3{]},2)}
& All unique pairs \\
functools & \texttt{reduce} &
\texttt{reduce(lambda\ x,y:\ x+y,\ {[}1,2,3{]})} & Cumulative
reduction \\
functools & \texttt{lru\_cache} & \texttt{@lru\_cache} & Cache results
for speed \\
functools & \texttt{partial} & \texttt{partial(func,\ arg=value)} &
Pre-fill arguments \\
\end{longtable}

\subsubsection{Tiny Code}\label{tiny-code-88}

\begin{Shaded}
\begin{Highlighting}[]
\ImportTok{from}\NormalTok{ itertools }\ImportTok{import}\NormalTok{ accumulate}
\BuiltInTok{print}\NormalTok{(}\BuiltInTok{list}\NormalTok{(accumulate([}\DecValTok{1}\NormalTok{, }\DecValTok{2}\NormalTok{, }\DecValTok{3}\NormalTok{, }\DecValTok{4}\NormalTok{])))  }\CommentTok{\# [1, 3, 6, 10]}
\end{Highlighting}
\end{Shaded}

\subsubsection{Why it Matters}\label{why-it-matters-88}

\texttt{itertools} and \texttt{functools} give you powerful building
blocks for iteration and function manipulation. They make complex tasks
simpler, faster, and more memory-efficient.

\subsubsection{Try It Yourself}\label{try-it-yourself-88}

\begin{enumerate}
\def\labelenumi{\arabic{enumi}.}
\tightlist
\item
  Use \texttt{itertools.combinations} to list all pairs from
  \texttt{{[}1,\ 2,\ 3,\ 4{]}}.
\item
  Create an infinite counter with \texttt{itertools.count()} and print
  the first 5 values.
\item
  Use \texttt{functools.reduce} to compute the sum of
  \texttt{{[}10,\ 20,\ 30{]}}.
\item
  Define a \texttt{cube} function using
  \texttt{functools.partial(power,\ exponent=3)}.
\end{enumerate}

\subsection{\texorpdfstring{90. Type Hints (\texttt{typing}
Module)}{90. Type Hints (typing Module)}}\label{type-hints-typing-module}

Type hints let you specify the expected data types of variables,
function arguments, and return values. They don't change how the code
runs, but they make it easier to read, maintain, and catch errors early
with tools like \texttt{mypy}.

\subsubsection{Deep Dive}\label{deep-dive-89}

Basic Function Hints

\begin{Shaded}
\begin{Highlighting}[]
\KeywordTok{def}\NormalTok{ greet(name: }\BuiltInTok{str}\NormalTok{) }\OperatorTok{{-}\textgreater{}} \BuiltInTok{str}\NormalTok{:}
    \ControlFlowTok{return} \StringTok{"Hello, "} \OperatorTok{+}\NormalTok{ name}
\end{Highlighting}
\end{Shaded}

\begin{itemize}
\tightlist
\item
  \texttt{name:\ str} means \texttt{name} should be a string.
\item
  \texttt{-\textgreater{}\ str} means the function returns a string.
\end{itemize}

Variable Hints

\begin{Shaded}
\begin{Highlighting}[]
\NormalTok{age: }\BuiltInTok{int} \OperatorTok{=} \DecValTok{25}
\NormalTok{pi: }\BuiltInTok{float} \OperatorTok{=} \FloatTok{3.14159}
\NormalTok{active: }\BuiltInTok{bool} \OperatorTok{=} \VariableTok{True}
\end{Highlighting}
\end{Shaded}

Using \texttt{List}, \texttt{Dict}, and \texttt{Tuple}

\begin{Shaded}
\begin{Highlighting}[]
\ImportTok{from}\NormalTok{ typing }\ImportTok{import}\NormalTok{ List, Dict, Tuple}

\NormalTok{numbers: List[}\BuiltInTok{int}\NormalTok{] }\OperatorTok{=}\NormalTok{ [}\DecValTok{1}\NormalTok{, }\DecValTok{2}\NormalTok{, }\DecValTok{3}\NormalTok{]}
\NormalTok{user: Dict[}\BuiltInTok{str}\NormalTok{, }\BuiltInTok{int}\NormalTok{] }\OperatorTok{=}\NormalTok{ \{}\StringTok{"Alice"}\NormalTok{: }\DecValTok{25}\NormalTok{, }\StringTok{"Bob"}\NormalTok{: }\DecValTok{30}\NormalTok{\}}
\NormalTok{point: Tuple[}\BuiltInTok{int}\NormalTok{, }\BuiltInTok{int}\NormalTok{] }\OperatorTok{=}\NormalTok{ (}\DecValTok{10}\NormalTok{, }\DecValTok{20}\NormalTok{)}
\end{Highlighting}
\end{Shaded}

Optional Values

\begin{Shaded}
\begin{Highlighting}[]
\ImportTok{from}\NormalTok{ typing }\ImportTok{import}\NormalTok{ Optional}

\KeywordTok{def}\NormalTok{ find\_user(}\BuiltInTok{id}\NormalTok{: }\BuiltInTok{int}\NormalTok{) }\OperatorTok{{-}\textgreater{}}\NormalTok{ Optional[}\BuiltInTok{str}\NormalTok{]:}
    \ControlFlowTok{if} \BuiltInTok{id} \OperatorTok{==} \DecValTok{1}\NormalTok{:}
        \ControlFlowTok{return} \StringTok{"Alice"}
    \ControlFlowTok{return} \VariableTok{None}
\end{Highlighting}
\end{Shaded}

Union Types

\begin{Shaded}
\begin{Highlighting}[]
\ImportTok{from}\NormalTok{ typing }\ImportTok{import}\NormalTok{ Union}

\KeywordTok{def}\NormalTok{ add(x: Union[}\BuiltInTok{int}\NormalTok{, }\BuiltInTok{float}\NormalTok{], y: Union[}\BuiltInTok{int}\NormalTok{, }\BuiltInTok{float}\NormalTok{]) }\OperatorTok{{-}\textgreater{}}\NormalTok{ Union[}\BuiltInTok{int}\NormalTok{, }\BuiltInTok{float}\NormalTok{]:}
    \ControlFlowTok{return}\NormalTok{ x }\OperatorTok{+}\NormalTok{ y}
\end{Highlighting}
\end{Shaded}

Type Aliases

\begin{Shaded}
\begin{Highlighting}[]
\NormalTok{UserID }\OperatorTok{=} \BuiltInTok{int}
\KeywordTok{def}\NormalTok{ get\_user(}\BuiltInTok{id}\NormalTok{: UserID) }\OperatorTok{{-}\textgreater{}} \BuiltInTok{str}\NormalTok{:}
    \ControlFlowTok{return} \StringTok{"User"} \OperatorTok{+} \BuiltInTok{str}\NormalTok{(}\BuiltInTok{id}\NormalTok{)}
\end{Highlighting}
\end{Shaded}

Callable (Functions as Arguments)

\begin{Shaded}
\begin{Highlighting}[]
\ImportTok{from}\NormalTok{ typing }\ImportTok{import}\NormalTok{ Callable}

\KeywordTok{def} \BuiltInTok{apply}\NormalTok{(func: Callable[[}\BuiltInTok{int}\NormalTok{, }\BuiltInTok{int}\NormalTok{], }\BuiltInTok{int}\NormalTok{], a: }\BuiltInTok{int}\NormalTok{, b: }\BuiltInTok{int}\NormalTok{) }\OperatorTok{{-}\textgreater{}} \BuiltInTok{int}\NormalTok{:}
    \ControlFlowTok{return}\NormalTok{ func(a, b)}

\BuiltInTok{print}\NormalTok{(}\BuiltInTok{apply}\NormalTok{(}\KeywordTok{lambda}\NormalTok{ x, y: x }\OperatorTok{+}\NormalTok{ y, }\DecValTok{2}\NormalTok{, }\DecValTok{3}\NormalTok{))  }\CommentTok{\# 5}
\end{Highlighting}
\end{Shaded}

Quick Summary Table

\begin{longtable}[]{@{}lll@{}}
\toprule\noalign{}
Type Hint & Example & Meaning \\
\midrule\noalign{}
\endhead
\bottomrule\noalign{}
\endlastfoot
Basic & \texttt{x:\ int}, \texttt{def\ f()-\textgreater{}str} & Simple
types \\
List, Dict & \texttt{List{[}int{]}}, \texttt{Dict{[}str,int{]}} &
Collections with types \\
Tuple & \texttt{Tuple{[}int,str{]}} & Fixed-size sequence \\
Optional & \texttt{Optional{[}str{]}} & String or None \\
Union & \texttt{Union{[}int,float{]}} & One of several types \\
Callable & \texttt{Callable{[}{[}int,int{]},int{]}} & Function type \\
Alias & \texttt{UserID\ =\ int} & Custom type name \\
\end{longtable}

\subsubsection{Tiny Code}\label{tiny-code-89}

\begin{Shaded}
\begin{Highlighting}[]
\ImportTok{from}\NormalTok{ typing }\ImportTok{import}\NormalTok{ List}

\KeywordTok{def}\NormalTok{ average(values: List[}\BuiltInTok{float}\NormalTok{]) }\OperatorTok{{-}\textgreater{}} \BuiltInTok{float}\NormalTok{:}
    \ControlFlowTok{return} \BuiltInTok{sum}\NormalTok{(values) }\OperatorTok{/} \BuiltInTok{len}\NormalTok{(values)}

\BuiltInTok{print}\NormalTok{(average([}\FloatTok{1.0}\NormalTok{, }\FloatTok{2.0}\NormalTok{, }\FloatTok{3.0}\NormalTok{]))  }\CommentTok{\# 2.0}
\end{Highlighting}
\end{Shaded}

\subsubsection{Why it Matters}\label{why-it-matters-89}

Type hints improve clarity and enable better error detection during
development. They help teams understand code faster and catch mistakes
before running the program.

\subsubsection{Try It Yourself}\label{try-it-yourself-89}

\begin{enumerate}
\def\labelenumi{\arabic{enumi}.}
\tightlist
\item
  Add type hints to a function \texttt{def\ square(x):\ return\ x*x}.
\item
  Write a function \texttt{join(names)} that expects a
  \texttt{List{[}str{]}} and returns a \texttt{str}.
\item
  Use \texttt{Optional{[}int{]}} for a function that may return
  \texttt{None}.
\item
  Create a function \texttt{operate} that accepts a
  \texttt{Callable{[}{[}int,int{]},int{]}} and applies it to two
  numbers.
\end{enumerate}

\section{Chapter 10. Python in
Practices}\label{chapter-10.-python-in-practices}

\subsection{91. REPL \& Interactive Mode}\label{repl-interactive-mode}

Python comes with an interactive environment called the REPL
(Read--Eval--Print Loop). It lets you type Python commands one at a time
and see results immediately, making it perfect for learning, testing,
and quick experiments.

\subsubsection{Deep Dive}\label{deep-dive-90}

Open a terminal and type:

\begin{Shaded}
\begin{Highlighting}[]
\ExtensionTok{python}
\end{Highlighting}
\end{Shaded}

or sometimes:

\begin{Shaded}
\begin{Highlighting}[]
\ExtensionTok{python3}
\end{Highlighting}
\end{Shaded}

You'll see a prompt like:

\begin{Shaded}
\begin{Highlighting}[]
\OperatorTok{\textgreater{}\textgreater{}\textgreater{}}
\end{Highlighting}
\end{Shaded}

where you can type Python code directly.

Basic Usage

\begin{Shaded}
\begin{Highlighting}[]
\OperatorTok{\textgreater{}\textgreater{}\textgreater{}} \DecValTok{2} \OperatorTok{+} \DecValTok{3}
\DecValTok{5}
\OperatorTok{\textgreater{}\textgreater{}\textgreater{}} \StringTok{"hello"}\NormalTok{.upper()}
\CommentTok{\textquotesingle{}HELLO\textquotesingle{}}
\end{Highlighting}
\end{Shaded}

The REPL evaluates each expression and prints the result instantly.

Multi-line Input For blocks like loops or functions, use indentation:

\begin{Shaded}
\begin{Highlighting}[]
\OperatorTok{\textgreater{}\textgreater{}\textgreater{}} \ControlFlowTok{for}\NormalTok{ i }\KeywordTok{in} \BuiltInTok{range}\NormalTok{(}\DecValTok{3}\NormalTok{):}
\NormalTok{...     }\BuiltInTok{print}\NormalTok{(i)}
\NormalTok{...}
\DecValTok{0}
\DecValTok{1}
\DecValTok{2}
\end{Highlighting}
\end{Shaded}

Exploring Objects You can quickly inspect functions and objects:

\begin{Shaded}
\begin{Highlighting}[]
\OperatorTok{\textgreater{}\textgreater{}\textgreater{}} \BuiltInTok{help}\NormalTok{(}\BuiltInTok{str}\NormalTok{)}
\OperatorTok{\textgreater{}\textgreater{}\textgreater{}} \BuiltInTok{dir}\NormalTok{(}\BuiltInTok{list}\NormalTok{)}
\end{Highlighting}
\end{Shaded}

Using the Underscore \texttt{\_} The REPL stores the last result in
\texttt{\_}:

\begin{Shaded}
\begin{Highlighting}[]
\OperatorTok{\textgreater{}\textgreater{}\textgreater{}} \DecValTok{5} \OperatorTok{*} \DecValTok{5}
\DecValTok{25}
\OperatorTok{\textgreater{}\textgreater{}\textgreater{}}\NormalTok{ \_ }\OperatorTok{+} \DecValTok{10}
\DecValTok{35}
\end{Highlighting}
\end{Shaded}

Exiting the REPL

\begin{itemize}
\tightlist
\item
  Press \texttt{Ctrl+D} (Linux/Mac) or \texttt{Ctrl+Z} + Enter
  (Windows).
\item
  Or type \texttt{exit()} or \texttt{quit()}.
\end{itemize}

Enhanced REPLs

\begin{itemize}
\tightlist
\item
  IPython → advanced REPL with colors, auto-complete, and history.
\item
  Jupyter Notebook → browser-based interactive coding environment.
\end{itemize}

Quick Summary Table

\begin{longtable}[]{@{}lll@{}}
\toprule\noalign{}
Feature & Example & Purpose \\
\midrule\noalign{}
\endhead
\bottomrule\noalign{}
\endlastfoot
Run REPL & \texttt{python} & Start interactive mode \\
Expression & \texttt{2\ +\ 3} → \texttt{5} & Immediate evaluation \\
Multi-line & \texttt{for\ i\ in\ ...} & Supports blocks of code \\
Inspect object & \texttt{dir(obj)}, \texttt{help(obj)} & Explore methods
\& docs \\
Last result & \texttt{\_} & Use last computed value \\
\end{longtable}

\subsubsection{Tiny Code}\label{tiny-code-90}

\begin{Shaded}
\begin{Highlighting}[]
\OperatorTok{\textgreater{}\textgreater{}\textgreater{}}\NormalTok{ x }\OperatorTok{=} \DecValTok{10}
\OperatorTok{\textgreater{}\textgreater{}\textgreater{}}\NormalTok{ y }\OperatorTok{=} \DecValTok{20}
\OperatorTok{\textgreater{}\textgreater{}\textgreater{}}\NormalTok{ x }\OperatorTok{+}\NormalTok{ y}
\DecValTok{30}
\OperatorTok{\textgreater{}\textgreater{}\textgreater{}}\NormalTok{ \_}
\DecValTok{30}
\OperatorTok{\textgreater{}\textgreater{}\textgreater{}}\NormalTok{ \_ }\OperatorTok{*} \DecValTok{2}
\DecValTok{60}
\end{Highlighting}
\end{Shaded}

\subsubsection{Why it Matters}\label{why-it-matters-90}

The REPL makes Python beginner-friendly and powerful for professionals.
It's like a live scratchpad where you can test ideas, debug small
snippets, or explore libraries interactively.

\subsubsection{Try It Yourself}\label{try-it-yourself-90}

\begin{enumerate}
\def\labelenumi{\arabic{enumi}.}
\tightlist
\item
  Start the Python REPL and calculate \texttt{7\ *\ 8}.
\item
  Use \texttt{help(int)} to see details about integers.
\item
  Assign a variable, then use \texttt{\_} to reuse its value.
\item
  Try an enhanced REPL like \texttt{ipython} for auto-completion.
\end{enumerate}

\subsection{\texorpdfstring{92. Debugging
(\texttt{pdb})}{92. Debugging (pdb)}}\label{debugging-pdb}

Python includes a built-in debugger called \texttt{pdb}. It allows you
to pause execution, step through code line by line, inspect variables,
and find bugs interactively.

\subsubsection{Deep Dive}\label{deep-dive-91}

Starting the Debugger Insert this line where you want to pause:

\begin{Shaded}
\begin{Highlighting}[]
\ImportTok{import}\NormalTok{ pdb}\OperatorTok{;}\NormalTok{ pdb.set\_trace()}
\end{Highlighting}
\end{Shaded}

When the program runs, it will stop there and open an interactive
debugging session.

Common \texttt{pdb} Commands

\begin{longtable}[]{@{}ll@{}}
\toprule\noalign{}
Command & Meaning \\
\midrule\noalign{}
\endhead
\bottomrule\noalign{}
\endlastfoot
\texttt{n} & Next line (step over) \\
\texttt{s} & Step into a function \\
\texttt{c} & Continue until next breakpoint \\
\texttt{l} & List source code around current line \\
\texttt{p\ var} & Print the value of \texttt{var} \\
\texttt{q} & Quit the debugger \\
\texttt{b\ num} & Set a breakpoint at line number \texttt{num} \\
\end{longtable}

Example Debugging Session

\begin{Shaded}
\begin{Highlighting}[]
\KeywordTok{def}\NormalTok{ divide(a, b):}
\NormalTok{    result }\OperatorTok{=}\NormalTok{ a }\OperatorTok{/}\NormalTok{ b}
    \ControlFlowTok{return}\NormalTok{ result}

\NormalTok{x }\OperatorTok{=} \DecValTok{10}
\NormalTok{y }\OperatorTok{=} \DecValTok{0}

\ImportTok{import}\NormalTok{ pdb}\OperatorTok{;}\NormalTok{ pdb.set\_trace()}
\BuiltInTok{print}\NormalTok{(divide(x, y))}
\end{Highlighting}
\end{Shaded}

When run:

\begin{verbatim}
(Pdb) p x
10
(Pdb) p y
0
(Pdb) n
ZeroDivisionError: division by zero
\end{verbatim}

Running a Script with Debug Mode You can also run the debugger directly
from the command line:

\begin{Shaded}
\begin{Highlighting}[]
\ExtensionTok{python} \AttributeTok{{-}m}\NormalTok{ pdb myscript.py}
\end{Highlighting}
\end{Shaded}

Modern Alternatives

\begin{itemize}
\tightlist
\item
  ipdb → improved pdb with colors and better interface.
\item
  debugpy → used in VS Code and IDEs for integrated debugging.
\end{itemize}

\subsubsection{Tiny Code}\label{tiny-code-91}

\begin{Shaded}
\begin{Highlighting}[]
\KeywordTok{def}\NormalTok{ greet(name):}
\NormalTok{    message }\OperatorTok{=} \StringTok{"Hello "} \OperatorTok{+}\NormalTok{ name}
    \ControlFlowTok{return}\NormalTok{ message}

\ImportTok{import}\NormalTok{ pdb}\OperatorTok{;}\NormalTok{ pdb.set\_trace()}
\BuiltInTok{print}\NormalTok{(greet(}\StringTok{"Alice"}\NormalTok{))}
\end{Highlighting}
\end{Shaded}

Inside pdb, type:

\begin{verbatim}
(Pdb) p name
(Pdb) n
\end{verbatim}

\subsubsection{Why it Matters}\label{why-it-matters-91}

Debugging with \texttt{pdb} helps you see \emph{exactly} what your
program is doing step by step. Instead of guessing where things go
wrong, you can inspect state directly and fix issues faster.

\subsubsection{Try It Yourself}\label{try-it-yourself-91}

\begin{enumerate}
\def\labelenumi{\arabic{enumi}.}
\tightlist
\item
  Write a function that divides two numbers and insert
  \texttt{pdb.set\_trace()} before the division. Step through and print
  variables.
\item
  Run a script with \texttt{python\ -m\ pdb\ file.py} and use \texttt{n}
  and \texttt{s} to move through code.
\item
  Try setting a breakpoint with \texttt{b} and continuing with
  \texttt{c}.
\item
  Experiment with inspecting variables using \texttt{p\ var} during
  debugging.
\end{enumerate}

\subsection{\texorpdfstring{93. Logging (\texttt{logging}
Module)}{93. Logging (logging Module)}}\label{logging-logging-module}

The \texttt{logging} module in Python is used to record messages about
what your program is doing. Unlike \texttt{print()}, logging is
flexible, configurable, and suitable for real-world applications.

\subsubsection{Deep Dive}\label{deep-dive-92}

Basic Logging

\begin{Shaded}
\begin{Highlighting}[]
\ImportTok{import}\NormalTok{ logging}

\NormalTok{logging.basicConfig(level}\OperatorTok{=}\NormalTok{logging.INFO)}
\NormalTok{logging.info(}\StringTok{"Program started"}\NormalTok{)}
\NormalTok{logging.warning(}\StringTok{"This is a warning"}\NormalTok{)}
\NormalTok{logging.error(}\StringTok{"An error occurred"}\NormalTok{)}
\end{Highlighting}
\end{Shaded}

Output:

\begin{verbatim}
INFO:root:Program started
WARNING:root:This is a warning
ERROR:root:An error occurred
\end{verbatim}

Log Levels Logging has different severity levels:

\begin{longtable}[]{@{}lll@{}}
\toprule\noalign{}
Level & Function & Meaning \\
\midrule\noalign{}
\endhead
\bottomrule\noalign{}
\endlastfoot
DEBUG & \texttt{logging.debug()} & Detailed information for devs \\
INFO & \texttt{logging.info()} & General program information \\
WARNING & \texttt{logging.warning()} & Something unexpected happened \\
ERROR & \texttt{logging.error()} & A serious problem occurred \\
CRITICAL & \texttt{logging.critical()} & Very severe error \\
\end{longtable}

Custom Formatting

\begin{Shaded}
\begin{Highlighting}[]
\NormalTok{logging.basicConfig(}
    \BuiltInTok{format}\OperatorTok{=}\StringTok{"}\SpecialCharTok{\%(asctime)s}\StringTok{ {-} }\SpecialCharTok{\%(levelname)s}\StringTok{ {-} }\SpecialCharTok{\%(message)s}\StringTok{"}\NormalTok{,}
\NormalTok{    level}\OperatorTok{=}\NormalTok{logging.DEBUG}
\NormalTok{)}

\NormalTok{logging.debug(}\StringTok{"Debugging details"}\NormalTok{)}
\end{Highlighting}
\end{Shaded}

Example output:

\begin{verbatim}
2025-09-14 19:30:01,234 - DEBUG - Debugging details
\end{verbatim}

Logging to a File

\begin{Shaded}
\begin{Highlighting}[]
\NormalTok{logging.basicConfig(filename}\OperatorTok{=}\StringTok{"app.log"}\NormalTok{, level}\OperatorTok{=}\NormalTok{logging.INFO)}
\NormalTok{logging.info(}\StringTok{"This message goes into the log file"}\NormalTok{)}
\end{Highlighting}
\end{Shaded}

Separate Logger Instances

\begin{Shaded}
\begin{Highlighting}[]
\NormalTok{logger }\OperatorTok{=}\NormalTok{ logging.getLogger(}\StringTok{"myapp"}\NormalTok{)}
\NormalTok{logger.setLevel(logging.DEBUG)}

\NormalTok{logger.info(}\StringTok{"App is running"}\NormalTok{)}
\end{Highlighting}
\end{Shaded}

Why Logging Instead of Print?

\begin{itemize}
\tightlist
\item
  \texttt{print()} always goes to stdout.
\item
  \texttt{logging} lets you choose where messages go: console, file,
  system log, etc.
\item
  You can control severity and disable logs without changing code.
\end{itemize}

Quick Summary Table

\begin{longtable}[]{@{}
  >{\raggedright\arraybackslash}p{(\linewidth - 4\tabcolsep) * \real{0.1970}}
  >{\raggedright\arraybackslash}p{(\linewidth - 4\tabcolsep) * \real{0.4848}}
  >{\raggedright\arraybackslash}p{(\linewidth - 4\tabcolsep) * \real{0.3182}}@{}}
\toprule\noalign{}
\begin{minipage}[b]{\linewidth}\raggedright
Feature
\end{minipage} & \begin{minipage}[b]{\linewidth}\raggedright
Example
\end{minipage} & \begin{minipage}[b]{\linewidth}\raggedright
Purpose
\end{minipage} \\
\midrule\noalign{}
\endhead
\bottomrule\noalign{}
\endlastfoot
Basic log & \texttt{logging.info("msg")} & Simple logging \\
Levels & \texttt{DEBUG}, \texttt{INFO}, \texttt{WARNING}, etc. & Control
importance \\
Formatting & \texttt{\%(asctime)s\ -\ \%(levelname)s...} & Add
timestamps, names \\
To file & \texttt{filename="app.log"} & Persist logs \\
Custom logger & \texttt{getLogger("name")} & Separate log sources \\
\end{longtable}

\subsubsection{Tiny Code}\label{tiny-code-92}

\begin{Shaded}
\begin{Highlighting}[]
\ImportTok{import}\NormalTok{ logging}

\NormalTok{logging.basicConfig(level}\OperatorTok{=}\NormalTok{logging.WARNING)}
\NormalTok{logging.debug(}\StringTok{"Hidden"}\NormalTok{)}
\NormalTok{logging.warning(}\StringTok{"Visible warning"}\NormalTok{)}
\end{Highlighting}
\end{Shaded}

\subsubsection{Why it Matters}\label{why-it-matters-92}

Logging is essential for debugging, monitoring, and auditing
applications. It helps you understand what your code does in production
without spamming users with print statements.

\subsubsection{Try It Yourself}\label{try-it-yourself-92}

\begin{enumerate}
\def\labelenumi{\arabic{enumi}.}
\tightlist
\item
  Write a script that logs an INFO message when it starts and an ERROR
  when something goes wrong.
\item
  Change log formatting to include the date and time.
\item
  Configure logging to write output to a file instead of the console.
\item
  Create two different loggers: one for \texttt{db} and one for
  \texttt{api}, with different log levels.
\end{enumerate}

\subsection{\texorpdfstring{94. Unit Testing
(\texttt{unittest})}{94. Unit Testing (unittest)}}\label{unit-testing-unittest}

Python's \texttt{unittest} module provides a framework for writing and
running automated tests. It helps you verify that your code works as
expected and prevents future changes from breaking existing
functionality.

\subsubsection{Deep Dive}\label{deep-dive-93}

Basic Test Case

\begin{Shaded}
\begin{Highlighting}[]
\ImportTok{import}\NormalTok{ unittest}

\KeywordTok{def}\NormalTok{ add(a, b):}
    \ControlFlowTok{return}\NormalTok{ a }\OperatorTok{+}\NormalTok{ b}

\KeywordTok{class}\NormalTok{ TestMath(unittest.TestCase):}
    \KeywordTok{def}\NormalTok{ test\_add(}\VariableTok{self}\NormalTok{):}
        \VariableTok{self}\NormalTok{.assertEqual(add(}\DecValTok{2}\NormalTok{, }\DecValTok{3}\NormalTok{), }\DecValTok{5}\NormalTok{)}

\ControlFlowTok{if} \VariableTok{\_\_name\_\_} \OperatorTok{==} \StringTok{"\_\_main\_\_"}\NormalTok{:}
\NormalTok{    unittest.main()}
\end{Highlighting}
\end{Shaded}

Running the script:

\begin{verbatim}
.
----------------------------------------------------------------------
Ran 1 test in 0.000s

OK
\end{verbatim}

Common Assertions

\begin{longtable}[]{@{}
  >{\raggedright\arraybackslash}p{(\linewidth - 4\tabcolsep) * \real{0.2651}}
  >{\raggedright\arraybackslash}p{(\linewidth - 4\tabcolsep) * \real{0.4458}}
  >{\raggedright\arraybackslash}p{(\linewidth - 4\tabcolsep) * \real{0.2892}}@{}}
\toprule\noalign{}
\begin{minipage}[b]{\linewidth}\raggedright
Method
\end{minipage} & \begin{minipage}[b]{\linewidth}\raggedright
Usage Example
\end{minipage} & \begin{minipage}[b]{\linewidth}\raggedright
Purpose
\end{minipage} \\
\midrule\noalign{}
\endhead
\bottomrule\noalign{}
\endlastfoot
\texttt{assertEqual(a,\ b)} & \texttt{assertEqual(x,\ 10)} & Check
equality \\
\texttt{assertNotEqual(a,\ b)} & \texttt{assertNotEqual(x,\ 5)} & Check
inequality \\
\texttt{assertTrue(x)} & \texttt{assertTrue(flag)} & Check condition is
True \\
\texttt{assertFalse(x)} & \texttt{assertFalse(flag)} & Check condition
is False \\
\texttt{assertIn(a,\ b)} & \texttt{assertIn(3,\ {[}1,2,3{]})} & Check
membership \\
\texttt{assertRaises(error)} &
\texttt{with\ self.assertRaises(ValueError):} & Check exception
raised \\
\end{longtable}

Testing Exceptions

\begin{Shaded}
\begin{Highlighting}[]
\KeywordTok{def}\NormalTok{ divide(a, b):}
    \ControlFlowTok{if}\NormalTok{ b }\OperatorTok{==} \DecValTok{0}\NormalTok{:}
        \ControlFlowTok{raise} \PreprocessorTok{ValueError}\NormalTok{(}\StringTok{"Cannot divide by zero"}\NormalTok{)}
    \ControlFlowTok{return}\NormalTok{ a }\OperatorTok{/}\NormalTok{ b}

\KeywordTok{class}\NormalTok{ TestDivide(unittest.TestCase):}
    \KeywordTok{def}\NormalTok{ test\_zero\_division(}\VariableTok{self}\NormalTok{):}
        \ControlFlowTok{with} \VariableTok{self}\NormalTok{.assertRaises(}\PreprocessorTok{ValueError}\NormalTok{):}
\NormalTok{            divide(}\DecValTok{5}\NormalTok{, }\DecValTok{0}\NormalTok{)}
\end{Highlighting}
\end{Shaded}

Grouping Multiple Tests

\begin{Shaded}
\begin{Highlighting}[]
\KeywordTok{class}\NormalTok{ TestStrings(unittest.TestCase):}
    \KeywordTok{def}\NormalTok{ test\_upper(}\VariableTok{self}\NormalTok{):}
        \VariableTok{self}\NormalTok{.assertEqual(}\StringTok{"hello"}\NormalTok{.upper(), }\StringTok{"HELLO"}\NormalTok{)}
    
    \KeywordTok{def}\NormalTok{ test\_isupper(}\VariableTok{self}\NormalTok{):}
        \VariableTok{self}\NormalTok{.assertTrue(}\StringTok{"HELLO"}\NormalTok{.isupper())}
        \VariableTok{self}\NormalTok{.assertFalse(}\StringTok{"Hello"}\NormalTok{.isupper())}
\end{Highlighting}
\end{Shaded}

Running Tests

\begin{itemize}
\tightlist
\item
  Run directly:
\end{itemize}

\begin{Shaded}
\begin{Highlighting}[]
\ExtensionTok{python}\NormalTok{ test\_file.py}
\end{Highlighting}
\end{Shaded}

\begin{itemize}
\tightlist
\item
  Or use:
\end{itemize}

\begin{Shaded}
\begin{Highlighting}[]
\ExtensionTok{python} \AttributeTok{{-}m}\NormalTok{ unittest discover}
\end{Highlighting}
\end{Shaded}

Quick Summary Table

\begin{longtable}[]{@{}
  >{\raggedright\arraybackslash}p{(\linewidth - 4\tabcolsep) * \real{0.2267}}
  >{\raggedright\arraybackslash}p{(\linewidth - 4\tabcolsep) * \real{0.4267}}
  >{\raggedright\arraybackslash}p{(\linewidth - 4\tabcolsep) * \real{0.3467}}@{}}
\toprule\noalign{}
\begin{minipage}[b]{\linewidth}\raggedright
Feature
\end{minipage} & \begin{minipage}[b]{\linewidth}\raggedright
Example
\end{minipage} & \begin{minipage}[b]{\linewidth}\raggedright
Purpose
\end{minipage} \\
\midrule\noalign{}
\endhead
\bottomrule\noalign{}
\endlastfoot
Test class & \texttt{class\ TestX(unittest.TestCase)} & Group related
tests \\
Assertion methods & \texttt{assertEqual}, \texttt{assertTrue} & Validate
expected behavior \\
Exception testing & \texttt{assertRaises} & Check error handling \\
Discover tests & \texttt{unittest\ discover} & Auto-run all tests \\
\end{longtable}

\subsubsection{Tiny Code}\label{tiny-code-93}

\begin{Shaded}
\begin{Highlighting}[]
\ImportTok{import}\NormalTok{ unittest}

\KeywordTok{class}\NormalTok{ TestBasics(unittest.TestCase):}
    \KeywordTok{def}\NormalTok{ test\_sum(}\VariableTok{self}\NormalTok{):}
        \VariableTok{self}\NormalTok{.assertEqual(}\BuiltInTok{sum}\NormalTok{([}\DecValTok{1}\NormalTok{,}\DecValTok{2}\NormalTok{,}\DecValTok{3}\NormalTok{]), }\DecValTok{6}\NormalTok{)}

\ControlFlowTok{if} \VariableTok{\_\_name\_\_} \OperatorTok{==} \StringTok{"\_\_main\_\_"}\NormalTok{:}
\NormalTok{    unittest.main()}
\end{Highlighting}
\end{Shaded}

\subsubsection{Why it Matters}\label{why-it-matters-93}

Unit tests catch bugs early, make code safer to change, and provide
confidence that your program works correctly. They are a cornerstone of
professional software development.

\subsubsection{Try It Yourself}\label{try-it-yourself-93}

\begin{enumerate}
\def\labelenumi{\arabic{enumi}.}
\tightlist
\item
  Write a function \texttt{multiply(a,\ b)} and a test to check
  \texttt{multiply(2,\ 5)\ ==\ 10}.
\item
  Add a test that verifies dividing by zero raises a
  \texttt{ValueError}.
\item
  Test that \texttt{"python".upper()} returns \texttt{"PYTHON"}.
\item
  Run your tests with \texttt{python\ -m\ unittest}.
\end{enumerate}

\subsection{95. Virtual Environments Best
Practice}\label{virtual-environments-best-practice}

A virtual environment is an isolated Python environment that allows you
to install packages without affecting the system-wide Python
installation. It's the best practice for managing dependencies in
projects.

\subsubsection{Deep Dive}\label{deep-dive-94}

Why Use Virtual Environments?

\begin{itemize}
\tightlist
\item
  Avoid conflicts between project dependencies.
\item
  Keep each project self-contained.
\item
  Easier to reproduce the same setup on another machine.
\end{itemize}

Creating a Virtual Environment

\begin{Shaded}
\begin{Highlighting}[]
\ExtensionTok{python} \AttributeTok{{-}m}\NormalTok{ venv venv}
\end{Highlighting}
\end{Shaded}

This creates a folder \texttt{venv/} that holds the environment.

Activating the Virtual Environment

\begin{itemize}
\tightlist
\item
  Linux / macOS:
\end{itemize}

\begin{Shaded}
\begin{Highlighting}[]
\BuiltInTok{source}\NormalTok{ venv/bin/activate}
\end{Highlighting}
\end{Shaded}

\begin{itemize}
\tightlist
\item
  Windows (cmd):
\end{itemize}

\begin{Shaded}
\begin{Highlighting}[]
\ExtensionTok{venv\textbackslash{}Scripts\textbackslash{}activate}
\end{Highlighting}
\end{Shaded}

After activation, your shell prompt changes, e.g.:

\begin{verbatim}
(venv) $
\end{verbatim}

Installing Packages Inside the environment, install packages as usual:

\begin{Shaded}
\begin{Highlighting}[]
\ExtensionTok{pip}\NormalTok{ install requests}
\end{Highlighting}
\end{Shaded}

Only this environment will have it.

Freezing Requirements Save dependencies to a file:

\begin{Shaded}
\begin{Highlighting}[]
\ExtensionTok{pip}\NormalTok{ freeze }\OperatorTok{\textgreater{}}\NormalTok{ requirements.txt}
\end{Highlighting}
\end{Shaded}

Reinstall them elsewhere:

\begin{Shaded}
\begin{Highlighting}[]
\ExtensionTok{pip}\NormalTok{ install }\AttributeTok{{-}r}\NormalTok{ requirements.txt}
\end{Highlighting}
\end{Shaded}

Deactivating the Environment

\begin{Shaded}
\begin{Highlighting}[]
\ExtensionTok{deactivate}
\end{Highlighting}
\end{Shaded}

Best Practices

\begin{enumerate}
\def\labelenumi{\arabic{enumi}.}
\tightlist
\item
  Always create a virtual environment for new projects.
\item
  Use \texttt{requirements.txt} for reproducibility.
\item
  Don't commit the \texttt{venv/} folder to version control.
\item
  Consider tools like pipenv or poetry for advanced dependency
  management.
\end{enumerate}

Quick Summary Table

\begin{longtable}[]{@{}ll@{}}
\toprule\noalign{}
Command & Purpose \\
\midrule\noalign{}
\endhead
\bottomrule\noalign{}
\endlastfoot
\texttt{python\ -m\ venv\ venv} & Create environment \\
\texttt{source\ venv/bin/activate} & Activate (Linux/macOS) \\
\texttt{venv\textbackslash{}Scripts\textbackslash{}activate} & Activate
(Windows) \\
\texttt{pip\ install\ package} & Install inside environment \\
\texttt{pip\ freeze\ \textgreater{}\ requirements.txt} & Save
dependencies \\
\texttt{deactivate} & Exit environment \\
\end{longtable}

\subsubsection{Tiny Code}\label{tiny-code-94}

\begin{Shaded}
\begin{Highlighting}[]
\ExtensionTok{python} \AttributeTok{{-}m}\NormalTok{ venv venv}
\BuiltInTok{source}\NormalTok{ venv/bin/activate}
\ExtensionTok{pip}\NormalTok{ install flask}
\ExtensionTok{pip}\NormalTok{ freeze }\OperatorTok{\textgreater{}}\NormalTok{ requirements.txt}
\end{Highlighting}
\end{Shaded}

\subsubsection{Why it Matters}\label{why-it-matters-94}

Virtual environments prevent dependency chaos. They ensure that one
project's libraries don't break another's, making projects portable and
maintainable.

\subsubsection{Try It Yourself}\label{try-it-yourself-94}

\begin{enumerate}
\def\labelenumi{\arabic{enumi}.}
\tightlist
\item
  Create a virtual environment called \texttt{myenv}.
\item
  Activate it and install the package \texttt{requests}.
\item
  Run \texttt{pip\ freeze} to confirm the installed package.
\item
  Deactivate the environment, then reactivate it.
\end{enumerate}

\subsection{96. Writing a Simple Script}\label{writing-a-simple-script}

Python scripts are just plain text files with \texttt{.py} extension.
They can contain functions, logic, and be executed directly from the
command line.

\subsubsection{Deep Dive}\label{deep-dive-95}

Hello World Script Create a file \texttt{hello.py}:

\begin{Shaded}
\begin{Highlighting}[]
\BuiltInTok{print}\NormalTok{(}\StringTok{"Hello, Python script!"}\NormalTok{)}
\end{Highlighting}
\end{Shaded}

Run it:

\begin{Shaded}
\begin{Highlighting}[]
\ExtensionTok{python}\NormalTok{ hello.py}
\end{Highlighting}
\end{Shaded}

Using \texttt{if\ \_\_name\_\_\ ==\ "\_\_main\_\_":} This ensures some
code only runs when the file is executed directly, not when imported as
a module.

\begin{Shaded}
\begin{Highlighting}[]
\KeywordTok{def}\NormalTok{ greet(name):}
    \ControlFlowTok{return} \SpecialStringTok{f"Hello, }\SpecialCharTok{\{}\NormalTok{name}\SpecialCharTok{\}}\SpecialStringTok{!"}

\ControlFlowTok{if} \VariableTok{\_\_name\_\_} \OperatorTok{==} \StringTok{"\_\_main\_\_"}\NormalTok{:}
    \BuiltInTok{print}\NormalTok{(greet(}\StringTok{"Alice"}\NormalTok{))}
\end{Highlighting}
\end{Shaded}

Running \texttt{python\ hello.py} prints:

\begin{verbatim}
Hello, Alice!
\end{verbatim}

But if you import it in another file:

\begin{Shaded}
\begin{Highlighting}[]
\ImportTok{import}\NormalTok{ hello}
\BuiltInTok{print}\NormalTok{(hello.greet(}\StringTok{"Bob"}\NormalTok{))}
\end{Highlighting}
\end{Shaded}

It won't run the main block automatically.

Accepting Command-Line Arguments Use the \texttt{sys} module:

\begin{Shaded}
\begin{Highlighting}[]
\ImportTok{import}\NormalTok{ sys}

\NormalTok{name }\OperatorTok{=}\NormalTok{ sys.argv[}\DecValTok{1}\NormalTok{] }\ControlFlowTok{if} \BuiltInTok{len}\NormalTok{(sys.argv) }\OperatorTok{\textgreater{}} \DecValTok{1} \ControlFlowTok{else} \StringTok{"World"}
\BuiltInTok{print}\NormalTok{(}\SpecialStringTok{f"Hello, }\SpecialCharTok{\{}\NormalTok{name}\SpecialCharTok{\}}\SpecialStringTok{!"}\NormalTok{)}
\end{Highlighting}
\end{Shaded}

Run it:

\begin{Shaded}
\begin{Highlighting}[]
\ExtensionTok{python}\NormalTok{ script.py Alice}
\CommentTok{\# Output: Hello, Alice!}
\end{Highlighting}
\end{Shaded}

Making the Script Executable (Linux/macOS) At the top of the file:

\begin{Shaded}
\begin{Highlighting}[]
\CommentTok{\#!/usr/bin/env python3}
\end{Highlighting}
\end{Shaded}

Then give permission:

\begin{Shaded}
\begin{Highlighting}[]
\FunctionTok{chmod}\NormalTok{ +x hello.py}
\ExtensionTok{./hello.py}
\end{Highlighting}
\end{Shaded}

Quick Summary Table

\begin{longtable}[]{@{}
  >{\raggedright\arraybackslash}p{(\linewidth - 4\tabcolsep) * \real{0.2727}}
  >{\raggedright\arraybackslash}p{(\linewidth - 4\tabcolsep) * \real{0.4205}}
  >{\raggedright\arraybackslash}p{(\linewidth - 4\tabcolsep) * \real{0.3068}}@{}}
\toprule\noalign{}
\begin{minipage}[b]{\linewidth}\raggedright
Concept
\end{minipage} & \begin{minipage}[b]{\linewidth}\raggedright
Example
\end{minipage} & \begin{minipage}[b]{\linewidth}\raggedright
Purpose
\end{minipage} \\
\midrule\noalign{}
\endhead
\bottomrule\noalign{}
\endlastfoot
Simple script & \texttt{print("Hello")} & First step in scripting \\
Main guard & \texttt{if\ \_\_name\_\_\ ==\ "\_\_main\_\_":} & Control
script vs import \\
Command-line arguments & \texttt{sys.argv} & Pass input via terminal \\
Executable script (Unix) & \texttt{\#!/usr/bin/env\ python3} +
\texttt{chmod\ +x} & Run without \texttt{python} prefix \\
\end{longtable}

\subsubsection{Tiny Code}\label{tiny-code-95}

\begin{Shaded}
\begin{Highlighting}[]
\ImportTok{import}\NormalTok{ sys}

\KeywordTok{def}\NormalTok{ square(n: }\BuiltInTok{int}\NormalTok{) }\OperatorTok{{-}\textgreater{}} \BuiltInTok{int}\NormalTok{:}
    \ControlFlowTok{return}\NormalTok{ n }\OperatorTok{*}\NormalTok{ n}

\ControlFlowTok{if} \VariableTok{\_\_name\_\_} \OperatorTok{==} \StringTok{"\_\_main\_\_"}\NormalTok{:}
\NormalTok{    num }\OperatorTok{=} \BuiltInTok{int}\NormalTok{(sys.argv[}\DecValTok{1}\NormalTok{]) }\ControlFlowTok{if} \BuiltInTok{len}\NormalTok{(sys.argv) }\OperatorTok{\textgreater{}} \DecValTok{1} \ControlFlowTok{else} \DecValTok{5}
    \BuiltInTok{print}\NormalTok{(}\SpecialStringTok{f"Square of }\SpecialCharTok{\{}\NormalTok{num}\SpecialCharTok{\}}\SpecialStringTok{ is }\SpecialCharTok{\{}\NormalTok{square(num)}\SpecialCharTok{\}}\SpecialStringTok{"}\NormalTok{)}
\end{Highlighting}
\end{Shaded}

\subsubsection{Why it Matters}\label{why-it-matters-95}

Scripts turn Python into a tool for automation. With just a few lines,
you can create utilities, batch jobs, or prototypes that are reusable
and shareable.

\subsubsection{Try It Yourself}\label{try-it-yourself-95}

\begin{enumerate}
\def\labelenumi{\arabic{enumi}.}
\tightlist
\item
  Write a script \texttt{greet.py} that prints
  \texttt{"Hello,\ Python\ learner!"}.
\item
  Add a function \texttt{double(x)} and use
  \texttt{if\ \_\_name\_\_\ ==\ "\_\_main\_\_":} to call it.
\item
  Modify the script to accept a number from the command line.
\item
  Make the script executable on Linux/macOS with a shebang line.
\end{enumerate}

\subsection{\texorpdfstring{97. CLI Arguments
(\texttt{argparse})}{97. CLI Arguments (argparse)}}\label{cli-arguments-argparse}

Python's \texttt{argparse} module makes it easy to build user-friendly
command-line interfaces (CLI). Instead of manually reading
\texttt{sys.argv}, you can define arguments, defaults, help text, and
parsing rules automatically.

\subsubsection{Deep Dive}\label{deep-dive-96}

Basic Example

\begin{Shaded}
\begin{Highlighting}[]
\ImportTok{import}\NormalTok{ argparse}

\NormalTok{parser }\OperatorTok{=}\NormalTok{ argparse.ArgumentParser()}
\NormalTok{parser.add\_argument(}\StringTok{"name"}\NormalTok{)}
\NormalTok{args }\OperatorTok{=}\NormalTok{ parser.parse\_args()}

\BuiltInTok{print}\NormalTok{(}\SpecialStringTok{f"Hello, }\SpecialCharTok{\{}\NormalTok{args}\SpecialCharTok{.}\NormalTok{name}\SpecialCharTok{\}}\SpecialStringTok{!"}\NormalTok{)}
\end{Highlighting}
\end{Shaded}

Run:

\begin{Shaded}
\begin{Highlighting}[]
\ExtensionTok{python}\NormalTok{ script.py Alice}
\CommentTok{\# Output: Hello, Alice!}
\end{Highlighting}
\end{Shaded}

Optional Arguments

\begin{Shaded}
\begin{Highlighting}[]
\NormalTok{parser }\OperatorTok{=}\NormalTok{ argparse.ArgumentParser()}
\NormalTok{parser.add\_argument(}\StringTok{"{-}{-}age"}\NormalTok{, }\BuiltInTok{type}\OperatorTok{=}\BuiltInTok{int}\NormalTok{, default}\OperatorTok{=}\DecValTok{18}\NormalTok{, }\BuiltInTok{help}\OperatorTok{=}\StringTok{"Your age"}\NormalTok{)}
\NormalTok{args }\OperatorTok{=}\NormalTok{ parser.parse\_args()}

\BuiltInTok{print}\NormalTok{(}\SpecialStringTok{f"Age: }\SpecialCharTok{\{}\NormalTok{args}\SpecialCharTok{.}\NormalTok{age}\SpecialCharTok{\}}\SpecialStringTok{"}\NormalTok{)}
\end{Highlighting}
\end{Shaded}

Run:

\begin{Shaded}
\begin{Highlighting}[]
\ExtensionTok{python}\NormalTok{ script.py }\AttributeTok{{-}{-}age}\NormalTok{ 25}
\CommentTok{\# Output: Age: 25}
\end{Highlighting}
\end{Shaded}

Multiple Arguments

\begin{Shaded}
\begin{Highlighting}[]
\NormalTok{parser }\OperatorTok{=}\NormalTok{ argparse.ArgumentParser()}
\NormalTok{parser.add\_argument(}\StringTok{"x"}\NormalTok{, }\BuiltInTok{type}\OperatorTok{=}\BuiltInTok{int}\NormalTok{)}
\NormalTok{parser.add\_argument(}\StringTok{"y"}\NormalTok{, }\BuiltInTok{type}\OperatorTok{=}\BuiltInTok{int}\NormalTok{)}
\NormalTok{args }\OperatorTok{=}\NormalTok{ parser.parse\_args()}

\BuiltInTok{print}\NormalTok{(args.x }\OperatorTok{+}\NormalTok{ args.y)}
\end{Highlighting}
\end{Shaded}

Run:

\begin{Shaded}
\begin{Highlighting}[]
\ExtensionTok{python}\NormalTok{ script.py 5 7}
\CommentTok{\# Output: 12}
\end{Highlighting}
\end{Shaded}

Flags (True/False)

\begin{Shaded}
\begin{Highlighting}[]
\NormalTok{parser }\OperatorTok{=}\NormalTok{ argparse.ArgumentParser()}
\NormalTok{parser.add\_argument(}\StringTok{"{-}{-}verbose"}\NormalTok{, action}\OperatorTok{=}\StringTok{"store\_true"}\NormalTok{)}
\NormalTok{args }\OperatorTok{=}\NormalTok{ parser.parse\_args()}

\ControlFlowTok{if}\NormalTok{ args.verbose:}
    \BuiltInTok{print}\NormalTok{(}\StringTok{"Verbose mode on"}\NormalTok{)}
\end{Highlighting}
\end{Shaded}

Run:

\begin{Shaded}
\begin{Highlighting}[]
\ExtensionTok{python}\NormalTok{ script.py }\AttributeTok{{-}{-}verbose}
\CommentTok{\# Output: Verbose mode on}
\end{Highlighting}
\end{Shaded}

Quick Summary Table

\begin{longtable}[]{@{}
  >{\raggedright\arraybackslash}p{(\linewidth - 4\tabcolsep) * \real{0.2206}}
  >{\raggedright\arraybackslash}p{(\linewidth - 4\tabcolsep) * \real{0.4265}}
  >{\raggedright\arraybackslash}p{(\linewidth - 4\tabcolsep) * \real{0.3529}}@{}}
\toprule\noalign{}
\begin{minipage}[b]{\linewidth}\raggedright
Feature
\end{minipage} & \begin{minipage}[b]{\linewidth}\raggedright
Example
\end{minipage} & \begin{minipage}[b]{\linewidth}\raggedright
Purpose
\end{minipage} \\
\midrule\noalign{}
\endhead
\bottomrule\noalign{}
\endlastfoot
Positional arg & \texttt{parser.add\_argument("name")} & Required
input \\
Optional arg & \texttt{-\/-age\ 25} & Extra info with defaults \\
Type conversion & \texttt{type=int} & Enforce types \\
Flags & \texttt{action="store\_true"} & On/off switches \\
Help text & \texttt{help="description"} & User guidance \\
\end{longtable}

\subsubsection{Tiny Code}\label{tiny-code-96}

\begin{Shaded}
\begin{Highlighting}[]
\ImportTok{import}\NormalTok{ argparse}

\NormalTok{parser }\OperatorTok{=}\NormalTok{ argparse.ArgumentParser(description}\OperatorTok{=}\StringTok{"Square a number"}\NormalTok{)}
\NormalTok{parser.add\_argument(}\StringTok{"num"}\NormalTok{, }\BuiltInTok{type}\OperatorTok{=}\BuiltInTok{int}\NormalTok{, }\BuiltInTok{help}\OperatorTok{=}\StringTok{"The number to square"}\NormalTok{)}
\NormalTok{args }\OperatorTok{=}\NormalTok{ parser.parse\_args()}

\BuiltInTok{print}\NormalTok{(args.num  }\DecValTok{2}\NormalTok{)}
\end{Highlighting}
\end{Shaded}

Run:

\begin{Shaded}
\begin{Highlighting}[]
\ExtensionTok{python}\NormalTok{ script.py 6}
\CommentTok{\# Output: 36}
\end{Highlighting}
\end{Shaded}

\subsubsection{Why it Matters}\label{why-it-matters-96}

With \texttt{argparse}, your Python scripts behave like real
command-line tools. They're more professional, self-documenting, and
easier to use.

\subsubsection{Try It Yourself}\label{try-it-yourself-96}

\begin{enumerate}
\def\labelenumi{\arabic{enumi}.}
\tightlist
\item
  Write a script that accepts a \texttt{-\/-name} argument and prints a
  greeting.
\item
  Add a \texttt{-\/-times} argument that repeats the greeting multiple
  times.
\item
  Create a flag \texttt{-\/-shout} that prints the greeting in
  uppercase.
\item
  Run your script with \texttt{-h} to see the auto-generated help
  message.
\end{enumerate}

\subsection{\texorpdfstring{98. Working with APIs
(\texttt{requests})}{98. Working with APIs (requests)}}\label{working-with-apis-requests}

APIs let programs talk to each other over the web. Python's
\texttt{requests} library makes sending HTTP requests simple and
readable. You can use it to fetch data, send data, or interact with web
services.

\subsubsection{Deep Dive}\label{deep-dive-97}

Making a GET Request

\begin{Shaded}
\begin{Highlighting}[]
\ImportTok{import}\NormalTok{ requests}

\NormalTok{response }\OperatorTok{=}\NormalTok{ requests.get(}\StringTok{"https://jsonplaceholder.typicode.com/posts/1"}\NormalTok{)}
\BuiltInTok{print}\NormalTok{(response.status\_code)   }\CommentTok{\# 200 means success}
\BuiltInTok{print}\NormalTok{(response.json())        }\CommentTok{\# Parse response as JSON}
\end{Highlighting}
\end{Shaded}

Query Parameters

\begin{Shaded}
\begin{Highlighting}[]
\NormalTok{url }\OperatorTok{=} \StringTok{"https://jsonplaceholder.typicode.com/posts"}
\NormalTok{params }\OperatorTok{=}\NormalTok{ \{}\StringTok{"userId"}\NormalTok{: }\DecValTok{1}\NormalTok{\}}
\NormalTok{response }\OperatorTok{=}\NormalTok{ requests.get(url, params}\OperatorTok{=}\NormalTok{params)}
\BuiltInTok{print}\NormalTok{(response.json())   }\CommentTok{\# All posts from userId=1}
\end{Highlighting}
\end{Shaded}

POST Request (Send Data)

\begin{Shaded}
\begin{Highlighting}[]
\NormalTok{data }\OperatorTok{=}\NormalTok{ \{}\StringTok{"title"}\NormalTok{: }\StringTok{"foo"}\NormalTok{, }\StringTok{"body"}\NormalTok{: }\StringTok{"bar"}\NormalTok{, }\StringTok{"userId"}\NormalTok{: }\DecValTok{1}\NormalTok{\}}
\NormalTok{response }\OperatorTok{=}\NormalTok{ requests.post(}\StringTok{"https://jsonplaceholder.typicode.com/posts"}\NormalTok{, json}\OperatorTok{=}\NormalTok{data)}
\BuiltInTok{print}\NormalTok{(response.json())}
\end{Highlighting}
\end{Shaded}

Handling Errors

\begin{Shaded}
\begin{Highlighting}[]
\NormalTok{response }\OperatorTok{=}\NormalTok{ requests.get(}\StringTok{"https://jsonplaceholder.typicode.com/invalid"}\NormalTok{)}
\ControlFlowTok{if}\NormalTok{ response.status\_code }\OperatorTok{!=} \DecValTok{200}\NormalTok{:}
    \BuiltInTok{print}\NormalTok{(}\StringTok{"Error:"}\NormalTok{, response.status\_code)}
\end{Highlighting}
\end{Shaded}

Headers and Authentication

\begin{Shaded}
\begin{Highlighting}[]
\NormalTok{headers }\OperatorTok{=}\NormalTok{ \{}\StringTok{"Authorization"}\NormalTok{: }\StringTok{"Bearer mytoken"}\NormalTok{\}}
\NormalTok{response }\OperatorTok{=}\NormalTok{ requests.get(}\StringTok{"https://api.example.com/data"}\NormalTok{, headers}\OperatorTok{=}\NormalTok{headers)}
\end{Highlighting}
\end{Shaded}

Quick Summary Table

\begin{longtable}[]{@{}lll@{}}
\toprule\noalign{}
Method & Example & Purpose \\
\midrule\noalign{}
\endhead
\bottomrule\noalign{}
\endlastfoot
GET & \texttt{requests.get(url)} & Retrieve data \\
POST & \texttt{requests.post(url,\ json=data)} & Send data \\
PUT & \texttt{requests.put(url,\ json=data)} & Update resource \\
DELETE & \texttt{requests.delete(url)} & Remove resource \\
params arg & \texttt{get(url,\ params=\{\})} & Add query string \\
headers arg & \texttt{get(url,\ headers=\{\})} & Set custom headers \\
\end{longtable}

\subsubsection{Tiny Code}\label{tiny-code-97}

\begin{Shaded}
\begin{Highlighting}[]
\ImportTok{import}\NormalTok{ requests}

\NormalTok{r }\OperatorTok{=}\NormalTok{ requests.get(}\StringTok{"https://api.github.com"}\NormalTok{)}
\BuiltInTok{print}\NormalTok{(}\StringTok{"Status:"}\NormalTok{, r.status\_code)}
\BuiltInTok{print}\NormalTok{(}\StringTok{"Headers:"}\NormalTok{, r.headers[}\StringTok{"content{-}type"}\NormalTok{])}
\end{Highlighting}
\end{Shaded}

\subsubsection{Why it Matters}\label{why-it-matters-97}

APIs are everywhere---from weather apps to payment systems. Knowing how
to interact with them lets you integrate external services into your
projects.

\subsubsection{Try It Yourself}\label{try-it-yourself-97}

\begin{enumerate}
\def\labelenumi{\arabic{enumi}.}
\tightlist
\item
  Use \texttt{requests.get} to fetch JSON from
  \texttt{https://jsonplaceholder.typicode.com/todos/1}.
\item
  Extract and print the \texttt{"title"} field from the response.
\item
  Send a \texttt{POST} request with your own JSON data.
\item
  Experiment with adding query parameters like \texttt{userId=2} to
  filter results.
\end{enumerate}

\subsection{\texorpdfstring{99. Basics of Web Scraping
(\texttt{BeautifulSoup})}{99. Basics of Web Scraping (BeautifulSoup)}}\label{basics-of-web-scraping-beautifulsoup}

Web scraping means extracting information from websites automatically.
In Python, this is commonly done using \texttt{requests} to fetch the
page and BeautifulSoup (\texttt{bs4}) to parse the HTML.

\subsubsection{Deep Dive}\label{deep-dive-98}

Installing BeautifulSoup

\begin{Shaded}
\begin{Highlighting}[]
\ExtensionTok{pip}\NormalTok{ install requests beautifulsoup4}
\end{Highlighting}
\end{Shaded}

Fetching a Webpage

\begin{Shaded}
\begin{Highlighting}[]
\ImportTok{import}\NormalTok{ requests}
\ImportTok{from}\NormalTok{ bs4 }\ImportTok{import}\NormalTok{ BeautifulSoup}

\NormalTok{url }\OperatorTok{=} \StringTok{"https://example.com"}
\NormalTok{response }\OperatorTok{=}\NormalTok{ requests.get(url)}
\NormalTok{soup }\OperatorTok{=}\NormalTok{ BeautifulSoup(response.text, }\StringTok{"html.parser"}\NormalTok{)}
\end{Highlighting}
\end{Shaded}

Extracting Data

\begin{itemize}
\tightlist
\item
  Get the page title:
\end{itemize}

\begin{Shaded}
\begin{Highlighting}[]
\BuiltInTok{print}\NormalTok{(soup.title.string)}
\end{Highlighting}
\end{Shaded}

\begin{itemize}
\tightlist
\item
  Find the first paragraph:
\end{itemize}

\begin{Shaded}
\begin{Highlighting}[]
\BuiltInTok{print}\NormalTok{(soup.p.text)}
\end{Highlighting}
\end{Shaded}

\begin{itemize}
\tightlist
\item
  Find all links:
\end{itemize}

\begin{Shaded}
\begin{Highlighting}[]
\ControlFlowTok{for}\NormalTok{ link }\KeywordTok{in}\NormalTok{ soup.find\_all(}\StringTok{"a"}\NormalTok{):}
    \BuiltInTok{print}\NormalTok{(link.get(}\StringTok{"href"}\NormalTok{))}
\end{Highlighting}
\end{Shaded}

Searching by CSS Class

\begin{Shaded}
\begin{Highlighting}[]
\NormalTok{soup.find\_all(}\StringTok{"div"}\NormalTok{, class\_}\OperatorTok{=}\StringTok{"article"}\NormalTok{)}
\end{Highlighting}
\end{Shaded}

Practical Example Scraping article headlines:

\begin{Shaded}
\begin{Highlighting}[]
\NormalTok{url }\OperatorTok{=} \StringTok{"https://news.ycombinator.com"}
\NormalTok{res }\OperatorTok{=}\NormalTok{ requests.get(url)}
\NormalTok{soup }\OperatorTok{=}\NormalTok{ BeautifulSoup(res.text, }\StringTok{"html.parser"}\NormalTok{)}

\NormalTok{titles }\OperatorTok{=}\NormalTok{ soup.find\_all(}\StringTok{"a"}\NormalTok{, class\_}\OperatorTok{=}\StringTok{"storylink"}\NormalTok{)}
\ControlFlowTok{for}\NormalTok{ t }\KeywordTok{in}\NormalTok{ titles[:}\DecValTok{5}\NormalTok{]:}
    \BuiltInTok{print}\NormalTok{(t.text)}
\end{Highlighting}
\end{Shaded}

Respect Robots.txt and Rules

\begin{itemize}
\tightlist
\item
  Always check if scraping is allowed (\texttt{/robots.txt}).
\item
  Don't overload websites with too many requests.
\end{itemize}

Quick Summary Table

\begin{longtable}[]{@{}
  >{\raggedright\arraybackslash}p{(\linewidth - 4\tabcolsep) * \real{0.4146}}
  >{\raggedright\arraybackslash}p{(\linewidth - 4\tabcolsep) * \real{0.2683}}
  >{\raggedright\arraybackslash}p{(\linewidth - 4\tabcolsep) * \real{0.3171}}@{}}
\toprule\noalign{}
\begin{minipage}[b]{\linewidth}\raggedright
Method
\end{minipage} & \begin{minipage}[b]{\linewidth}\raggedright
Example
\end{minipage} & \begin{minipage}[b]{\linewidth}\raggedright
Purpose
\end{minipage} \\
\midrule\noalign{}
\endhead
\bottomrule\noalign{}
\endlastfoot
\texttt{soup.title.string} & Get title & Page metadata \\
\texttt{soup.p.text} & Get first \texttt{\textless{}p\textgreater{}}
text & Paragraphs \\
\texttt{soup.find\_all("a")} & Extract all links & Navigation,
references \\
\texttt{soup.find\_all("div",\ class\_="x")} & Find elements by class &
Structured data extraction \\
\end{longtable}

\subsubsection{Tiny Code}\label{tiny-code-98}

\begin{Shaded}
\begin{Highlighting}[]
\ImportTok{import}\NormalTok{ requests}
\ImportTok{from}\NormalTok{ bs4 }\ImportTok{import}\NormalTok{ BeautifulSoup}

\NormalTok{res }\OperatorTok{=}\NormalTok{ requests.get(}\StringTok{"https://example.com"}\NormalTok{)}
\NormalTok{soup }\OperatorTok{=}\NormalTok{ BeautifulSoup(res.text, }\StringTok{"html.parser"}\NormalTok{)}

\BuiltInTok{print}\NormalTok{(}\StringTok{"Title:"}\NormalTok{, soup.title.string)}
\BuiltInTok{print}\NormalTok{(}\StringTok{"First paragraph:"}\NormalTok{, soup.p.text)}
\end{Highlighting}
\end{Shaded}

\subsubsection{Why it Matters}\label{why-it-matters-98}

Web scraping lets you automate data collection from websites---useful
for research, market analysis, or building datasets when APIs aren't
available.

\subsubsection{Try It Yourself}\label{try-it-yourself-98}

\begin{enumerate}
\def\labelenumi{\arabic{enumi}.}
\tightlist
\item
  Scrape the title of \texttt{https://example.com}.
\item
  Extract and print all \texttt{\textless{}h1\textgreater{}} headers
  from the page.
\item
  Collect all links (\texttt{href}) on the page.
\item
  Try scraping a news site (like Hacker News) and print the first 10
  headlines.
\end{enumerate}

\subsection{100. Next Steps: Where to Go from
Here}\label{next-steps-where-to-go-from-here}

Now that you've mastered the Python flashcards, you have the foundation
to build almost anything. The next step is to choose a direction and
deepen your skills in areas that interest you most.

\subsubsection{Deep Dive}\label{deep-dive-99}

\begin{enumerate}
\def\labelenumi{\arabic{enumi}.}
\tightlist
\item
  Data Science \& Machine Learning
\end{enumerate}

\begin{itemize}
\tightlist
\item
  Libraries: \texttt{numpy}, \texttt{pandas}, \texttt{matplotlib},
  \texttt{scikit-learn}
\item
  Learn to analyze datasets, build models, and visualize results.
\item
  Progress into deep learning with \texttt{tensorflow} or
  \texttt{pytorch}.
\end{itemize}

\begin{enumerate}
\def\labelenumi{\arabic{enumi}.}
\setcounter{enumi}{1}
\tightlist
\item
  Web Development
\end{enumerate}

\begin{itemize}
\tightlist
\item
  Frameworks: \texttt{flask}, \texttt{django}, \texttt{fastapi}
\item
  Learn to build APIs, web apps, and services.
\item
  Explore front-end integration with JavaScript frameworks.
\end{itemize}

\begin{enumerate}
\def\labelenumi{\arabic{enumi}.}
\setcounter{enumi}{2}
\tightlist
\item
  Automation \& Scripting
\end{enumerate}

\begin{itemize}
\tightlist
\item
  Use Python to automate repetitive tasks (file handling, Excel reports,
  web scraping).
\item
  Explore \texttt{selenium} for browser automation.
\end{itemize}

\begin{enumerate}
\def\labelenumi{\arabic{enumi}.}
\setcounter{enumi}{3}
\tightlist
\item
  Systems \& DevOps
\end{enumerate}

\begin{itemize}
\tightlist
\item
  Learn about Python in DevOps: \texttt{fabric}, \texttt{ansible}, or
  working with Docker/Kubernetes APIs.
\item
  Use Python for cloud services (AWS, GCP, Azure SDKs).
\end{itemize}

\begin{enumerate}
\def\labelenumi{\arabic{enumi}.}
\setcounter{enumi}{4}
\tightlist
\item
  Computer Science Foundations
\end{enumerate}

\begin{itemize}
\tightlist
\item
  Study algorithms and data structures with Python.
\item
  Explore competitive programming and problem-solving platforms
  (LeetCode, HackerRank).
\end{itemize}

Learning Pathways

\begin{itemize}
\tightlist
\item
  Books: \emph{Fluent Python}, \emph{Automate the Boring Stuff with
  Python}, \emph{Python Crash Course}.
\item
  Online platforms: Coursera, edX, freeCodeCamp.
\item
  Open-source projects: contribute on GitHub to gain real experience.
\end{itemize}

Quick Summary Table

\begin{longtable}[]{@{}
  >{\raggedright\arraybackslash}p{(\linewidth - 4\tabcolsep) * \real{0.1948}}
  >{\raggedright\arraybackslash}p{(\linewidth - 4\tabcolsep) * \real{0.3506}}
  >{\raggedright\arraybackslash}p{(\linewidth - 4\tabcolsep) * \real{0.4545}}@{}}
\toprule\noalign{}
\begin{minipage}[b]{\linewidth}\raggedright
Direction
\end{minipage} & \begin{minipage}[b]{\linewidth}\raggedright
Libraries / Tools
\end{minipage} & \begin{minipage}[b]{\linewidth}\raggedright
Example Goal
\end{minipage} \\
\midrule\noalign{}
\endhead
\bottomrule\noalign{}
\endlastfoot
Data Science & \texttt{numpy}, \texttt{pandas}, \texttt{scikit} & Build
a recommendation system \\
Web Development & \texttt{flask}, \texttt{django} & Create a blog or
API \\
Automation & \texttt{requests}, \texttt{selenium} & Automate a daily
reporting workflow \\
DevOps \& Cloud & \texttt{boto3}, \texttt{ansible} & Deploy an app to
AWS automatically \\
CS Foundations & \texttt{heapq}, \texttt{collections} & Implement
algorithms in Python \\
\end{longtable}

\subsubsection{Tiny Code (Automation
Example)}\label{tiny-code-automation-example}

\begin{Shaded}
\begin{Highlighting}[]
\ImportTok{import}\NormalTok{ requests}

\KeywordTok{def}\NormalTok{ get\_weather(city):}
\NormalTok{    url }\OperatorTok{=} \SpecialStringTok{f"https://wttr.in/}\SpecialCharTok{\{}\NormalTok{city}\SpecialCharTok{\}}\SpecialStringTok{?format=3"}
\NormalTok{    res }\OperatorTok{=}\NormalTok{ requests.get(url)}
    \ControlFlowTok{return}\NormalTok{ res.text}

\BuiltInTok{print}\NormalTok{(get\_weather(}\StringTok{"London"}\NormalTok{))}
\end{Highlighting}
\end{Shaded}

\subsubsection{Why it Matters}\label{why-it-matters-99}

Python is not just a language---it's a gateway. Whether you're
interested in AI, finance, web apps, or automating your own life, Python
is a tool that grows with you.

\subsubsection{Try It Yourself}\label{try-it-yourself-99}

\begin{enumerate}
\def\labelenumi{\arabic{enumi}.}
\tightlist
\item
  Choose one domain (web, data, AI, automation).
\item
  Install the relevant libraries
  (\texttt{pip\ install\ flask\ pandas\ torch}, etc.).
\item
  Build a small project (e.g., a to-do app, data analysis notebook, or
  web scraper).
\item
  Share your project on GitHub to start building a portfolio.
\end{enumerate}




\end{document}
